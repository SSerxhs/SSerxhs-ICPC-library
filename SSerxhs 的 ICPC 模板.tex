\documentclass[12pt]{ctexart}
\usepackage{amsfonts,amssymb}
\usepackage{booktabs}
\usepackage{graphicx}
\usepackage{amsmath}
\usepackage{listings}
\usepackage{geometry}
\usepackage[usenames,dvipsnames]{xcolor}
\usepackage{setspace}
\usepackage{hyperref}
\usepackage{float}
\setstretch{1.1} 
\geometry{a4paper}
\geometry{top=2cm} 
\geometry{bottom=1cm} 
\geometry{left=1.5cm, right=1.5cm}
\tolerance=1000
\definecolor{mygreen}{rgb}{0,0.6,0}
\definecolor{mygray}{rgb}{0.5,0.5,0.5}
\definecolor{mymauve}{rgb}{0.58,0,0.82}
\lstset{ %
backgroundcolor=\color{white},   % choose the background color
basicstyle=\footnotesize\ttfamily,        % size of fonts used for the code
columns=fullflexible,
inputencoding=utf8,
extendedchars=true, 
breaklines=true,                 % automatic line breaking only at whitespace
captionpos=b,                    % sets the caption-position to bottom
tabsize=4,
commentstyle=\color{mygreen},    % comment style
escapeinside={\%*}{*)},          % if you want to add LaTeX within your code
keywordstyle=\color{blue},       % keyword style
stringstyle=\color{mymauve}\ttfamily,     % string literal style
frame=single,
rulesepcolor=\color{red!20!green!20!blue!20},
% identifierstyle=\color{red},
language=c++,
}
\newcommand*{\dif}{\mathop{}\!\mathrm{d}}
\author{SSerxhs}
\title{SSerxhs 的 ICPC 模板}

\begin{document}

\maketitle

\centerline{ver: 3.10.2}

\tableofcontents

\newpage

\section{前言}

此模板的初衷是个人使用,因此已有的模板可能未列出。建议结合 Heltion 模板和 HDU 模板使用。

模板需要的版本为 cpp17 或 cpp20。

大部分情况下,涉及取模的都需要使用 \verb|unsigned long long|,即使类型名是 \verb|ll|。这是因为值域较大有利于合理减少取模次数。

\verb|optional| 的用法:一个 \verb|optional| 变量 \verb|r| 可以用 \verb|if (r)| 判断其是否有值。取出值的方法是 \verb|*r|。常见于包含无解又包含空集解的代码中,便于区分无解和空集解。

常见的被漏掉的初始代码:
\begin{lstlisting}
#define all(x) (x).begin(),(x).end()
using lll=__int128;
template<class T1, class T2> bool cmin(T1 &x, const T2 &y) { if (y<x) { x=y; return 1; } return 0; }
template<class T1, class T2> bool cmax(T1 &x, const T2 &y) { if (x<y) { x=y; return 1; } return 0; }
\end{lstlisting}

常见的缺漏算法:

回文自动机。

\newpage

\section{数据结构}

\subsection{树状数组}

支持单点修改、求前缀和、二分前缀和大于等于 x 的第一个位置。

二分这部分没有验证过。

\begin{lstlisting}
template<typename typC> struct bit
{
	vector<typC> a;
	int n;
	bit() { }
	bit(int nn):n(nn), a(nn+1) { }
	template<typename T> bit(int nn, T *b):n(nn), a(nn+1)
	{
		for (int i=1; i<=n; i++) a[i]=b[i];
		for (int i=1; i<=n; i++) if (i+(i&-i)<=n) a[i+(i&-i)]+=a[i];
	}
	void add(int x, typC y)
	{
		//cerr<<"add "<<x<<" by "<<y<<endl;
		assert(1<=x&&x<=n);
		a[x]+=y;
		while ((x+=x&-x)<=n) a[x]+=y;
	}
	typC sum(int x)
	{
		//cerr<<"sum "<<x;
		assert(0<=x&&x<=n);
		typC r=a[x];
		while (x^=x&-x) r+=a[x];
		//cerr<<"= "<<r<<endl;
		return r;
	}
	typC sum(int x, int y)
	{
		return sum(y)-sum(x-1);
	}
	int lower_bound(typC x)
	{
		if (n==0) return 0;
		int i=__lg(n), j=0;
		for (; i>=0; i--) if ((1<<i|j)<=n&&a[1<<i|j]<x) j|=1<<i, x-=a[j];
		return j+1;
	}
};
\end{lstlisting}

\subsection{线段树}

包含标记的线段树,支持线段树上二分,采用左闭右闭。但只支持求左侧第一个符合条件的下标。

要求:具有 \verb|info+info|,\verb|info+=tag|,\verb|tag+=tag|。\verb|info|,\verb|tag| 需要有默认构造,但不必有正确的值。

\begin{lstlisting}
	
template<class info, class tag> struct sgt
{
	int n, shift;
	info *a;
	info tmp;
	vector<info> s;
	vector<tag> tg;
	vector<int> lz;
	bool flg;
	void build(int x, int l, int r)
	{
		if (l==r)
		{
			s[x]=(flg?tmp:a[l]);
			return;
		}
		int c=x*2, m=l+r>>1;
		build(c, l, m); build(c+1, m+1, r);
		s[x]=s[c]+s[c+1];
	}
	sgt(info *b, int L, int R):n(R-L+1), shift(L-1), a(b+L-1), s(R-L+1<<2), tg(R-L+1<<2), lz(R-L+1<<2)
	{
		flg=0;
		build(1, 1, n);
	}//[L,R]
	sgt(info b, int L, int R):n(R-L+1), shift(L-1), s(R-L+1<<2), tg(R-L+1<<2), lz(R-L+1<<2)
	{
		tmp=b;
		flg=1;
		build(1, 1, n);
	}//[L,R]
	int z, y;
	info res;
	tag dt;
	bool fir;
private:
	void _modify(int x, int l, int r)
	{
		if (z<=l&&r<=y)
		{
			s[x]+=dt;
			if (lz[x]) tg[x]+=dt; else tg[x]=dt;
			lz[x]=1;
			return;
		}
		int c=x*2, m=l+r>>1;
		if (lz[x])
		{
			if (lz[c]) tg[c]+=tg[x]; else tg[c]=tg[x];
			lz[c]=1; s[c]+=tg[x]; c^=1;
			if (lz[c]) tg[c]+=tg[x]; else tg[c]=tg[x];
			lz[c]=1; s[c]+=tg[x]; c^=1;
			lz[x]=0;
		}
		if (z<=m) _modify(c, l, m);
		if (m<y) _modify(c+1, m+1, r);
		s[x]=s[c]+s[c+1];
	}
	void ask(int x, int l, int r)
	{
		if (z<=l&&r<=y)
		{
			res=fir?s[x]:res+s[x];
			fir=0;
			return;
		}
		int c=x*2, m=l+r>>1;
		if (lz[x])
		{
			if (lz[c]) tg[c]+=tg[x]; else tg[c]=tg[x];
			lz[c]=1; s[c]+=tg[x]; c^=1;
			if (lz[c]) tg[c]+=tg[x]; else tg[c]=tg[x];
			lz[c]=1; s[c]+=tg[x]; c^=1;
			lz[x]=0;
		}
		if (z<=m) ask(c, l, m);
		if (m<y) ask(c+1, m+1, r);
	}
	function<bool(info)> check;
	void find_left_most(int x, int l, int r)
	{
		if (r<z||!check(s[x])) return;
		if (l==r) { y=l; res=s[x]; return; }
		int c=x*2, m=l+r>>1;
		if (lz[x])
		{
			if (lz[c]) tg[c]+=tg[x]; else tg[c]=tg[x];
			lz[c]=1; s[c]+=tg[x]; c^=1;
			if (lz[c]) tg[c]+=tg[x]; else tg[c]=tg[x];
			lz[c]=1; s[c]+=tg[x]; c^=1;
			lz[x]=0;
		}
		find_left_most(c, l, m);
		if (y==n+1) find_left_most(c+1, m+1, r);
	}
	void find_right_most(int x, int l, int r)
	{
		if (l>y||!check(s[x])) return;
		if (l==r) { z=l; res=s[x]; return; }
		int c=x*2, m=l+r>>1;
		if (lz[x])
		{
			if (lz[c]) tg[c]+=tg[x]; else tg[c]=tg[x];
			lz[c]=1; s[c]+=tg[x]; c^=1;
			if (lz[c]) tg[c]+=tg[x]; else tg[c]=tg[x];
			lz[c]=1; s[c]+=tg[x]; c^=1;
			lz[x]=0;
		}
		find_right_most(c+1, m+1, r);
		if (z==0) find_right_most(c, l, m);
	}
public:
	void modify(int l, int r, const tag &x)//[l,r]
	{
		z=l-shift; y=r-shift; dt=x;
		// cerr<<"modify ["<<l<<','<<r<<"] "<<'\n';
		assert(1<=z&&z<=y&&y<=n);
		_modify(1, 1, n);
	}
	void modify(int pos, const info &o)
	{
		pos-=shift;
		int l=1, r=n, m, c, x=1;
		while (l<r)
		{
			c=x*2; m=l+r>>1;
			if (lz[x])
			{
				if (lz[c]) tg[c]+=tg[x]; else tg[c]=tg[x];
				lz[c]=1; s[c]+=tg[x]; c^=1;
				if (lz[c]) tg[c]+=tg[x]; else tg[c]=tg[x];
				lz[c]=1; s[c]+=tg[x]; c^=1;
				lz[x]=0;
			}
			if (pos<=m) x=c, r=m; else x=c+1, l=m+1;
		}
		s[x]=o;
		while (x>>=1) s[x]=s[x*2]+s[x*2+1];
	}
	info ask(int l, int r)//[l,r]
	{
		z=l-shift; y=r-shift; fir=1;
		// cerr<<"ask ["<<l<<','<<r<<"] "<<'\n';
		assert(1<=z&&z<=y&&y<=n);
		ask(1, 1, n);
		return res;
	}
	pair<int, info> find_left_most(int l, const function<bool(info)> &_check)
	{
		check=_check;
		z=l-shift; y=n+1;
		assert(1<=z&&z<=n+1);
		find_left_most(1, 1, n);
		return {y+shift, res};
	}
	pair<int, info> find_right_most(int r, const function<bool(info)> &_check)
	{
		check=_check;
		z=0; y=r-shift;
		assert(0<=y&&y<=n);
		find_right_most(1, 1, n);
		return {z+shift, res};
	}
};
\end{lstlisting}


\subsection{珂朵莉树}
支持区间赋值、单点访问。维护每个连续段的范围和值。

如果希望维护所有连续段的整体信息(如长度的最大值),修改 \verb|add| 和 \verb|del| 函数即可,分别表示连续段被加入和被删去。

特别注意一开始 \verb|insert| 的不会触发 \verb|add|,只有 \verb|modify| 会触发。
\begin{lstlisting}
namespace chtholly_tree
{
	using T=int;//可以把 T 修改为任意想要的类型。
	struct node
	{
		int l;
		mutable int r;
		mutable T v;
		int len() const { return r-l+1; }
		bool operator<(const node &x) const { return l<x.l; }
	};
	void add(const node &a) {}
	void del(const node &a) {}
	class odt: public set<node>
	{
	public:
		typedef odt::iterator iter;
		iter split(int x)
		{
			iter it=lower_bound({x});
			if (it!=end()&&it->l==x) return it;
			node t=*--it,a={t.l,x-1,t.v},b={x,t.r,t.v};
			del(*it); add(a); add(b);
			erase(it); insert(a);
			return insert(b).first;
		}
		void modify(int l,int r,T v)//[l,r]
		{
			iter lt,rt,it;
			rt=r==rbegin()->r?end():split(r+1); lt=split(l);//[lt,rt)
			while (lt!=begin()&&(it=prev(lt))->v==v) l=(lt=it)->l;
			while (rt!=end()&&rt->v==v) r=(rt++)->r;
			for (it=lt; it!=rt; it++) del(*it);
			add({l,r,v});
			erase(lt,rt); insert({l,r,v});
		}
		T operator[](const int x) const { return prev(upper_bound({x}))->v; }//直接访问单点
		iter find(int x) const {return prev(upper_bound({x}));}//找到对应的线段
	};
}
using chtholly_tree::node,chtholly_tree::odt;
typedef odt::iterator iter;
int main()
{
	odt s;
	s.insert({0,5,1}); 	// 先 insert({L,R,x}) 表示整个下标范围和初始值。 左闭右闭。
						// s={1,1,1,1,1,1}
	s.modify(2,3,2); 	// 左闭右闭。s={1,1,2,2,1,1}
	for (auto [l,r,v]:s)
	{
		//(l,r,v)=(0,1,1)
		//(l,r,v)=(2,3,2)
		//(l,r,v)=(4,5,1)
	}
}
\end{lstlisting}


\subsection{带删堆}

本质是额外维护一个堆 \verb|q| 表示要被删除的元素,当 \verb|p| 的最值和 \verb|q| 一样时删除。

需要保证每次 \verb|pop| 的元素都存在于堆中。

本代码的用法和 \verb|priority_queue| 一致。

\begin{lstlisting}
template<class T, class T1=vector<T>, class T2=less<T>> struct heap
{
private:
	priority_queue<T, T1, T2> p, q;
public:
	void push(const T &x)
	{
		if (!q.empty()&&q.top()==x)
		{
			q.pop();
			while (!q.empty()&&q.top()==p.top()) p.pop(), q.pop();
		}
		else p.push(x);
	}
	void pop()
	{
		p.pop();
		while (!q.empty()&&p.top()==q.top()) p.pop(), q.pop();
	}
	void pop(const T &x)
	{
		if (p.top()==x)
		{
			p.pop();
			while (!q.empty()&&p.top()==q.top()) p.pop(), q.pop();
		}
		else q.push(x);
	}
	T top() const { return p.top(); }
	int size() const { return p.size()-q.size(); }
	bool empty() const { return p.empty(); }
	vector<T> to_vector() const
	{
		vector<T> a;
		auto P=p, Q=q;
		while (P.size())
		{
			a.push_back(P.top()); P.pop();
			while (Q.size()&&P.top()==Q.top()) P.pop(), Q.pop();
		}
		return a;
	}
};

\end{lstlisting}
\subsection{前 $k$ 大的和}

本质是用小根堆维护前 $k$ 大的数,用大根堆维护其余数。

如果需要支持删除,结合前面一个使用,或者直接用 \verb|multiset| 进行 \verb|extract|。

为了方便起见,直接给出支持删除的版本,并且使用 \verb|long long|。如果不需要支持删除,类型改为优先队列并去掉 \verb|pop| 函数即可。

注意:复杂度为 $O(k-k')$,其中 $k'$ 是上一次询问的 $k$。也就是说,多组询问时询问的 $k$ 的差值应该尽可能小。

其用法与 \verb|priority_queue| 保持一致,可以用同样的方法改写成前 $k$ 小。

\begin{lstlisting}
using ll=long long;
template<class T, class T1=vector<T>, class T2=less<T>> struct ksum_pop
{
private:
	struct __cmp
	{
		bool operator()(const T &x, const T &y) const
		{
			return x!=y&&!T2()(x, y);
		}
	};
	heap<T, T1, __cmp> p;
	heap<T, T1, T2> q;
	ll cur;
public:
	ksum_pop():cur(0) { }
	void push(const T &x)
	{
		if (!q.size()||!T2()(x, q.top())) p.push(x), cur+=x; else q.push(x);
	}
	int size() const { return p.size()+q.size(); }
	void pop(const T &x)
	{
		if (q.size()&&!T2()(q.top(), x)) q.pop(x);
		else p.pop(x), cur-=x;
	}
	ll sum(int k)
	{
		while (p.size()<k)
		{
			cur+=q.top();
			p.push(q.top());
			q.pop();
		}
		while (p.size()>k)
		{
			cur-=p.top();
			q.push(p.top());
			p.pop();
		}
		return cur;
	}
};



\end{lstlisting}



\subsection{可持久化数组}

历史遗留产物,无意义,仅作留存,不会更新。

$O((n+q)\log(n))$,$O((n+q)\log (n))$。

\begin{lstlisting}
struct arr
{
	int c[M][2],rt[O],s[M],b[N];
	int ds,n,ver,v,p,i;
	void build(int &x,int l,int r)
	{
		x=++ds;
		if (l==r) {s[x]=b[l];return;}
		build(c[x][0],l,l+r>>1);
		build(c[x][1],(l+r>>1)+1,r);
	}
	void rebuild(int &x,int pre)
	{
		x=++ds;int l=1,r=n,mid,now=x;
		while (l<r)
		{
			mid=l+r>>1;
			if (mid>=p){c[now][1]=c[pre][1];now=c[now][0]=++ds;r=mid;pre=c[pre][0];} else {c[now][0]=c[pre][0];now=c[now][1]=++ds;l=mid+1;pre=c[pre][1];}
		}
		s[now]=v;
	}
	void init(int *a,int nn)
	{
		n=nn;
		for (i=1;i<=n;i++) b[i]=a[i];
		build(rt[0],1,n);
	}
	int mdf(int pv,int pos,int val)
	{
		p=pos,v=val;
		rebuild(rt[++ver],rt[pv]);
		return ver;
	}
	int ask(int ve,int pos)
	{
		int l=1,r=n,x=rt[ve],mid;
		rt[++ver]=rt[ve];
		while (l<r)
		{
			mid=l+r>>1;
			if (mid>=pos) {x=c[x][0];r=mid;} else {x=c[x][1];l=mid+1;}
		}
		return s[x];
	}
};
\end{lstlisting}

\subsection{左偏树/可并堆}

建议不要使用。\verb|pb_ds| 可以替代这个功能。我完全没有使用过这个板子。

$O((n+q)\log n)$,$O(n)$。

\begin{lstlisting}
struct left_tree//小根堆,大根堆需要改的地方注释了
{
	int jl[N],v[N],f[N],c[N][2],tf[N],n;//tf只有删非堆顶才用
	bool ed[N];
	void init(const int nn,const int *a)
	{
		jl[0]=-1;n=nn;
		memset(jl+1,0,n<<2);
		memset(tf+1,0,n<<2);//同上
		memset(c+1,0,n<<3);
		memset(ed+1,0,n);
		for (int i=1;i<=n;i++) v[f[i]=i]=a[i];
	}
	int mg(int x,int y)
	{
		if (!(x&&y)) return x|y;
		if (v[x]>v[y]||v[x]==v[y]&&x>y) swap(x,y);//改
		tf[c[x][1]=mg(c[x][1],y)]=x;//同上
		if (jl[c[x][0]]<jl[c[x][1]]) swap(c[x][0],c[x][1]);
		jl[x]=jl[c[x][1]]+1;
		return x;
	}
	int getf(int x)
	{
		if (f[x]==x) return x;
		return f[x]=getf(f[x]);
	}
	int merge(int x,int y)
	{
		if (ed[x]||ed[y]||(x=getf(x))==(y=getf(y))) return x;
		int z=mg(x,y);return f[x]=f[y]=z;
	}
	int getv(int x)//需要自行判断是否存在
	{
		return v[getf(x)];
	}
	int del(int x)//删除堆内最值
	{
		tf[c[x][0]]=tf[c[x][1]]=0;
		f[c[x][0]]=f[c[x][1]]=f[x]=mg(c[x][0],c[x][1]);
		ed[x]=1;c[x][0]=c[x][1]=tf[x]=0;return f[x];
	}
	int del_all(int x)//删除堆内非最值(没验证过)
	{
		int fa=tf[x];
		if (f[c[x][0]]==x) f[c[x][0]]=getf(tf[x]);
		if (f[c[x][1]]==x) f[c[x][1]]=f[tf[x]];
		tf[x]=tf[c[x][0]]=tf[c[x][1]]=0;
		tf[c[fa][c[fa][1]==x]=mg(c[x][0],c[x][1])]=fa;
		c[x][0]=c[x][1]=0;
		while (jl[c[fa][0]]<jl[c[fa][1]])
		{
			swap(c[fa][0],c[fa][1]);
			jl[fa]=jl[c[fa][1]]+1;
			fa=tf[fa];
		}
	}
	void out(int n)
	{
		for (int i=1;i<=n;i++) printf("%d: c%d&%d f%d v%d\n",i,c[i][0],c[i][1],f[i],v[i]);
	}
};
\end{lstlisting}

\subsection{树状数组区间加区间求和}

本质:$a_n$ 区间加等价于差分数组 $d_n$ 的单点加。

$\sum\limits_{i=1}^ma_i =\sum\limits_{i=1}^m\sum\limits_{j=1}^id_j=\sum\limits_{j=1}^md_j(m-j+1)=((m+1)\sum\limits_{j=1}^md_j)-(\sum\limits_{j=1}^mjd_j)$。

分别维护 $d_j$ 和 $jd_j$ 的前缀和。

$O(n)\sim O(q\log n)$,$O(n)$。

\begin{lstlisting}
struct bit
{
	ll a[N],b[N],s[N];//有初始值
	int n;
	void init(int nn,int *a)//初始值
	{
		n=nn;s[0]=0;
		for (int i=1;i<=n;i++) s[i]=s[i-1]+a[i];
	}
	void mdf(int l,int r,ll dt)
	{
		int i;++r;
		ll j=dt*l;
		a[l]+=dt;b[l]+=j;
		while ((l+=l&-l)<=n)
		{
			a[l]+=dt;
			b[l]+=j;
		}
		if (r<=n)
		{
			j=dt*r;
			a[r]-=dt;b[r]-=j;
			while ((r+=r&-r)<=n)
			{
				a[r]-=dt;
				b[r]-=j;
			}
		}
	}
	ll presum(int x)
	{
		ll r=a[x],rr=b[x];
		int y=x;
		while (x^=x&-x)
		{
			r+=a[x];
			rr+=b[x];
		}
		return r*(y+1)-rr+s[y];
	}
	ll sum(int l,int r)
	{
		return presum(r)-presum(l-1);
	}
};
\end{lstlisting}

\subsection{二维树状数组矩形加矩形求和}

本质还是差分,只不过这次要维护 $d_{i,j},d_{i,j}i,d_{i,j}j,d_{i,j}ij$。

$O(n^2)\sim O(q\log^2n)$,$O(n^2)$

\begin{lstlisting}
struct bit2
{
	ll a[2050][2050],b[2050][2050],c[2050][2050],d[2050][2050];
	int n,m;
	private:
	void cha(ll a[][2050],int x,int y,int z)
	{
		int i,j;
		for (i=x;i<=n;i+=(i&(-i))) for (j=y;j<=m;j+=(j&(-j))) a[i][j]+=z;
	}
	ll he(int x,int y)
	{
		if ((x<=0)||(y<=0)) return 0;
		int i,j;
		ll z=0,w=0;
		for (i=x;i;i-=(i&(-i))) for (j=y;j;j-=(j&(-j))) z+=a[i][j];
		z*=(x+1)*(y+1);
		w=0;
		for (i=x;i;i-=(i&(-i))) for (j=y;j;j-=(j&(-j))) w+=b[i][j];
		z-=w*(y+1);
		w=0;
		for (i=x;i;i-=(i&(-i))) for (j=y;j;j-=(j&(-j))) w+=c[i][j];
		z-=w*(x+1);
		for (i=x;i;i-=(i&(-i))) for (j=y;j;j-=(j&(-j))) z+=d[i][j];
		return z;
	}
	public:
	void init(int x,int y)
	{
		n=x;m=y;
	}
	void add(int u,int v,int x,int y,int z)//(x1,y1,x2,y2,dt)
	{
		cha(a,u,v,z);
		cha(b,u,v,u*z);//小心乘爆
		cha(c,u,v,v*z);
		cha(d,u,v,u*v*z);
		++x;++y;
		if (x<=n)
		{
			cha(a,x,v,-z);
			cha(b,x,v,-z*x);
			cha(c,x,v,-z*v);
			cha(d,x,v,-z*x*v);
		}
		if (y<=m)
		{
			cha(a,u,y,-z);
			cha(b,u,y,-z*u);
			cha(c,u,y,-z*y);
			cha(d,u,y,-z*u*y);
			if (x<=n)
			{
				cha(a,x,y,z);
				cha(b,x,y,z*x);
				cha(c,x,y,z*y);
				cha(d,x,y,z*x*y);
			}
		}
	}
	ll sum(int u,int v,int x,int y)//(x1,y1,x2,y2)
	{
		--u;--v;
		return (he(x,y)+he(u,v)-he(u,y)-he(x,v));
	}
};
\end{lstlisting}

\subsection{带修莫队(功能:区间数有多少种不同的数字)}

按照 $n^{\frac{2}{3}}$ 分块,排序关键字是 $l,r,t$ 所在的块($t$ 是版本号,每次修改都会增加一个版本),可以奇偶分块优化。

相比于传统莫队多了一个 \verb|modify|。

$O(n^{\frac {5}{3}})$,$O(n)$。

\begin{lstlisting}
#include "bits/stdc++.h"
using namespace std;
typedef long long ll;
#define all(x) (x).begin(),(x).end()
const int N=1.4e5,M=1e6+2;
int a[N],ans[N],bel[N],cnt[M],sum,z,y,cur;
struct P
{
	int p,v;
};
struct Q
{
	int l,r,t,p;
	bool operator<(const Q &o) const
	{
		if (bel[l]!=bel[o.l]) return bel[l]<bel[o.l];
		if (bel[r]!=bel[o.r]) return (bel[l]&1)^bel[r]<bel[o.r];
		return (bel[r]&1)?t<o.t:t>o.t;
	}
};
Q b[N];
P d[N];
void add(const int &x) {sum+=!(cnt[a[x]]++);}
void del(const int &x) {sum-=!(--cnt[a[x]]);}
void mdf(const int &x)
{
	auto &[p,v]=d[x];
	if (z<=p&&p<=y) del(p);
	swap(a[p],v);
	if (z<=p&&p<=y) add(p);
}
int main()
{
	ios::sync_with_stdio(0);cin.tie(0);
	int n,m,q1=0,q2=0,i,ksiz;
	cin>>n>>m;
	for (i=1;i<=n;i++) cin>>a[i];
	for (i=1;i<=m;i++)
	{
		char c;
		int l,r;
		cin>>c>>l>>r;
		if (c=='Q') ++q1,b[q1]={l,r,q2,q1};
		else d[++q2]={l,r};
	}
	ksiz=max(1.0,round(cbrt((ll)n*n)));
	for (i=1;i<=n;i++) bel[i]=i/ksiz;
	sort(b+1,b+q1+1);
	z=b[1].l;y=z-1;cur=0;
	for (i=1;i<=q1;i++)
	{
		auto [l,r,t,p]=b[i];
		while (z>l) add(--z);
		while (y<r) add(++y);
		while (z<l) del(z++);
		while (y>r) del(y--);
		while (cur<t) mdf(++cur);
		while (cur>t) mdf(cur--);
		ans[p]=sum;
	}
	for (i=1;i<=q1;i++) cout<<ans[i]<<'\n';
}

\end{lstlisting}

\subsection{二次离线莫队}

直接摘录题解,用途不大。

$O(n\sqrt n)$,$O(n)$。

珂朵莉给了你一个序列 $a$,每次查询给一个区间 $[l,r]$,查询 $l \leq i< j \leq r$,且 $a_i \oplus a_j$ 的二进制表示下有 $k$ 个 $1$ 的二元组 $(i,j)$ 的个数。$\oplus$ 是指按位异或。

二次离线莫队,通过扫描线,再次将更新答案的过程离线处理,降低时间复杂度。假设更新答案的复杂度为 $O(k)$,它将莫队的复杂度从 $O(nk\sqrt n)$ 降到了 $O(nk + n\sqrt n)$,大大简化了计算。
设 $x$ 对区间 $[l,r]$ 的贡献为 $f(x,[l,r])$,我们考虑区间端点变化对答案的影响:以 $[l..r]$ 变成 $[l..(r+k)]$ 为例,$\forall x \in [r+1,r+k]$ 求 $f(x,[l,x-1])$。
我们可以进行差分:$f(x,[l,x-1])=f(x,[1,x-1])-f(x,[1,l-1])$,这样转化为了一个数对一个前缀的贡献。保存下来所有这样的询问,从左到右扫描数组计算就可以了。
但是这样做,空间是 $O(n\sqrt n)$ 的,不太优秀,而且时间常数巨大。。
这样的贡献分为两类:

1. 减号左边的贡献永远是一个前缀 和它后面一个数的贡献。这可以预处理出来。
2. 减号右边的贡献对于一次移动中所有的 $x$ 来说,都是不变的。我们打标记的时候,可以只标记左右端点。

这样,减小时间常数的同时,空间降为了 $O(n)$ 级别。是一个很优秀的算法了。处理前缀询问的时候,我们利用异或运算的交换律,即 $a~\mathrm{xor}~b=c \Longleftrightarrow a~\mathrm{xor}~c=b$
开一个桶 $t$,$t[i]$ 表示当前前缀中与 $i$ 异或有 $k$ 个数位为 $1$ 的数有多少个。
则每加入一个数 $a[i]$,对于所有 $\mathrm{popcount}(x)=k$ 的 $x$,$t[a[i]\operatorname{xor}x]\gets t[a[i]\operatorname{xor}x]+1$ 即可。

\begin{lstlisting}
typedef long long ll;
const int N = 1e5 + 2, M = 1 << 14;
ll f[N], ans[N], ta[N];
int a[N], cnt[M], bel[N], pc[M], st[N];
int n, m, ksiz;
struct Q
{
	int z, y, wz;
	bool operator<(const Q &x) const { return (bel[z] < bel[x.z]) || (bel[z] == bel[x.z]) && ((y < x.y) && (bel[z] & 1) || (y > x.y) && (1 ^ bel[z] & 1)); }
};
Q mq(const int x, const int y, const int z)
{
	Q a;
	a.z = x; a.y = y; a.wz = z;
	return a;
}
Q q[N];
vector<Q> b[N];
int main()
{
	ios::sync_with_stdio(false);
	cin.tie(0);
	int i, j, k, l = 1, r = 0, tp = 0, x, na;
	cin >> n >> m >> k; ksiz = sqrt(n);
	for (i = 1; i <= n; i++) { cin >> a[i]; bel[i] = (i - 1) / ksiz + 1; }
	if (k == 0) st[++tp] = 0;
	for (i = 1; i < 16384; i++)
	{
		if (i & 1) pc[i] = pc[i >> 1] + 1; else pc[i] = pc[i >> 1];
		if (pc[i] == k) st[++tp] = i;
	}
	for (i = 1; i <= n; i++)
	{
		j = tp + 1; f[i] = f[i - 1];
		while (--j) f[i] += cnt[st[j] ^ a[i]];
		++cnt[a[i]];
	}
	for (i = 1; i <= m; i++) { cin >> q[i].z >> q[q[i].wz = i].y; }
	sort(q + 1, q + m + 1);
	for (i = 1; i <= m; i++)
	{
		ans[i] = f[q[i].y] - f[r] + f[q[i].z - 1] - f[l - 1];
		if (k == 0) ans[i] += q[i].z - l;
		if (r < q[i].y)
		{
			b[l - 1].push_back(mq(r + 1, q[i].y, -i));
			r = q[i].y;
		}
		if (l > q[i].z)
		{
			b[r].push_back(mq(q[i].z, l - 1, i));
			l = q[i].z;
		}
		if (r > q[i].y)
		{
			b[l - 1].push_back(mq(q[i].y + 1, r, i));
			r = q[i].y;
		}
		if (l < q[i].z)
		{
			b[r].push_back(mq(l, q[i].z - 1, -i));
			l = q[i].z;
		}
	}
	memset(cnt, 0, sizeof(cnt));
	for (i = 1; i <= n; i++)
	{
		j = tp + 1; x = a[i];
		while (--j) ++cnt[x ^ st[j]];
		for (j = 0; j < b[i].size(); j++)
		{
			na = 0; l = b[i][j].z; r = b[i][j].y;
			for (k = l; k <= r; k++) na += cnt[a[k]];
			if (b[i][j].wz > 0) ans[b[i][j].wz] += na; else ans[-b[i][j].wz] -= na;
		}
	}
	for (i = 2; i <= m; i++) ans[i] += ans[i - 1];
	for (i = 1; i <= m; i++) ta[q[i].wz] = ans[i];
	for (i = 1; i <= m; i++) printf("%lld\n", ta[i]);
}

\end{lstlisting}

\subsection{回滚莫队}

不删除的莫队,比如求 $\max$。

做法:块内询问暴力。对于 $l$ 所在块相同的询问,按照 $r$ 升序排序,并且将左指针固定在 $l$ 所在块的最右侧。(由于块内询问暴力,这不会导致左指针更大)

回答每个询问的时候,先右端点右移到 $r$,然后左端点左移到 $l$。询问完成后,把左端点移回去。移回去的过程虽然涉及删除,但不需要维护答案变成什么了(因为在左端点左移之前已经求过了)。换句话说,相当于“撤销”而不是删除,完全可以记录移动过程中的所有变化来撤销。

$O(n\sqrt n)$,$O(n)$。

\begin{lstlisting}
#include "bits/stdc++.h"
using namespace std;
const int N = 2e5 + 2;
int a[N], z[N], y[N], wz[N], b[N], d[N], bel[N], ans[N], st[N][2], pos[N][2];
void qs(int l, int r)
{
	int i = l, j = r, m = bel[z[l + r >> 1]], mm = y[l + r >> 1];
	while (i <= j)
	{
		while ((bel[z[i]] < m) || (bel[z[i]] == m) && (y[i] < mm)) ++i;
		while ((bel[z[j]] > m) || (bel[z[j]] == m) && (y[j] > mm)) --j;
		if (i <= j)
		{
			swap(wz[i], wz[j]);
			swap(z[i], z[j]);
			swap(y[i++], y[j--]);
		}
	}
	if (i < r) qs(i, r);
	if (l < j) qs(l, j);
}
int main()
{
	ios::sync_with_stdio(false);
	cin.tie(0);
	cin >> n;
	ksiz = sqrt(n);
	for (i = 1; i <= n; i++) { cin >> a[i]; b[i] = a[i]; bel[i] = (i - 1) / ksiz + 1; }
	sort(b + 1, b + n + 1);
	d[gs = 1] = b[1];
	for (i = 2; i <= n; i++) if (b[i] != b[i - 1]) d[++gs] = b[i];
	for (i = 1; i <= n; i++) a[i] = lower_bound(d + 1, d + gs + 1, a[i]) - d;
	cin >> m; assert(int(n / sqrt(m)));
	for (i = 1; i <= m; i++)  cin >> z[i] >> y[wz[i] = i];
	qs(1, m);
	for (i = 1; i <= m; i++)
	{
		if (bel[z[i]] > bel[z[i - 1]])
		{
			while (l <= r) { pos[a[l]][0] = pos[a[l]][1] = 0; ++l; }na = 0;
			if (bel[z[i]] == bel[y[i]])
			{
				for (j = z[i]; j <= y[i]; j++) if (pos[a[j]][0]) na = max(na, j - pos[a[j]][0]); else pos[a[j]][0] = j;
				ans[wz[i]] = na; for (j = z[i]; j <= y[i]; j++) pos[a[j]][0] = 0; na = 0; l = ksiz * bel[z[i]]; r = l - 1;
				continue;
			}
			l = ksiz * bel[z[i]]; r = l - 1; na = 0;
		}
		if (bel[z[i]] == bel[y[i]])
		{
			while (l <= r) { pos[a[l]][0] = pos[a[l]][1] = 0; ++l; }na = 0;
			for (j = z[i]; j <= y[i]; j++) if (pos[a[j]][0]) na = max(na, j - pos[a[j]][0]); else pos[a[j]][0] = j;
			ans[wz[i]] = na; for (j = z[i]; j <= y[i]; j++) pos[a[j]][0] = 0;
			l = ksiz * bel[z[i]]; r = l - 1; na = 0;
			continue;
		}
		while (r < y[i])
		{
			x = a[++r]; pos[x][1] = r;
			if (!pos[x][0]) pos[x][0] = r; else na = max(na, r - pos[x][0]);
		}c = na;
		while (l > z[i])
		{
			x = a[--l]; st[++tp][0] = x; st[tp][1] = pos[x][0];
			pos[x][0] = l;
			if (!pos[x][1])
			{
				st[++tp][0] = x + n; st[tp][1] = 0;
				pos[x][1] = l;
			}
			else na = max(na, pos[x][1] - l);
		}
		ans[wz[i]] = na; na = c; ++tp; l = ksiz * bel[z[i]];
		while (--tp) if (st[tp][0] <= n) pos[st[tp][0]][0] = st[tp][1]; else pos[st[tp][0] - n][1] = st[tp][1];
	}
	for (i = 1; i <= m; i++) cout << ans[i] << "\n";
}

\end{lstlisting}

\subsection{李超树}

题意:插入线段,查询某个 $x$ 的最大 $y$(输出最小编号)

算法核心:修改时,线段树每个点只维护在中点取值最大的线段,中点取值较小的线段只会在至多一侧有用,递归下去插入,复杂度 $O(\log^2)$。查询时询问线段树上 $\log$ 个点的线段中最大的。

\begin{lstlisting}
struct Q
{
	int x0,y0,dx,dy,id;
	Q():x0(0),y0(-1),dx(1),dy(0),id(-1){}//y>=0
	Q(int a,int b,int c,int d,int e):x0(a),y0(b),dx(c),dy(d),id(e){}
	bool contains(const int &x) const {return x0<=x&&x<=x0+dx;}
};
bool cmp(const Q &a,const Q &b,int x)//小心数值爆炸
{
	ll A=((ll)a.y0*a.dx+(ll)(x-a.x0)*a.dy)*b.dx,B=((ll)b.y0*b.dx+(ll)(x-b.x0)*b.dy)*a.dx;
	if (A!=B) return A<B;
	return a.id>b.id;
}
bool cmp2(const Q &a,const Q &b)
{
	if (a.y0+a.dy!=b.y0+b.dy) return a.y0+a.dy<b.y0+b.dy;
	return a.id>b.id;
}
const int inf=1e9;
int ans;
namespace seg
{
	const int N=4e4+2,M=N*4;
	Q s[M],X[N];
	int n,z,y;
	void init(int nn) {n=nn;for (int i=1;i<=n*4;i++) s[i]=Q();}
	void insert(int x,int l,int r,Q dt)
	{
		int c=x*2,m=l+r>>1;
		if (z<=l&&r<=y)
		{
			if (cmp(s[x],dt,m)) swap(s[x],dt);
			if (l==r) return;
			if (cmp(s[x],dt,l)) insert(c,l,m,dt);
			else if (cmp(s[x],dt,r)) insert(c+1,m+1,r,dt);
			return;
		}
		if (z<=m) insert(c,l,m,dt);
		if (y>m) insert(c+1,m+1,r,dt);
	}
	void insert(const Q &o)
	{
		z=o.x0;y=z+o.dx;
		assert(1<=z&&z<=y&&y<=n);
		if (z==y)
		{
			if (cmp2(X[z],o)) X[z]=o;
			return;
		}
		insert(1,1,n,o);
	}
	Q askmax(int p)
	{
		Q ans=s[1].contains(p)?s[1]:Q();
		int x=1,l=1,r=n,c,m;
		while (l<r)
		{
			c=x*2,m=l+r>>1;
			if (p<=m) x=c,r=m; else x=c+1,l=m+1;
			if (s[x].contains(p)&&cmp(ans,s[x],p)) ans=s[x];
		}
		Q o(X[p].x0,X[p].y0+X[p].dy,1,0,0);
		return cmp(ans,o,p)?X[p]:ans;
	}
}
int main()
{
	ios::sync_with_stdio(0);cin.tie(0);
	cout<<setiosflags(ios::fixed)<<setprecision(15);
	int n=4e4,m,i;
	seg::init(n);
	cin>>m;
	while (m--)
	{
		int op;
		cin>>op;
		if (op)
		{
			int x[2],y[2];
			cin>>x[0]>>y[0]>>x[1]>>y[1];
			for (int &v:x) v=(v+ans-1)%39989+1;
			for (int &v:y) v=(v+ans-1)%inf+1;
			if (x[0]>x[1]||x[0]==x[1]&&y[0]>y[1]) swap(x[0],x[1]),swap(y[0],y[1]);
			static int id;
			seg::insert({x[0],y[0],x[1]-x[0],y[1]-y[0],++id});
		}
		else
		{
			int x;
			cin>>x;
			x=(x+ans-1)%39989+1;
			cout<<(ans=max(0,seg::askmax(x).id))<<'\n';
		}
	}
}
\end{lstlisting}

\subsection{李超树(动态开点)}
\begin{lstlisting}
struct Q
{
	int k;
	ll b;
	ll y(const int &x) const {return (ll)k*x+b;}
};
const int inf=1e9;
const ll INF=1e18;
struct seg//可以析构,不能并行
{
	const static int N=4e5+2,M=N*8*8+(1<<23);
	const static ll npos=9e18;
	static Q s[M];
	static int c[M][2],id;
	int z,y,L,R;
	seg(int l,int r)
	{
		L=l;R=r;id=1;
		s[1]={0,npos};
		assert(L<=R&&(ll)R-L<1ll<<32);
	}
private:
	void insert(int &x,int l,int r,Q o)
	{
		if (!x)
		{
			x=++id;
			assert(id<M);
			s[x]={0,npos};
		}
		int m=l+(r-l>>1);
		if (z<=l&&r<=y)
		{
			if (s[x].y(m)>o.y(m)) swap(s[x],o);
			if (s[x].y(l)>o.y(l)) insert(c[x][0],l,m,o);
			else if (s[x].y(r)>o.y(r)) insert(c[x][1],m+1,r,o);
			return;
		}
		if (z<=m) insert(c[x][0],l,m,o);
		if (y>m) insert(c[x][1],m+1,r,o);
	}
public:
	void insert(const Q &x,const int &l,const int &r)//[l,r]
	{
		z=l;y=r;int tmp=1;
		insert(tmp,L,R,x);
		assert(tmp==1);
	}
	ll askmin(const int &p)
	{
		ll res=s[1].y(p);
		int l=L,r=R,m,x=1;
		while (l<r)
		{
			m=l+(r-l>>1);
			if (p<=m) x=c[x][0],r=m; else x=c[x][1],l=m+1;
			if (!x) return res;
			res=min(res,s[x].y(p));
		}
		return res;
	}
	~seg()
	{
		++id;
		while (--id) c[id][0]=c[id][1]=0;
	}
};
Q seg::s[seg::M];
int seg::c[seg::M][2],seg::id;
\end{lstlisting}


\subsection{区间线性基}

$O((n+q)\log a)$,$O(n\log a)$。

\begin{lstlisting}
template<class T,int M=sizeof(T)*8> struct base//线性基
{
	array<T,M> a;
	base():a{ } { }
	bool insert(T x)//线性基插入
	{
		if (x==0) return 0;
		for (int i=__lg(x); x; i=__lg(x))
		{
			if (!a[i])
			{
				a[i]=x;
				return 1;
			}
			x^=a[i];
		}
		return 0;
	}
	base &operator+=(const base &o)//合并线性基
	{
		for (ll x:o.a) if (x) insert(x);
		return *this;
	}
	base operator+(base o) const { return o+=*this; }//合并线性基
	bool contains(T x) const//查询是否能 xor 出 x
	{
		if (x==0) return 1;
		for (int i=__lg(x); x; i=__lg(x))
		{
			if (!a[i]) return 0;
			x^=a[i];
		}
		return 1;
	}
	T max(T x=0) const//查询子集 xor 的最大值。若有传入参数 x,表示子集 xor x 的最大值。 
	{
		for (int i=M-1; i>=0; i--) if (1^x>>i&1) x^=a[i];
		return x;
	}
};
template<class T=ll,int M=sizeof(T)*8> struct rangebase//[0,...)
{
	vector<array<pair<T,int>,M>> a;
	rangebase():a{{ }} { }
	rangebase(const vector<T> &b):a{{ }} { for (T x:b) insert(x); }//直接用一个 vector 构造
	void push_back(T x)//在最后插入 x
	{
		int n=a.size()-1;
		a.push_back(a.back());
		if (x==0) return;
		for (int i=__lg(x); x; i=__lg(x))
		{
			auto &[v,p]=a.back()[i];
			if (v)
			{
				if (n>p)
				{
					swap(x,v);
					swap(n,p);
				}
				x^=v;
			}
			else
			{
				v=x;
				p=n;
				return;
			}
		}
	}
	base<T,M> ask(int l,int r)//查询 $[l,r)$ 元素构成的线性基。下标从 0 开始(同 vector)
	{
		assert(0<=l&&l<=r&&r<=a.size());
		base<T,M> res;
		for (int i=0; i<M; i++)
		{
			auto [v,p]=a[r][i];
			if (v&&p>=l) res.a[i]=v;
		}
		return res;
	}
};
\end{lstlisting}

\subsection{splay 重构}

$O(n)$,$O((n+q)\log n)$。

\begin{lstlisting}
template<class info,class tag> struct splay
{
#define _rev
	struct node
	{
		node *c[2],*f;
		int siz;
		info s,v;
		tag t;
		node():c{},f(0),siz(1),s(),v(),t() {}
		node(info x):c{},f(0),siz(1),s(x),v(x),t() {}
		void operator+=(const tag &o)
		{
			s+=o; v+=o; t+=o;
#ifdef _rev
			if (o.rev) swap(c[0],c[1]);
#endif
		}
		void pushup()
		{
			if (c[0]) s=c[0]->s+v,siz=c[0]->siz+1; else s=v,siz=1;
			if (c[1]) s=s+c[1]->s,siz+=c[1]->siz;
		}
		void pushdown()
		{
			for (auto x:c) if (x) *x+=t;
			t={};
		}
		void zigzag()
		{
			node *y=f,*z=y->f;
			int typ=y->c[0]==this;
			if (z) z->c[z->c[1]==y]=this;
			f=z; y->f=this;
			y->c[typ^1]=c[typ];
			if (c[typ]) c[typ]->f=y;
			c[typ]=y;
			y->pushup();
		}
		void splay(node *tar)//不要在 makeroot 以外调用
		{
			for (node *y=f; y!=tar; zigzag(),y=f) if (node *z=y->f; z!=tar) (z->c[1]==y^y->c[1]==this?this:y)->zigzag();
			pushup();
		}
		void clear()
		{
			for (node *x:c) if (x) x->clear();
			delete this;
		}
	};
	node *rt;
	void debug()
	{
		map<node *,int> id;
		id[0]=0; id[rt]=1;
		int cnt=1;
		function<void(node *)> out=[&](node *x)
		{
			if (!x) return;
			for (auto y:x->c) if (!id.count(y)) id[y]=++cnt;
			cerr<<id[x]<<' '<<id[x->c[0]]<<' '<<id[x->c[1]]<<' '<<id[x->f]<<' '<<x->siz<<'\n';
			for (auto y:x->c) out(y);
		};
		out(rt);
	}
	node *build(info *a,int n)
	{
		if (n==0) return 0;
		int m=n-1>>1;
		node *x=new node(a[m]);
		x->c[0]=build(a,m);
		x->c[1]=build(a+m+1,n-1-m);
		for (node *y:x->c) if (y) y->f=x;
		x->pushup();
		return x;
	}
	splay()
	{
		rt=new node;
		rt->c[1]=new node;
		rt->c[1]->f=rt;
		rt->siz=2;
	}
	int shift;
	splay(info *a,int l,int r)//[l,r)
	{
		shift=l-1;
		rt=new node;
		rt->c[1]=new node;
		rt->c[1]->f=rt;
		if (l<r)
		{
			rt->c[1]->c[0]=build(a+l,r-l);
			rt->c[1]->c[0]->f=rt->c[1];
		}
		rt->c[1]->pushup();
		rt->pushup();
	}
	void makeroot(node *u,node *tar)
	{
		if (!tar) rt=u;
		u->splay();
	}
	void findnth(int k,node *tar)
	{
		node *x=rt;
		while (1)
		{
			x->pushdown();
			int v=x->c[0]?x->c[0]->siz:0;
			if (v+1==k) { x->splay(tar); if (!tar) rt=x; return; }
			if (v>=k) x=x->c[0]; else x=x->c[1],k-=v+1;
		}
	}
	void split(int l,int r)
	{
		assert(1<=l&&r<=rt->siz-2&&l-1<=r);
		findnth(l,0);
		findnth(r+2,rt);
	}
#ifdef _rev
	void reverse(int l,int r)
	{
		l-=shift; r-=shift+1;
		if (l-1==r) return;
		assert(1<=l&&l<=r&&r<=rt->siz-2);
		split(l,r);
		*(rt->c[1]->c[0])+=tag(1);
	}
#endif
	void insert(int pos,info x)//insert before pos
	{
		pos-=shift;
		assert(1<=pos&&pos<=rt->siz-1);
		split(pos,pos-1);
		rt->c[1]->c[0]=new node(x);
		rt->c[1]->c[0]->f=rt->c[1];
		rt->c[1]->pushup();
		rt->pushup();
	}
	void insert(int pos,info *a,int n)//insert before pos, [1,n]
	{
		pos-=shift;
		assert(1<=pos&&pos<=rt->siz-1);
		split(pos,pos-1);
		rt->c[1]->c[0]=build(a,n);
		rt->c[1]->c[0]->f=rt->c[1];
		rt->c[1]->pushup();
		rt->pushup();
	}
	void erase(int pos)
	{
		pos-=shift;
		assert(1<=pos&&pos<=rt->siz-2);
		split(pos,pos);
		delete rt->c[1]->c[0];
		rt->c[1]->c[0]=0;
		rt->c[1]->pushup();
		rt->pushup();
	}
	void erase(int l,int r)
	{
		l-=shift;  r-=shift+1;
		if (l-1==r) return;
		assert(1<=l&&l<=r&&r<=rt->siz-2);
		split(l,r);
		rt->c[1]->c[0]->clear();
		rt->c[1]->c[0]=0;
		rt->c[1]->pushup();
		rt->pushup();
	}
	void modify(int pos,info x)//not checked
	{
		pos-=shift;
		assert(1<=pos&&pos<=rt->siz-2);
		findnth(pos+1,0);
		rt->v=x; rt->pushup();
	}
	void modify(int l,int r,tag w)
	{
		l-=shift; r-=shift+1;
		if (l-1==r) return;
		assert(1<=l&&l<=r&&r<=rt->siz-2);
		split(l,r);
		node *x=rt->c[1]->c[0];
		*x+=w;
		rt->c[1]->pushup();
		rt->pushup();
	}
	info ask(int l,int r)
	{
		l-=shift; r-=shift+1;
		assert(1<=l&&l<=r&&r<=rt->siz-2);
		split(l,r);
		return rt->c[1]->c[0]->s;
	}
	~splay() { rt->clear(); }
#undef _rev
};
struct Q
{
	bool rev;
	Q():rev(0) {}
	Q(bool c):rev(c) {}
	void operator+=(const Q &o)
	{
		rev^=o.rev;
	}
};
struct P
{
	ll s;
	void operator+=(const Q &o) const
	{
	}
	P operator+(const P &o) const { return{s+o.s}; }
};

\end{lstlisting}

\subsection{第 $k$ 大线性基}

注意数字大于 $2^{50}$ 时可能要修改循环范围。

$O((n+q)\log a)$,$O(\log a)$。

\begin{lstlisting}
void ins(ll x)
{
	if (x==0) return con=1,void();//con=1:有0
	int i;
	for (i=50;x;i--) if (x>>i&1)
	{
		if (!ji[i]) {ji[i]=x;i=-1;break;}x^=ji[i];
	}
	if (!x) con=1;
}
ll kmax(ll x)//查询第 k 大(本质不同,不允许空集)的 xor 结果,若有初始值改 r 即可
{
	ll r=0;
	int m=0,i;
	for (i=50;~i;i--) if (ji[i]) a[++m]=i;
	if (1ll<<m<=x-con) return -1;//个数少于k
	x=(1ll<<m)-x;
	for (i=1;i<=m;i++) if ((x>>m-i^r>>a[i])&1) r^=ji[a[i]];
	return r;
}
ll kmin(ll x)//查询第 k 小(本质不同,不允许空集)的 xor 结果,若有初始值改 r 即可
{
	ll r=0;
	int m=0,i;
	for (i=50;~i;i--) if (ji[i]) a[++m]=i;
	x-=con;
	if (1ll<<m<=x) return -1;//个数少于k
	for (i=1;i<=m;i++) if ((x>>m-i^r>>a[i])&1) r^=ji[a[i]];
	return r;
}

\end{lstlisting}

\subsection{fhq-treap}
洛谷模板:普通平衡树。

$O((n+q)\log n)$,$O(n)$。

\begin{lstlisting}
const int N = 1.1e6 + 2;
int c[N][2], v[N], w[N], s[N];
int n, i, x, y, ds, val, kth, p, q, z, rt, la, m, ans;
void pushup(const int x)
{
	s[x] = s[c[x][0]] + s[c[x][1]] + 1;
}
void split_val(int now, int &x, int &y)//调用外部val,相等归入y
{
	if (!now) return x = y = 0, void();
	if (val <= v[now]) split_val(c[y = now][0], x, c[now][0]);
	else split_val(c[x = now][1], c[now][1], y);
	pushup(now);
}
void split_kth(int now, int &x, int &y)//调用外部kth,左子树大小为 kth
{
	if (!now) return x = y = 0, void();
	if (kth <= s[c[now][0]]) split_kth(c[y = now][0], x, c[now][0]);
	else kth -= s[c[now][0]] + 1, split_kth(c[x = now][1], c[now][1], y);
	pushup(now);
}
int merge(int x, int y)//小根ver.
{
	if (!(x && y)) return x | y;
	if (w[x] < w[y]) { c[x][1] = merge(c[x][1], y); pushup(x); return x; }
	else { c[y][0] = merge(x, c[y][0]); pushup(y); return y; }
}
int main()
{
	cin>>n>>m; srand(998244353);
	for (i = 1; i <= n; i++)
	{
		cin >> x;
		val = v[++ds] = x;
		w[ds] = rand();
		s[ds] = 1;
		split_val(rt, p, q);
		rt = merge(merge(p, ds), q);
	}
	while (m--)
	{
		cin >> y >> x;
		x ^= la;
		if (y == 4)//找到第 x 小的
		{
			kth = x; split_kth(rt, p, q); x = p;
			while (c[x][1]) x = c[x][1];
			ans ^= (la = v[x]); rt = merge(p, q);
			continue;
		}
		val = x;//注意这一步
		if (y == 1)//插入 x
		{
			v[++ds] = x; w[ds] = rand(); s[ds] = 1;
			split_val(rt, p, q); rt = merge(merge(p, ds), q);
			continue;
		}
		if (y == 2)//删除一个 x
		{
			split_val(rt, p, q); kth = 1; split_kth(q, i, z);
			rt = merge(p, z); continue;
		}
		if (y == 3)//询问 x 的排名(比 x 小的数字个数 +1)
		{
			split_val(rt, p, q); ans ^= (la = s[p] + 1);
			rt = merge(p, q); continue;
		}
		if (y == 5)//询问比 x 小的最大值
		{
			split_val(rt, p, q); x = p;
			while (c[x][1]) x = c[x][1]; ans ^= (la = v[x]);
			rt = merge(p, q); continue;
		}
		++val; split_val(rt, p, q); x = q;//询问比 x 大的最小值
		while (c[x][0]) x = c[x][0];
		ans ^= (la = v[x]); rt = merge(p, q);
	}
	cout<<ans<<endl;
}

\end{lstlisting}

\subsection{笛卡尔树的线性建树}

$p[1,2,\ldots,n]$ 是原序列,\verb|c| 表示子结点。

笛卡尔树满足堆性质(权值小于等于子结点权值),并且中序遍历是原序列。

$O(n)$,$O(n)$。

\begin{lstlisting}
int c[N][2],p[N],st[N];
int main()
{
	ios::sync_with_stdio(false);
	cin.tie(0);
	int i,n,tp=0;
	ll la=0,ra=0;
	cin>>n;
	for (i=1;i<=n;i++)
	{
		cin>>p[i];st[tp+1]=0;
		while ((tp)&&(p[st[tp]]>p[i])) --tp;
		c[c[st[tp]][1]=i][0]=st[tp+1];st[++tp]=i;
	}
	for (i=1;i<=n;i++) la^=(ll)i*(c[i][0]+1);
	for (i=1;i<=n;i++) ra^=(ll)i*(c[i][1]+1);
	cout<<la<<' '<<ra<<endl;
}
\end{lstlisting}

\subsection{扫描线}

求矩形并的面积和周长(包括内周长)

$O((n+q)\log n)$,$O(n+q)$。

\begin{lstlisting}
	using T=ll;
	vector<T> fun(vector<tuple<T, T, T, T>> &a)
	{
		vector<T> x;
		for (auto [x1, y1, x2, y2]:a)
		{
			x.push_back(x1);
			x.push_back(x2);
		}
		sort(all(x)); x.resize(unique(all(x))-x.begin());
		for (auto &[x1, y1, x2, y2]:a)
		{
			x1=lower_bound(all(x), x1)-x.begin();
			x2=lower_bound(all(x), x2)-x.begin();
		}
		return x;
	}
	struct sgt
	{
		int n, z, y, d;
		vector<T> cnt, &p;
		vector<int> mn, lz;
		void build(int x, int l, int r)
		{
			cnt[x]=p[min(r, n-1)]-p[l];
			if (l+1==r) return;
			int c=x*2, m=l+r>>1;
			build(c, l, m); build(c+1, m, r);
		}
		sgt(vector<T> &p):n(p.size()), p(p), cnt(n*4), mn(n*4), lz(n*4) { build(1, 0, n); }
		void dfs(int x, int l, int r)
		{
			if (z<=l&&r<=y)
			{
				mn[x]+=d;
				lz[x]+=d;
				return;
			}
			int c=x*2, m=l+r>>1;
			if (lz[x])
			{
				lz[c]+=lz[x]; lz[c+1]+=lz[x];
				mn[c]+=lz[x]; mn[c+1]+=lz[x];
				lz[x]=0;
			}
			if (z<m) dfs(c, l, m);
			if (m<y) dfs(c+1, m, r);
			mn[x]=min(mn[c], mn[c+1]);
			cnt[x]=cnt[c]*(mn[x]==mn[c])+cnt[c+1]*(mn[x]==mn[c+1]);
		}
		void modify(int l, int r, int dt)
		{
			z=l;
			y=r;
			d=dt;
			dfs(1, 0, n);
		}
	};
	T area(vector<tuple<T, T, T, T>> a)//[x1,y1,x2,y2], x1<y1, x2<y2
	{
		int n=a.size(), i;
		auto X=fun(a);
		vector<tuple<T, int, T, T>> b(n*2);
		for (i=0; i<n; i++)
		{
			auto [x1, y1, x2, y2]=a[i];
			b[i]={y1, -1, x1, x2};
			b[i+n]={y2, 1, x1, x2};
		}
		sort(all(b), greater<>());
		sgt s(X);
		T lst=0, ans=0;
		for (auto [y, d, l, r]:b)
		{
			ans+=(lst-y)*(X.back()-X[0]-s.cnt[1]);
			s.modify(l, r, d);
			lst=y;
		}
		return ans;
	}
	T perimeter_x(vector<tuple<T, T, T, T>> a)
	{
		int n=a.size(), i;
		auto X=fun(a);
		vector<tuple<T, int, T, T>> b(n*2);
		for (i=0; i<n; i++)
		{
			auto [x1, y1, x2, y2]=a[i];
			b[i]={y1, -1, x1, x2};
			b[i+n]={y2, 1, x1, x2};
		}
		sort(all(b), greater<>());
		sgt s(X);
		T lst=s.cnt[1], ans=0;
		for (auto [y, d, l, r]:b)
		{
			s.modify(l, r, d);
			T cur=s.cnt[1];
			ans+=abs(lst-cur);
			lst=cur;
		}
		return ans;
	}
	T perimeter(vector<tuple<T, T, T, T>> a)//[x1,y1,x2,y2], x1<y1, x2<y2
	{
		T ansx=perimeter_x(a);
		for (auto &[x1, y1, x2, y2]:a)
		{
			swap(x1, y1);
			swap(x2, y2);
		}
		T ansy=perimeter_x(a);
		return ansx+ansy;
	}
\end{lstlisting}

\subsection{Segmenttree Beats!}
核心是 P(tag) 和 Q(info) 的维护。线段树部分是套的模板,并非全都有用。
\begin{enumerate}
	\item $l,r,k$:对于所有的 $i\in[l,r]$,将 $A_i$ 加上 $k$($k$ 可以为负数)。
	\item $l,r,v$:对于所有的 $i\in[l,r]$,将 $A_i$ 变成 $\min(A_i,v)$。
	\item $l,r$:求 $\sum\limits_{i=l}^{r}A_i$。
	\item $l,r$:对于所有的 $i\in[l,r]$,求 $A_i$ 的最大值。
	\item $l,r$:对于所有的 $i\in[l,r]$,求 $B_i$ 的最大值。
\end{enumerate}
其中 $B_i$ 是 $A_i$ 的历史最大值。
\begin{lstlisting}
struct P
{
	ll tg,L,R;
	P(ll a=0,ll b=-inf,ll c=inf):tg(a),L(b),R(c) { }
	void operator+=(P o)
	{
		o.L-=tg; o.R-=tg; tg+=o.tg;
		if (L>=o.R) L=R=o.R;
		else if (R<=o.L) L=R=o.L;
		else cmax(L,o.L),cmin(R,o.R);
	}
};
struct Q
{
	ll mx0,cmx,mx1,mn0,cmn,mn1,cnt,sum;
	Q():mx0(-inf),cmx(0),mx1(-inf),mn0(inf),cmn(0),mn1(inf),cnt(0),sum(0) { }
	Q(ll x):mx0(x),cmx(1),mx1(-inf),mn0(x),cmn(1),mn1(inf),cnt(1),sum(x) { }
	bool operator+=(const P &o)
	{
		if (o.L==o.R)
		{
			ll c=cnt;
			*this=Q(o.L+o.tg);
			cnt=cmx=cmn=c;
			sum=cnt*(o.L+o.tg);
			return 1;
		}
		if (o.L>=mn1||o.R<=mx1) return 0;
		if (mx0==mn0)
		{
			mn0=min(o.R,max(mx0,o.L));
			sum+=cnt*(mn0-mx0);
			mx0=mn0;
		}
		else
		{
			if (o.L>mn0)
			{
				sum+=(o.L-mn0)*cmn;
				mn0=o.L;
				cmax(mx1,o.L);
			}
			if (o.R<mx0)
			{
				sum+=(o.R-mx0)*cmx;
				mx0=o.R;
				cmin(mn1,o.R);
			}
		}
		if (o.tg)
		{
			sum+=o.tg*cnt;
			mx0+=o.tg;
			mx1+=o.tg;
			mn0+=o.tg;
			mn1+=o.tg;
		}
		return 1;
	}
};
Q operator+(const Q &a,const Q &b)
{
	Q res;
	res.sum=a.sum+b.sum;
	res.cnt=a.cnt+b.cnt;
	res.mx0=max(a.mx0,b.mx0);
	res.mx1=max(a.mx1,b.mx1);
	if (res.mx0==a.mx0) res.cmx+=a.cmx; else cmax(res.mx1,a.mx0);
	if (res.mx0==b.mx0) res.cmx+=b.cmx; else cmax(res.mx1,b.mx0);

	res.mn0=min(a.mn0,b.mn0);
	res.mn1=min(a.mn1,b.mn1);
	if (res.mn0==a.mn0) res.cmn+=a.cmn; else cmin(res.mn1,a.mn0);
	if (res.mn0==b.mn0) res.cmn+=b.cmn; else cmin(res.mn1,b.mn0);

	return res;
}
template<class info,class tag> struct sgt
{
	int n,shift;
	vector<info> s;
	vector<tag> tg;
	vector<char> lz;
	template<class T> void build(T *a,int x,int l,int r)
	{
		if (l==r)
		{
			s[x]=a[l];
			return;
		}
		int c=x*2,m=l+r>>1;
		build(a,c,l,m); build(a,c+1,m+1,r);
		s[x]=s[c]+s[c+1];
	}
	template<class T> sgt(T *b,int L,int R):n(R-L+1),shift(L-1),s(R-L+1<<2),tg(R-L+1<<2),lz(R-L+1<<2)
	{
		build(b+L-1,1,1,n);
	}//[L,R]
	int z,y;
	info res;
	tag dt;
	bool fir;
private:
	void pushdown(int x)
	{
		int c=x*2;
		if (lz[x])
		{
			if (lz[c]) tg[c]+=tg[x]; else tg[c]=tg[x];
			lz[c]=1;
			if (!(s[c]+=tg[x]))
			{
				pushdown(c);
				s[c]=s[c*2]+s[c*2+1];
			}
			c^=1;
			if (lz[c]) tg[c]+=tg[x]; else tg[c]=tg[x];
			lz[c]=1;
			if (!(s[c]+=tg[x]))
			{
				pushdown(c);
				s[c]=s[c*2]+s[c*2+1];
			}
			c^=1;
			lz[x]=0;
		}
	}
	void _modify(int x,int l,int r)
	{
		if (z<=l&&r<=y)
		{
			if (lz[x]) tg[x]+=dt; else tg[x]=dt;
			lz[x]=1;
			if (!(s[x]+=dt))
			{
				pushdown(x);
				s[x]=s[x*2]+s[x*2+1];
			}
			return;
		}
		int c=x*2,m=l+r>>1;
		pushdown(x);
		if (z<=m) _modify(c,l,m);
		if (m<y) _modify(c+1,m+1,r);
		s[x]=s[c]+s[c+1];
	}
	void ask(int x,int l,int r)
	{
		if (z<=l&&r<=y)
		{
			res=fir?s[x]:res+s[x];
			fir=0;
			return;
		}
		int c=x*2,m=l+r>>1;
		pushdown(x);
		if (z<=m) ask(c,l,m);
		if (m<y) ask(c+1,m+1,r);
	}
	function<bool(info)> check;
	void find_left_most(int x,int l,int r)
	{
		if (r<z||!check(s[x])) return;
		if (l==r) { y=l; res=s[x]; return; }
		int c=x*2,m=l+r>>1;
		pushdown(x);
		find_left_most(c,l,m);
		if (y==n+1) find_left_most(c+1,m+1,r);
	}
	void find_right_most(int x,int l,int r)
	{
		if (l>y||!check(s[x])) return;
		if (l==r) { z=l; res=s[x]; return; }
		int c=x*2,m=l+r>>1;
		pushdown(x);
		find_right_most(c+1,m+1,r);
		if (z==0) find_right_most(c,l,m);
	}
public:
	void modify(int l,int r,const tag &x)//[l,r]
	{
		z=l-shift; y=r-shift; dt=x;
		// cerr<<"modify ["<<l<<','<<r<<"] "<<'\n';
		assert(1<=z&&z<=y&&y<=n);
		_modify(1,1,n);
	}
	void modify(int pos,const info &o)
	{
		pos-=shift;
		int l=1,r=n,m,c,x=1;
		while (l<r)
		{
			c=x*2; m=l+r>>1;
			pushdown(x);
			if (pos<=m) x=c,r=m; else x=c+1,l=m+1;
		}
		s[x]=o;
		while (x>>=1) s[x]=s[x*2]+s[x*2+1];
	}
	info ask(int l,int r)//[l,r]
	{
		z=l-shift; y=r-shift; fir=1;
		// cerr<<"ask ["<<l<<','<<r<<"] "<<'\n';
		assert(1<=z&&z<=y&&y<=n);
		ask(1,1,n);
		return res;
	}
	pair<int,info> find_left_most(int l,const function<bool(info)> &_check)//y=n+1 第二个参数是乱给的
	{
		check=_check;
		z=l-shift; y=n+1;
		assert(1<=z&&z<=n+1);
		find_left_most(1,1,n);
		return {y+shift,res};
	}
	pair<int,info> find_right_most(int r,const function<bool(info)> &_check)//z=0 第二个参数是乱给的
	{
		check=_check;
		z=0; y=r-shift;
		assert(0<=y&&y<=n);
		find_right_most(1,1,n);
		return {z+shift,res};
	}
};
//要求:具有 info+info,info+=tag,tag+=tag。info,tag 需要拥有默认构造,但不必拥有正确的值。
//采用左闭右闭
mt19937 rnd(345);
int main()
{
	ios::sync_with_stdio(0); cin.tie(0);
	cout<<fixed<<setprecision(15);
	int n,q,i;
	cin>>n>>q;
	vector<ll> a(n);
	cin>>a;
	sgt<Q,P> s(a.data(),0,n-1);
	while (q--)
	{
		int op,l,r;
		cin>>op>>l>>r;
		--r;
		if (op==3)
		{
			ll res=s.ask(l,r).sum;
			cout<<res<<'\n';
		}
		else
		{
			ll b;
			cin>>b;
			if (op==0) s.modify(l,r,{0,-inf,b});
			else if (op==1) s.modify(l,r,{0,b});
			else s.modify(l,r,{b});
		}
	}
}

	
\end{lstlisting}

\subsection{$k$-d 树(二进制分组)}

均摊 $O(\log ^2n)$ 插入,$O(\sqrt n)$ 矩形查询。

\begin{lstlisting}
#define tmpl template<class T>
typedef long long ll;
tmpl struct P
{
	ll x,y;
	T v;
};
tmpl struct Q
{
	ll x[2],y[2];
	bool t;
	T s;
	Q() {}
	Q(const P<T> &a)
	{
		x[0]=x[1]=a.x;
		y[0]=y[1]=a.y;
		s=a.v;
	}
};
tmpl bool cmp0(const P<T> &a,const P<T> &b) { return a.x<b.x; }
tmpl bool cmp1(const P<T> &a,const P<T> &b) { return a.y<b.y; }
tmpl struct kdt
{
	vector<P<T>> c;
	vector<Q<T>> a;
	ll m,u,d,l,r;
	T ans;
	bool fir;
	void build(int x,P<T> *b,int n)
	{
		if (x==1)
		{
			a.resize(m=n<<1);
			a[x].t=0;
			c.resize(n);
			for (int i=0; i<n; i++) c[i]=b[i];
		}
		if (n==1)
		{
			a[x]=Q<T>(b[0]);
			return;
		}
		int mid=n>>1,c=x<<1;
		nth_element(b,b+mid,b+n,a[x].t?cmp1<T>:cmp0<T>);
		a[c].t=a[c|1].t=a[x].t^1;
		build(c,b,mid);
		build(c|1,b+mid,n-mid);
		a[x].s=a[c].s+a[c|1].s;
		a[x].x[0]=min(a[c].x[0],a[c|1].x[0]);
		a[x].x[1]=max(a[c].x[1],a[c|1].x[1]);
		a[x].y[0]=min(a[c].y[0],a[c|1].y[0]);
		a[x].y[1]=max(a[c].y[1],a[c|1].y[1]);
	}
	void find(int x)
	{
		if (x>=m||a[x].x[1]<u||a[x].x[0]>d||a[x].y[1]<l||a[x].y[0]>r) return;
		if (u<=a[x].x[0]&&a[x].x[1]<=d&&l<=a[x].y[0]&&a[x].y[1]<=r)
		{
			ans=fir?a[x].s:ans+a[x].s;
			fir=0;
			return;
		}
		find(x<<1); find(x<<1|1);
	}
	pair<bool,T> find(ll x1,ll y1,ll x2,ll y2)
	{
		fir=1;
		ans={};
		u=x1; d=x2;
		l=y1; r=y2;
		find(1);
		return {!fir,ans};
	}
};
const int N=2e5+2,M=18;
tmpl struct KDT
{
	kdt<T> s[M];
	P<T> a[N];
	int n,m,i;
	KDT() { n=0; }
	KDT(int N,ll *x,ll *y,T *w)//[0,n)
	{
		n=N;
		int i,j;
		for (i=0; i<n; i++) a[i]={x[i],y[i],w[i]};
		for (i=j=0; n>>i; i++) if (n>>i&1) s[i].build(1,a+j,1<<i),j+=1<<i;
	}
	void insert(ll x,ll y,T w)//插入 (x,y) 的一个数 w
	{
		a[0]={x,y,w}; m=1;
		for (i=0; n&1<<i; i++) for (auto u:s[i].c) a[m++]=u;
		s[i].build(1,a,m);
		++n;
	}
	pair<bool,T> ask(ll x,ll y,ll xx,ll yy)//查询 [x,xx]*[y,yy] 的和
	{
		T ans;
		bool fir=1;
		for (i=0; 1<<i<=n; i++) if (1<<i&n)
		{
			auto [_,tmp]=s[i].find(x,y,xx,yy);
			if (!_) continue;
			ans=fir?tmp:ans+tmp;
			fir=0;
		}
		return {!fir,ans};
	}
};
int x[N],y[N],w[N];
int main()
{
	ios::sync_with_stdio(0); cin.tie(0); cout.tie(0);
	int n,q,i;
	cin>>n>>q;
	for (i=0; i<n; i++) cin>>x[i]>>y[i]>>w[i];
	KDT<ll> s(n,x,y,w);
	while (q--)
	{
		int op,x,y,w;
		cin>>op>>x>>y>>w;
		if (op==0) s.insert(x,y,w); else
		{
			cin>>op;
			cout<<s.ask(x,y,w-1,op-1)<<'\n';
		}
	}
	return 0;
}
\end{lstlisting}

\subsection{双端队列全局查询}

对一个支持结合律的信息 T,维护 \verb|deque| 内信息的和。总复杂度线性。

\begin{lstlisting}
template<class T> struct dq
{
	vector<T> l,sl,r,sr;
	void push_front(const T &o)
	{
		sl.push_back(sl.size()?o+sl.back():o);
		l.push_back(o);
	}
	void push_back(const T &o)
	{
		sr.push_back(sr.size()?sr.back()+o:o);
		r.push_back(o);
	}
	void pop_front()
	{
		if (l.size()) sl.pop_back(),l.pop_back();
		else
		{
			assert(r.size());
			int n=r.size(),m,i;
			if (m=n-1>>1)
			{
				l.resize(m); sl.resize(m);
				for (i=1; i<=m; i++) l[m-i]=r[i];
				sl[0]=l[0];
				for (i=1; i<m; i++) sl[i]=l[i]+sl[i-1];
			}
			for (i=m+1; i<n; i++) r[i-(m+1)]=r[i];
			m=n-(m+1);
			r.resize(m); sr.resize(m);
			if (m)
			{
				sr[0]=r[0];
				for (i=1; i<m; i++) sr[i]=sr[i-1]+r[i];
			}
		}
	}
	void pop_back()
	{
		if (r.size()) sr.pop_back(),r.pop_back();
		else
		{
			assert(l.size());
			int n=l.size(),m,i;
			if (m=n-1>>1)
			{
				r.resize(m); sr.resize(m);
				for (i=1; i<=m; i++) r[m-i]=l[i];
				sr[0]=r[0];
				for (i=1; i<m; i++) sr[i]=sr[i-1]+r[i];
			}
			for (i=m+1; i<n; i++) l[i-(m+1)]=l[i];
			m=n-(m+1);
			l.resize(m); sl.resize(m);
			if (m)
			{
				sl[0]=l[0];
				for (i=1; i<m; i++) sl[i]=l[i]+sl[i-1];
			}
		}
	}
	template<class TT> TT ask(TT r)
	{
		if (sl.size()) r=r+sl.back();
		if (sr.size()) r=r+sr.back();
		return r;
	}
	T ask()
	{
		assert(sl.size()||sr.size());
		if (sl.size()&&sr.size()) return sl.back()+sr.back();
		return sl.size()?sl.back():sr.back();
	}
};//参数:类型。结合使用 + 运算符

\end{lstlisting}

\subsection{静态矩形加矩形和}

\begin{lstlisting}
const ll p=998244353;
struct Q
{
	int n,m;
	ll w;
	int typ;
	bool operator<(const Q &o) const
	{
		if (n!=o.n) return n<o.n;
		return typ<o.typ;
	}
};
template<class T> struct tork
{
	vector<T> a;
	int n;
	tork(const vector<T> &b):a(all(b))
	{
		sort(all(a));
		a.resize(unique(all(a))-a.begin());
		n=a.size();
	}
	tork(const T *first,const T *last):a(first,last)
	{
		sort(all(a));
		a.resize(unique(all(a))-a.begin());
		n=a.size();
	}
	void get(T &x) { x=lower_bound(all(a),x)-a.begin()+1; }
	T operator[](const int &x) { return a[x]; }
};
struct bit
{
	vector<ll> a;
	int n;
	bit() {}
	bit(int nn):n(nn),a(nn+1) {}
	template<class T> bit(int nn,T *b):n(nn),a(nn+1)
	{
		for (int i=1; i<=n; i++) a[i]=b[i];
		for (int i=1; i<=n; i++) if (i+(i&-i)<=n) a[i+(i&-i)]+=a[i];
	}
	void add(int x,ll y)
	{
		// cerr<<"add "<<x<<" by "<<y<<endl;
		assert(1<=x&&x<=n);
		if ((a[x]+=y)>=p) a[x]-=p;
		while ((x+=x&-x)<=n) if ((a[x]+=y)>=p) a[x]-=p;
	}
	ll sum(int x)
	{
		// cerr<<"sum "<<x;
		assert(0<=x&&x<=n);
		ll r=a[x];
		while (x^=x&-x) r+=a[x];
		// cerr<<"= "<<r<<endl;
		return r%p;
	}
	ll sum(int x,int y)
	{
		return (sum(y)+p-sum(x-1))%p;
	}
};
struct matrix
{
	int l,d,r,u;
	ll w;
};
vector<ll> rec_add_rec_sum(const vector<matrix> &op,const vector<matrix> &query)
{
	vector<Q> a[4];
	int n=op.size(),m=query.size(),i;
	for (auto &v:a) v.reserve(n+m<<2);
	for (auto [l,d,r,u,w]:op)//[l,r)*[d,u) += w
	{
		a[0].push_back({l,d,w*l%p*d%p,-1});
		a[1].push_back({l,d,w*l%p,-1});
		a[2].push_back({l,d,w*d%p,-1});
		a[3].push_back({l,d,w,-1});
		w=(p-w)%p;
		a[0].push_back({l,u,w*l%p*u%p,-1});
		a[1].push_back({l,u,w*l%p,-1});
		a[2].push_back({l,u,w*u%p,-1});
		a[3].push_back({l,u,w,-1});
		a[0].push_back({r,d,w*r%p*d%p,-1});
		a[1].push_back({r,d,w*r%p,-1});
		a[2].push_back({r,d,w*d%p,-1});
		a[3].push_back({r,d,w,-1});
		w=(p-w)%p;
		a[0].push_back({r,u,w*r%p*u%p,-1});
		a[1].push_back({r,u,w*r%p,-1});
		a[2].push_back({r,u,w*u%p,-1});
		a[3].push_back({r,u,w,-1});
	}
	i=0;
	for (auto [l,d,r,u,w]:query)//ask sum of [l,r)*[d,u)
	{
		a[0].push_back({l,d,1,i});
		a[1].push_back({l,d,(p*2-d)%p,i});
		a[2].push_back({l,d,(p*2-l)%p,i});
		a[3].push_back({l,d,(ll)l*d%p,i});
		a[0].push_back({l,u,p-1,i});
		a[1].push_back({l,u,u%p,i});
		a[2].push_back({l,u,l%p,i});
		a[3].push_back({l,u,(p*2-l)*u%p,i});
		a[0].push_back({r,u,1,i});
		a[1].push_back({r,u,(p*2-u)%p,i});
		a[2].push_back({r,u,(p*2-r)%p,i});
		a[3].push_back({r,u,(ll)u*r%p,i});
		a[0].push_back({r,d,p-1,i});
		a[1].push_back({r,d,d%p,i});
		a[2].push_back({r,d,r%p,i});
		a[3].push_back({r,d,(p*2-d)*r%p,i});
		++i;
	}
	assert(a[0].size()==n+m<<2);
	vector<ll> ans(m);
	auto cal=[&](vector<Q> a)
	{
		int n=a.size(),i;
		vector<int> b(n);
		for (i=0; i<n; i++) b[i]=(a[i].m-=a[i].typ>=0),a[i].n-=a[i].typ>=0;
		sort(all(a));
		tork t(b);
		for (i=0; i<n; i++) t.get(a[i].m);
		int m=t.a.size();
		bit s(m);
		for (auto [n,m,w,typ]:a) if (typ>=0) ans[typ]=(ans[typ]+s.sum(m)*w)%p; else s.add(m,w);
	};
	for (auto &v:a) cal(v);
	return ans;
}
int main()
{
	ios::sync_with_stdio(0); cin.tie(0);
	cout<<setiosflags(ios::fixed)<<setprecision(15);
	int n,m,i;
	cin>>n>>m;
	vector<matrix> a(n),b(m);
	for (auto &[l,d,r,u,w]:a) cin>>l>>d>>r>>u>>w;
	for (auto &[l,d,r,u,w]:b) cin>>l>>d>>r>>u;
	auto ans=rec_add_rec_sum(a,b);
	for (i=0; i<m; i++) cout<<ans[i]<<'\n';
}

\end{lstlisting}

\subsection{线段树分裂}

\begin{lstlisting}
namespace sgt
{
#define ask_kth
	int L=0,R=1e9;
	void set_bound(int l,int r) { L=l; R=r; }
	typedef ll info;
	const info E=0;//找不到会返回 E
	const int N=8e6+5;
#define lc(x) (a[x].lc)
#define rc(x) (a[x].rc)
#define s(x) (a[x].s)
	struct node
	{
		int lc,rc;
		info s;
	};
	node a[N];
	vector<int> id;
	int ids=0,pos,z,y;
	bool fir;
	info tmp;
	int npt()
	{
		int x;
		if (id.size()) x=id.back(),id.pop_back();
		else x=++ids;
		lc(x)=rc(x)=0;
		return x;
	}
	void pushup(int &x)
	{
		if (lc(x)&&rc(x)) s(x)=s(lc(x))+s(rc(x));
		else if (lc(x)) s(x)=s(lc(x));
		else if (rc(x)) s(x)=s(rc(x));
		else id.push_back(x),x=0;
	}
	void insert(int &x,int l,int r)
	{
		if (l==r)
		{
			if (!x) x=npt(),s(x)=tmp;
			else s(x)=s(x)+tmp;
			return;
		}
		if (!x) x=npt();
		int mid=l+r>>1;
		if (pos<=mid)
		{
			insert(lc(x),l,mid);
			if (rc(x)) s(x)=s(lc(x))+s(rc(x)); else s(x)=s(lc(x));
		}
		else
		{
			insert(rc(x),mid+1,r);
			if (lc(x)) s(x)=s(lc(x))+s(rc(x)); else s(x)=s(rc(x));
		}
	}
	void modify(int &x,int l,int r)
	{
		if (!x) x=npt();
		if (l==r)
		{
			s(x)=tmp;
			return;
		}
		int mid=l+r>>1;
		if (pos<=mid)
		{
			insert(lc(x),l,mid);
			if (rc(x)) s(x)=s(lc(x))+s(rc(x)); else s(x)=s(lc(x));
		}
		else
		{
			insert(rc(x),mid+1,r);
			if (lc(x)) s(x)=s(lc(x))+s(rc(x)); else s(x)=s(rc(x));
		}
	}
	int merge(int x1,int x2,int l,int r)
	{
		if (!(x1&&x2)) return x1|x2;
		if (l==r) { s(x1)=s(x1)+s(x2); return x1; }
		int mid=l+r>>1;
		lc(x1)=merge(lc(x1),lc(x2),l,mid);
		rc(x1)=merge(rc(x1),rc(x2),mid+1,r);
		pushup(x1);
		return x1;
	}
	void ask(int x,int l,int r)
	{
		if (!x) return;
		if (z<=l&&r<=y)
		{
			if (fir) tmp=s(x),fir=0; else tmp=tmp+s(x);
			return;
		}
		int mid=l+r>>1;
		if (z<=mid) ask(lc(x),l,mid);
		if (y>mid) ask(rc(x),mid+1,r);
	}
	void split(int &x1,int &x2,int l,int r)
	{
		assert(!x1);
		if (!x2) return;
		if (z<=l&&r<=y) { x1=x2; x2=0; return; }
		x1=npt();
		int mid=l+r>>1;
		if (z<=mid) split(lc(x1),lc(x2),l,mid);
		if (y>mid) split(rc(x1),rc(x2),mid+1,r);
		pushup(x1); pushup(x2);
	}
	info *b;
	void build(int &x,int l,int r)
	{
		x=npt();
		if (l==r) { s(x)=b[l]; return; }
		int mid=l+r>>1;
		build(lc(x),l,mid); build(rc(x),mid+1,r);
		s(x)=s(lc(x))+s(rc(x));
	}
	struct set
	{
		int rt;
		set():rt(0) {}
		set(info *a):rt(0) { b=a; build(rt,L,R); }
		void modify(int p,const info &o) { pos=p; tmp=o; sgt::modify(rt,L,R); }
		void insert(int p,const info &o) { pos=p; tmp=o; sgt::insert(rt,L,R); }
		void join(const set &o) { rt=merge(rt,o.rt,L,R); }
		info ask(int l,int r)
		{
			z=l; y=r; fir=1;
			sgt::ask(rt,L,R);
			return fir?E:tmp;
		}
		set split(int l,int r)
		{
			z=l; y=r; set p;
			sgt::split(p.rt,rt,L,R);
			return p;
		}
#ifdef ask_kth
		int kth(info k)
		{
			int x=rt,l=L,r=R,mid;
			if (k>s(x)) return -1;
			s(0)=0;
			while (l<r)
			{
				mid=l+r>>1;
				if (s(lc(x))>=k) x=lc(x),r=mid;
				else k-=s(lc(x)),x=rc(x),l=mid+1;
			}
			return l;
		}
#endif
	};
#undef lc
#undef rc
#undef s
}
typedef sgt::set tree;
\end{lstlisting}

\subsection{bitset(手写,未验证)}

\begin{lstlisting}
struct Bitset
{
	typedef unsigned int ui;
	typedef unsigned long long ll;
#define all(x) (x).begin(),(x).end()
	const static ll B=-1llu;
	vector<ll> a;
	int n;
	Bitset() { }
	Bitset(int _n):n(_n), a(_n+63>>6) { }
	bool test(int x) const { assert(x>=0&&x<n); return a[x>>6]>>(x&63)&1; }
	bool operator[](int x) const { return test(x); }
	void set(int x, bool y) { assert(x>=0&&x<n); a[x>>6]=(a[x>>6]&(B^1llu<<(x&63)))|((ll)y<<(x&63)); }
	void set(int x) { assert(x>=0&&x<n); a[x>>6]|=1llu<<(x&63); }
	void set() { memset(a.data(), 0xff, a.size()*sizeof a[0]); a.back()&=(1llu<<1+(n-1&63))-1; }
	void reset(int x) { assert(x>=0&&x<n); a[x>>6]&=~(1llu<<(x&63)); }
	void reset() { memset(a.data(), 0, a.size()*sizeof a[0]); }
	int count() const
	{
		int r=0;
		for (ll x:a) r+=__builtin_popcountll(x);
		return r;
	}
	Bitset &operator|=(const Bitset &o)
	{
		assert(n==o.n);
		for (int i=0; i<a.size(); i++) a[i]|=o.a[i];
		return *this;
	}
	Bitset operator|(Bitset o) { o|=*this; return o; }
	Bitset &operator&=(const Bitset &o)
	{
		assert(n==o.n);
		for (int i=0; i<a.size(); i++) a[i]&=o.a[i];
		return *this;
	}
	Bitset operator&(Bitset o) { o&=*this; return o; }
	Bitset &operator^=(const Bitset &o)
	{
		assert(n==o.n);
		for (int i=0; i<a.size(); i++) a[i]^=o.a[i];
		return *this;
	}
	Bitset operator^(Bitset o) { o^=*this; return o; }
	Bitset operator~() const
	{
		auto r=*this;
		for (ll &x:r.a) x=~x;
		return r;
	}
	Bitset &operator<<=(int x)
	{
		if (x>=n)
		{
			fill(all(a), 0);
			return *this;
		}
		assert(x>=0);
		int y=x>>6;
		x&=63;
		if (x==0)
		{
			for (int i=(int)a.size()-1; i>=y; i--) a[i]=a[i-y]<<x;
			if (n&63) a.back()&=(1llu<<1+(n-1&63))-1;
			memset(a.data(), 0, y*sizeof a[0]);
			return *this;
		}
		for (int i=(int)a.size()-1; i>y; i--) a[i]=a[i-y]<<x|a[i-y-1]>>64-x;
		a[y]=a[0]<<x;
		memset(a.data(), 0, y*sizeof a[0]);
		// fill_n(a.begin(),y,0);
		if (n&63) a.back()&=(1llu<<1+(n-1&63))-1;
		return *this;
	}
	Bitset operator<<(int x)
	{
		auto r=*this;
		r<<=x;
		return r;
	}
	Bitset &operator>>=(int x)
	{
		if (x>=n)
		{
			fill(all(a), 0);
			return *this;
		}
		assert(x>=0);
		int y=x>>6, R=(int)a.size()-y-1;
		x&=63;
		for (int i=0; i<R; i++) a[i]=a[i+y]>>x|a[i+y+1]<<64-x;
		a[R]=a.back()>>x;
		memset(a.data()+R+1, 0, y*sizeof a[0]);
		// fill(R+1+all(a),0);
		return *this;
	}
	Bitset operator>>(int x)
	{
		auto r=*this;
		r>>=x;
		return r;
	}
	void range_set(int l, int r)//[l,r) to 1
	{
		if (l>>6==r>>6)
		{
			a[l>>6]|=(1llu<<r-l)-1<<(l&63);
			return;
		}
		if (l&63)
		{
			a[l>>6]|=~((1llu<<(l&63))-1);//[l&63,64)
			l=(l>>6)+1<<6;
		}
		if (r&63)
		{
			a[r>>6]|=(1llu<<(r&63))-1;
			r=(r>>6)-1<<6;
		}
		memset(a.data()+(l>>6), 0xff, (r-l>>6)*sizeof a[0]);
	}
	void range_reset(int l, int r)//[l,r) to 0
	{
		if (l>>6==r>>6)
		{
			a[l>>6]&=~((1llu<<r-l)-1<<(l&63));
			return;
		}
		if (l&63)
		{
			a[l>>6]&=(1llu<<(l&63))-1;//[l&63,64)
			l=(l>>6)+1<<6;
		}
		if (r&63)
		{
			a[r>>6]&=~((1llu<<(r&63))-1);
			r=(r>>6)-1<<6;
		}
		memset(a.data()+(l>>6), 0, (r-l>>6)*sizeof a[0]);
	}
	void range_set(int l, int r, bool x)//[l,r)
	{
		if (x) range_set(l, r);
		else range_reset(l, r);
	}
	int size() const { return n; }
	int _Find_first() const
	{
		for (int i=0; i<a.size(); i++) if (a[i]) return i*64+__lg(a[i]&-a[i]);
		return n;
	}
};
istream &operator>>(istream &cin, Bitset &o)
{
	string s;
	cin>>s;
	int n=s.size(), i;
	assert(n<=o.size());
	for (i=0; i<n; i++) o.set(i, s[n-i-1]-'0');
	return cin;
}
ostream &operator<<(ostream &cout, const Bitset &o)
{
	int n=o.size(), i;
	string s(n, '0');
	for (i=0; i<n; i++) s[n-i-1]+=o.test(i);
	return cout;
}	
	
\end{lstlisting}

\subsection{区间众数}

\begin{lstlisting}
template<class T> struct mode//[0,n)
{
	int n,ksz,m;
	vector<T> b;
	vector<vector<int>> pos,f;
	vector<int> a,blk,id,l;
	mode(const vector<T> &c):n(c.size()),ksz(max<int>(1,sqrt(n))),m((n+ksz-1)/ksz),b(c),
		pos(n),f(m,vector<int>(m)),a(n),blk(n),id(n),l(m+1)
	{
		int i,j,k;
		sort(all(b)); b.resize(unique(all(b))-b.begin());
		for (i=0; i<n; i++)
		{
			a[i]=lower_bound(all(b),c[i])-b.begin();
			id[i]=pos[a[i]].size();
			pos[a[i]].push_back(i);
		}
		for (i=0; i<n; i++) blk[i]=i/ksz;
		for (i=0; i<=m; i++) l[i]=min(i*ksz,n);
		vector<int> cnt(b.size());
		for (i=0; i<m; i++)
		{
			fill(all(cnt),0);
			pair<int,int> cur={0,0};
			for (j=i; j<m; j++)
			{
				for (k=l[j]; k<l[j+1]; k++) cmax(cur,pair{++cnt[a[k]],a[k]});
				f[i][j]=cur.second;
			}
		}
	}
	pair<T,int> ask(int L,int R)//返回最大众数
	{
		assert(0<=L&&L<R&&R<=n);
		int val=blk[L]==blk[R-1]?0:f[blk[L]+1][blk[R-1]-1],i;
		int cnt=lower_bound(all(pos[val]),R)-lower_bound(all(pos[val]),L);
		for (i=min(R,l[blk[L]+1])-1; i>=L; i--)
		{
			auto &v=pos[a[i]];
			while (id[i]+cnt<v.size()&&v[id[i]+cnt]<R) ++cnt,val=a[i];
			if (a[i]>val&&id[i]+cnt-1<v.size()&&v[id[i]+cnt-1]<R) val=a[i];
		}
		for (i=max(L,l[blk[R-1]]); i<R; i++)
		{
			auto &v=pos[a[i]];
			while (id[i]>=cnt&&v[id[i]-cnt]>=L) ++cnt,val=a[i];
			if (a[i]>val&&id[i]>=cnt-1&&v[id[i]-cnt+1]>=L) val=a[i];
		}
		return {b[val],cnt};
	}
};
\end{lstlisting}

\subsection{表达式树}

传入表达式,输出表达式树。

输入的第二个参数是全体括号以外的运算符,每个运算符要记录字符优先级和是否右结合。优先级数字越大,越优先计算,且优先级必须为正整数。

输出的第一个参数是子结点数组,第二个参数是每个结点对应的字符,第三个参数是根。结点编号从 $1$ 开始。

输出的表达式树满足每个结点对应一个字符。若包含数字串,则视为相邻数码之间加一个井号,表示“数码链接”这个运算符。你不需要,也不应该手动加入这个井号。

如果表达式非法,将返回根为 $0$。不允许一元运算符(负号),不允许省略乘号,不允许出现字母(除非字母是运算符)。

如果需要支持字母作为数字,修改所有包含 \verb|isdigit| 的部分。

由于存在“数码链接”,在 dfs 树的时候最好记录一下子树大小,便于链接时计算(你不能在链接时直接看右子树的数字大小,因为有可能有前导 $0$)。

\begin{lstlisting}
struct Q
{
	char ch;
	int prec;
	bool right;
};
tuple<vector<array<int, 2>>, vector<char>, int> parse_expr(string s, vector<Q> op) {
	static int idx[128];
	int maxp = 0, pos = 0, n, err = 0, i;
	{
		string t;
		for (char c : s)
		{
			if (t.size() && isdigit(t.back()) && isdigit(c)) t += '#';
			t += c;
		}
		swap(s, t);
		n = s.size();
	}
	for (i = 0; i < op.size(); ++i)
	{
		idx[op[i].ch] = i + 1;
		cmax(maxp, op[i].prec);
	}
	op.push_back({'#', ++maxp, 0});
	idx['#'] = op.size();
	vector<array<int, 2>> c(1);
	vector<char> ch(1);
	auto node = [&](char x) {
		c.push_back({0, 0});
		ch.push_back(x);
		return c.size() - 1;
	};
	function<int(int)> parse = [&](int lv) -> int {
		int u;
		if (lv > maxp)
		{
			if (pos < n && s[pos] == '(')
			{
				pos++;
				u = parse(1);
				if (err |= (pos >= n || s[pos++] != ')')) return 0;
				return u;
			}
			else if (pos < n && isdigit(s[pos])) return u = node(s[pos++]);
			else return err = 1, 0;
		}
		else
		{
			u = parse(lv + 1);
			while (!err && pos < n)
			{
				char ch = s[pos];
				int i = idx[ch] - 1;
				if (i >= 0 && op[i].prec == lv)
				{
					++pos;
					int v = node(ch), w = parse(lv + !op[i].right);
					c[v] = {u, w};
					u = v;
				}
				else break;
			}
			return u;
		}
	};
	int root = parse(0);
	for (auto [ch, _, __] : op) idx[ch] = 0;
	if (err || pos != n) return {{ }, { }, 0};
	return {c, ch, root};
}
int main()
{
	ios::sync_with_stdio(0); cin.tie(0);
	cout << fixed << setprecision(15);
	string s;
	getline(cin, s);
	vector<Q> op = {
		{'|', 1, 0},
		{'&', 2, 0},
	};
	auto [c, ch, root] = parse_expr(s, op);
	assert(root);
	function<array<int, 3>(int)> dfs = [&](int u)->array<int, 3> {
		if (isdigit(ch[u])) return {ch[u] - '0', 0, 0};
		auto [l, r1, r2] = dfs(c[u][0]);
		if (ch[u] == '|')
		{
			if (l) return {1, r1, r2 + 1};
			auto [r, r3, r4] = dfs(c[u][1]);
			return {r, r1 + r3, r2 + r4};
		}
		else
		{
			if (!l) return {0, r1 + 1, r2};
			auto [r, r3, r4] = dfs(c[u][1]);
			return {r, r1 + r3, r2 + r4};
		}
	};
	auto [r0, r1, r2] = dfs(root);
	cout << r0 << endl << r1 << ' ' << r2 << endl;
}

\end{lstlisting}

\newpage

\section{数学}

\subsection{任意模数矩阵求逆(未验证)}

$O(n^3)$,$O(n^2)$。

原理和任意模数行列式类似,辗转相除。注意仍然要求对角线元素是有逆的。

\begin{lstlisting}
int ksm(int x,int y)
{
	int r=1;
	while (y)
	{
		if (y&1) r=(ll)r*x%p;
		y>>=1;x=(ll)x*x%p;
	}
	return r;
}
int phi(int n)
{
	int r=n;
	for (int i=2;i*i<=n;i++) if (n%i==0)
	{
		r=r/i*(i-1);n/=i;
		while (n%i==0) n/=i;
	}
	if (n>1) r=r/n*(n-1);
	return r;
}
void cal(int a[][N],int b[][N],int n)
{
	int i,j,k,r,ph=phi(p);
	for (i=1;i<=n;i++) memset(b+1,0,n<<2);
	for (i=1;i<=n;i++) b[i][i]=1;
	for (i=1;i<=n;i++)
	{
		k=i;
		for (j=i+1;j<=n;j++) if (a[j][i]&&a[j][i]<a[k][i]) k=j;
		if (!a[k][i]) {puts("No Solution");exit(0);}
		swap(a[i],a[k]);swap(b[i],b[k]);
		for (j=i+1;j<=n;j++) if (a[j][i])
		{
			r=p-a[j][i]/a[i][i];
			for (k=i;k<=n;k++) a[j][k]=(a[j][k]+(ll)r*a[i][k])%p;
			for (k=1;k<=n;k++) b[j][k]=(b[j][k]+(ll)r*b[i][k])%p;
			while (a[j][i])
			{
				swap(a[i],a[j]);swap(b[i],b[j]);
				r=p-a[j][i]/a[i][i];
				for (k=i;k<=n;k++) a[j][k]=(a[j][k]+(ll)r*a[i][k])%p;
				for (k=1;k<=n;k++) b[j][k]=(b[j][k]+(ll)r*b[i][k])%p;
			}
		}
		if (__gcd(a[i][i],p)!=1) {puts("No Solution");exit(0);}
		r=ksm(a[i][i],ph-1);
		for (j=i;j<=n;j++) a[i][j]=(ll)a[i][j]*r%p;
		for (j=1;j<=n;j++) b[i][j]=(ll)b[i][j]*r%p;
		assert(a[i][i]==1);
		for (j=1;j<i;j++)
		{
			r=p-a[j][i];
			for (k=i;k<=n;k++) a[j][k]=(a[j][k]+(ll)r*a[i][k])%p;
			for (k=1;k<=n;k++) b[j][k]=(b[j][k]+(ll)r*b[i][k])%p;
		}
	}
}
\end{lstlisting}

\subsection{矩阵类(较新)}

\begin{lstlisting}
using ll = unsigned long long;
const ll p = 998244353;
ll ksm(ll x, ll y)
{
	ll r = 1;
	while (y)
	{
		if (y & 1) r = r * x % p;
		x = x * x % p; y >>= 1;
	}
	return r;
}
struct matrix;
matrix E(int n);
struct matrix :vector<vector<ll>>
{
	explicit matrix(int n = 0, int m = 0) :vector(n, vector<ll>(m)) { }
	pair<int, int> sz() const { if (size()) return {size(), back().size()}; return {0, 0}; }
	matrix &operator+=(const matrix &b)
	{
		assert(sz() == b.sz());
		auto [n, m] = sz();
		for (int i = 0; i < n; i++) for (int j = 0; j < m; j++) ((*this)[i][j] += b[i][j]) %= p;
		return *this;
	}
	matrix &operator-=(const matrix &b)
	{
		assert(sz() == b.sz());
		auto [n, m] = sz();
		for (int i = 0; i < n; i++) for (int j = 0; j < m; j++) ((*this)[i][j] += p - b[i][j]) %= p;
		return *this;
	}
	matrix operator*(const matrix &b) const
	{
		auto [n, m] = sz();
		auto [_, q] = b.sz();
		assert(m == _);
		int i, j, k;
		matrix c(n, q);
		for (k = 0; k < m; k++)
		{
			for (i = 0; i < n; i++) for (j = 0; j < q; j++) c[i][j] += (*this)[i][k] * b[k][j];
			if (!((k ^ q - 1) & 15)) for (auto &v : c) for (ll &x : v) x %= p;
		}
		static_assert(-1llu / p / p > 17);
		return c;
	}
	matrix operator+(const matrix &b) const { auto a = *this; return a += b; }
	matrix operator-(const matrix &b) const { auto a = *this; return a -= b; }
	matrix &operator*=(const matrix &b) { return *this = *this * b; }
	matrix &operator*=(ll k) { for (auto &v : *this) for (ll &x : v) x = x * k % p; return *this; }
	matrix operator*(ll k) const { auto a = *this; return a *= k; }
	matrix transpose() const
	{
		auto [n, m] = sz();
		matrix res(m, n);
		for (int i = 0; i < n; i++) for (int j = 0; j < m; j++) res[j][i] = (*this)[i][j];
		return res;
	}
	int rank() const
	{
		auto [n, m] = sz();
		vector<vector<ll>> a = n <= m ? *this : transpose();
		if (n > m) ::swap(n, m);
		int i, j, k, l, r = 0;
		for (i = 0, j = 0; i < n && j < m; j++)
		{
			for (k = i; k < n; k++) if (a[k][j]) break;
			if (k == n) continue;
			::swap(a[i], a[k]);
			ll iv = ksm(a[i][j], p - 2);
			for (k = j; k < m; k++) a[i][k] = a[i][k] * iv % p;
			for (k = i + 1; k < n; k++) for (l = j + 1; l < m; l++) a[k][l] = (a[k][l] + (p - a[k][j]) * a[i][l]) % p;
			++i; ++r;
		}
		return r;
	}
	vector<ll> poly() const
	{
		auto [n, m] = sz();
		vector<vector<ll>> a = *this;
		assert(n == m);
		int i, j, k;
		for (i = 1; i < n; i++)
		{
			for (j = i; j < n && !a[j][i - 1]; j++);
			if (j == n) continue;
			if (j > i)
			{
				::swap(a[i], a[j]);
				for (k = 0; k < n; k++) ::swap(a[k][j], a[k][i]);
			}
			ll r = a[i][i - 1];
			for (j = 0; j < n; j++) a[j][i] = a[j][i] * r % p;
			r = ksm(r, p - 2);
			for (j = i - 1; j < n; j++) a[i][j] = a[i][j] * r % p;
			for (j = i + 1; j < n; j++)
			{
				r = a[j][i - 1];
				for (k = 0; k < n; k++) a[k][i] = (a[k][i] + a[k][j] * r) % p;
				r = p - r;
				for (k = i - 1; k < n; k++) a[j][k] = (a[j][k] + a[i][k] * r) % p;
			}
		}
		vector g(n + 1, vector<ll>(n + 1));
		g[0][0] = 1;
		for (i = 0; i < n; i++)
		{
			ll r = p - 1, rr;
			for (j = i; j >= 0; j--)//第 j 行选第 n 列
			{
				rr = r * a[j][i] % p;
				for (k = 0; k <= j; k++) g[i + 1][k] = (g[i + 1][k] + rr * g[j][k]) % p;
				if (j) r = r * a[j][j - 1] % p;
			}
			for (k = 1; k <= i + 1; k++) (g[i + 1][k] += g[i][k - 1]) %= p;
		}
		auto f = g[n];
		//if (n&1) for (i=0;i<=n;i++) if (f[i]) f[i]=p-f[i];//若注释掉则为 |kE-A|
		return f;
	}
	ll det() const
	{
		auto [n, m] = sz();
		vector<vector<ll>> a = *this;
		assert(n == m);
		int i, j, k;
		ll r = 1;
		for (i = 0; i < n; i++)
		{
			for (j = i; j < n; j++) if (a[j][i]) break;
			if (j == n) return 0;
			if (i != j) r = p - r, ::swap(a[i], a[j]);
			(r *= a[i][i]) %= p;
			ll iv = ksm(a[i][i], p - 2);
			for (j = i; j < n; j++) a[i][j] = a[i][j] * iv % p;
			for (j = i + 1; j < n; j++) for (k = i + 1; k < n; k++) a[j][k] = (a[j][k] + (p - a[i][k]) * a[j][i]) % p;
		}
		return r % p;
	}
	tuple<int, vector<ll>, vector<vector<ll>>> gauss(const vector<ll> &b) const//Ax=b, rank of base, one sol, base
	{
		auto [n, m] = sz();
		if (b.size() != n) return {-1, { }, { }};
		vector<vector<ll>> a = *this;
		int i, j, k, R = m;
		for (i = 0; i < n; i++) a[i].push_back(b[i]);
		vector<int> fix(m, -1);
		for (i = k = 0; i < m; i++)
		{
			for (j = k; j < n; j++) if (a[j][i]) break;
			if (j == n) continue;
			fix[i] = k; --R;
			::swap(a[k], a[j]);
			auto &u = a[k];
			ll x = ksm(u[i], p - 2);
			for (j = i; j <= m; j++) u[j] = u[j] * x % p;
			for (auto &v : a) if (v.data() != u.data())
			{
				x = p - v[i];
				for (j = i; j <= m; j++) v[j] = (v[j] + x * u[j]) % p;
			}
			++k;
		}
		for (i = k; i < n; i++) if (a[i][m]) return {-1, { }, { }};
		vector<ll> r(m);
		vector<vector<ll>> c;
		for (i = 0; i < m; i++) if (fix[i] != -1) r[i] = a[fix[i]][m];
		for (i = 0; i < m; i++) if (fix[i] == -1)
		{
			vector<ll> r(m);
			r[i] = 1;
			for (j = 0; j < m; j++) if (fix[j] != -1) r[j] = (p - a[fix[j]][i]) % p;
			c.push_back(r);
		}
		return {R, r, c};
	}
	optional<matrix> inverse() const
	{
		auto [n, m] = sz();
		assert(n == m);
		vector<int> ih(n, -1), jh(n, -1);
		matrix a = *this;
		int i, j, k;
		for (k = 0; k < n; k++)
		{
			for (i = k; i < n; i++) if (ih[k] == -1) for (j = k; j < n; j++) if (a[i][j])
			{
				ih[k] = i;
				jh[k] = j;
				break;
			}
			if (ih[k] == -1) return { };
			::swap(a[k], a[ih[k]]);
			for (i = 0; i < n; i++) ::swap(a[i][k], a[i][jh[k]]);
			if (!a[k][k]) return { };
			a[k][k] = ksm(a[k][k], p - 2);
			for (i = 0; i < n; i++) if (i != k) (a[k][i] *= a[k][k]) %= p;
			for (i = 0; i < n; i++) if (i != k) for (j = 0; j < n; j++) if (j != k)
				(a[i][j] += (p - a[i][k]) * a[k][j]) %= p;
			for (i = 0; i < n; i++) if (i != k) (a[i][k] *= p - a[k][k]) %= p;
		}
		for (k = n - 1; k >= 0; k--)
		{
			::swap(a[k], a[jh[k]]);
			for (i = 0; i < n; i++) ::swap(a[i][k], a[i][ih[k]]);
		}
		return a;
	}
	matrix adjugate() const
	{
		auto [n, m] = sz();
		assert(n == m);
		int R = rank();
		if (n == 1) return E(1);
		if (R == n) return *inverse() * det();
		if (R == n - 1)
		{
			int i, j, k, l;
			auto [_, x, dx] = gauss(vector<ll>(n));
			auto [__, y, dy] = transpose().gauss(vector<ll>(n));
			if (count(all(x), 0) == n) x = dx[0];
			if (count(all(y), 0) == n) y = dy[0];
			for (k = 0; k < n; k++) if (x[k]) break;
			for (l = 0; l < n; l++) if (y[l]) break;
			assert(k < n && l < n);
			matrix res(n, n), c(n - 1, n - 1);
			for (i = 0; i < n; i++) if (i != l) for (j = 0; j < n; j++) if (j != k) c[i - (i > l)][j - (j > k)] = (*this)[i][j];
			for (i = 0; i < n; i++) for (j = 0; j < n; j++) res[i][j] = x[i] * y[j] % p;
			ll t = c.det() * ksm((k + l & 1) ? p - res[k][l] : res[k][l], p - 2) % p;
			assert(res[k][l]);
			assert(c.det());
			assert(t);
			return res * t;
		}
		return matrix(n, n);
	}
};
istream &operator>>(istream &cin, matrix &r) { for (auto &v : r) for (ll &x : v) cin >> x; return cin; }
ostream &operator<<(ostream &cout, const matrix &r) { auto [n, m] = r.sz(); for (int i = 0; i < n; i++) for (int j = 0; j < m; j++) cout << r[i][j] << " \n"[j + 1 == m]; return cout; }
matrix E(int n) { matrix r(n, n); for (int i = 0; i < n; i++) r[i][i] = 1; return r; }
matrix pow(matrix a, long long k)
{
	assert(k >= 0);
	auto [n, m] = a.sz();
	assert(n == m);
	matrix r = k & 1 ? a : E(n);
	k >>= 1;
	while (k)
	{
		a *= a;
		if (k & 1) r *= a;
		k >>= 1;
	}
	return r;
}
matrix pow2(matrix a, long long k)
{
	vector<ll> f = a.poly();
	int n = f.size() - 1, i, j;
	if (!n) return matrix();
	if (n == 1) return E(1) * ksm(a[0][0], k);
	assert(f[n] == 1);
	vector<ll> r(n), x(n), t(n * 2);
	r[0] = x[1] = 1;
	for (ll &x : f) x = (p - x) % p;
	reverse(all(f));
	fill(all(t), 0);
	if (k & 1)
	{
		for (i = 0; i < n; i++) for (j = 0; j < n; j++) t[i + j] = (t[i + j] + r[i] * x[j]) % p;
		for (i = n * 2 - 2; i >= n; i--) for (j = 1; j <= n; j++) t[i - j] = (t[i - j] + f[j] * t[i]) % p;
		for (i = 0; i < n; i++) r[i] = t[i];
	}
	k >>= 1;
	while (k)
	{
		fill(all(t), 0);
		for (i = 0; i < n; i++) for (j = 0; j < n; j++) t[i + j] = (t[i + j] + x[i] * x[j]) % p;
		for (i = n * 2 - 2; i >= n; i--) for (j = 1; j <= n; j++) t[i - j] = (t[i - j] + f[j] * t[i]) % p;
		for (i = 0; i < n; i++) x[i] = t[i];
		if (k & 1)
		{
			fill(all(t), 0);
			for (i = 0; i < n; i++) for (j = 0; j < n; j++) t[i + j] = (t[i + j] + r[i] * x[j]) % p;
			for (i = n * 2 - 2; i >= n; i--) for (j = 1; j <= n; j++) t[i - j] = (t[i - j] + f[j] * t[i]) % p;
			for (i = 0; i < n; i++) r[i] = t[i];
		}
		k >>= 1;
	}
	matrix res(n, n);
	int b = ceil(sqrt(n));
	vector<matrix> s(b + 1);
	s[0] = E(n); s[1] = a;
	for (i = 2; i <= b; i++) s[i] = s[i - 1] * a;
	for (i = b - 1; i >= 0; i--)
	{
		res *= s[b];
		for (j = min(n, (i + 1) * b) - 1; j >= i * b; j--) res += s[j - i * b] * r[j];
	}
	return res;
}
\end{lstlisting}

\subsection{最短递推式(BM 算法)}

给定 $\{a\}$,求最短的 $\{r\}$ 满足 $\sum\limits_{j=0}^{m-1} a_{i-j-1}r_j=a_i$。

\begin{lstlisting}
vector<ui> bm(const vector<ui> &a)
{
	vector<ui> r,lst;
	int n=a.size(),m=0,q=0,i,j,k=-1;
	ui D=0;
	for (i=0;i<n;i++)
	{
		ui cur=0;
		for (j=0;j<m;j++) cur=(cur+(ll)a[i-j-1]*r[j])%p;
		cur=(a[i]+p-cur)%p;
		if (!cur) continue;
		if (k==-1)
		{
			k=i;
			D=cur;
			r.resize(m=i+1);
			continue;
		}
		auto v=r;
		ui x=(ll)cur*ksm(D,p-2)%p;
		if (m<q+i-k) r.resize(m=q+i-k);
		(r[i-k-1]+=x)%=p;
		ui *b=r.data()+i-k;
		x=(p-x)%p;
		for (j=0;j<q;j++) b[j]=(b[j]+(ll)x*lst[j])%p;
		if (v.size()+k<lst.size()+i)
		{
			lst=v;
			q=v.size();
			k=i;
			D=cur;
		}
	}
	return r;
}
\end{lstlisting}

\subsection{在线 $O(1)$ 逆元}

预处理复杂度为 $O(p^{\frac{2}{3}})$。

\begin{lstlisting}
namespace online_inv
{
	typedef unsigned int ui;
	typedef unsigned long long ll;
	const ll p=1e9+7,n=1010,m=n*n,N=m+2;
	int l[N],r[N];
	ll y[N];
	bool s[N];
	ll _inv[N*2],i,j,k;
	void init_inv()
	{
		assert(n*n*n>p);
		_inv[1]=1;
		for (i=2;i<m*2;i++)
		{
			j=p/i;
			_inv[i]=(p-j)*_inv[p-i*j]%p;
		}
		s[0]=y[0]=1;
		for (i=1;i<n;i++) for (j=i;j<n;j++) if (!s[k=i*m/j])
		{
			y[k]=j;
			s[k]=1;
		}
		l[0]=1;
		for (i=1;i<=m;i++) l[i]=s[i]?y[i]:l[i-1];
		r[m]=1;
		for (i=m-1;~i;i--) r[i]=s[i]?y[i]:r[i+1];
		for (i=0;i<=m;i++) y[i]=min(l[i],r[i]);
	}
	inline ll inv(const ll &x)
	{
		assert(x&&x<p);
		if (x<m*2) return _inv[x];
		k=x*m/p;
		j=y[k]*x%p;
		return (j<m*2?_inv[j]:p-_inv[p-j])*y[k]%p;
	}
}
using online_inv::init_inv,online_inv::inv,online_inv::p;
\end{lstlisting}

\subsection{Strassen 矩阵乘法}

没用,不如卡常。$O(n^{\log_27})$。

\begin{lstlisting}
#include "bits/stdc++.h"
using namespace std;
typedef unsigned int ui;
typedef unsigned long long ull;
const ui p=998244353;
const ull fh=1ull<<31;
struct Q
{
	ui **a;
	int n;
	Q(){n=0;}
	void clear()
	{
		for (int i=0;i<n;i++) delete a[i];
		if (n) delete a;n=0;
	}
	Q(int nn)//不能传入不是 2 的幂的数!
	{
		n=nn;
		assert(n==(n&-n));
		a=new ui*[n];
		for (int i=0;i<n;i++) a[i]=new ui[n],memset(a[i],0,n*sizeof a[0][0]);
	}
	const Q & operator=(const Q& b)
	{
		clear();n=b.n;
		a=new ui*[n];
		for (int i=0;i<n;i++) a[i]=new ui[n],memcpy(a[i],b.a[i],n*sizeof a[0][0]);
		return *this;
	}
	~Q(){clear();}
	Q operator+(const Q &b)
	{
		Q c(n);
		for (int i=0;i<n;i++) for (int j=0;j<n;j++) if ((c.a[i][j]=a[i][j]+b.a[i][j])>=p) c.a[i][j]-=p;
		return c;
	}
	Q operator-(const Q &b)
	{	
		Q c(n);
		for (int i=0;i<n;i++) for (int j=0;j<n;j++) if ((c.a[i][j]=a[i][j]-b.a[i][j])&fh) c.a[i][j]+=p;
		return c;
	}
	Q operator*(Q &b)
	{
		Q c(n);
		if (n<=128)
		{
			for (int i=0;i<n;i++) for (int k=0;k<n;k++) for (int j=0;j<n;j++) c.a[i][j]=(c.a[i][j]+(ull)a[i][k]*b.a[k][j])%p;
			return c;
		}
		Q A[2][2],B[2][2],s[10],p[5];
		n>>=1;
		int i,j,k,l;
		for (i=0;i<2;i++) for (j=0;j<2;j++)
		{
			A[i][j]=Q(n);
			for (k=0;k<n;k++) memcpy(A[i][j].a[k],a[k+i*n]+j*n,n*sizeof a[0][0]);
			B[i][j]=Q(n);
			for (k=0;k<n;k++) memcpy(B[i][j].a[k],b.a[k+i*n]+j*n,n*sizeof a[0][0]);
		}
		s[0]=B[0][1]-B[1][1];
		s[1]=A[0][0]+A[0][1];
		s[2]=A[1][0]+A[1][1];
		s[3]=B[1][0]-B[0][0];
		s[4]=A[0][0]+A[1][1];
		s[5]=B[0][0]+B[1][1];
		s[6]=A[0][1]-A[1][1];
		s[7]=B[1][0]+B[1][1];
		s[8]=A[0][0]-A[1][0];
		s[9]=B[0][0]+B[0][1];
		p[0]=A[0][0]*s[0];
		p[1]=s[1]*B[1][1];
		p[2]=s[2]*B[0][0];
		p[3]=A[1][1]*s[3];
		p[4]=s[4]*s[5];
		A[0][0]=p[4]+p[3]-p[1]+s[6]*s[7];
		A[0][1]=p[0]+p[1];
		A[1][0]=p[2]+p[3];
		A[1][1]=p[4]+p[0]-p[2]-s[8]*s[9];
		for (i=0;i<2;i++) for (j=0;j<2;j++)	for (k=0;k<n;k++) memcpy(c.a[k+i*n]+j*n,A[i][j].a[k],n*sizeof a[0][0]);
		n<<=1;
		return c;
	}
};
int main()
{
	int i,j,n,m,k;
	ios::sync_with_stdio(0);cin.tie(0);
	cin>>n>>m>>k;
	int N=1<<32-min({__builtin_clz(n-1),__builtin_clz(m-1),__builtin_clz(k-1)});
	Q a(N),b(N);
	for (i=0;i<n;i++) for (j=0;j<m;j++) cin>>a.a[i][j];
	for (i=0;i<m;i++) for (j=0;j<k;j++) cin>>b.a[i][j];
	a=a*b;
	for (i=0;i<n;i++) for (j=0;j<k;j++) cout<<a.a[i][j]<<" \n"[j+1==k]; 
}
\end{lstlisting}

\subsection{扩展欧拉定理}

求 $a\uparrow\uparrow b \bmod c$。前面的 Prime 命名空间只是求 $\varphi$ 用的。

\begin{lstlisting}
namespace Prime
{
	typedef unsigned int ui;
	typedef unsigned long long ll;
	const int N=1e6+2;
	const ll M=(ll)(N-1)*(N-1);
	ui pr[N],mn[N],phi[N],cnt;
	int mu[N];
	void init_prime()
	{
		ui i,j,k;
		phi[1]=mu[1]=1;
		for (i=2;i<N;i++)
		{
			if (!mn[i])
			{
				pr[cnt++]=i;
				phi[i]=i-1;mu[i]=-1;
				mn[i]=i;
			}
			for (j=0;(k=i*pr[j])<N;j++)
			{
				mn[k]=pr[j];
				if (i%pr[j]==0)
				{
					phi[k]=phi[i]*pr[j];
					break;
				}
				phi[k]=phi[i]*(pr[j]-1);
				mu[k]=-mu[i];
			}
		}
		//for (i=2;i<N;i++) if (mu[i]<0) mu[i]+=p;
	}
	template<class T> T getphi(T x)
	{
		assert(M>=x);
		T r=x;
		for (ui i=0;i<cnt&&(T)pr[i]*pr[i]<=x&&x>=N;i++) if (x%pr[i]==0)
		{
			ui y=pr[i],tmp;
			x/=y;
			while (x==(tmp=x/y)*y) x=tmp;
			r=r/y*(y-1);
		}
		if (x>=N) return r/x*(x-1);
		while (x>1)
		{
			ui y=mn[x],tmp;
			x/=y;
			while (x==(tmp=x/y)*y) x=tmp;
			r=r/y*(y-1);
		}
		return r;
	}
	template<class T> vector<pair<T,ui>> getw(T x)
	{
		assert(M>=x);
		vector<pair<T,ui>> r;
		for (ui i=0;i<cnt&&(T)pr[i]*pr[i]<=x&&x>=N;i++) if (x%pr[i]==0)
		{
			ui y=pr[i],z=1,tmp;
			x/=y;
			while (x==(tmp=x/y)*y) x=tmp,++z;
			r.push_back({y,z});
		}
		if (x>=N)
		{
			r.push_back({x,1});
			return r;
		}
		while (x>1)
		{
			ui y=mn[x],z=1,tmp;
			x/=y;
			while (x==(tmp=x/y)*y) x=tmp,++z;
			r.push_back({y,z});
		}
		return r;
	}
}
using Prime::pr,Prime::phi,Prime::getw,Prime::getphi;
using Prime::mu,Prime::init_prime;
ui ksm(ll x,ui y,ui p)
{
	x=x%p+(x>=p)*p;
	ll r=1;
	while (y)
	{
		if (y&1)
		{
			if ((r*=x)>=p) r=r%p+p; else r%=p;
		}
		if ((x*=x)>=p) x=x%p+p; else x%=p;
		y>>=1;
	}
	return r;
}
struct Q
{
	vector<ui> p;
	Q(const ui &P)
	{
		p.push_back(P);
		while (p.back()>1) p.push_back(getphi(p.back()));
	}
	ui operator()(ll a,ll b)
	{
		if (!a) return (1^b&1)%p[0];
		ui r=1;
		int i=min(b,(ll)p.size());
		while ((--i)>=0) r=ksm(a,r,p[i]);
		return r%p[0];
	}
};
int main()
{
	ios::sync_with_stdio(0);cin.tie(0);
	cout<<setiosflags(ios::fixed)<<setprecision(15);
	int n,i;
	init_prime();
	int T;
	cin>>T;
	while (T--)
	{
		ui a,b,c;
		cin>>a>>b>>c;
		cout<<Q(c)(a,b)<<'\n';
	}
}
\end{lstlisting}

\subsection{exgcd}

$O(\log p)$,$O(\log p)$。

递归版:

\begin{lstlisting}
int exgcd(int a,int b,int c)//ax+by=c,return x
{
	if (a==0) return c/b;
	return (c-(ll)b*exgcd(b%a,a,c))/a%b;
}
\end{lstlisting}

递推重构版:

\begin{lstlisting}
pair<ll,ll> exgcd(ll a,ll b,ll c)//ax+by=c,{-1,-1} 无解,b=0 返回 {c/a,0},否则返回最小非负 x
{
	assert(a||b);
	if (!b) return {c/a,0};
	if (a<0) a=-a,b=-b,c=-c;
	ll d=gcd(a,b);
	if (c%d) return {-1,-1};
	ll x=1,x1=0,p=a,q=b,k;
	b=abs(b);
	while (b)
	{
		k=a/b;
		x-=k*x1;a-=k*b;
		swap(x,x1);
		swap(a,b);
	}
	b=abs(q/d);
	x=(c/d)%b*(x%b)%b;
	if (x<0) x+=b;
	return {x, (ll)((c-(lll)p*x)/q)};
}
ll fun(ll a, ll b, ll p)//ax=b(mod p)
{
	return exgcd(a, -p, b).first%p;
}
\end{lstlisting}

\subsection{exCRT}

实现了一个类 \verb|Q|,表示一条方程,支持合并。

\begin{lstlisting}
namespace CRT
{
	typedef long long ll;
	pair<ll,ll> exgcd(ll a,ll b,ll c)
	{
		assert(a||b);
		if (!b) return {c/a,0};
		ll d=gcd(a,b);
		if (c%d) return {-1,-1};
		ll x=1,x1=0,p=a,q=b,k;
		b=abs(b);
		while (b)
		{
			k=a/b;
			x-=k*x1;a-=k*b;
			swap(x,x1);
			swap(a,b);
		}
		b=abs(q/d);
		x=x*(c/d)%b;
		if (x<0) x+=b;
		return {x,(c-p*x)/q};
	}
	struct Q
	{
		ll p,r;//0<=r<p
		Q operator+(const Q &o) const
		{
			if (p==0||o.p==0) return {0,0};
			auto [x,y]=exgcd(p,-o.p,r-o.r);
			if (x==-1&&y==-1) return {0,0};
			ll q=lcm(p,o.p);
			return {q,((r-x*p)%q+q)%q};
		}
	};
}
using CRT::Q;
\end{lstlisting}


\subsection{exBSGS}

$O(\sqrt n)$。哈希表 ht 可以用 map 代替。

\begin{lstlisting}
namespace BSGS
{
	typedef unsigned int ui;
	typedef unsigned long long ll;
	template<int N,class T,class TT> struct ht//个数,定义域,值域
	{
		const static int p=1e6+7,M=p+2;
		TT a[N];
		T v[N];
		int fir[p+2],nxt[N],st[p+2];//和模数相适应
		int tp,ds;//自定义模数
		ht(){memset(fir,0,sizeof fir);tp=ds=0;}
		void mdf(T x,TT z)//位置,值
		{
			ui y=x%p;
			for (int i=fir[y];i;i=nxt[i]) if (v[i]==x) return a[i]=z,void();//若不可能重复不需要 for
			v[++ds]=x;a[ds]=z;
			if (!fir[y]) st[++tp]=y;
			nxt[ds]=fir[y];fir[y]=ds;
		}
		TT find(T x)
		{
			ui y=x%p;
			int i;
			for (i=fir[y];i;i=nxt[i]) if (v[i]==x) return a[i];
			return 0;//返回值和是否判断依据要求决定
		}
		void clear()
		{
			++tp;
			while (--tp) fir[st[tp]]=0;
			ds=0;
		}
	};
	const int N=5e4;
	ht<N,ui,ui> s;
	int exgcd(int a,int b)
	{
		if (a==1) return 1;
		return (1-(long long)b*exgcd(b%a,a))/a;//not ll
	}
	int bsgs(ui a,ui b,ui p)
	{
		s.clear();
		a%=p;b%=p;
		if (!a) return 1-min((int)b,2);//含 -1
		ui i,j,k,x,y;
		x=sqrt(p)+2;
		for (i=0,j=1;i<x;i++,j=(ll)j*a%p)
		{
			if (j==b) return i;
			s.mdf((ll)j*b%p,i+1);
		}
		k=j;
		for (i=1;i<=x;i++,j=(ll)j*k%p) if (y=s.find(j)) return (ll)i*x-y+1;
		return -1;
	}
	bool isprime(ui p)
	{
		if (p<=1) return 0;
		for (ui i=2;i*i<=p;i++) if (p%i==0) return 0;
		return 1;
	}
	int exbsgs(ui a,ui b,ui p)//a^x=b(mod p)
	{
		//if (isprime(p)) return bsgs(a,b,p);
		a%=p;b%=p;
		ui i,j,k,x,y=__lg(p),cnt=0;
		for (i=0,j=1%p;i<=y;i++,j=(ll)j*a%p) if (j==b) return i;
		y=1;
		while (1)
		{
			if ((x=gcd(a,p))==1) break;
			if (b%x) return -1;//no sol
			++cnt;
			p/=x;b/=x;
			y=(ll)y*(a/x)%p;
		}
		a%=p;
		b=(ll)b*(p+exgcd(y,p))%p;
		int r=bsgs(a,b,p);
		return r==-1?-1:r+cnt;
	}
}
using BSGS::bsgs,BSGS::exbsgs;
\end{lstlisting}

\subsection{exLucas}

求组合数。包含多个不同的版本,按需使用。

\begin{lstlisting}
namespace exlucas
{
	typedef long long ll;
	typedef pair<int,int> pa;
	int P,p,q,i;
	vector<pa> a;
	vector<vector<int> > b;
	vector<int> ph;
	vector<int> xs;
	int ksm(unsigned int x,ll y,const unsigned int p)
	{
		unsigned int r=1;
		while (y)
		{
			if (y&1) r=(unsigned long long)r*x%p;
			x=(unsigned long long)x*x%p;
			y>>=1;
		}
		return r;
	}
	void init(int x)//分解质因数,如有必要可以使用更快的方法
	{
		a.clear();b.clear();
		int i,y,z;
		vector<int> v;
		for (i=2;i*i<=x;i++) if (x%i==0)
		{
			z=i;x/=i;
			while (1)
			{
				y=x/i;
				if (i*y==x) x=y; else break;
				z*=i;
			}
			a.push_back(pa(i,z));
			b.push_back(v);
		}
		if (x>1) a.push_back(pa(x,x)),b.push_back(v);
		ph.resize(a.size());
		xs.resize(a.size());
		for (int k=0;k<a.size();k++)
		{
			tie(q,p)=a[k];
			ph[k]=p/q*(q-1);
			xs[k]=(ll)ksm(P/p,ph[k]-1,p)*(P/p)%P;
		}
	}
	void spinit(int x)//O(p) space
	{
		for (int k=0;k<a.size();k++)
		{
			int q,p;
			tie(q,p)=a[k];
			b[k].resize(p);
			b[k][0]=1;
			for (int i=1,j=q;i<p;i++) if (i==j) j+=q,b[k][i]=b[k][i-1]; else b[k][i]=(ll)b[k][i-1]*i%p;
		}
	}
	ll g(ll n)
	{
		ll r=0,s;
		while (n>=q)
		{
			n/=q;
			r+=n;
		}
		return r;
	}
	// int f(ll n)
	// {
	// 	if (n==0) return 1;
	// 	int r=1;//若 p>1e9 j 要 unsigned
	// 	for (int i=1,j=q;i<p;i++) if (i==j) j+=q; else r=(ll)r*i%p;
	// 	r=(ll)ksm(r,n/p,p)*f(n/q)%p;
	// 	n%=p;
	// 	for (int i=1,j=q;i<=n;i++) if (i==j) j+=q; else r=(ll)r*i%p;
	// 	return r;
	// }//O(T\sum p) time,O(1) space ver.
	int f(ll n)
	{
		int r=1;
		ll cs=0;
		while (n)
		{
			r=(ll)r*b[i][n%p]%p;
			cs+=n/p;
			n/=q;
		}
		return (ll)ksm(b[i][p-1],cs%ph[i],p)*r%p;
	}//O(\sum p) time,O(p) space ver.
	int C(ll n,ll m,int M)
	{
		if (n<m) return 0;
		int r=0,w;
		if (P!=M) init(P=M),spinit(P);//sp for O(p) space
		for (i=0;i<a.size();i++)
		{
			tie(q,p)=a[i];
			w=(ll)ksm(q,g(n)-g(m)-g(n-m),p)*f(n)%p*ksm((ll)f(m)*f(n-m)%p,ph[i]-1,p)%p;
			r=(r+(ll)xs[i]*w)%M;
		}
		return r;
	}
}
#define C(x,y,z) exlucas::C(x,y,z)
\end{lstlisting}

\subsection{杜教筛}

求 $\varphi(n)$ 的前缀和。

核心:构造 $g$ 满足 $h(n)=\sum\limits_{d|n}f(d)g(\frac{n}{d})$ 容易计算,

则有 $\sum\limits_{i=1}^n h(i)=\sum\limits_{i=1}^n g(i)\sum\limits_{j=1}^{\lfloor n/i\rfloor}f(j)$,

故 $g(1)\sum\limits_{j=1}^n f(j)=\sum\limits_{i=1}^n h(i)-\sum\limits_{i=2}^n g(i)\sum\limits_{j=1}^{\lfloor n/i\rfloor}f(j)$,

则 $f$ 前缀和可以递归求解。

\begin{lstlisting}
namespace du_seive
{
	typedef unsigned int ui;
	typedef unsigned long long ll;
	unordered_map<ll,ui> mp;
	const int N=1e7+2;
	const ui p=998244353;
	ui pr[N],phi[N];
	ui cnt;
	void init()
	{
		cnt=0;phi[1]=1;
		int i,j;
		for (i=2;i<N;i++)
		{
			if (!phi[i])
			{
				pr[cnt++]=i;
				phi[i]=i-1;
			}
			for (j=0;i*pr[j]<N;j++)
			{
				if (i%pr[j]==0)
				{
					phi[i*pr[j]]=phi[i]*pr[j];
					break;
				}
				phi[i*pr[j]]=phi[i]*(pr[j]-1);
			}
			if ((phi[i]+=phi[i-1])>=p) phi[i]-=p;
		}
	}
	ui get_phi_sum(ll n)
	{
		if (n<N) return phi[n];
		if (mp.count(n)) return mp[n];
		ui sum=0;
		for (ll i=2,j,k;i<=n;i=j+1)
		{
			j=n/(k=n/i);
			sum=(sum+(ll)get_phi_sum(k)*(j-i+1))%p;
		}
		ui nn=n%p;
		sum=(nn*(nn+1ll)/2+p-sum)%p;
		mp[n]=sum;
		return sum;
	}
}
using du_seive::init,du_seive::get_phi_sum;
\end{lstlisting}

\subsection{$\mu^2(n)$ 前缀和}

$10^{18}$,0.46s。

$\mu^2(n)=\sum\limits_{d^2|n}\mu(d)$

\begin{lstlisting}
const int N = 5e7 + 5;
int pr[N / 8], cnt, mu[N];
bool ed[N];
void init()
{
	ui i, j, k;
	mu[1] = 1;
	for (i = 2; i < N; i++)
	{
		if (!ed[i]) pr[++cnt] = i, mu[i] = -1;
		for (j = 1; pr[j] * i < N; j++)
		{
			ed[pr[j] * i] = 1;
			if (i % pr[j] == 0) break;
			mu[pr[j] * i] = -mu[i];
		}
		mu[i] += mu[i - 1];
	}
}
ll sum_mu(ll n)
{
	if (n < N) return mu[n];
	ll r = 1, i, j, k;
	for (i = 2; i <= n; i = j + 1)
	{
		j = n / (k = n / i);
		r -= sum_mu(k) * (j - i + 1);
	}
	return r;
}
ll sum_mu2(ll n)
{
	ll r = 0, i, j, k, l, s = 0, t;
	for (i = 1; i * i <= n; i = j + 1)
	{
		k = n / (i * i);
		j = sqrtl(n / k);
		t = sum_mu(j);
		r += k * (t - s);
		s = t;
	}
	return r;
}
int main()
{
	ll n;
	init();
	cin >> n;
	cout << sum_mu2(n) << endl;
}
\end{lstlisting}
\subsection{线性规划}
用法:构造函数指明目标函数系数,add 函数增加限制。额外的限制是 $x_i\ge 0$。
\begin{lstlisting}
typedef long double db;//__float128
struct linear
{
	static const int N=45;//n+m
	db r[N][N];
	int col[N],row[N];
	const db eps=1e-10,inf=1e9;//1e-17
	int n,m;
	template<class T> linear(const vector<T> &a)//target: maximize \sum a(i-1)xi
	{
		memset(r,0,sizeof r);
		memset(col,0,sizeof col);
		memset(row,0,sizeof row);
		n=a.size();m=0;
		for (int i=1;i<=n;i++) r[0][i]=-a[i-1];
	}
	template<class T> void add(const vector<T> &a,db b)//limit: \sum a(i-1)xi<=b
	{
		assert(a.size()==n);
		++m;
		for (int i=1;i<=n;i++) r[m][i]=-a[i-1];
		r[m][0]=b;
	}
	void pivot(int k, int t)
	{
		swap(row[k+n],row[t]);
		db rkt=-r[k][t];
		int i,j;
		for (i=0;i<=n;i++) r[k][i]/=rkt;
		r[k][t]=-1/rkt;
		for (i=0;i<=m;i++) if (i!=k)
		{
			db rit=r[i][t];
			if (rit>=-eps&&rit<=eps) continue;
			for (j=0;j<=n;j++) if (j!=t) r[i][j]+=rit*r[k][j];
			r[i][t]=r[k][t]*rit;
		}
	}
	bool init()
	{
		int i;
		for (i=1;i<=n+m;i++) row[i]=i;
		while(1)
		{
			int q=1;
			auto b_min=r[1][0];
			for (i=2;i<=m;i++) if (r[i][0]<b_min) b_min=r[i][0],q=i;
			if (b_min+eps>=0) return 1;
			int p=0;
			for (i=1;i<=n;i++) if (r[q][i]>eps&&(!p||row[i]>row[p])) p=i;
			if (!p) break;
			pivot(q,p);
		}
		return 0;
	}
	bool simplex()
	{
		while (1)
		{
			int t=1,k=0,i;
			for (i=2;i<=n;i++) if (r[0][i]<r[0][t]) t=i;
			if (r[0][t]>=-eps) return 1;
			db ratio_min=inf;
			for (i=1;i<=m;i++) if (r[i][t]<-eps)
			{
				db ratio=-r[i][0]/r[i][t];
				if (!k||ratio<ratio_min||ratio<=ratio_min+eps&&row[i]>row[k])
				{
					ratio_min=ratio;
					k=i;
				}
			}
			if (!k) break;
			pivot(k,t);
		}
		return 0;
	}
	void solve(int type)
	{
		if (!init())
		{
			cout<<"Infeasible\n";
			return;
		}
		if (!simplex())
		{
			cout<<"Unbounded\n";
			return;
		}
		cout<<(long double)(-r[0][0])<<'\n';
		if (type)
		{
			int i;
			memset(col+1,0,n*sizeof col[0]);
			for (i=n+1;i<=n+m;i++) col[row[i]]=i;
			for (i=1;i<=n;i++) cout<<(long double)(col[i]?r[col[i]-n][0]:0)<<" \n"[i==n];
		}
	}
};
\end{lstlisting}

\subsection{线性插值($k$ 次幂和)}

$O(m)$,$O(m)$。

\begin{lstlisting}
ll interpolation(vector<ll> a, ll n)
{
	int m = a.size(), i;
	vector<ll> ans(2);
	n %= p;
	if (n < m) return a[n];
	ll k = ifac[m - 1];
	for (i = m - 1;i >= 0;i--)
	{
		(a[i] *= k) %= p;
		(k *= n - i) %= p;
	}
	k = 1;
	for (i = 0;i < m;i++)
	{
		(ans[(m ^ i) & 1] += a[i] * k) %= p;
		k = k * inv[i + 1] % p * (n - i) % p * (m - i - 1) % p;
	}
	return (ans[1] + p - ans[0]) % p;
}
ll sum_of_kth_power(ll n, ll k)
{
	if (n == 0) return 0;
	ll m = min(n + 1, k + 2);
	int i;
	vector<ll> s(m);
	vector<int> pr, ed(m);pr.reserve(m / 4);
	s[1] = 1;
	for (i = 2;i < m;i++)
	{
		if (!ed[i]) s[i] = ksm(i, k);
		for (int j : pr) if (i * j < m)
		{
			s[i * j] = s[i] * s[j] % p;
			if (i % j == 0) break;
		}
		else break;
	}
	for (i = 1;i < m;i++) (s[i] += s[i - 1]) %= p;
	return interpolation(s, n);
}
\end{lstlisting}

\subsection{单原根(仅手动验证质数)}

\begin{lstlisting}
namespace get_root
{
	typedef unsigned int ui;
	typedef unsigned long long ll;
	ui ksm(ui x,ui y,ui p)
	{
		ui r=1;
		while (y)
		{
			if (y&1) r=(ll)r*x%p;
			x=(ll)x*x%p;y>>=1;
		}
		return r;
	}
	vector<ui> getw(ui n)
	{
		vector<ui> w;
		for (ui i=2;i*i<=n;i++) if (n%i==0)
		{
			w.push_back(i);
			n/=i;
			for (ui j=n/i;n==i*j;j=n/i) n/=i;
		}
		if (n>1) w.push_back(n);
		return w;
	}
	int getrt(ui n)
	{
		if (n<=2) return n-1;
		auto w=getw(n);
		ui ph=n;
		for (ui x:w) ph=ph/x*(x-1);
		w=getw(ph);
		for (ui &x:w) x=ph/x;
		for (ui i=2;i<n;i++) if (gcd(i,n)==1)
		{
			for (ui x:w) if (ksm(i,x,n)==1) goto no;
			return i;
			no:;
		}
		return -1;
	}
}
using get_root::getrt;
\end{lstlisting}

\subsection{稍快单原根(仅验证质数)}

\begin{lstlisting}
namespace get_root
{
	typedef unsigned int ui;
	typedef unsigned long long ll;
	bool ied=0;
	const int N=1e5+5;
	vector<ui> pr;
	bool ed[N];
	void init()
	{
		pr.reserve(N);
		for (ui i=2;i<N;i++)
		{
			if (!ed[i]) pr.push_back(i);
			for (ui x:pr)
			{
				if (i*x>=N) break;
				ed[i*x]=1;
				if (i%x==0) break;
			}
		}
	}
	ui ksm(ui x,ui y,ui p)
	{
		ui r=1;
		while (y)
		{
			if (y&1) r=(ll)r*x%p;
			x=(ll)x*x%p;y>>=1;
		}
		return r;
	}
	vector<ui> getw(ui n)
	{
		vector<ui> w;
		for (ui x:pr)
		{
			if (x*x>n) break;
			if (n%x==0)
			{
				w.push_back(x);
				n/=x;
				for (ui i=n/x;n==x*i;i=n/x) n/=x;
			}
		}
		if (n>1) w.push_back(n);
		return w;
	}
	int getrt(ui n)
	{
		if (n<=2) return n-1;
		if (!ed[4]) init();
		auto w=getw(n);
		ui ph=n;
		for (ui x:w) ph=ph/x*(x-1);
		w=getw(ph);
		for (ui &x:w) x=ph/x;
		for (ui i=2;i<n;i++) if (gcd(i,n)==1)
		{
			for (ui x:w) if (ksm(i,x,n)==1) goto no;
			return i;
			no:;
		}
		return -1;
	}
}
using get_root::getrt;
\end{lstlisting}

\subsection{筛全部原根}

\begin{lstlisting}
#include "bits/stdc++.h"
using namespace std;
typedef long long ll;
const int N=1e6+2;
int ss[N],mn[N],fmn[N],phi[N];
int t,n,gs,i,d;
bool ed[N],av[N],yg[N],hv[N];
double inv[N];
void getfac(int x,int *a,int &n)
{
	int y=x,z;
	if (1^x&1)
	{
		a[n=1]=2;x>>=1;while (1^x&1) x>>=1;
	}
	while (x>1)
	{
		x=1e-9+(x*inv[a[++n]=z=mn[x]]);
		while (x%z==0) x=1e-9+x*inv[z];
	}
	for (i=1;i<=n;i++) av[a[i]]=0,a[i]=1e-9+(y*inv[a[i]]);
}
int ksm(int x,int y,int p)
{
	int r=1;
	while (y)
	{
		if (y&1) r=(ll)r*x%p;
		x=(ll)x*x%p;y>>=1;
	}
	return r;
}
bool ck(int x,int *a,int n,int p)
{
	for (int i=1;i<=n;i++) if (ksm(x,a[i],p)==1) return 0;
	return 1;
} 
void getrt(int x,int d)
{
	if (!hv[x]) return puts("0\n"),void();
	static int a[30];
	int n=0,y,i,g=0,c=d;y=phi[x];
	fill(av+1,av+y+1,1);
	getfac(y,a,n);
	for (i=1;i<x;i++) if (__gcd(i,x)==1&&ck(i,a,n,x)) break;
	yg[g=i]=1;//g就是最小原根
	int j=(ll)g*g%x;
	for (i=2;i<y;i++,j=(ll)j*g%x) yg[j]=av[i]=av[mn[i]]&av[fmn[i]]; 
	printf("%d\n",phi[y]);
	for (i=1;i<x;i++) if (yg[i]) 
	{
		yg[i]=0;
		if (--c==0) printf("%d ",i),c=d;
	}puts("");
}
void init()
{
	int i,j,k,n=N-1;
	mn[1]=phi[1]=1;
	for (i=1;i<=n;i++) inv[i]=1.0/i;
	for (i=2;i<=n;i++)
	{
		if (!ed[i]) phi[mn[i]=ss[++gs]=i]=i-1,hv[i]=1;
		for (j=1;j<=gs&&(k=ss[j]*i)<=n;j++)
		{
			ed[k]=1;mn[k]=ss[j];
			if (i%ss[j]==0) {phi[k]=phi[i]*ss[j];hv[k]=hv[i];break;}
			phi[k]=phi[i]*(ss[j]-1);
		}
	}
	for (i=n;i;i--) fmn[i]=1e-9+(i*inv[mn[i]]),hv[i]|=(1^i&1)&&hv[i>>1];
	for (i=8;i<=n;i<<=1) hv[i]=0;
}
int main()
{
	init();
	scanf("%d",&t);
	while (t--)
	{
		scanf("%d%d",&n,&d);
		getrt(n,d);
	}
}
\end{lstlisting}

\subsection{高斯消元(列主元)}

$O(n^3)$,$O(n^2)$。

浮点数的版本。

\begin{lstlisting}
namespace Gauss
{
	typedef double db;
	const db eps=1e-8;
	template<class T> pair<vector<db>,int> solve(const vector<vector<T>> &A)//和为 0。返回秩,负数无解
	{
		assert(A.size());
		int n=A.size(),m=A[0].size()-1,i,j,k,l,r,fg=1;
		db a[n][m+1],b;
		for (i=0;i<n;i++) for (j=0;j<=m;j++) a[i][j]=A[i][j];
		for (i=l=r=0;i<n&&l<m;i++,l++)
		{
			k=i;
			for (j=i+1;j<n;j++) if (fabs(a[j][l])>fabs(a[k][l])) k=j;
			if (fabs(a[k][l])<eps) {--i;continue;}
			if (i!=k) for (j=l;j<=m;j++) swap(a[i][j],a[k][j]);
			b=1/a[i][l];++r;a[i][l]=1;
			for (j=l+1;j<=m;j++) a[i][j]*=b;
			for (j=0;j<n;j++) if (i!=j)
			{
				b=a[j][l];a[j][l]=0;
				for (k=l+1;k<=m;k++) a[j][k]-=b*a[i][k];
			}
		}
		vector<db> X(m);
		for (j=0;j<l;j++) for (k=0;k<i;k++) if (a[k][j]==1)
		{
			X[j]=-a[k][m];
			break;
		}
		for (j=i;j<n&&~fg;j++)
		{
			b=a[j][m];
			for (k=0;k<m;k++) b+=X[k]*a[j][k];
			if (fabs(b)>eps) fg=-1;
		}
		return {X,r*fg};
	}
}
\end{lstlisting}

\subsection{行列式求值(任意模数)}

$O(n^3)$,$O(n^2)$。

原理:辗转相除。注意这个 $\log p$ 并不在 $n^3$ 上。

\begin{lstlisting}
#include "bits/stdc++.h"
using namespace std;
typedef long long ll;
const int N=502,p=998244353;
int cal(int a[][N],int n)
{
	int i,j,k,r=1,fh=0,l;
	for (i=1;i<=n;i++)
	{
		k=i;
		for (j=i+1;j<=n;j++) if (a[j][i]) {k=j;break;}
		if (a[k][i]==0) return 0;
		if (i!=k) {swap(a[k],a[i]);fh^=1;}
		for (j=i+1;j<=n;j++)
		{
			if (a[j][i]>a[i][i]) swap(a[j],a[i]),fh^=1;
			while (a[j][i])
			{
				l=a[i][i]/a[j][i];
				for (k=i;k<=n;k++) a[i][k]=(a[i][k]+(ll)(p-l)*a[j][k])%p;
				swap(a[j],a[i]);fh^=1;
			}
		}
		r=(ll)r*a[i][i]%p;
	}
	if (fh) return (p-r)%p;
	return r;
}
int main()
{
	ios::sync_with_stdio(0);cin.tie(0);
	int n,i,j;
	static int a[N][N];
	cin>>n;
	for (i=1;i<=n;i++) for (j=1;j<=n;j++) cin>>a[i][j];
	cout<<cal(a,n)<<endl;
}
\end{lstlisting}

\subsection{稀疏矩阵系列}

safe 宏用于验证结果正确性,可不定义。实现了稀疏矩阵的行列式和求解方程组。

\begin{lstlisting}
vector<ui> bm(const vector<ui> &a)
{
	vector<ui> r,lst;
	int n=a.size(),m=0,q=0,i,j,k=-1;
	ui D=0;
	for (i=0;i<n;i++)
	{
		ui cur=0;
		for (j=0;j<m;j++) cur=(cur+(ll)a[i-j-1]*r[j])%p;
		cur=(a[i]+p-cur)%p;
		if (!cur) continue;
		if (k==-1)
		{
			k=i;
			D=cur;
			r.resize(m=i+1);
			continue;
		}
		auto v=r;
		ui x=(ll)cur*ksm(D,p-2)%p;
		if (m<q+i-k) r.resize(m=q+i-k);
		(r[i-k-1]+=x)%=p;
		ui *b=r.data()+i-k;
		x=(p-x)%p;
		for (j=0;j<q;j++) b[j]=(b[j]+(ll)x*lst[j])%p;
		if (v.size()+k<lst.size()+i)
		{
			lst=v;
			q=v.size();
			k=i;
			D=cur;
		}
	}
	return r;
}
#define safe
struct Q
{
	int x,y;
	ui w;
};
mt19937_64 rnd(9980);
vector<ui> minpoly(int n,const vector<Q> &a)//[0,n),max:1
{
	for (auto [x,y,w]:a) assert(min(x,y)>=0&&max(x,y)<n);
	vector<ui> u(n),v(n),b(n*2+1),tmp(n);
	int i;
	for (ui &x:u) x=rnd()%p;
	for (ui &x:v) x=rnd()%p;
	assert(*min_element(all(u))&&*min_element(all(v)));
	for (ui &r:b)
	{
		for (i=0;i<n;i++) r=(r+(ll)u[i]*v[i])%p;
		fill(all(tmp),0);
		for (auto [x,y,w]:a) tmp[x]=(tmp[x]+(ll)w*v[y])%p;
		swap(v,tmp);
	}
	auto r=bm(b);
	#ifdef safe
		for (ui &x:u) x=rnd()%p;
		for (ui &x:v) x=rnd()%p;
		for (ui &r:b)
		{
			for (i=0;i<n;i++) r=(r+(ll)u[i]*v[i])%p;
			fill(all(tmp),0);
			for (auto [x,y,w]:a) tmp[x]=(tmp[x]+(ll)w*v[y])%p;
			swap(v,tmp);
		}
		auto rr=bm(b);
		assert(r==rr);
	#endif
	reverse(all(r));
	for (ui &x:r) if (x) x=p-x;
	r.push_back(1);
	return r;
}
ui det(int n,vector<Q> a)//[0,m)
{
	vector<ui> b(n);
	for (ui &x:b) x=rnd()%p;
	assert(*min_element(all(b)));
	for (auto &[x,y,w]:a) w=(ll)w*b[x]%p;
	ui r=minpoly(n,a)[0],tmp=1;
	for (ui x:b) tmp=(ll)tmp*x%p;
	r=(ll)r*ksm(tmp,p-2)%p;
	#ifdef safe
		for (ui &x:b) x=rnd()%p;
		assert(*min_element(all(b)));
		for (auto &[x,y,w]:a) w=(ll)w*b[x]%p;
		ui rr=minpoly(n,a)[0],tmpp=1;
		for (ui x:b) tmpp=(ll)tmpp*x%p;
		rr=(ll)rr*ksm(tmpp,p-2)%p*ksm(tmp,p-2)%p;
		assert(r==rr);
	#endif
	return n&1?(p-r)%p:r;
}
vector<ui> gauss(const vector<Q> &a,vector<ui> v)
{
	int n=v.size(),i,j;
	for (auto [x,y,w]:a) assert(0<=x&&x<n&&0<=y&&y<n);
	vector<ui> u(n),b(2*n+1),tmp(n),tv=v;
	for (ui &x:u) x=rnd()%p;
	assert(*min_element(all(u)));
	for (ui &r:b)
	{
		for (i=0;i<n;i++) r=(r+(ll)u[i]*v[i])%p;
		fill(all(tmp),0);
		for (auto [x,y,w]:a) tmp[x]=(tmp[x]+(ll)w*v[y])%p;
		swap(v,tmp);
	}
	auto f=bm(b);
	f.insert(f.begin(),p-1);
	int m=(int)f.size()-2;
	v=tv;fill(all(u),0);
	ui x;
	for (i=0;i<=m;i++)
	{
		x=f[m-i];
		for (j=0;j<n;j++) u[j]=(u[j]+(ll)v[j]*x)%p;
		fill(all(tmp),0);
		for (auto [x,y,w]:a) tmp[x]=(tmp[x]+(ll)w*v[y])%p;
		swap(v,tmp);
	}
	x=ksm((p-f.back())%p,p-2);
	for (ui &y:u) y=(ll)y*x%p;
	#ifdef safe
		for (auto [x,y,w]:a) tv[x]=(tv[x]+(ll)(p-w)*u[y])%p;
		assert(!*min_element(all(tv)));
	#endif
	return u;
}
\end{lstlisting}



\subsection{Min\_25筛}

$f(p^k)=p^k(p^k-1)$,求 $\sum\limits_{i=1}^n f(i)$。这个的原理我了解的不多,因此没有更多注释。

\begin{lstlisting}
const int N=1e5+2,p=1e9+7,i6=166666668;
ll fs[N<<1],m;
int ss[N],ys[N<<1],s[N],f[N<<1],g[N<<1],ls[N<<1],cs[N<<1];
int gs,n,i,j,k,cnt,ct,ans,sq;
bool ed[N];
int S(ll n,int x)
{
	int r,i,j,l;
	ll k;
	if (ss[x]>=n) return 0;
	if (n>sq) r=g[ys[m/n]]; else r=g[n];
	if ((r=r-s[x])<0) r+=p;
	for (i=x+1;(ll)ss[i]*ss[i]<=n;i++) for (j=1,k=ss[i];k<=n;j++,k*=ss[i])
	{
		l=(k-1)%p;
		r=(r+(ll)l*(l+1)%p*((j!=1)+S(n/k,i)))%p;
	}
	return r;
}
int main()
{
	n=1e5;
	for (i=2;i<=n;i++)
	{
		if (!ed[i]) ss[++gs]=i;
		for (j=1;(j<=gs)&&(i*ss[j]<=n);j++)
		{
			ed[i*ss[j]]=1;
			if (i%ss[j]==0) break;
		}
	}ss[gs+1]=1e6;
	s[1]=ss[1]*ss[1];
	for (i=2;i<=gs;i++) s[i]=(s[i-1]+(ll)ss[i]*ss[i])%p;//s 是多项式在素数位置的前缀和
	memcpy(cs,s,sizeof(s));
	ll i,j,k,x,z; scanf("%lld",&m);
	sq=n=sqrt(m);while ((ll)(n+1)*(n+1)<=m) ++n;
	cnt=n-1;
	for (i=n;i<=m;i=j+1) {j=m/(m/i);++cnt;}ct=cnt++;
	for (i=1;i<=m;i=j+1)
	{
		j=m/(k=m/i);
		if (k<=n) g[fs[k]=k]=(k*(k+1)*(k<<1|1)/6-1)%p;//这里是多项式前缀和(不含1)
		else
		{
			z=k%p;//一样
			g[ys[j]=--cnt]=(z*(z+1)%p*(z<<1|1)%p+p-6)*i6%p;fs[cnt]=k;
		}
	}
	cnt=ct;
	for (j=1;(j<=gs)&&(z=(ll)ss[j]*ss[j]);j++) for (i=cnt;z<=fs[i];i--)
	{
		x=fs[i]/ss[j];if (x>n) x=ys[m/x];
		g[i]=(g[i]+(ll)(p-ss[j])*ss[j]%p*(g[x]-s[j-1]+p))%p;//另一处需要修改的
	}
	memcpy(ls,g,sizeof(g));
	s[1]=ss[1];
	for (i=2;i<=gs;i++) s[i]=s[i-1]+ss[i];
	cnt=n-1;
	for (i=n;i<=m;i=j+1) {j=m/(m/i);++cnt;}ct=cnt++;
	for (i=1;i<=m;i=j+1)
	{
		j=m/(k=m/i);
		if (k<=n) g[fs[k]=k]=((k*(k+1)>>1)-1)%p;
		else
		{
			z=k%p;
			g[ys[j]=--cnt]=(z*(z+1)-2>>1)%p;fs[cnt]=k;
		}
	}
	cnt=ct;
	for (j=1;(j<=gs)&&(z=(ll)ss[j]*ss[j]);j++) for (i=cnt;z<=fs[i];i--)
	{
		x=fs[i]/ss[j];if (x>n) x=ys[m/x];
		g[i]=(g[i]+(ll)(p-ss[j])*(g[x]-s[j-1]+p))%p;
	}
	for (i=1;i<=cnt;i++) if ((g[i]=ls[i]-g[i])<0) g[i]+=p;
	for (i=1;i<=gs;i++) if ((s[i]=cs[i]-s[i])<0) s[i]+=p;
	ans=S(m,0)+1;if (ans==p) ans=0;printf("%d",ans);
}
\end{lstlisting}

\subsection{Min\_25筛(卡常,素数个数,注意评测机 double 性能)}

\begin{lstlisting}
#include "bits/stdc++.h"
using namespace std;
typedef long long ll;
const int N=3.2e5+2;
ll s[N];
int ss[N],ys[N],gs=0;
bool ed[N];
ll cal(ll m)
{
	static ll g[N<<1],fs[N<<1];
	ll i,j,k,x;
	int n;
	int p,q,cnt;
	n=round(sqrt(m));
	q=lower_bound(ss+1,ss+gs+1,n)-ss;
	memset(g,0,sizeof(g));memset(ys,0,sizeof(ys));cnt=n-1;
	for (i=n;i<=m;i=j+1) {j=m/(m/i);++cnt;}int ct=cnt++;
	for (i=1;i<=m;i=j+1)
	{
		j=m/(k=m/i);
		if (k<=n) g[fs[k]=k]=k-1; else {g[ys[j]=--cnt]=k-1;fs[cnt]=k;}
	}cnt=ct;
	for (j=1;j<=q;j++) for (i=cnt;(ll)ss[j]*ss[j]<=fs[i];i--)
	{
		x=fs[i]/ss[j];if (x>n) x=ys[m/x];
		g[i]-=g[x]-j+1;
	}
	return g[cnt];//这里 g[cnt-i+1] 表示的是 [1,m/i] 的答案
}
int main()
{
	int n,i,j,t;
	n=3.2e5;
	for (i=2;i<=n;i++)
	{
		if (!ed[i]) ss[++gs]=i;
		for (j=1;(j<=gs)&&(i*ss[j]<=n);j++)
		{
			ed[i*ss[j]]=1;
			if (i%ss[j]==0) break;
		}
	}
	s[1]=ss[1];
	for (i=2;i<=gs;i++) s[i]=s[i-1]+ss[i];
	t=1;
	ll m;
	while (t--) cin>>m,cout<<cal(m)<<'\n';
}
\end{lstlisting}

\subsection{扩展 min-max 容斥(重返现世)}

$k\text{-th}\max\{S\}=\sum\limits_{T\subseteq S}(-1)^{|T|-k}\tbinom{|T|-1}{k-1}\min\{T\}$

\begin{lstlisting}
	scanf("%d%d%d",&n,&q,&m);inv[1]=1;q=n+1-q;
	for (i=2;i<=m;i++) inv[i]=p-(ll)p/i*inv[p%i]%p;
	for (i=1;i<=n;i++) scanf("%d",a+i);f[0][0]=1;
	for (j=1;j<=n;j++) for (i=q;i;i--) for (k=m;k>=a[j];k--) if ((f[i][k]=f[i][k]+f[i-1][k-a[j]]-f[i][k-a[j]])>=p) f[i][k]-=p; else if (f[i][k]<0) f[i][k]+=p;
	for (i=1;i<=m;i++) ans=(ans+(ll)f[q][i]*inv[i])%p;
	ans=(ll)ans*m%p;printf("%d",ans);
\end{lstlisting}

\subsection{模数为偶数 FWT/光速乘}

$O(n2^n)$,$O(2^n)$。

原理:让模数变为 $p2^n$,就可以正常做除法了。

\begin{lstlisting}
const int N=1<<20,M=21;
int x[M];
ll p,f[N],g[N];
int n,m,c;
ll mul(ll x,ll y)
{
	x=x*y-(ll)((ldb)x/p*y+1e-8)*p;
	if (x<0) return x+p;return x;
}
void dft(ll *a)
{
	int i,j,k,l;
	ll b;
	for (i=1;i<n;i=l)
	{
		l=i<<1;
		for (j=0;j<n;j+=l) for (k=0;k<i;k++)
		{
			b=a[j|k|i];
			a[j|k|i]=(a[j|k]-b+p)%p;
			a[j|k]=(a[j|k]+b)%p;
		}
	}
}
int main()
{
    ios::sync_with_stdio(0);cin.tie(0);
	ll t;int i;
	cin>>m>>t>>p;p*=(n=1<<m);
	for (i=0;i<n;i++) cin>>f[i];
	dft(f);
	for (i=0;i<=m;i++) cin>>x[i];
	for (i=1;i<n;i++) g[i]=g[i>>1]+(i&1);
	for (i=0;i<n;i++) g[i]=x[g[i]];dft(g);
	while (t)
	{
		if (t&1) for (i=0;i<n;i++) f[i]=mul(f[i],g[i]);
		for (i=0;i<n;i++) g[i]=mul(g[i],g[i]);t>>=1;
	}
	dft(f);
	for (i=0;i<n;i++) cout<<(f[i]>>m)<<'\n';
}
\end{lstlisting}

\subsection{二次剩余}

\begin{lstlisting}
namespace cipolla
{
	typedef unsigned int ui;
	typedef unsigned long long ll;
	ui p,w;
	struct Q
	{
		ll x,y;
		Q operator*(const Q &o) const {return {(x*o.x+y*o.y%p*w)%p,(x*o.y+y*o.x)%p};}
	};
	ui ksm(ll x,ui y)
	{
		ll r=1;
		while (y)
		{
			if (y&1) r=r*x%p;
			x=x*x%p;y>>=1;
		}
		return r;
	}
	Q ksm(Q x,ui y)
	{
		Q r={1,0};
		while (y)
		{
			if (y&1) r=r*x;
			x=x*x;y>>=1;
		}
		return r;
	}
	ui mosqrt(ui x,ui P)//0<=x<P
	{
		if (x==0||P==2) return x;
		p=P;
		if (ksm(x,p-1>>1)!=1) return -1;
		ui y;
		mt19937 rnd(chrono::steady_clock::now().time_since_epoch().count());
		do y=rnd()%p,w=((ll)y*y+p-x)%p; while (ksm(w,p-1>>1)<=1);//not for p=2
		y=ksm({y,1},p+1>>1).x;
		if (y*2>p) y=p-y;//两解取小
		return y;
	}
}
using cipolla::mosqrt;
\end{lstlisting}

\subsection{$k$ 次剩余}
\begin{lstlisting}
namespace get_root
{
	typedef unsigned int ui;
	typedef unsigned long long ll;
	bool ied=0;
	const int N=1e5+5;
	vector<ui> pr;
	bool ed[N];
	void init()
	{
		pr.reserve(N);
		for (ui i=2;i<N;i++)
		{
			if (!ed[i]) pr.push_back(i);
			for (ui x:pr)
			{
				if (i*x>=N) break;
				ed[i*x]=1;
				if (i%x==0) break;
			}
		}
	}
	ui ksm(ui x,ui y,ui p)
	{
		ui r=1;
		while (y)
		{
			if (y&1) r=(ll)r*x%p;
			x=(ll)x*x%p;y>>=1;
		}
		return r;
	}
	vector<ui> getw(ui n)
	{
		vector<ui> w;
		for (ui x:pr)
		{
			if (x*x>n) break;
			if (n%x==0)
			{
				w.push_back(x);
				n/=x;
				for (ui i=n/x;n==x*i;i=n/x) n/=x;
			}
		}
		if (n>1) w.push_back(n);
		return w;
	}
	int getrt(ui n)
	{
		if (n<=2) return n-1;
		if (!ed[4]) init();
		auto w=getw(n);
		ui ph=n;
		for (ui x:w) ph=ph/x*(x-1);
		w=getw(ph);
		for (ui &x:w) x=ph/x;
		for (ui i=2;i<n;i++) if (gcd(i,n)==1)
		{
			for (ui x:w) if (ksm(i,x,n)==1) goto no;
			return i;
			no:;
		}
		return -1;
	}
}
namespace BSGS
{
	typedef unsigned int ui;
	typedef unsigned long long ll;
	template<int N,class T,class TT> struct ht//个数,定义域,值域
	{
		const static int p=1e6+7,M=p+2;
		TT a[N];
		T v[N];
		int fir[p+2],nxt[N],st[p+2];//和模数相适应
		int tp,ds;//自定义模数
		ht(){memset(fir,0,sizeof fir);tp=ds=0;}
		void mdf(T x,TT z)//位置,值
		{
			ui y=x%p;
			for (int i=fir[y];i;i=nxt[i]) if (v[i]==x) return a[i]=z,void();//若不可能重复不需要 for
			v[++ds]=x;a[ds]=z;
			if (!fir[y]) st[++tp]=y;
			nxt[ds]=fir[y];fir[y]=ds;
		}
		TT find(T x)
		{
			ui y=x%p;
			int i;
			for (i=fir[y];i;i=nxt[i]) if (v[i]==x) return a[i];
			return 0;//返回值和是否判断依据要求决定
		}
		void clear()
		{
			++tp;
			while (--tp) fir[st[tp]]=0;
			ds=0;
		}
	};
	const int N=5e4;
	ht<N,ui,ui> s;
	int exgcd(int a,int b)
	{
		if (a==1) return 1;
		return (1-(long long)b*exgcd(b%a,a))/a;//not ll
	}
	int bsgs(ui a,ui b,ui p)
	{
		s.clear();
		a%=p;b%=p;
		if (!a) return 1-min((int)b,2);//含 -1
		ui i,j,k,x,y;
		x=sqrt(p)+2;
		for (i=0,j=1;i<x;i++,j=(ll)j*a%p)
		{
			if (j==b) return i;
			s.mdf((ll)j*b%p,i+1);
		}
		k=j;
		for (i=1;i<=x;i++,j=(ll)j*k%p) if (y=s.find(j)) return (ll)i*x-y+1;
		return -1;
	}
	bool isprime(ui p)
	{
		if (p<=1) return 0;
		for (ui i=2;i*i<=p;i++) if (p%i==0) return 0;
		return 1;
	}
	int exbsgs(ui a,ui b,ui p)//a^x=b(mod p)
	{
		//if (isprime(p)) return bsgs(a,b,p);
		a%=p;b%=p;
		ui i,j,k,x,y=__lg(p),cnt=0;
		for (i=0,j=1%p;i<=y;i++,j=(ll)j*a%p) if (j==b) return i;
		y=1;
		while (1)
		{
			if ((x=gcd(a,p))==1) break;
			if (b%x) return -1;//no sol
			++cnt;
			p/=x;b/=x;
			y=(ll)y*(a/x)%p;
		}
		a%=p;
		b=(ll)b*(p+exgcd(y,p))%p;
		int r=bsgs(a,b,p);
		return r==-1?-1:r+cnt;
	}
}
pair<ll,ll> exgcd(ll a,ll b,ll c)//ax+by=c,{-1,-1} 无解,b=0 返回 {c/a,0},否则返回最小非负 x
{
	assert(a||b);
	if (!b) return {c/a,0};
	if (a<0) a=-a,b=-b,c=-c;
	ll d=gcd(a,b);
	if (c%d) return {-1,-1};
	ll x=1,x1=0,p=a,q=b,k;
	b=abs(b);
	while (b)
	{
		k=a/b;
		x-=k*x1;a-=k*b;
		swap(x,x1);
		swap(a,b);
	}
	b=abs(q/d);
	x=x*(c/d)%b;
	if (x<0) x+=b;
	return {x,(c-p*x)/q};
}
ll fun(ll a,ll b,ll p)//ax=b(mod p)
{
	return exgcd(-p,a,b).second%p;
}
using get_root::getrt;
using BSGS::bsgs,BSGS::exbsgs;
int nth_root(ui k,ui y,ui p)//x^k=y(mod p)
{
	if (k==0) return y==1?0:-1;
	if (y==0) return 0;
	ui g=getrt(p);
	ui z=bsgs(g,y,p);
	ll x=fun(k,z,p-1);
	if (x==-1) return -1;
	return get_root::ksm(g,x,p);
}
\end{lstlisting}

网上的超快版本

\begin{lstlisting}
#define popcount __builtin_popcount
using namespace std;
typedef long long int ll;
//using ll=__int128_t;
typedef pair<ll, int> P;
ll gcd(ll a, ll b){
	if (b==0) return a;
	return gcd(b, a%b);
}
ll powmod(ll a, ll k, ll mod){
	ll ap=a, ans=1;
	while(k){
		if (k&1){
			ans*=ap;
			ans%=mod;
		}
		ap=ap*ap;
		ap%=mod;
		k>>=1;
	}
	return ans;
}
ll inv(ll a, ll m){
	ll b=m, x=1, y=0;
	while(b>0){
		ll t=a/b;
		swap(a-=t*b, b);
		swap(x-=t*y, y);
	}
	return (x%m+m)%m;
}
vector<P> fac(ll x){
	vector<P> ret;
	for(ll i=2; i*i<=x; i++){
		if (x%i==0){
			int e=0;
			while(x%i==0){
				x/=i;
				e++;
			}
			ret.push_back({i, e});
		}
	}
	if (x>1) ret.push_back({x, 1});
	return ret;
}
//mt19937_64 mt(334);
mt19937 mt(334);
ll solve1(ll p, ll q, int e, ll a){
	int s=0;
	ll r=p-1, qs=1, qp=1;
	while(r%q==0){
		r/=q;
		qs*=q;
		s++;
	}
	for(int i=0; i<e; i++) qp*=q;
	ll d=qp-inv(r%qp, qp);
	ll t=(d*r+1)/qp;
	ll at=powmod(a, t, p), inva=inv(a, p);
	if (e>=s){
		if (powmod(at, qp, p)!=a) return -1;
		else return at;
	}
	//uniform_int_distribution<long long> rnd(1, p-1);
	uniform_int_distribution<> rnd(1, p-1);
	ll rv;
	while(1){
		rv=powmod(rnd(mt), r, p);
		if (powmod(rv, qs/q, p)!=1) break;
	}
	int i=0;
	ll qi=1, sq=1;
	while(sq*sq<q) sq++;
	while(i<s-e){
		ll qq=qs/qp/qi/q;
		vector<P> v(sq);
		ll rvi=powmod(rv, qp*qq*(p-2)%(p-1), p), rvp=powmod(rv, sq*qp*qq, p);
		ll x=powmod(powmod(at, qp, p)*inva%p, qq*(p-2)%(p-1), p), y=1;
		for(int j=0; j<sq; j++){
			v[j]=P(x, j);
			(x*=rvi)%=p;
		}
		sort(v.begin(), v.end());
		ll z=-1;
		for(int j=0; j<sq; j++){
			int l=lower_bound(v.begin(), v.end(), P(y, 0))-v.begin();
			if (v[l].first==y){
				z=v[l].second+j*sq;
				break;
			}
			(y*=rvp)%=p;
		}
		if (z==-1) return -1;
		(at*=powmod(rv, z, p))%=p;
		i++;
		qi*=q;
		rv=powmod(rv, q, p);
	}
	return at;
}
ll solve0(ll p, ll q, ll r, ll a){
	ll d=q-inv(r%q, q);
	ll t=(d*r+1)/q;
	ll at=powmod(a, t, p), inva=inv(a, p);
	if (powmod(at, q, p)!=a) return -1;
	else return at;
}
ll solve(ll p, ll k, ll a)//p k y
{
	if (k==0)
	{
		if (a==1) return 1;
		return -1;
	}
	if (a==0) return 0;
	if (p==2 || a==1) return 1;
	ll a1=a;
	ll g=gcd(p-1, k);
	ll c=inv(k/g%((p-1)/g), (p-1)/g);
	a=powmod(a, c, p);
	if (g==1){
		if (powmod(a, k, p)==a1) return a;
		else return -1;
	}
	ll g1=gcd(g, (p-1)/g), g2=g;
	vector<P> f1=fac(g1), f;
	for(auto r:f1){
		ll q=r.first;
		int e=0;
		while(g2%q==0){
			g2/=q;
			e++;
		}
		f.push_back({q, e});
	}
	ll ret=1, gp=1;
	if (g2>1){
		ll x=solve0(p, g2, (p-1)/g2, a);
		if (x==-1) return -1;
		ret=x, gp*=g2;
	}
	for(auto r:f){
		ll qp=1;
		for(int i=0; i<r.second; i++) qp*=r.first;
		ll x=solve1(p, r.first, r.second, a);
		if (x==-1) return -1;
		if (gp==1){
			ret=x, gp*=qp;
			continue;
		}
		ll s=inv(gp%qp, qp), t=(1-gp*s)/qp;
		if (t>=0) ret=powmod(ret, t, p);
		else ret=powmod(ret, p-1+t%(p-1), p);
		if (s>=0) x=powmod(x, s, p);
		else x=powmod(x, p-1+s%(p-1), p);
		(ret*=x)%=p;
		gp*=qp;
	}
	if (powmod(ret, k, p)!=a1) return -1;
	return ret;
}
\end{lstlisting}

\subsection{FWT/子集卷积}

$O(n2^n)$,$O(2^n)$。注意全都是无符号的。

这里混合了两个版本的代码,但只有 ui 和 ull 的差异。容易自行调整。

\begin{lstlisting}
void fwt_and(vector<ll> &A)//本质:母集和
{
	ll n=A.size(), *a=A.data(), i, j, k, l, *f, *g;
	for (i=1; i<n; i=l)
	{
		l=i*2;
		for (j=0; j<n; j+=l)
		{
			f=a+j; g=a+j+i;
			for (k=0; k<i; k++) f[k]+=g[k];
		}
		if (l==n||i==1<<10) for (ll &x:A) x%=p;
	}
}
void ifwt_and(vector<ll> &A)
{
	ll n=A.size(), *a=A.data(), i, j, k, l, *f, *g;
	for (i=1; i<n; i=l)
	{
		l=i*2;
		for (j=0; j<n; j+=l)
		{
			f=a+j; g=a+j+i;
			for (k=0; k<i; k++) f[k]+=p*i-g[k];
		}
		if (l==n||i==1<<10) for (ll &x:A) x%=p;
	}
}
void fwt_or(vector<ll> &A)//本质:子集和
{
	ll n=A.size(), *a=A.data(), i, j, k, l, *f, *g;
	for (i=1; i<n; i=l)
	{
		l=i*2;
		for (j=0; j<n; j+=l)
		{
			f=a+j; g=a+j+i;
			for (k=0; k<i; k++) g[k]+=f[k];
		}
		if (l==n||i==1<<10) for (ll &x:A) x%=p;
	}
}
void ifwt_or(vector<ll> &A)
{
	ll n=A.size(), *a=A.data(), i, j, k, l, *f, *g;
	for (i=1; i<n; i=l)
	{
		l=i*2;
		for (j=0; j<n; j+=l)
		{
			f=a+j; g=a+j+i;
			for (k=0; k<i; k++) g[k]+=p*i-f[k];
		}
		if (l==n||i==1<<10) for (ll &x:A) x%=p;
	}
}
void fwt_xor(vector<ui> &A)
{
	ui n=A.size(),*a=A.data(),i,j,k,l,*f,*g;
	for (i=1;i<n;i=l)
	{
		l=i*2;
		for (j=0;j<n;j+=l)
		{
			f=a+j;g=a+j+i;
			for (k=0;k<i;k++)
			{
				if ((f[k]+=g[k])>=p) f[k]-=p;
				g[k]=(f[k]+2*(p-g[k]))%p;
			}
		}
	}
}
void ifwt_xor(vector<ui> &A)
{
	ui n=A.size(),*a=A.data(),i,j,k,l,*f,*g,x=p+1>>1,y=1;
	for (i=1;i<n;i=l)
	{
		l=i*2;
		for (j=0;j<n;j+=l)
		{
			f=a+j;g=a+j+i;
			for (k=0;k<i;k++)
			{
				if ((f[k]+=g[k])>=p) f[k]-=p;
				g[k]=(f[k]+2*(p-g[k]))%p;
			}
		}
		y=(ll)y*x%p;
	}
	for (i=0;i<n;i++) a[i]=(ll)a[i]*y%p;
}
vector<ui> fst(const vector<ui> &s,const vector<ui> &t)
{
	int n=s.size(),m=__builtin_ctz(n),i,j,k;
	vector<ui> a[m+1],b[m+1],c[m+1],r(n);
	for (i=0;i<=m;i++) a[i].resize(n),b[i].resize(n),c[i].resize(n);
	for (i=0;i<n;i++)
	{
		k=__builtin_popcount(i);
		a[k][i]=s[i];
		b[k][i]=t[i];
	}
	for (i=0;i<m;i++) fwt_or(a[i]),fwt_or(b[i]);
	for (i=0;i<=m;i++) for (j=0;j<=i;j++) for (k=0;k<n;k++) c[i][k]=(c[i][k]+(ll)a[j][k]*b[i-j][k])%p;
	for (i=1;i<=m;i++) ifwt_or(c[i]);
	for (i=0;i<n;i++) r[i]=c[__builtin_popcount(i)][i];
	return r;
}
\end{lstlisting}

\subsection{NTT}

一种较快的 NTT(尤其是对于卷积以外的用途),但不推荐在不熟悉的情况下直接使用。一般的卷积可以参照字符串部分通配符的字符串匹配,其余的用途可以参照其他板子。

如果确实需要卡常,建议先抄写需要的函数,并递归地找到需要补的内容。

注意事项:所有 \verb|ll| 为无符号。始终保证数组大小为 $2^n$,不应当使用 \verb|resize| 而应该使用取模来调整长度。三种卷积对应的运算符见注释。

需要特别小心其长度的变化,注意不要越界。如果修改模数,\verb|dft| 和 \verb|hf_dft| 处有一个参数也要修改。

常见函数如下(带 new 的基本上都是较快但较长的):

卷积 \verb|operator*|,循环卷积 \verb|operator&|,差卷积 \verb|operator^|,求逆 \verb|operator~/|
(包含一个较短版,被注释了),分治 \verb|cdq|,对数 \verb|ln|,指数 \verb|exp,exp_cdq,exp_new|,
开方 \verb|sqrt,sqrt_new|,幂函数 \verb|pow(Q,ll),pow(Q,string),pow2(Q,ll),pow(Q,ll,Q)|,
整除与取模 \verb|div,mod,div_mod|,线性递推 \verb|recurrent,recurrent_new,recurrent_interval|,
连乘 \verb|prod,prod_new|,\newline
多点求值 \verb|evaluation,evaluation_new|,阶乘 \verb|factorial|,
快速插值 \verb|interpolation|,复合(逆)\verb|comp,comp_inv|,多项式平移 \verb|shift|,
区间点值平移 \verb|shift|,Z 变换 \verb|Z_transform|,贝尔数($[n]$ 划分等价类方案数)\verb|Bell|,
斯特林数 \verb|S1_row,S1_column,S2_row,S2_column,signed_S1_row|,伯努利数 \verb|Bernoulli|,
划分数 \verb|Partition|,最大公因式 \verb|gcd|,求根 \verb|root|,模多项式意义的逆 \verb|inverse|。

\begin{lstlisting}
#include <optional>
namespace NTT
{
	using ll = unsigned long long;
	const ll g = 3, p = 998244353;
	const int N = 1 << 22;//务必修改
	ll inv[N], fac[N], ifac[N];//非必要
	void getfac(int n)//非必要
	{
		static int pre = -1;
		if (pre == -1) pre = 1, ifac[0] = fac[0] = fac[1] = ifac[1] = inv[1] = 1;
		if (n <= pre) return;
		for (int i = pre + 1, j; i <= n; i++)
		{
			j = p / i;
			inv[i] = (p - j) * inv[p - i * j] % p;
			fac[i] = fac[i - 1] * i % p;
			ifac[i] = ifac[i - 1] * inv[i] % p;
		}
		pre = n;
	}
	ll w[N];
	int r[N];
	ll ksm(ll x, ll y)
	{
		ll r = 1;
		while (y)
		{
			if (y & 1) r = r * x % p;
			x = x * x % p;
			y >>= 1;
		}
		return r;
	}
	void init(int n)
	{
		static int pr = 0, pw = 0;
		if (pr == n) return;
		int b = __lg(n) - 1, i, j, k;
		for (i = 1; i < n; i++) r[i] = r[i >> 1] >> 1 | (i & 1) << b;
		if (pw < n)
		{
			for (j = 1; j < n; j = k)
			{
				k = j * 2;
				ll wn = ksm(g, (p - 1) / k);
				w[j] = 1;
				for (i = j + 1; i < k; i++) w[i] = w[i - 1] * wn % p;
			}
			pw = n;
		}
		pr = n;
	}
	int cal(int x) { return 1 << __lg(max(x, 1) * 2 - 1); }
	struct Q :vector<ll>
	{
		bool flag;
		Q& operator%=(int n) { assert((n & -n) == n); resize(n); return *this; }
		Q operator%(int n) const
		{
			assert((n & -n) == n);
			if (size() <= n)
			{
				auto f = *this;
				return f %= n;
			}
			return Q(vector(begin(), begin() + n));
		}
		int deg() const
		{
			int n = size() - 1;
			while (n >= 0 && begin()[n] == 0) --n;
			return n;
		}
		explicit Q(int x = 1, bool f = 0) :flag(f), vector<ll>(cal(x)) { }//小心:{}会调用这条而非下一条
		Q(const vector<ll>& o, bool f = 0) :Q(o.size(), f) { copy(all(o), begin()); }
		Q(const initializer_list<ll>& o, bool f = 0) :Q(vector(o), f) { }
		ll fx(ll x)
		{
			ll r = 0;
			for (auto it = rbegin(); it != rend(); ++it) r = (r * x + *it) % p;
			return r;
		}
		void dft()
		{
			int n = size(), i, j, k;
			ll y, * f, * g, * wn, * a = data();
			init(n);
			for (i = 1; i < n; i++) if (i < r[i]) ::swap(a[i], a[r[i]]);
			for (k = 1; k < n; k *= 2)
			{
				wn = w + k;
				for (i = 0; i < n; i += k * 2)
				{
					g = (f = a + i) + k;
					for (j = 0; j < k; j++)
					{
						y = g[j] * wn[j] % p;
						g[j] = f[j] + p - y;
						f[j] += y;
					}
				}//此处要求 14*p*p<=2^64。如果调整模数,需要修改 12。
				if (__lg(n / k) % 12 == 1) for (i = 0; i < n; i++) a[i] %= p;
			}
			if (flag)
			{
				y = ksm(n, p - 2);
				for (i = 0; i < n; i++) a[i] = a[i] * y % p;
				reverse(a + 1, a + n);
			}
			flag ^= 1;
		}
		void hf_dft()
		{
			assert(size() >= 2 && flag);
			int n = size() / 2, i, j, k;
			ll x, y, * f, * g, * wn, * a = data();
			init(n);
			for (i = 1; i < n; i++) if (i < r[i]) ::swap(a[i], a[r[i]]);
			for (k = 1; k < n; k *= 2)
			{
				wn = w + k;
				for (i = 0; i < n; i += k * 2)
				{
					g = (f = a + i) + k;
					for (j = 0; j < k; j++)
					{
						y = g[j] * wn[j] % p;
						g[j] = f[j] + p - y;
						f[j] += y;
					}
				}
				if (__lg(n / k) % 12 == 1) for (i = 0; i < n; i++) a[i] %= p;
			}
			if (flag)
			{
				x = ksm(n, p - 2);
				for (i = 0; i < n; i++) a[i] = a[i] * x % p;
				reverse(a + 1, a + n);
			}
			flag ^= 1;
		}
		Q operator<<(int m) const
		{
			int n = deg(), i;
			Q r(n + m + 1);
			for (i = 0; i <= n; i++) r[i + m] = at(i);
			return r;
		}
		Q operator>>(int m) const
		{
			int n = deg(), i;
			if (n < m) return Q();
			Q r(n + 1 - m);
			for (i = m; i <= n; i++) r[i - m] = at(i);
			return r;
		}
	};
	Q shrink(Q f) { return f %= cal(f.deg() + 1); }
	ostream& operator<<(ostream& cout, const Q& o)
	{
		int n = o.deg();
		if (n < 0) return cout << "[0]";
		cout << "[" << o[n];
		for (int i = n - 1; i >= 0; i--) cout << ", " << o[i];
		return cout << "]";
	}
	Q der(const Q& f)
	{
		ll n = f.size(), i;
		Q r(n);
		for (i = 1; i < n; i++) r[i - 1] = f[i] * i % p;
		return r;
	}
	Q integral(const Q& f)
	{
		ll n = f.size(), i;
		getfac(n);
		Q r(n);
		for (i = 1; i < n; i++) r[i] = f[i - 1] * inv[i] % p;
		return r;
	}
	Q& operator+=(Q& f, ll x) { (f[0] += x) %= p; return f; }
	Q operator+(Q f, ll x) { return f += x; }
	Q& operator-=(Q& f, ll x) { (f[0] += p - x) %= p; return f; }
	Q operator-(Q f, ll x) { return f -= x; }
	Q& operator*=(Q& f, ll x) { for (ll& y : f) (y *= x) %= p; return f; }
	Q operator*(Q f, ll x) { return f *= x; }
	Q& operator+=(Q& f, const Q& g)
	{
		f %= max(f.size(), g.size());
		for (int i = 0; i < g.size(); i++) f[i] = (f[i] + g[i]) % p;
		return f;
	}
	Q operator+(Q f, const Q& g) { return f += g; }
	Q& operator-=(Q& f, const Q& g)
	{
		f %= max(f.size(), g.size());
		for (int i = 0; i < g.size(); i++) f[i] = (f[i] + p - g[i]) % p;
		return f;
	}
	Q operator-(Q f, const Q& g) { return f -= g; }
	Q& operator*=(Q& f, Q g)//卷积
	{
		if (f.flag | g.flag)
		{
			int n = f.size(), i;
			assert(n == g.size());
			if (!f.flag) f.dft();
			if (!g.flag) g.dft();
			for (i = 0; i < n; i++) (f[i] *= g[i]) %= p;
			f.dft();
		}
		else
		{
			int n = cal(f.size() + g.size() - 1), i, j;
			int m1 = f.deg(), m2 = g.deg();
			if ((ll)m1 * m2 > (ll)n * __lg(n) * 8)
			{
				(f %= n).dft(); (g %= n).dft();
				for (i = 0; i < n; i++) (f[i] *= g[i]) %= p;
				f.dft();
			}
			else
			{
				vector<ll> r(max(0, m1 + m2 + 1));
				for (i = 0; i <= m1; i++) for (j = 0; j <= m2; j++) (r[i + j] += f[i] * g[j]) %= p;
				f = Q(n);
				copy(all(r), f.begin());
			}
		}
		return f;
	}
	Q operator*(Q f, const Q& g) { return f *= g; }
	Q& operator&=(Q& f, Q g)//循环卷积
	{
		assert(f.size() == g.size());
		int n = f.size(), i;
		if (!f.flag) f.dft();
		if (!g.flag) g.dft();
		for (i = 0; i < n; i++) (f[i] *= g[i]) %= p;
		f.dft();
		return f;
	}
	Q operator&(Q f, const Q& g) { return f &= g; }
	Q& operator^=(Q& f, Q g)//差卷积
	{
		int n = f.size();
		g %= n;
		reverse(all(g));
		f *= g;
		rotate(f.begin(), n - 1 + all(f));
		return f %= n;
	}
	Q operator^(Q f, const Q& g) { return f ^= g; }
	Q sqr(Q f)
	{
		assert(!f.flag);
		int n = f.size() * 2, i;
		(f %= n).dft();
		for (i = 0; i < n; i++) f[i] = f[i] * f[i] % p;
		f.dft();
		return f;
	}
	/*Q operator~(const Q &f)
	{
		Q r;
		r[0]=ksm(f[0],p-2);
		for (int i=1; i<=f.size(); i*=2) r=(-((f%i)*r-2)*r)%i;
		return r;
	}//trivial, 5e5 750ms*/
	Q operator~(const Q& f)
	{
		Q q, r, g;
		int n = f.size(), i, j, k;
		r[0] = ksm(f[0], p - 2);
		for (j = 2; j <= n; j *= 2)
		{
			k = j / 2;
			g = (r %= j) % k;
			r.dft();
			q = f % j * r;
			fill_n(q.begin(), k, 0);
			r *= q;
			copy(all(g), r.begin());
			for (i = k; i < j; i++) r[i] = (p - r[i]) % p;
		}
		return r;
	}//5e5 200ms, inv(1 6 3 4 9)=(1 998244347 33 998244169 1020)
	Q& operator/=(Q& f, const Q& g) { int n = f.size(); return (f *= ~g) %= n; }
	Q operator/(Q f, const Q& g) { return f /= g; }
	void cdq(Q& f, Q& g, int l, int r)//g_0=1,i*g_i=g_{i-j}*f_j,use for cdq
	{
		static vector<Q> cd;
		int i, m = l + r >> 1, n = r - l, nn = n >> 1;
		if (r - l == f.size())
		{
			getfac(n - 1);
			g = Q(n);
			cd.clear();
			for (i = 2; i <= n; i *= 2)
			{
				cd.emplace_back(i);
				Q& h = cd.back();
				h %= i;
				copy_n(f.begin(), i, h.begin());
				h.dft();
			}
		}
		if (l + 1 == r)
		{
			g[l] = l ? g[l] * inv[l] % p : 1;
			return;
		}
		cdq(f, g, l, m);
		Q h(n);
		copy_n(g.begin() + l, nn, h.begin());
		h *= cd[__lg(n) - 1];
		for (i = m; i < r; i++) (g[i] += h[i - l]) %= p;
		cdq(f, g, m, r);
	}
	Q exp_cdq(Q f)
	{
		Q g;
		int n = f.size(), i;
		for (i = 1; i < n; i++) f[i] = f[i] * i % p;
		cdq(f, g, 0, n);
		return g;
	}//5e5 455ms
	Q ln(const Q& f) { return integral(der(f) / f); }
	//5e5 330ms, ln(1 2 3 4 5)=(0 2 1 665496236 499122177)
	Q exp(Q f)
	{
		Q r; r[0] = 1;
		for (int i = 1; i <= f.size(); i *= 2) (r *= f % i - ln(r % i) + 1) %= i;
		return r;
	}//5e5 700ms, exp(0 4 2 3 5)=(1 4 10 665496257 665496281)
	Q exp_new(Q b)
	{
		Q h, f, r, u, v, bj;
		int n = b.size(), i, j, k;
		r[0] = h[0] = 1;
		for (j = 2; j <= n; j *= 2)
		{
			f = bj = der(b % j); k = j / 2; fill(k + all(bj), 0);
			h.dft(); u = der(r) & h;
			v = (r & h) % j - 1 & bj;
			for (i = 0; i < k; i++) f[i + k] = (p * p + u[i] - v[i] - f[i] - f[i + k]) % p, f[i] = 0;
			f[k - 1] = (f[j - 1] + v[k - 1]) % p;
			u = (r %= j) & integral(f);
			for (i = k; i < j; i++) r[i] = (p - u[i]) % p;
			if (j < n) h = ~r;
		}
		return r;
	}//5e5 420ms
	optional<ll> mosqrt(ll x)
	{
		static mt19937 rnd(chrono::steady_clock::now().time_since_epoch().count());
		static ll W;
		struct P
		{
			ll x, y;
			P operator*(const P& a) const
			{
				return {(x * a.x + y * a.y % p * W) % p, (x * a.y + y * a.x) % p};
			}
		};
		if (x == 0) return {0};
		if (ksm(x, p - 1 >> 1) != 1) return { };
		ll y;
		do y = rnd() % p; while (ksm(W = (y * y % p + p - x) % p, p - 1 >> 1) <= 1);//not for p=2
		y = [&](P x, ll y)
			{
				P r{1, 0};
				while (y)
				{
					if (y & 1) r = r * x;
					x = x * x; y >>= 1;
				}
				return r.x;
			}({y, 1}, p + 1 >> 1);
		return {y * 2 < p ? y : p - y};
	}
	optional<Q> sqrt(Q f)
	{
		const static ll i2 = p + 1 >> 1;
		Q r;
		int n = f.size(), i, l;

		for (i = 0; i < n; i++) if (f[i]) break;
		if (i == n) return f;
		if (i & 1) return { };
		l = i / 2;
		copy(i + all(f), f.begin());
		fill(n - i + all(f), 0);

		auto rt = mosqrt(f[0]);
		if (rt) r[0] = rt.value(); else return { };
		for (i = 2; i <= n; i *= 2) r = (sqr(r) + f % i) / (r % i) % i * i2;

		copy_backward(all(r) - l, r.end());
		fill_n(r.begin(), l, 0);

		return {r};
	}//5e5 530ms, sqrt(0 0 4 2 3)=(0 2 499122177 311951361 171573248)
	optional<Q> sqrt_new(Q f)
	{
		const static ll i2 = p + 1 >> 1;
		Q q, r;
		int n = f.size(), i, j, k, l;

		for (i = 0; i < n; i++) if (f[i]) break;
		if (i == n) return f;
		if (i & 1) return { };
		l = i / 2;
		copy(i + all(f), f.begin());
		fill(n - i + all(f), 0);

		auto rt = mosqrt(f[0]);
		if (rt) r[0] = rt.value(); else return { };
		for (j = 2; j <= n; j *= 2)
		{
			k = j / 2; (q = r).dft(); (q &= q) %= j;
			for (i = k; i < j; i++) q[i] = (q[i - k] + p * 2 - f[i] - f[i - k]) * i2 % p, q[i - k] = 0;
			q &= ~r % j; r %= j;
			for (i = k; i < j; i++) r[i] = (p - q[i]) % p;
		}

		copy_backward(all(r) - l, r.end());
		fill_n(r.begin(), l, 0);

		return {r};
	}//5e5 280ms
	Q pow(Q b, ll m)//不应传入超过 int 内容
	{
		assert(m <= 1llu << 32);
		int n = b.size(), i, j = n, k;
		for (i = 0; i < n; i++) if (b[i]) { j = i; break; }
		if (j == n) return b[0] = !m, b;
		if (j * m >= n) return Q(n);
		copy(j + all(b), b.begin());
		fill(n - j + all(b), 0);
		k = b[0]; j *= m;
		b = exp_new(ln(b * ksm(k, p - 2)) * m) * ksm(k, m);
		copy_backward(all(b) - j, b.end());
		fill_n(b.begin(), j, 0);
		return b;
	}
	Q pow(Q b, string s)
	{
		int n = b.size(), i, j = n, k;
		for (i = 0; i < n; i++) if (b[i]) { j = i; break; }
		if (j == n) return b[0] = s == "0", b;
		if (j && (s.size() > 8 || j * stoll(s) >= n)) return Q(n);
		ll m0 = 0, m1 = 0;
		for (auto c : s) m0 = (m0 * 10 + c - '0') % p, m1 = (m1 * 10 + c - '0') % (p - 1);
		copy(j + all(b), b.begin());
		fill(n - j + all(b), 0);
		k = b[0]; j *= m0;
		b = exp_new(ln(b * ksm(k, p - 2)) * m0) * ksm(k, m1);
		copy_backward(all(b) - j, b.end());
		fill_n(b.begin(), j, 0);
		return b;
	}//5e5 1e18 700ms
	Q pow2(Q b, ll m)
	{
		int n = b.size();
		Q r(n); r[0] = 1;
		while (m)
		{
			if (m & 1) (r *= b) %= n;
			if (m >>= 1) b = sqr(b) % n;
		}
		return r;
	}//5e5 1e18 7425ms
	Q div(Q f, Q g)
	{
		int n = 0, m = 0, i;
		for (i = f.size() - 1; i >= 0; i--) if (f[i]) { n = i + 1; break; }
		for (i = g.size() - 1; i >= 0; i--) if (g[i]) { m = i + 1; break; }
		assert(m);
		if (n < m) return Q(1);
		reverse(f.begin(), f.begin() + n);
		reverse(g.begin(), g.begin() + m);
		n = n - m + 1; m = cal(n);
		f = (f % m) / (g % m) % m;
		fill(n + all(f), 0);
		reverse(f.begin(), f.begin() + n);
		return f;
	}
	Q mod(const Q& a, const Q& b)
	{
		if (a.deg() < b.deg()) return shrink(a);
		Q r = (a - b * div(a, b));
		return shrink(r %= min(r.size(), b.size()));
	}
	Q pow(Q x, ll y, Q f)
	{
		Q r(1);
		r[0] = 1;
		while (y)
		{
			if (y & 1) r = mod(r * x, f);
			if (y >>= 1) x = mod(sqr(x), f);
		}
		return r;
	}
	pair<Q, Q> div_mod(const Q& a, const Q& b) { Q q = div(a, b); Q r = (a - b * q); return {q, r %= min(r.size(), b.size())}; }
	//5e5 430ms (1 2 3 4)=(916755018 427819009)*(5 6 7)+(407446676 346329673)
	// Q cdq_inv(const Q &f) { return (~(f-1))*(p-1); }//g_0=1,g_i=g_{i-j}*f_j ?
	ll recurrent(const vector<ll>& f, const vector<ll>& a, ll m)//常系数齐次线性递推,find a_m,a_n=a_{n-i}*f_i,f_1...k,a_0...k-1
	{
		if (m < a.size()) return a[m];
		assert(f.size() == a.size() + 1 && f[0] == 0);
		int k = a.size(), n = cal(k + 1) * 2, i;
		ll ans = 0;
		Q h(n), g(2);
		for (i = 1; i <= k; i++) h[k - i] = (p - f[i]) % p;
		h[k] = g[1] = 1;
		Q r = pow(g, m, h);
		k = min(k, (int)r.size());
		for (i = 0; i < k; i++) ans = (ans + a[i] * r[i]) % p;
		return ans;
	}//1e5 1e18 8500ms
	ll recurrent_new(const vector<ll>& f, const vector<ll>& a, ll m)//常系数齐次线性递推,find a_m,a_n=a_{n-i}*f_i,f_1...k,a_0...k-1
	{
		const static ll i2 = p + 1 >> 1;
		if (m < a.size()) return a[m];
		assert(f.size() == a.size() + 1 && f[0] == 0);
		int k = a.size(), n = cal(k + 1), i;
		Q g(n * 2), h(n * 2);
		for (h[0] = i = 1; i <= k; i++) h[i] = (p - f[i]) % p;
		copy(all(a), g.begin());
		g &= h; fill(k++ + all(g), 0);
		vector<ll> res(n);
		while (m)
		{
			if (m & 1)
			{
				ll x = p - g[0];
				for (i = 1; i < k; i += 2) res[i >> 1] = x * h[i] % p;
				copy_n(g.begin() + 1, k - 1, g.begin());
				g[k - 1] = 0;
			}
			g.dft(); h.dft();
			ll* a = g.data(), * b = h.data(), * c = a + n, * d = b + n;
			for (i = 0; i < n; i++) g[i] = (a[i] * d[i] + b[i] * c[i]) % p * i2 % p;
			for (i = 0; i < n; i++) h[i] = h[i] * h[i ^ n] % p;
			g.hf_dft(); h.hf_dft();
			fill(k + all(g), 0);
			if (m & 1) for (i = 0; i < k; i++) (g[i] += res[i]) %= p;
			fill(k + all(h), 0);
			m >>= 1;
		}
		assert(h[0] == 1);
		return g[0];
	}//1e5 1e18 1000ms
	vector<ll> recurrent_interval(const vector<ll>& f, const vector<ll>& a, ll L, ll R)//常系数齐次线性递推,find a_[L,R),a_n=a_{n-i}*f_i,f_1...k,a_0...k-1
	{
		assert(f.size() == a.size() + 1 && f[0] == 0);
		int k = a.size(), n = cal(k + 1) * 2, i, len = R - L;
		ll ans = 0, m = L;
		Q h(n), g(2), r;
		for (i = 1; i <= k; i++) h[k - i] = (p - f[i]) % p;
		h[k] = g[1] = r[0] = 1;
		while (m)
		{
			if (m & 1) r = mod(r * g, h);
			if (m >>= 1) g = mod(sqr(g), h);
		}
		Q F(f), A(a);
		F[0] = p - 1;
		A *= F;
		A %= cal(k);
		fill(k + all(A), 0);
		n = cal(len + k);
		F %= n;
		A *= ~F;
		r %= cal(k);
		reverse(r.begin(), r.begin() + k);
		r *= A;
		r.erase(r.begin(), r.begin() + k - 1);
		r.resize(len);
		return r;
	}//1e5 1e18 5e5 10000ms
	Q prod(const vector<Q>& a)
	{
		if (!a.size()) return {1};
		function<Q(int, int)> dfs = [&](int l, int r)
			{
				if (r - l == 1) return a[l];
				int m = l + r >> 1;
				return shrink(dfs(l, m) * dfs(m, r));
			};
		return dfs(0, a.size());
	}//not check
	Q prod_new(const vector<Q>& a)
	{
		if (!a.size()) return {1};
		struct cmp
		{
			bool operator()(const Q& f, const Q& g) const { return f.size() > g.size(); }
		};
		priority_queue<Q, vector<Q>, cmp> q(all(a));
		while (q.size() > 1)
		{
			auto f = q.top(); q.pop();
			f = shrink(f * q.top()); q.pop();
			q.push(f);
		}
		return q.top();
	}//not check
	vector<ll> evaluation(const Q& f, const vector<ll>& X)
	{
		int m = X.size(), n = f.size() - 1, i, j;
		vector<Q> pro(m * 4 + 4);
		while (n > 1 && !f[n]) --n;
		vector<ll> y(m);
		function<void(int, int, int)> build = [&](int x, int l, int r)
			{
				if (l + 1 == r)
				{
					pro[x] = Q(vector{(p - X[l]) % p, 1llu});
					return;
				}
				int mid = l + r >> 1, c = x * 2;
				build(c, l, mid); build(c + 1, mid, r);
				pro[x] = shrink(pro[c] * pro[c + 1]);
			};
		function<void(int, int, int, Q, int)> dfs = [&](int x, int l, int r, Q f, int d)
			{
				const static int limit = 256;
				if (d >= r - l) f = shrink(mod(f, pro[x]));
				if (r - l < limit)
				{
					for (int i = l; i < r; i++) y[i] = f.fx(X[i]);
					return;
				}
				int mid = l + r >> 1, c = x * 2;
				dfs(c, l, mid, f, d);
				dfs(c + 1, mid, r, f, d);
			};
		build(1, 0, m);
		dfs(1, 0, m, f, n);
		return y;
	}//131072 880ms
	vector<ll> evaluation_new(Q f, const vector<ll>& X)//多项式多点求值
	{
		int m = X.size(), i, j;
		vector<ll> y(m);
		if (X.size() <= 10)
		{
			for (i = 0; i < m; i++) y[i] = f.fx(X[i]);
			return y;
		}
		int n = f.size();
		while (n > 1 && !f[n - 1]) --n;
		f.resize(cal(n));
		vector<Q> pro(m * 4 + 4);
		function<void(int, int, int)> build = [&](int x, int l, int r)
			{
				if (l == r)
				{
					pro[x] = Q(vector{1llu, (p - X[l]) % p});
					return;
				}
				int m = l + r >> 1, c = x * 2;
				build(c, l, m); build(c + 1, m + 1, r);
				pro[x] = shrink(pro[c] * pro[c + 1]);
			};
		function<void(int, int, int, Q)> dfs = [&](int x, int l, int r, Q f)
			{
				const static int limit = 30;
				if (r - l + 1 <= limit)
				{
					int m = r - l + 1, m1, m2, mid = l + r >> 1, i, j, k;
					static ll g[limit + 2], g1[limit + 2], g2[limit + 2];
					m1 = m2 = r - l;
					copy_n(f.data(), m, g1);
					copy_n(g1, m, g2);
					for (i = mid + 1; i <= r; i++, --m1) for (k = 0; k < m1; k++) g1[k] = (g1[k] + g1[k + 1] * (p - X[i])) % p;
					for (i = l; i <= mid; i++, --m2) for (k = 0; k < m2; k++) g2[k] = (g2[k] + g2[k + 1] * (p - X[i])) % p;
					for (i = l; i <= mid; i++)
					{
						copy_n(g1, (m = m1) + 1, g);
						for (j = l; j <= mid; j++) if (i != j)
						{
							for (k = 0; k < m; k++) g[k] = (g[k] + g[k + 1] * (p - X[j])) % p;
							--m;
						}
						y[i] = g[0];
					}
					for (i = mid + 1; i <= r; i++)
					{
						copy_n(g2, (m = m2) + 1, g);
						for (j = mid + 1; j <= r; j++) if (i != j)
						{
							for (k = 0; k < m; k++) g[k] = (g[k] + g[k + 1] * (p - X[j])) % p;
							--m;
						}
						y[i] = g[0];
					}
					return;
				}
				int mid = l + r >> 1, c = x * 2, n = f.size();
				f.dft();
				for (auto [x, len] : {pair{c, r - mid}, {c + 1, mid - l + 1}})
				{
					pro[x] %= n;
					reverse(all(pro[x])); pro[x] &= f;
					rotate(all(pro[x]) - 1, pro[x].end());
					pro[x] %= cal(len);
					fill(len + all(pro[x]), 0);
				}
				dfs(c, l, mid, pro[c + 1]);
				dfs(c + 1, mid + 1, r, pro[c]);
			};
		build(1, 0, m - 1);
		pro[1] %= f.size();
		(f ^= ~pro[1]) %= cal(m);
		fill(min(m, n) + all(f), 0);
		dfs(1, 0, m - 1, f);
		return y;
	}//131072 460ms
	ll factorial(ll n)
	{
		if (n >= p) return 0;
		if (n <= 1) return 1 % p;
		ll B = ::sqrt(n), i;
		vector F(B, Q({0, 1}));
		for (i = 0; i < B; i++) F[i][0] = i + 1;
		auto f = prod(F);
		vector<ll> x(B);
		for (i = 0; i < B; i++) x[i] = i * B;
		ll r = 1;
		auto y = evaluation(f, x);
		for (i = 0; i < B; i++) r = r * y[i] % p;
		for (i = B * B + 1; i <= n; i++) r = r * i % p;
		return r;
	}//998244352 170ms
	vector<ll> getinvs(vector<ll> a)
	{
		int n = a.size(), i;
		if (n <= 2)
		{
			for (i = 0; i < n; i++) a[i] = ksm(a[i], p - 2);
			return a;
		}
		vector<ll> l(n), r(n);
		l[0] = a[0]; r[n - 1] = a[n - 1];
		for (i = 1; i < n; i++) l[i] = l[i - 1] * a[i] % p;
		for (i = n - 2; i; i--) r[i] = r[i + 1] * a[i] % p;
		ll x = ksm(l[n - 1], p - 2);
		a[0] = x * r[1] % p; a[n - 1] = x * l[n - 2] % p;
		for (i = 1; i < n - 1; i++) a[i] = x * l[i - 1] % p * r[i + 1] % p;
		return a;
	}
	Q interpolation(const vector<ll>& X, const vector<ll>& y)//多项式快速插值
	{
		assert(X.size() == y.size());
		int n = X.size(), i, j;
		if (n <= 1) return Q(y);
		if (1)
		{
			auto vv = X; sort(all(vv));
			assert(unique(all(vv)) - vv.begin() == n);
		}
		vector<Q> sum(4 * n + 4), pro(4 * n + 4);
		function<void(int, int, int)> build = [&](int x, int l, int r)
			{
				if (l == r)
				{
					sum[x] = Q(vector{(p - X[l]) % p, 1llu});
					return;
				}
				int mid = l + r >> 1, c = x * 2;
				build(c, l, mid); build(c + 1, mid + 1, r);
				sum[x] = shrink(sum[c] * sum[c + 1]);
			};
		build(1, 0, n - 1);
		auto v = evaluation_new(sum[1] = der(sum[1]), X);
		assert(v.size() == n);
		auto Y = getinvs(v);
		for (i = 0; i < n; i++) Y[i] = Y[i] * y[i] % p;
		function<void(int, int, int)> dfs = [&](int x, int l, int r)
			{
				if (l == r)
				{
					pro[x][0] = Y[l];
					return;
				}
				int c = x * 2, mid = l + r >> 1;
				dfs(c, l, mid); dfs(c | 1, mid + 1, r);
				pro[x] = shrink((pro[c] * sum[c | 1]) + (pro[c | 1] * sum[c]));
			};
		dfs(1, 0, n - 1);
		return pro[1] %= cal(n);
	}//131072 1150ms
	Q comp(const Q& f, Q g)//多项式复合 f(g(x))=[x^i]f(x)g(x)^i
	{
		int n = f.size(), l = ceil(::sqrt(n)), i, j;
		assert(n >= g.size());//返回 n-1 次多项式
		vector<Q> a(l + 1), b(l);
		a[0] %= n; a[0][0] = 1; a[1] = g;
		g %= n * 2;
		Q u = g, v(n);
		g.dft();
		for (i = 2; i <= l; i++) a[i] = ((u &= g) %= n), u %= n * 2;
		for (i = 2; i < l; i++)
		{
			u.dft(); b[i - 1] = u;
			u &= b[1]; fill(n + all(u), 0);
		}
		u.dft(); b[l - 1] = u;
		for (i = 0; i < l; i++)
		{
			fill(all(v), 0);
			for (j = 0; j < l; j++) if (i * l + j < n) v += a[j] * f[i * l + j];
			if (i == 0) u = v; else u += ((v %= n * 2) &= b[i]) %= n;
		}
		return u;
	}//n^2+n\sqrt n\log n,8000 350ms
	Q comp_inv(Q f)//多项式复合逆 g(f(x))=x,求 g,[x^n]g=([x^{n-1}](x/f)^n)/n,要求常数 0 一次非 0
	{
		assert(!f[0] && f[1]);
		int n = f.size(), l = ceil(::sqrt(n)), i, j, k, m;//l>=2
		rotate(f.begin(), 1 + all(f));
		f = ~f;
		getfac(n * 2);
		vector<Q> a(l + 1), b(l);
		Q u, v;
		u = a[1] = f;
		u %= n * 2; (v = u).dft();
		for (i = 2; i <= l; i++)
		{
			u &= v;
			fill(n + all(u), 0);
			a[i] = u;
		}
		b[0] %= n; b[0][0] = 1; b[1] = u; (v = u).dft();
		for (i = 2; i < l; i++)
		{
			u &= v;
			fill(n + all(u), 0);
			b[i] = u;
		}
		u %= n; u[0] = 0;
		for (i = 0; i < l; i++) for (j = 1; j <= l; j++) if (i * l + j < n)
		{
			m = i * l + j - 1;
			ll r = 0, * f = b[i].data(), * g = a[j].data();
			for (k = 0; k <= m; k++) r = (r + f[k] * g[m - k]) % p;
			u[m + 1] = r * inv[m + 1] % p;
		}
		return u;
	}//8000 200ms
	Q shift(Q f, ll c)//get f(x+c),c\in [0,p)
	{
		int n = f.size(), i, j;
		Q g(n);
		getfac(n);
		for (i = 0; i < n; i++) (f[i] *= fac[i]) %= p;
		g[0] = 1;
		for (i = 1; i < n; i++) g[i] = g[i - 1] * c % p;
		for (i = 0; i < n; i++) (g[i] *= ifac[i]) %= p;
		f ^= g;
		for (i = 0; i < n; i++) (f[i] *= ifac[i]) %= p;
		return f;
	}//5e5 200ms (1 2 3 4 5) 3 -> (547 668 309 64 5)
	vector<ll> shift(vector<ll> y, ll c, ll m)//[0,n) 点值 -> [c,c+m) 点值
	{
		assert(y.size());
		if (y.size() == 1) return vector(m, y[0]);
		vector<ll> r, res;
		r.reserve(m);
		int n = y.size(), i, j, mm = m;
		while (c < n && m) r.push_back(y[c++]), --m;
		if (c + m > p)
		{
			res = shift(y, 0, c + m - p);
			m = p - c;
		}
		if (!m) { r.insert(r.end(), all(res)); return r; }
		int len = cal(m + n - 1), l = m + n - 1;
		for (i = n & 1; i < n; i += 2) y[i] = (p - y[i]) % p;
		getfac(n);
		for (i = 0; i < n; i++) y[i] = y[i] * ifac[i] % p * ifac[n - 1 - i] % p;
		y.resize(len);
		Q f, g;
		vector<ll> v(m + n - 1);
		c -= n - 1;
		for (i = 0; i < l; i++) v[i] = (c + i) % p;
		f = Q(y); g = Q(getinvs(v)) % len;
		f *= g;
		vector<ll> u(m);
		for (i = n - 1; i < l; i++) u[i - (n - 1)] = f[i];
		v.resize(m);
		for (i = 0; i < m; i++) v[i] = c + i;
		v = getinvs(v); c += n;
		ll tmp = 1;
		for (i = c - n; i < c; i++) tmp = tmp * i % p;
		for (i = 0; i < m; i++) (u[i] *= tmp) %= p, tmp = tmp * (c + i) % p * v[i] % p;
		r.insert(r.end(), all(u));
		r.insert(r.end(), all(res));
		assert(r.size() == mm);
		return r;
	}//5e5 430ms, (1 4 9 16) 3 5 -> (16 25 36 49 64)
	vector<ll> Z_transform(Q f, ll c, ll m)//求 f(c^[0,m))。核心 ij=C(i+j,2)-C(i,2)-C(j,2)
	{
		const static ll B = 1e5;
		static ll a[B + 2], b[B + 2];
		int i, n = f.size();
		if (n * m < B * 5)
		{
			vector<ll> r(m);
			ll j;
			for (i = 0, j = 1; i < m; i++) r[i] = f.fx(j), j = j * c % p;
			return r;
		}
		auto mic = [&](ll x) { return a[x % B] * b[x / B] % p; };
		ll l = cal(m += n - 1);
		Q g(l);
		assert(B * B > p);
		a[0] = b[0] = g[0] = g[1] = 1;
		for (i = 1; i <= B; i++) a[i] = a[i - 1] * c % p;
		for (i = 1; i <= B; i++) b[i] = b[i - 1] * a[B] % p;
		for (i = 2; i < n; i++) f[i] = f[i] * mic((p * 2 - 2 - i) * (i - 1) / 2 % (p - 1)) % p;
		for (i = 2; i < m; i++) g[i] = mic(i * (i - 1llu) / 2 % (p - 1));
		reverse(all(f)); (f %= l) &= g;
		vector<ll> r(f.begin() + n - 1, f.begin() + m); m -= n - 1;
		for (i = 2; i < m; i++) r[i] = r[i] * mic((p * 2 - 2 - i) * (i - 1) / 2 % (p - 1)) % p;
		return r;
	}//luogu 1e6 500ms
	vector<ll> Bell(int n)//B(0...n)
	{
		++n;
		getfac(n - 1);
		Q f(n);
		int i;
		for (i = 1; i < n; i++) f[i] = ifac[i];
		f = exp_new(f);
		for (i = 2; i < n; i++) f[i] = f[i] * fac[i] % p;
		return vector<ll>(f.begin(), f.begin() + n);
	}//not check
	vector<ll> S1_row(int n, int m)//S1(n,0...m),O(nlogn),unsigned
	{
		int cm = cal(++m);
		if (n == 0)
		{
			vector<ll> r(m);
			r[0] = 1;
			return r;
		}
		function<Q(int)> dfs = [&](int n)
			{
				if (n == 1)
				{
					Q f(2);
					f[1] = 1;
					return f;
				}
				Q f = dfs(n / 2);
				f *= shift(f, n / 2);
				if (n & 1)
				{
					f %= cal(n + 1);
					for (int i = n; i; i--) f[i] = f[i - 1];
					// for (int i=1; i<=n; i++) f[i]=f[i-1];
					--n;
					for (int i = 0; i <= n; i++) f[i] = (f[i] + f[i + 1] * n) % p;
				}
				if (f.size() > cm) f %= cm;
				return f;
			};
		Q f = dfs(n);
		if (f.size() < cm) f %= cm;
		return vector<ll>(f.begin(), f.begin() + m);
	}
	vector<ll> S1_column(int n, int m)//S1(0...n,m),O(nlogn)
	{
		if (m == 0)
		{
			vector<ll> r(n + 1);
			r[0] = 1;
			return r;
		}
		Q f(n + 1);
		getfac(max(n, m));
		int i;
		for (i = 1; i <= n; i++) f[i] = inv[i];
		f = pow(f, m);
		for (i = m; i <= n; i++) f[i] = f[i] * fac[i] % p * ifac[m] % p;
		return vector<ll>(f.begin(), f.begin() + n + 1);
	}
	vector<ll> S2_row(int n, int m)//S2(n,0...m),O(mlogm)
	{
		int tm = ++m, i, j, cnt = 0;
		if (n == 0)
		{
			vector<ll> r(m);
			r[0] = 1;
			return r;
		}
		m = min(m, n + 1);
		vector<ll> pr(m), pw(m);
		pw[1] = 1;
		for (i = 2; i < m; i++)
		{
			if (!pw[i]) pr[cnt++] = i, pw[i] = ksm(i, n);
			for (j = 0; i * pr[j] < m; j++)
			{
				pw[i * pr[j]] = pw[i] * pw[pr[j]] % p;
				if (i % pr[j] == 0) break;
			}
		}
		getfac(m - 1);
		Q f(m), g(m);
		for (i = 0; i < m; i += 2) f[i] = ifac[i];
		for (i = 1; i < m; i += 2) f[i] = p - ifac[i];
		// for (i=1; i<m; i++) g[i]=pw[i]*ifac[i]%p;
		for (i = 1; i < m; i++) g[i] = ksm(i, n) * ifac[i] % p;
		f *= g;
		vector<ll> r(f.begin(), f.begin() + m);
		r.resize(tm);
		return r;
	}//5e5 150ms
	vector<ll> S2_column(int n, int m)//S2(0...n,m),O(nlogn)
	{
		if (m == 0)
		{
			vector<ll> r(n + 1);
			r[0] = 1;
			return r;
		}
		Q f(n + 1);
		getfac(max(n, m));
		int i;
		for (i = 1; i <= n; i++) f[i] = ifac[i];
		f = pow(f, m);
		for (i = m; i <= n; i++) f[i] = f[i] * fac[i] % p * ifac[m] % p;
		return vector<ll>(f.begin(), f.begin() + n + 1);
	}//5e5 640ms
	vector<ll> signed_S1_row(int n, int m)
	{
		auto v = S1_row(n, m);
		for (int i = 1 ^ n & 1; i <= m; i += 2) v[i] = (p - v[i]) % p;
		return v;
	}//5e5 190ms
	vector<ll> Bernoulli(int n)//B(0...n)
	{
		getfac(++n);
		int i;
		Q f(n);
		for (i = 0; i < n; i++) f[i] = ifac[i + 1];
		f = ~f;
		for (i = 0; i < n; i++) f[i] = f[i] * fac[i] % p;
		return vector<ll>(f.begin(), f.begin() + n);
	}//5e5 180ms
	vector<ll> Partition(int n)//P(0...n),拆分数
	{
		Q f(++n);
		int i, l = 0, r = 0;
		while (--l) if (3 * l * l - l >= n * 2) break;
		while (++r) if (3 * r * r - r >= n * 2) break;
		++l;
		for (i = l + abs(l) % 2; i < r; i += 2) f[3 * i * i - i >> 1] = 1;
		for (i = l + abs(l + 1) % 2; i < r; i += 2) f[3 * i * i - i >> 1] = p - 1;
		f = ~f;
		return vector<ll>(f.begin(), f.begin() + n);
	}//5e5 150ms
	struct reg
	{
		Q a00, a01, a10, a11;
		reg operator*(const reg& o) const
		{
			return {
				shrink(a00 * o.a00 + a01 * o.a10),
				shrink(a00 * o.a01 + a01 * o.a11),
				shrink(a10 * o.a00 + a11 * o.a10),
				shrink(a10 * o.a01 + a11 * o.a11)};
		}
		pair<Q, Q> operator*(const pair<Q, Q>& o) const
		{
			const auto& [b0, b1] = o;
			return {shrink(a00 * b0 + a01 * b1), shrink(a10 * b0 + a11 * b1)};
		}
	} E = {{vector{1llu}}, Q(), Q(), {vector{1llu}}};
	ostream& operator<<(ostream& cout, const reg& o)
	{
		return cout << "[" << o.a00 << ", " << o.a01 << "]\n"
			<< "[" << o.a10 << ", " << o.a11 << "]\n";
	}
	reg hgcd(Q a, Q b)
	{
		int m = a.deg() + 1 >> 1;
		if (b.deg() < m) return E;
		reg r = hgcd(a >> m, b >> m);
		auto [c, d] = r * pair{a, b};
		if (d.deg() < m) return r;
		auto [q, e] = div_mod(c, d);
		r.a00 -= shrink(q * r.a10);
		r.a01 -= shrink(q * r.a11);
		swap(r.a00, r.a10);
		swap(r.a01, r.a11);
		if (e.deg() < m) return r;
		int k = 2 * m - d.deg();
		auto s = hgcd(d >> k, e >> k);
		return s * r;
	}
	Q gcd(Q a, Q b)
	{
		if (a.deg() < b.deg()) swap(a, b);
		while (b.deg() >= 0)
		{
			a = mod(a, b);
			swap(a, b);
			auto tmp = hgcd(a, b);
			tie(a, b) = tmp * pair{a, b};
		}
		if (a.deg() == -1) return a;
		ll k = ksm(a[a.deg()], p - 2);
		for (int i = 0; i < a.size(); i++) a[i] = a[i] * k % p;
		return a;
	}
	vector<ll> root(Q f)
	{
		Q x(2);
		x[1] = 1;
		x = pow(x, p, f);
		if (x.size() < 2) x %= 2;
		(x[1] += p - 1) %= p;
		f = gcd(f, x);
		vector<ll> res;
		static mt19937 rnd(chrono::steady_clock::now().time_since_epoch().count());
		function<void(Q)> dfs = [&](Q f)
			{
				int n = f.deg(), i;
				if (n <= 0) return;
				if (n == 1)
				{
					res.push_back((p - f[0]) % p);
					return;
				}
				Q g(n);
				for (i = 0; i < n; i++) g[i] = rnd() % p;
				g = gcd(pow(g, (p - 1) / 2, f) - 1, f);
				dfs(g); dfs(div(f, g));
			};
		dfs(f);
		sort(all(res));
		assert(unique(all(res)) == res.end());
		return res;
	}//4000 950ms
	optional<Q> inverse(Q a, Q m)
	{
		Q b = m;
		vector<pair<reg, Q>> buf;
		a = mod(a, b);
		swap(a, b);
		while (b.deg() >= 0)
		{
			auto [q, r] = div_mod(a, b);
			swap(a, r); swap(a, b);
			auto tmp = hgcd(a, b);
			tie(a, b) = tmp * pair{a, b};
			buf.emplace_back(move(tmp), q);
		}
		if (a.deg()) return { };
		reg res = E;
		reverse(all(buf));
		for (const auto& [tmp, q] : buf)
		{
			res = res * tmp;
			res.a00 -= shrink(q * res.a01);
			res.a10 -= shrink(q * res.a11);
			swap(res.a00, res.a01);
			swap(res.a10, res.a11);
		}
		return {res.a01 * ksm(a[0], p - 2)};
	}//5e4 950ms
}
using NTT::p;
using poly = NTT::Q;
\end{lstlisting}


\subsection{MTT}

\begin{lstlisting}
namespace MTT
{
	template<ll p> constexpr ll ksm(ll x,ll y=p-2)
	{
		ll r=1;
		while (y)
		{
			if (y&1) r=r*x%p;
			x=x*x%p;
			y>>=1;
		}
		return r;
	}
	int cal(int x) { return 1<<__lg(max(x,1)*2-1); }
	const int N=1<<22;
	const ll p=1e9+7,g=3,
		p1=469'762'049,p2=998'244'353,p3=1004'535'809,//三模,原根都是 3,非常好
		inv_p1=ksm<p2>(p1),inv_p12=ksm<p3>(p1*p2%p3),_p12=p1*p2%p;//三模,1 关于 2 逆,1*2 关于 3 逆,1*2 mod 3
	int r[N];
	struct P
	{
		ll v1,v2,v3;
		P operator+(const P &o) const { return {v1+o.v1,v2+o.v2,v3+o.v3}; }
		P operator-(const P &o) const { return {v1+p1-o.v1,v2+p2-o.v2,v3+p3-o.v3}; }
		P operator*(const P &o) const { return {v1*o.v1,v2*o.v2,v3*o.v3}; }
		void operator+=(const P &o) { v1+=o.v1,v2+=o.v2,v3+=o.v3; }
		void operator-=(const P &o) { v1+=p1-o.v1,v2+=p2-o.v2,v3+=p3-o.v3; }
		void operator*=(const P &o) { v1*=o.v1,v2*=o.v2,v3*=o.v3; }
		void mod() { v1%=p1,v2%=p2,v3%=p3; }
	};
	P w[N];
	void init(int n)
	{
		static int pr=0,pw=0;
		if (pr==n) return;
		int b=__lg(n)-1,i,j,k;
		for (i=1; i<n; i++) r[i]=r[i>>1]>>1|(i&1)<<b;
		if (pw<n)
		{
			for (j=1; j<n; j=k)
			{
				k=j*2;
				P wn={ksm<p1>(g,(p1-1)/k),ksm<p2>(g,(p2-1)/k),ksm<p3>(g,(p3-1)/k)};
				w[j]={1,1,1};
				for (i=j+1; i<k; i++) w[i]=w[i-1]*wn,w[i].mod();
			}
			pw=n;
		}
		pr=n;
	}
	void dft(vector<P> &a,int o=0)
	{
		int n=a.size(),i,j,k;
		P *f,*g,*wn,*b=a.data(),x,y;
		init(n);
		for (i=1; i<n; i++) if (i<r[i]) swap(a[i],a[r[i]]);
		for (k=1; k<n; k*=2)
		{
			wn=w+k;
			for (i=0; i<n; i+=k*2)
			{
				f=b+i; g=b+i+k;
				for (j=0; j<k; j++)
				{
					y=g[j]*wn[j];
					y.mod();
					g[j]=f[j]-y;
					f[j]+=y;
				}
			}
			if (k*2==n||k==1<<14) for (P &x:a) x.mod();
		}
		if (o)
		{
			x={ksm<p1>(n),ksm<p2>(n),ksm<p3>(n)};
			for (P &y:a) y*=x,y.mod();
			reverse(1+all(a));
		}
	}
	struct Q:vector<ll>
	{
		Q(int x=1):vector(x) { }
		Q &operator%=(int n) { resize(n); return *this; }
	};
	Q &operator*=(Q &f,const Q &g)
	{
		int n=f.size()+g.size()-1,m=cal(n),i;
		vector<P> F(m,{0,0,0}),G(m,{0,0,0});
		for (i=0; i<f.size(); i++) F[i]={f[i]%p1,f[i]%p2,f[i]%p3};
		for (i=0; i<g.size(); i++) G[i]={g[i]%p1,g[i]%p2,g[i]%p3};
		dft(F); dft(G);
		for (i=0; i<m; i++) F[i]*=G[i],F[i].mod();
		dft(F,1);
		f%=n;
		ll x;
		for (i=0; i<n; i++)
		{
			auto [r1,r2,r3]=F[i];
			x=(r2+p2-r1)*inv_p1%p2*p1+r1;
			f[i]=((x+p3-r3)%p3*(p3-inv_p12)%p3*_p12+x)%p;
		}
		return f;
	}//5e5 440ms
	Q operator*(Q f,const Q &g) { return f*=g; }
}
using MTT::p;
using poly=MTT::Q;
\end{lstlisting}
\subsection{FFT}

\begin{lstlisting}
namespace FFT
{
	#define all(x) (x).begin(),(x).end()
	typedef double db;
	const int N=1<<21;
	const db pi=3.14159265358979323846;
	struct comp
	{
		db x,y;
		comp operator+(const comp &o) const {return {x+o.x,y+o.y};}
		comp operator-(const comp &o) const {return {x-o.x,y-o.y};}
		comp operator*(const comp &o) const {return {x*o.x-y*o.y,o.x*y+x*o.y};}
		comp operator*(const db &o) const {return {x*o,y*o};}
		void operator*=(const comp &o) {*this={x*o.x-y*o.y,o.x*y+x*o.y};}
		void operator*=(const db &o) {x*=o;y*=o;}
		void operator/=(const db &o) {x/=o;y/=o;}
		comp operator/(const comp &o) const
		{
			db z=1/(o.x*o.x+o.y*o.y);
			return {z*(x*o.x+y*o.y),z*(o.x*y-x*o.y)};
		}//not necessary, no check
	};
	long long dtol(const double &x) {return fabs(round(x));}
	const comp I{0,-1};
	ostream & operator<<(ostream &cout,const comp &o) {cout<<o.x;if (o.y>=0) cout<<'+';return cout<<o.y<<'i';}
	int r[N];
	char c;
	comp Wn[N];
	void init(int n)
	{
		static int preone=-1;
		if (n==preone) return;
		preone=n;
		int b,i;
		b=__builtin_ctz(n)-1;
		for (i=1;i<n;i++) r[i]=r[i>>1]>>1|(i&1)<<b;
		for (i=0;i<n;i++) Wn[i]={cos(pi*i/n),sin(pi*i/n)};
	}
	int cal(int x) {return 1u<<32-__builtin_clz(max(x,2)-1);}
	struct Q
	{
		vector<comp> a;
		int deg;
		comp* pt() {return a.data();}
		Q(int n=0)
		{
			deg=n;
			a.resize(cal(n));
		}
		void dft(int xs=0)//1,0
		{
			int i,j,k,l,n=a.size(),d;
			comp w,wn,b,c,*f=pt(),*g,*a=f;
			init(n);
			if (xs) reverse(a+1,a+n);//spe
			for (i=0;i<n;i++) if (i<r[i]) swap(a[i],a[r[i]]);
			for (i=1,d=0;i<n;i=l,d++)
			{
				//wn={cos(pi/i),(xs?-1:1)*sin(pi/i)};
				l=i<<1;
				for (j=0;j<n;j+=l)
				{
					//w={1,0};
					f=a+j;g=f+i;
					for (k=0;k<i;k++)
					{
						w=Wn[k*(n>>d)];
						b=f[k];c=g[k]*w;
						f[k]=b+c;
						g[k]=b-c;
						//w*=wn;
					}
				}
			}
			if (xs) for (i=0;i<n;i++) a[i]/=n;
		}
		void operator|=(Q o)
		{
			int n=deg+o.deg-1,m=cal(n),i;
			a.resize(m);o.a.resize(m);
			dft();o.dft();
			for (i=0;i<m;i++) a[i]*=o.a[i];
			dft(1);
			for (i=n;i<m;i++) a[i]={};
			deg=n;
		}
		Q operator|(Q o) const {o|=*this;return o;}
	};
	Q mul(Q a,const Q &b)//三次变两次,仅实数,注意精度
	{
		int n=a.deg+b.deg-1,m=cal(n),i;
		a.a.resize(m);
		for (i=0;i<b.deg;i++) a.a[i]={a.a[i].x,b.a[i].x};
		a.dft();
		for (i=0;i<m;i++) a.a[i]*=a.a[i];
		a.dft(1);
		for (i=0;i<n;i++) a.a[i]={a.a[i].y*.5};
		for (i=n;i<m;i++) a.a[i]={};
		a.deg=n;
		return a;
	}
	void ddt(Q &a,Q &b)//double dft,仅实数,注意精度
	{
		comp x,y;
		int n=a.a.size(),i;
		assert(n==b.a.size());
		for (i=0;i<n;i++) a.a[i]={a.a[i].x,b.a[i].x};
		a.dft();
		for (i=0;i<n;i++) b.a[i]={a.a[i].x,-a.a[i].y};
		reverse(b.pt()+1,b.pt()+n);
		for (i=0;i<n;i++)
		{
			x=a.a[i];y=b.a[i];
			a.a[i]=(x+y)*.5;
			b.a[i]=(y-x)*.5*I;
		}
	}
}
using FFT::dtol;
\end{lstlisting}

\subsection{约数个数和}

$O(\sqrt[3]n\log n)$。

\begin{lstlisting}
#include"bits/stdc++.h"
#define ll long long
#define lll __int128
using namespace std;

void myw(lll x){
	if(!x) return;
	myw(x/10);printf("%d",(int)(x%10));
}

struct vec{
	ll x,y;
	vec (ll x0=0,ll y0=0){x=x0,y=y0;}
	vec operator +(const vec b){return vec(x+b.x,y+b.y);}
};

ll N;
vec stk[1000005];int len;
vec P;
vec L,R; 

bool ninR(vec a){return N<(lll)a.x*a.y;}
bool steep(ll x,vec a){return (lll)N*a.x<=(lll)x*x*a.y;}

lll Solve(){
	len=0;
	ll cbr=cbrt(N),sqr=sqrt(N);
	P.x=N/sqr,P.y=sqr+1;
	lll ans=0;
	stk[++len]=vec(1,0);stk[++len]=vec(1,1);
	while(1){
		L=stk[len--];
		while(ninR(vec(P.x+L.x,P.y-L.y)))
			ans+=(lll)P.x*L.y+(lll)(L.y+1)*(L.x-1)/2,
			P.x+=L.x,P.y-=L.y;
		if(P.y<=cbr) break;
		R=stk[len];
		while(!ninR(vec(P.x+R.x,P.y-R.y))) L=R,R=stk[--len];
		while(1){
			vec mid=L+R;
			if(ninR(vec(P.x+mid.x,P.y-mid.y))) R=stk[++len]=mid;
			else if(steep(P.x+mid.x,R)) break;
			else L=mid;
		}
	}
	for(int i=1;i<P.y;i++) ans+=N/i;
	return ans*2-sqr*sqr;
}

int T;

int main(){
	scanf("%d",&T);
	while(T--){
		scanf("%lld",&N);
		myw(Solve());printf("\n");
	}
}
\end{lstlisting}



\subsection{万能欧几里得/min of mod of linear}

题意:$\sum\limits_{x=0}^{n-1}\lfloor \dfrac {ax+b}m\rfloor$ ($0\le a,b<m$)

\includegraphics[scale=0.8]{1.png}

原理:考虑紧贴着斜线的折线的答案。每个 \verb|nd| 表示的是一段折线,你需要实现 \verb|operator+| 来计算出拼接两个折线之后的答案。除此以外的原理不必了解。

你需要传入的 \verb|dx| 和 \verb|dy| 表示向上和向右的折线的答案(也就是边界)。

\begin{lstlisting}
struct nd
{
	ll x, y, sy;
	nd operator+(const nd &o) const
	{
		return {x + o.x, y + o.y, sy + o.sy + y * o.x};
	}
};
nd ksm(nd a, int k)
{
	nd res{ };
	while (k)
	{
		if (k & 1) res = res + a;
		a = a + a; k >>= 1;
	}
	return res;
}
nd sol(int a, int b, int m, int n, nd dx, nd dy)//[0,n] (ax+b)/m 0<=b<m
{
	if (!n) return { };
	if (a >= m) return sol(a % m, b, m, n, ksm(dy, a / m) + dx, dy);
	int c = ((ll)n * a + b) / m;
	if (!c) return ksm(dx, n);
	int cnt = n - ((ll)m * c - b - 1) / a;
	return ksm(dx, (m - b - 1) / a) + dy + sol(m, (m - b - 1) % a, a, c - 1, dy, dx) + ksm(dx, cnt);
}
ll sum_of_floor_of_linear(int a, int b, int m, int n)//[0,n] sum((ax+b)/m)
{
	nd dx = {1, 0, 0}, dy = {0, 1, 0};
	int nb = (b % m + m) % m;
	return sol(a, nb, m, n, dx, dy).sy + (ll)(b - nb) / m * (n + 1);
}
int min_of_mod_of_linear(int a, int b, int p, int n)//[0,n] min((ax+b) mod p)
{
	ll s = sum_of_floor_of_linear(a, b, p, n);
	int l = 0, r = p - 1, mid;
	while (l < r)
	{
		mid = (l + r + 1) / 2;
		if (sum_of_floor_of_linear(a, b - mid, p, n) >= s) l = mid;
		else r = mid - 1;
	}
	return l;
}
\end{lstlisting}


\subsection{高斯整数类}

圆上整点的基础。

\begin{lstlisting}
ll roundiv(ll x,ll y)
{
	return x>=0?(x+y/2)/y:(x-y/2)/y;
}
struct Q
{
	ll x,y;
	Q operator~() const { return {x,-y}; }
	ll len2() const { return x*x+y*y; }
	Q operator+(const Q &o) const { return {x+o.x,y+o.y}; }
	Q operator-(const Q &o) const { return {x-o.x,y-o.y}; }
	Q operator*(const Q &o) const { return {x*o.x-y*o.y,x*o.y+y*o.x}; }
	Q operator/(const Q &o) const
	{
		Q t=*this*~o;
		ll l=o.len2();
		return {roundiv(t.x,l),roundiv(t.y,l)};
	}
	Q operator%(const Q &o) const { return *this-*this/o*o; }
};
Q gcd(Q a,Q b)
{
	if (a.len2()>b.len2()) swap(a,b);
	while (a.len2())
	{
		b=b%a;
		swap(a,b);
	}
	return b;
}

\end{lstlisting}

\subsection{Miller Rabin/Pollard Rho}

1s:$200$ 组 $10^{18}$。

如果你只需要做 \verb|int| 以内的分解,你可以改为

\begin{lstlisting}
typedef int ll;
typedef long long lll;
\end{lstlisting}

\begin{lstlisting}
namespace pr
{
	typedef long long ll;
	typedef __int128 lll;
	typedef pair<ll,int> pa;
	ll ksm(ll x,ll y,const ll p)
	{
		ll r=1;
		while (y)
		{
			if (y&1) r=(lll)r*x%p;
			x=(lll)x*x%p; y>>=1;
		}
		return r;
	}
	namespace miller
	{
		const int p[7]={2,3,5,7,11,61,24251};
		ll s,t;
		bool test(ll n,int p)
		{
			if (p>=n) return 1;
			ll r=ksm(p,t,n),w;
			for (int j=0; j<s&&r!=1; j++)
			{
				w=(lll)r*r%n;
				if (w==1&&r!=n-1) return 0;
				r=w;
			}
			return r==1;
		}
		bool prime(ll n)
		{
			if (n<2||n==46'856'248'255'981) return 0;
			for (int i=0; i<7; ++i) if (n%p[i]==0) return n==p[i];
			s=__builtin_ctz(n-1); t=n-1>>s;
			for (int i=0; i<7; ++i) if (!test(n,p[i])) return 0;
			return 1;
		}
	}
	using miller::prime;
	mt19937_64 rnd(chrono::steady_clock::now().time_since_epoch().count());
	namespace rho
	{
		void nxt(ll &x,ll &y,ll &p) { x=((lll)x*x+y)%p; }
		ll find(ll n,ll C)
		{
			ll l,r,d,p=1;
			l=rnd()%(n-2)+2,r=l;
			nxt(r,C,n);
			int cnt=0;
			while (l^r)
			{
				p=(lll)p*llabs(l-r)%n;
				if (!p) return gcd(n,llabs(l-r));
				++cnt;
				if (cnt==127)
				{
					cnt=0;
					d=gcd(llabs(l-r),n);
					if (d>1) return d;
				}
				nxt(l,C,n); nxt(r,C,n); nxt(r,C,n);
			}
			return gcd(n,p);
		}
		vector<pa> w;
		vector<ll> d;
		void dfs(ll n,int cnt)
		{
			if (n==1) return;
			if (prime(n)) return w.emplace_back(n,cnt),void();
			ll p=n,C=rnd()%(n-1)+1;
			while (p==1||p==n) p=find(n,C++);
			int r=1; n/=p;
			while (n%p==0) n/=p,++r;
			dfs(p,r*cnt); dfs(n,cnt);
		}
		vector<pa> getw(ll n)
		{
			w=vector<pa>(0); dfs(n,1);
			if (n==1) return w;
			sort(w.begin(),w.end());
			int i,j;
			for (i=1,j=0; i<w.size(); i++) if (w[i].first==w[j].first) w[j].second+=w[i].second; else w[++j]=w[i];
			w.resize(j+1);
			return w;
		}
		void dfss(int x,ll n)
		{
			if (x==w.size()) return d.push_back(n),void();
			dfss(x+1,n);
			for (int i=1; i<=w[x].second; i++) dfss(x+1,n*=w[x].first);
		}
		vector<ll> getd(ll n)
		{
			getw(n); d=vector<ll>(0); dfss(0,1);
			sort(d.begin(),d.end());
			return d;
		}
	}
	using rho::getw,rho::getd;
	using miller::prime;
}
using pr::getw,pr::getd,pr::prime;
\end{lstlisting}


\newpage

\section{字符串}

\subsection{字典树(trie 树)}



\begin{lstlisting}
struct trie
{
	const static int N=3e6+2, M=62;
	int c[N][M], sz[N];//sz 维护有多少个以当前字符串为前缀的字符串。
	int cnt;
	void insert(string s)
	{
		int u=0;
		++sz[u];
		for (char ch:s)
		{
			assert(ch>=0&&ch<M);
			int &v=c[u][ch];
			if (!v) v=++cnt;
			u=v;
			++sz[u];
		}
		//此时 u 是字符串结束位置。你可以在此存储结点信息。
	}
	int match(string s)//返回字符串结束位置。可能为 0。
	{
		int u=0;
		for (char ch:s)
		{
			assert(ch>=0&&ch<M);
			u=c[u][ch];
			if (!u) return 0;
		}
		return u;
	}
	void clear()
	{
		memset(c, 0, (cnt+1)*sizeof c[0]);
		memset(sz, 0, (cnt+1)*sizeof sz[0]);
		cnt=0;
	}
} s;
\end{lstlisting}

\subsection{AC 自动机}

注意 AC 自动机与 trie 不同的地方在于,根必须是 0。

题意:给你一个文本串 $S$ 和 $n$ 个模式串 $T_{1 \sim n}$,请你分别求出每个模式串 $T_i$ 在 $S$ 中出现的次数。

\begin{lstlisting}
struct AC
{
	const static int N=3e6+2, M=26;
	int c[N][M], sz[N], pos[N], f[N], app[N];//sz 维护有多少个以当前字符串为前缀的字符串。
	int cnt=0, id=0;
	vector<int> q;
	void insert(string s)
	{
		int u=0;
		++sz[u];
		for (char ch:s)
		{
			assert(ch>=0&&ch<M);
			int &v=c[u][ch];
			if (!v) v=++cnt;
			u=v;
			++sz[u];
		}
		pos[id++]=u;
		//此时 u 是字符串结束位置。你可以在此存储结点信息。
	}
	vector<int> match(string s)//返回答案。复杂度 O(结点数)
	{
		int u=0, i;
		for (char ch:s)
		{
			assert(ch>=0&&ch<M);
			u=c[u][ch];
			++app[u];
		}
		for (int u:q) app[f[u]]+=app[u];
		vector<int> r(id);
		for (i=0; i<id; i++) r[i]=app[pos[i]];
		memset(app, 0, (cnt+1)*sizeof app[0]);
		return r;
	}
	void clear()
	{
		memset(c, 0, (cnt+1)*sizeof c[0]);
		memset(f, 0, (cnt+1)*sizeof f[0]);
		memset(sz, 0, (cnt+1)*sizeof sz[0]);
		cnt=id=0;
	}
	void build()
	{
		q.clear();
		int ql=0;
		for (int i=0; i<M; i++) if (c[0][i]) q.push_back(c[0][i]);
		while (ql<q.size())
		{
			int u=q[ql++];
			for (int i=0; i<M; i++) if (c[u][i])
			{
				q.push_back(c[u][i]);
				f[c[u][i]]=c[f[u]][i];
			}
			else c[u][i]=c[f[u]][i];
		}
		reverse(all(q));
	}
} s;
int main()
{
	ios::sync_with_stdio(0); cin.tie(0);
	int n, i;
	cin>>n;
	while (n--)
	{
		string t;
		cin>>t;
		for (char &c:t) c-='a';
		s.insert(t);
	}
	s.build();
	string t;
	cin>>t;
	for (char &c:t) c-='a';
	auto res=s.match(t);
	for (int x:res) cout<<x<<'\n';
}

\end{lstlisting}

\subsection{hash}

在调试时,可以把 base 设置为 10 的幂方便输出。可能建议把第一个模数也设置为 1,但未测试是否有奇怪的问题。但要注意,此时不应当使用接近 10 的幂次的模数。

$O(n)$,$O(n)$。

双模数版本:注意使用的是无符号数,效率比 int128 高,但不卡常建议抄 int128 版本。

特别注意这里 m 数组预处理的不是幂次,而是幂次的相反数。如果有复杂的变换需要建议用 int128 版本。

其返回值是两个 32 位数拼接而成的,要改动比较麻烦。

\begin{lstlisting}
namespace sh
{
	typedef unsigned int ui;
	typedef unsigned long long ll;
	const int N=1e6+5;
	const ll p1=2'034'452'107, p2=2'013'074'419;
	struct pa
	{
		ll v1, v2;
		pa(ll v=0):v1(v), v2(v) { }
		pa(ll v1, ll v2):v1(v1), v2(v2) { }
		pa operator*(const pa &o) const { return {v1*o.v1%p1, v2*o.v2%p2}; }
	};
	pa fma(const pa &a, const pa &b, const pa &c) { return {(a.v1*b.v1+c.v1)%p1, (a.v2*b.v2+c.v2)%p2}; }
	const pa b={137, 149}, inv={1'603'801'661, 1'024'053'074};
	pa m[N];
	void init()
	{
		m[0]={p1-1, p2-1};
		for (int i=1; i<N; i++) m[i]=m[i-1]*b;
	}
	int i=(init(), 0);
	struct str
	{
		int n;
		vector<pa> a;
		template<class T> str(const vector<T> &s):n(s.size()), a(n+1)
		{
			for (i=0; i<n; i++) a[i+1]=fma(a[i], b, s[i]);
		}
		template<class T> str(const basic_string<T> &s):n(s.size()), a(n+1)//直接去掉模板换成 string 也可以
		{
			for (i=0; i<n; i++) a[i+1]=fma(a[i], b, s[i]);
		}
		ll getv(int l, int r)//[l,r)
		{
			auto [x, y]=fma(a[l], m[r-l], a[r]);
			return x<<32|y;
		}
	};
}
using sh::str;
int main()
{
	ios::sync_with_stdio(0); cin.tie(0);
	int T; cin>>T;
	set<ull> s;
	while (T--)
	{
		string t;
		cin>>t;
		s.insert(str(t).getv(0, t.size()));
	}
	cout<<s.size()<<endl;
}
\end{lstlisting}

\verb|__int128| 版本:

\begin{lstlisting}
namespace sh
{
	typedef __uint128_t lll;
	const int N=1e6+5;
	const lll p=1'80'143'985'094'819'841, b=137;
	lll m[N];
	void init()
	{
		m[0]=1;
		for (int i=1; i<N; i++) m[i]=m[i-1]*b%p;
	}
	int i=(init(), 0);
	struct str
	{
		int n;
		vector<lll> a;
		template<class T> str(const vector<T> &s):n(s.size()), a(n+1)
		{
			for (i=0; i<n; i++) a[i+1]=(a[i]*b+s[i])%p;
		}
		template<class T> str(const basic_string<T> &s):n(s.size()), a(n+1)//直接去掉模板换成 string 也可以
		{
			for (i=0; i<n; i++) a[i+1]=(a[i]*b+s[i])%p;
		}
		lll getv(int l, int r)//[l,r)
		{
			return (a[r]+(p-a[l])*m[r-l])%p;
		}
	};
}
using sh::str;
\end{lstlisting}

\subsection{KMP}

$O(n)$,$O(n)$。

\begin{lstlisting}
struct str
{
	vector<int> nxt,s;
	int n;
	str(int *S,int _n)//[1,n]
	{
		n=_n;
		nxt.resize(n+1);
		s=vector<int>(S,S+n+1);
		int i,j=0;
		nxt[1]=0;
		for (i=2;i<=n;i++)
		{
			while (j&&s[i]!=s[j+1]) j=nxt[j];
			nxt[i]=j+=s[i]==s[j+1];
		}
	}
	vector<int> match(int *t,int m)//find s(str) in t (start pos)
	{
		vector<int> r;
		int i,j=0;
		for (i=1;i<=m;i++)
		{
			while (j&&t[i]!=s[j+1]) j=nxt[j];
			if ((j+=t[i]==s[j+1])==n) j=nxt[j],r.push_back(i-n+1);
		}
		return r;
	}
};
int main()
{
	ios::sync_with_stdio(0);cin.tie(0);
	string s,t;
	cin>>s>>t;
	int n=s.size(),m=t.size(),i;
	vector<int> a(n+1),b(m+1);
	for (i=1;i<=n;i++) a[i]=s[i-1];
	for (i=1;i<=m;i++) b[i]=t[i-1];
	str q(b.data(),m);
	auto r=q.match(a.data(),n);
	for (int x:r) cout<<x<<'\n';
	for (i=1;i<=m;i++) cout<<q.nxt[i]<<" \n"[i==m];
}

\end{lstlisting}

\subsection{KMP(重构,未验证)}

$O(n)$,$O(n)$。

\begin{lstlisting}
struct str//[0,n)
{
	vector<int> nxt,s;
	int n;
	str(const vector<int> &_s):nxt(_s.size(),-1),s(all(_s)),n(_s.size())
	{
		int i,j=-1;
		for (i=1;i<n;i++)
		{
			while (j!=-1&&s[i]!=s[j+1]) j=nxt[j];
			nxt[i]=j+=s[i]==s[j+1];
		}
	}
	vector<int> match(const vector<int> &t)//find s(str) in t (start pos)
	{
		int m=t.size();
		vector<int> r;
		int i,j=-1;
		for (i=0;i<m;i++)
		{
			while (j!=-1&&t[i]!=s[j+1]) j=nxt[j];
			if ((j+=t[i]==s[j+1])==n-1) j=nxt[j],r.push_back(i-n+1);
		}
		return r;
	}
};

	
\end{lstlisting}

\subsection{manacher}

$O(n)$,$O(n)$。

\begin{lstlisting}
vector<int> manacher(const string &t)//ex[i](total length) centered at i/2
{
	string S="$#";
	int n=t.size(),i,r=1,m=0;
	for (i=0;i<n;i++) S+=t[i],S+='#';
	S+='#';
	char *s=S.data()+2;
	n=n*2-1;
	vector<int> ex(n);
	ex[0]=2;
	for (i=1;i<n;i++)
	{
		ex[i]=i<r?min(ex[m*2-i],r-i+1):1;
		while (s[i+ex[i]]==s[i-ex[i]]) ++ex[i];
		if (i+ex[i]-1>r) r=i+ex[m=i]-1;
	}
	for (i=0;i<n;i++) --ex[i];
	return ex;
}
\end{lstlisting}

\subsection{SA}

$O((n+\sum)\log n)$,$O(n+\sum)$。

功能:查询两个后缀的 lcp。单次询问复杂度 $O(1)$。

下标从 $0$ 开始。

\begin{lstlisting}
struct SA
{
	int n;
	vector<vector<int>> st;
	vector<int> sa, rk, h;
	int lcp(int x, int y)
	{
		if (x == y) return n - x;
		x = rk[x]; y = rk[y];
		if (x > y) swap(x, y);
		++x;
		int z = __lg(y - x + 1);
		return min(st[z][x], st[z][y - (1 << z) + 1]);
	}
	SA(vector<int> a) :n(a.size()), st(__lg(n) + 1, vector<int>(n + 1)), sa(n), h(n)
	{
		const static int N = 2e6 + 2;
		static int s[N];
		int i, j, m, cnt;
		m = *min_element(all(a));
		for (int &x : a) x -= m;
		m = *max_element(all(a)) + 1;
		assert(max(n, m) < N);
		a.resize(n * 2);
		for (i = 0; i < n; i++) a[i + n] = -i - 1;
		vector<int> id(n * 2);
		rk = a;
		for (i = 0; i < n; i++) ++s[a[i]];
		for (i = 1; i < m; i++) s[i] += s[i - 1];
		for (i = n - 1; i >= 0; i--) sa[--s[rk[i]]] = i;
		memset(s, 0, m * sizeof s[0]);
		for (j = 1; j <= n; j <<= 1)
		{
			cnt = 0;
			for (i = n - j; i < n; i++) id[cnt++] = i;
			for (i = 0; i < n; i++) if (sa[i] >= j) id[cnt++] = sa[i] - j;
			for (i = 0; i < n; i++) ++s[rk[i]];
			for (i = 1; i < m; i++) s[i] += s[i - 1];
			for (i = n - 1; i >= 0; i--) sa[--s[rk[id[i]]]] = id[i];
			id[sa[0]] = cnt = 0;
			memset(s, 0, m * sizeof s[0]);
			for (i = 1; i < n; i++)
				if (rk[sa[i]] == rk[sa[i - 1]] && rk[sa[i] + j] == rk[sa[i - 1] + j])
					id[sa[i]] = cnt;
				else
					id[sa[i]] = ++cnt;
			swap(rk, id);
			if ((m = cnt + 1) == n) break;
		}
		j = 0;
		for (i = 0; i < n; i++) if (rk[i])
		{
			cnt = sa[rk[i] - 1];
			while (a[i + j] == a[cnt + j]) ++j;
			h[rk[i]] = j;
			if (j) --j;
		}
		st[0] = h;
		for (j = 0; j < __lg(n); j++)
			for (i = 0, m = n - (1 << j + 1); i <= m; i++)
				st[j + 1][i] = min(st[j][i], st[j][i + (1 << j)]);
	}
};
\end{lstlisting}

\subsection{SAM}

$O(n\sum)$,$O(2n\sum )$。

\begin{lstlisting}
template<int M> struct sam//M:字符集大小
{
	vector<array<int,M>> c;
	vector<int> len,fa,ep;
	int np,cd;
	sam():c(2),len(2),fa(2),ep(2),np(1),cd(0) { }
	void insert(int ch)
	{
		int p=np,q,nq;
		np=c.size();
		len.push_back(++cd);
		fa.push_back(0);
		c.push_back({ });
		ep.push_back(cd);
		while (p&&!c[p][ch]) c[p][ch]=np,p=fa[p];
		if (!p)
		{
			fa[np]=1;
			return;
		}
		q=c[p][ch];
		if (len[q]==len[p]+1)
		{
			fa[np]=q;
			return;
		}
		nq=c.size();
		len.push_back(len[p]+1);
		c.push_back(c[q]);
		fa.push_back(fa[q]);
		ep.push_back(ep[q]);
		fa[np]=fa[q]=nq;
		c[p][ch]=nq;
		while (c[p=fa[p]][ch]==q) c[p][ch]=nq;
	}
	vector<int> match(const string &s)//返回每个前缀最长匹配长度
	{
		vector<int> r;
		r.reserve(s.size());
		int p=1,nl=0;
		for (auto ch:s)
		{
			if (c[p][ch]) ++nl,p=c[p][ch];
			else
			{
				while (p&&c[p][ch]==0) p=fa[p];
				if (p==0) p=1,nl=0; else nl=len[p]+1,p=c[p][ch];
			}
			r.push_back(nl);
		}
		return r;
	}
	array<int,3> max_match(const string &s)//返回长度,结尾(开)
	{
		array<int,3> r{0,0,0};
		int p=1,nl=0,i=0;
		for (auto ch:s)
		{
			if (c[p][ch]) ++nl,p=c[p][ch];
			else
			{
				while (p&&c[p][ch]==0) p=fa[p];
				if (p==0) p=1,nl=0; else nl=len[p]+1,p=c[p][ch];
			}
			cmax(r,array{nl,ep[p],i+1});
			++i;
		}
		if (r[0]==0) return { };
		return r;
	}
};
\end{lstlisting}

\subsection{ukkonen 后缀树}

$O(n)$,$O(2n\sum )$。

\begin{lstlisting}
void dfs(int x,int lf)
{
	if (!fir[x])
	{
		siz[x][1]=1;
		return;
	}
	int i,j;
	for (i=fir[x];i;i=nxt[i])
	{
		j=c[x][lj[i]];
		if ((f[j]<=m)&&(t[j]>=m)) ++siz[x][0];
		dfs(zd[j],t[j]-f[j]+1);
		siz[x][0]+=siz[zd[j]][0];
		siz[x][1]+=siz[zd[j]][1];
		if ((t[j]==n)&&(f[j]<=m)) --siz[x][1];
	}
	ans+=(ll)siz[x][0]*siz[x][1]*lf;
}
void add(int a,int b,int cc,int d)
{
	zd[++bbs]=b;
	t[bbs]=d;
	c[a][s[f[bbs]=cc]]=bbs;
}
void add(int x,int y)
{
	lj[++bs]=y;
	nxt[bs]=fir[x];
	fir[x]=bs;
}
	s[++m]=26;
	fa[1]=point=ds=1;
	for (i=1;i<=m;i++)
	{
		ad=0;++remain;
		while (remain)
		{
			if (r==0) edge=i;
			if ((j=c[point][s[edge]])==0)
			{
				fa[++ds]=1;
				fa[ad]=point;
				add(ad=point,ds,edge,m);
				add(point,s[edge]);
			}
			else
			{
				if ((t[j]!=m)&&(t[j]-f[j]+1<=r))
				{
					r-=t[j]-f[j]+1;
					edge+=t[j]-f[j]+1;
					point=zd[j];
					continue;
				}
				if (s[f[j]+r]==s[i]) {++r;fa[ad]=point;break;}
				fa[fa[ad]=++ds]=1;
				add(ad=ds,zd[j],f[j]+r,t[j]);
				add(ds,s[i]);add(ds,s[f[j]+r]);fa[++ds]=1;
				add(ds-1,ds,i,m);
				zd[j]=ds-1;t[j]=f[j]+r-1;
			}
			--remain;
			if ((r)&&(point==1))
			{
				--r;edge=i-remain+1;
			} else point=fa[point];
		}
	}
	for (i=1;i<=ds;i++) for (j=fir[i];j;j=nxt[j]) {len[j]=t[c[i][lj[j]]]-f[c[i][lj[j]]]+1;lj[j]=zd[c[i][lj[j]]];}
\end{lstlisting}

\subsection{ukkonen 后缀树(重构)}

\begin{lstlisting}
struct suffixtree
{
	const static int M=27;
	struct P
	{
		int v,w;
	};
	struct Q
	{
		int f,t,v;//t=0: n
	};
	vector<Q> edges;
	vector<vector<P>> e;
	vector<array<int,M>> c;
	vector<int> s,fa,dep,siz;
	int n,point,ds,remain,r,edge;
	bool bd;
	suffixtree():c(2),fa({0,1}),edges(1),e(2)
	{
		n=remain=r=edge=bd=0;
		point=ds=1;
	}
	suffixtree(const string &s):c(2),fa({0,1}),edges(1),e(2)
	{
		n=remain=r=edge=bd=0;
		point=ds=1;
		reserve(s.size());
		for (auto c:s) insert(c-'a');
		insert(26);
	}
	void reserve(int len)
	{
		++len;
		s.reserve(len);
		len=len*2+2;
		c.reserve(len);
		fa.reserve(len);
		e.reserve(len);
	}
	inline void add(int a,int b,int cc,int d)
	{
		assert(edges.size());
		c[a][s[cc]]=edges.size();
		edges.push_back({cc,d,b});
	}
	void insert(int ch)//[0,|S|)
	{
		assert(ds==fa.size()-1&&ds==c.size()-1&&n==s.size()&&ds==e.size()-1);
		assert(ch>=0&&ch<M);
		s.push_back(ch);
		int ad=0;
		++remain;
		while (remain)
		{
			if (!r) edge=n;
			if (int m=c[point][s[edge]];!m)
			{
				assert(!m);
				fa.push_back(1);c.push_back({});e.push_back({});
				fa[ad]=point;
				add(ad=point,++ds,edge,-1);
				e[point].push_back({s[edge]});
				//add(point,s[edge]);
			}
			else
			{
				assert(m);
				auto [f,t,v]=edges[m];
				if (t>=0&&t-f+1<=r)
				{
					assert(t!=n);
					r-=t-f+1;
					edge+=t-f+1;
					point=v;
					continue;
				}
				assert(f+r<=n);
				if (s[f+r]==s[n])
				{
					++r;
					fa[ad]=point;
					break;
				}
				fa.push_back(1);c.push_back({});e.push_back({});
				fa.push_back(1);c.push_back({});e.push_back({});
				fa[ad]=++ds;
				add(ad=ds,v,f+r,t);
				e[ds].push_back({s[n]});
				e[ds].push_back({s[f+r]});
				//add(ds,s[n]);add(ds,s[f+r]);
				++ds;add(ds-1,ds,n,-1); 
				edges[m]={f,f+r-1,ds-1};
			}
			--remain;
			if (r&&point==1)
			{
				--r;
				edge=n-remain+1;
			} else point=fa[point];
		}
		++n;
	}
	void build_edge()
	{
		bd=1;

		//其余信息
		dep.resize(ds+1);
		siz.resize(ds+1);

		int i,j;
		for (i=1;i<=ds;i++) for (auto &[v,w]:e[i])
		{
			j=c[i][v];
			v=edges[j].v;
			w=(edges[j].t>=0?edges[j].t:n-1)-edges[j].f+1;
		}
	}
	void out()
	{
		int i;
		for (i=1;i<=ds;i++) for (int j:c[i]) if (j)
		{
			auto [f,t,v]=edges[j];
			if (t==-1) t=n-1;
			cerr<<i<<' '<<v<<' ';
			//cerr<<i<<" -> "<<v<<": ";
			for (int k=f;k<=t;k++) cerr<<char('a'+s[k]);
			cerr<<endl;
		}
	}
	ll ans;
	void dfs(int u)
	{
		assert(bd);
		++ans;
		for (auto [v,w]:e[u])
		{
			//dep[v]=dep[u]+w;
			dfs(v);
			ans+=w-1;
		}
	}
	ll fun()
	{
		ans=0;
		build_edge();
		dfs(1);
		return ans-n;
	}
};
\end{lstlisting}



\subsection{Z 函数}

表示每个后缀和母串的 lcp。

\begin{lstlisting}
vector<int> Z(const string &s)
{
	int n=s.size(),i,l,r;
	vector<int> z(n);
	z[0]=n;
	for (i=1,l=r=0; i<n; i++)
	{
		if (i<=r&&z[i-l]<r-i+1) z[i]=z[i-l];
		else
		{
			z[i]=max(0,r-i+1);
			while (i+z[i]<n&&s[i+z[i]]==s[z[i]]) ++z[i];
		}
		if (i+z[i]-1>r) l=i,r=i+z[i]-1;
	}
	return z;
}
\end{lstlisting}



\subsection{最小表示法}

找到一个串的循环同构串中字典序最小的那个,将这个串直接变过去。常见应用:环哈希(基环树哈希)。

如果只需要找到起点下标,在 \verb|rotate| 前返回 $\min\{i,j\}$ 即可。

$O(n)$,$O(1)$。

\begin{lstlisting}
template<class T> void min_order(vector<T>& a)
{
	int n = a.size(), i, j, k;
	a.resize(n * 2);
	for (i = 0;i < n;i++) a[i + n] = a[i];
	i = k = 0;j = 1;
	while (i < n && j < n && k < n)
	{
		T x = a[i + k], y = a[j + k];
		if (x == y) ++k; else
		{
			(x > y ? i : j) += k + 1;
			j += (i == j);
			k = 0;
		}
	}
	a.resize(n);
	//[min(i,j),n)+[0,min(i,j))
	rotate(a.begin(), min(i, j) + all(a));
}
\end{lstlisting}
\subsection{带通配符的字符串匹配}

原理:匹配等价于 $\sum (f_i-g_i)^2=0$。带通配符等价于 $\sum f_ig_i(f_i-g_i)^2=0$,展开即可。

这里也是较为推荐的 NTT 版本,直接实现任意长度的多项式相乘,便于一般情况的运用。不需要提前做任何 init。

\begin{lstlisting}
namespace NTT
{
	typedef unsigned ui;
	typedef unsigned long long ll;
	const int N=1<<22;
	const ui p=998244353, g=3;
	inline ui ksm(ui x, ui y)
	{
		ui ans=1;
		while (y)
		{
			if (y&1) ans=1llu*ans*x%p;
			y>>=1; x=1llu*x*x%p;
		}
		return ans;
	}
	ui r[N], w[N];
	void ntt(vector<ui> &a)
	{
		int n=a.size(), i, j, k;
		for (i=0; i<n; i++) if (i<r[i]) swap(a[i], a[r[i]]);
		for (k=1; k<n; k<<=1)
		{
			for (i=0; i<n; i+=k<<1)
			{
				for (j=0; j<k; j++)
				{
					ui x=a[i+j], y=1llu*a[i+j+k]*w[j+k]%p;
					a[i+j]=(x+y)%p; a[i+j+k]=(x+p-y)%p;
				}
			}
		}
	}
	vector<ui> mul(vector <ui> a, vector <ui> b)
	{
		if (a.size()==0||b.size()==0) return { };
		int m=a.size()+b.size()-1;
		int n=1<<__lg(m*2-1);
		int i, j, base=__lg(n)-1;
		ui inv=ksm(n, p-2);
		for (i=1; i<n; i++) r[i]=r[i>>1]>>1|(i&1)<<base;
		for (j=1; j<n; j<<=1)
		{
			ui wn=ksm(3, (p-1)/(j<<1));
			w[j]=1;
			for (i=1; i<j; i++) w[j+i]=1llu*w[j+i-1]*wn%p;
		}
		a.resize(n); b.resize(n);
		ntt(a); ntt(b);
		for (i=0; i<n; i++) a[i]=1llu*a[i]*b[i]%p;
		ntt(a); reverse(1+all(a)); a.resize(n=m);
		for (i=0; i<n; i++) a[i]=1llu*a[i]*inv%p;
		return a;
	}
}
vector<int> match(const string &s, const string &t)
{
	using NTT::p, NTT::mul;
	static mt19937 rnd(chrono::steady_clock::now().time_since_epoch().count());
	static array<ui, 256> c;
	static bool inited=0;
	if (!inited)
	{
		inited=1;
		for (ui &x:c) x=rnd()%NTT::p;
		c['*']=0;//通配符
	}
	int n=s.size(), m=t.size(), i, j;
	if (n<m) return { };
	vector<int> ans;
	vector<ui> f(n), ff(n), fff(n), g(m), gg(m), ggg(m);
	for (i=0; i<n; i++)
	{
		f[i]=c[s[i]];
		ff[i]=1llu*f[i]*f[i]%p;
		fff[i]=1llu*ff[i]*f[i]%p;
	}
	for (i=0; i<m; i++)
	{
		g[i]=c[t[m-i-1]];
		gg[i]=1llu*g[i]*g[i]%p;
		ggg[i]=1llu*gg[i]*g[i]%p;
	}
	auto fffg=mul(fff, g), ffgg=mul(ff, gg), fggg=mul(f, ggg);
	for (i=0; i<=n-m; i++) if ((fffg[m-1+i]+fggg[m-1+i]+2*(NTT::p-ffgg[m-1+i]))%NTT::p==0) ans.push_back(i);
	return ans;
}

\end{lstlisting}

快一些的版本,手动拆开了多项式乘法。

\begin{lstlisting}
const int N=1<<22;
const ui p=998244353, g=3;
inline ui ksm(ui x, ui y)
{
	ui ans=1;
	while (y)
	{
		if (y&1) ans=1llu*ans*x%p;
		y>>=1; x=1llu*x*x%p;
	}
	return ans;
}
ui r[N], w[N];
void ntt(vector <ui> &a)
{
	int n=a.size(), i, j, k;
	for (k=1; k<n; k<<=1)
	{
		for (i=0; i<n; i+=k<<1)
		{
			for (j=0; j<k; j++)
			{
				ui x=a[i+j], y=1llu*a[i+j+k]*w[j+k]%p;
				a[i+j]=(x+y)%p; a[i+j+k]=(x+p-y)%p;
			}
		}
	}
}
vector<int> match(string s, string t, char ch='*')
{
	static mt19937 rnd(chrono::steady_clock::now().time_since_epoch().count());
	static array<ui, 256> c;
	static bool inited=0;
	if (!inited)
	{
		inited=1;
		for (ui &x:c) x=rnd()%p;
		// for (int i=0; i<256; i++) c[i]=i-96;
		c[ch]=0;//通配符
	}
	int n=s.size(), m=t.size(), i, j;
	if (n<m) return { };
	vector<int> ans;
	int N=1<<__lg(n*2-1), base=__lg(N)-1;
	vector<ui> f(N), ff(N), fff(N), g(N), gg(N), ggg(N);
	reverse(all(t));
	s.resize(N, ch), t.resize(N, ch);
	for (i=0; i<N; i++)
	{
		r[i]=r[i>>1]>>1|(i&1)<<base;
		if (i<r[i])
		{
			swap(s[i], s[r[i]]);
			swap(t[i], t[r[i]]);
		}
	}
	for (j=1; j<N; j<<=1)
	{
		ui wn=ksm(3, (p-1)/(j<<1));
		w[j]=1;
		for (i=1; i<j; i++) w[j+i]=1llu*w[j+i-1]*wn%p;
	}
	for (i=0; i<N; i++)
	{
		f[i]=c[s[i]];
		ff[i]=1llu*f[i]*f[i]%p;
		fff[i]=1llu*ff[i]*f[i]%p;
		g[i]=c[t[i]];
		gg[i]=1llu*g[i]*g[i]%p;
		ggg[i]=1llu*gg[i]*g[i]%p;
	}
	ntt(f); ntt(ff); ntt(fff); ntt(g); ntt(gg); ntt(ggg);
	for (i=0; i<N; i++) f[i]=(1llu*fff[i]*g[i]+1llu*f[i]*ggg[i]+2llu*(p-ff[i])*gg[i])%p;
	for (i=0; i<N; i++) if (i<r[i]) swap(f[i], f[r[i]]);
	ntt(f); reverse(1+all(f));
	for (i=0; i<=n-m; i++) if (f[m+i-1]==0) ans.push_back(i);
	return ans;
}
\end{lstlisting}
\newpage

\section{图论}

\subsection{最小密度环}

求所有环中边权和除以边数最少的,$O(nm)$。更常用的做法是二分 spfa。

\begin{lstlisting}
#include "bits/stdc++.h"
using namespace std;
const int N=3e3+5,M=1e4+5;
const double inf=1e18;
int u[M],v[M];
double f[N][N],w[M];
int main()
{
	ios::sync_with_stdio(0);cin.tie(0);
	cout<<setiosflags(ios::fixed)<<setprecision(8);
	int n,m,i,j;
	cin>>n>>m;
	for (i=1;i<=m;i++) cin>>u[i]>>v[i]>>w[i];
	++n;
	for (i=1;i<=n;i++)
	{
		fill_n(f[i]+1,n,inf);
		for (j=1;j<=m;j++) f[i][v[j]]=min(f[i][v[j]],f[i-1][u[j]]+w[j]);
	}
	double ans=inf;
	for (i=1;i<n;i++) if (f[n][i]!=inf)
	{
		double r=-inf;
		for (j=1;j<n;j++) r=max(r,(f[n][i]-f[j][i])/(n-j));
		ans=min(ans,r);
	}
	cout<<ans<<endl;
}
\end{lstlisting}

\subsection{全源最短路与判负环}

使用 floyd 实现全源最短路与判负环。注意边权较大时可能需要考虑 int128.

\begin{lstlisting}
#include "bits/stdc++.h"
using namespace std;
typedef long long ll;
typedef pair<int,int> pa;
typedef tuple<int,int,int> tp;
const int N=152;
const ll inf=5e8;
ll dis[N][N],d[N][N];
int main()
{
	ios::sync_with_stdio(0);cin.tie(0);
	while (1)
	{
		int n,m,q,i,j,k;
		cin>>n>>m>>q;
		if (tp(n,m,q)==tp(0,0,0)) return 0;
		for (i=0;i<n;i++) fill_n(dis[i],n,inf*inf);
		for (i=0;i<n;i++) dis[i][i]=0;
		while (m--)
		{
			int u,v,w;
			cin>>u>>v>>w;
			dis[u][v]=min(dis[u][v],(ll)w);
		}
		for (k=0;k<n;k++) for (i=0;i<n;i++) for (j=0;j<n;j++) dis[i][j]=max(min(dis[i][j],dis[i][k]+dis[k][j]),-inf*2);
		for (i=0;i<n;i++) copy_n(dis[i],n,d[i]);
		for (k=0;k<n;k++) for (i=0;i<n;i++) for (j=0;j<n;j++) dis[i][j]=max(min(dis[i][j],dis[i][k]+dis[k][j]),-inf*2);
		while (q--)
		{
			int u,v;
			cin>>u>>v;
			if (d[u][v]>inf) cout<<"Impossible\n"; else if (dis[u][v]!=d[u][v]||d[u][v]<-inf) cout<<"-Infinity\n"; else cout<<d[u][v]<<'\n';
		}
		cout<<'\n';
	}
}
\end{lstlisting}

\subsection{三/四元环计数}

不能处理有重边和自环的情况。

$O(m\sqrt m)$,$O(n+m)$。

注意四元环数的是边四元环。点四元环需要去掉四点完全图个数*2,似乎不太能做?

三元环是可以枚举的,你可以在 ans 改变处记录三元环 $(i,u,v)$。

\begin{lstlisting}
ll triple(const vector<pair<int,int>> &edges)//start from 0
{
	int n=0,i;
	for (auto [u,v]:edges) n=max({n,u,v});
	++n;
	vector<int> d(n),id(n),rk(n),cnt(n);
	vector<vector<int>> e(n);
	for (auto [u,v]:edges) ++d[u],++d[v];
	iota(all(id),0); sort(all(id),[&](int x,int y) { return d[x]<d[y]; });
	for (i=0; i<n; i++) rk[id[i]]=i;
	for (auto [u,v]:edges)
	{
		if (rk[u]>rk[v]) swap(u,v);
		e[u].push_back(v);
	}
	ll ans=0;
	for (i=0; i<n; i++)
	{
		for (int u:e[i]) cnt[u]=1;
		for (int u:e[i]) for (int v:e[u]) ans+=cnt[v];
		for (int u:e[i]) cnt[u]=0;
	}
	return ans;
}
ll quadruple(const vector<pair<int,int>> &edges)
{
	int n=0,i;
	for (auto [u,v]:edges) n=max({n,u,v});
	++n;
	vector<int> d(n),id(n),rk(n),cnt(n);
	vector<vector<int>> e(n),lk(n);
	for (auto [u,v]:edges) ++d[u],++d[v];
	iota(all(id),0); sort(all(id),[&](int x,int y) { return d[x]<d[y]; });
	for (i=0; i<n; i++) rk[id[i]]=i;
	for (auto [u,v]:edges)
	{
		if (rk[u]>rk[v]) swap(u,v);
		e[u].push_back(v);
		lk[u].push_back(v);
		lk[v].push_back(u);
	}
	ll ans=0;
	for (i=0; i<n; i++)
	{
		for (int u:lk[i]) for (int v:e[u]) if (rk[v]>rk[i]) ans+=cnt[v]++;
		for (int u:lk[i]) for (int v:e[u]) cnt[v]=0;
	}
	return ans;
}
map<pair<int, int>, ll> quadruple(vector<pair<int, int>> edges)
{
	int n = 0, i;
	for (auto [u, v] : edges) n = max({n, u, v});
	++n;
	map<pair<int, int>, int> ec;
	for (auto [u, v] : edges)
	{
		if (u > v) swap(u, v);
		++ec[{u, v}];
	}
	vector<ll> c;
	edges.clear();
	for (auto [_, cc] : ec) edges.push_back(_), c.push_back(cc);
	vector d(n, 0), id(d), rk(d);
	vector<ll> cnt(n);
	vector<vector<pair<int, int>>> e(n), lk(n);
	for (auto [u, v] : edges) ++d[u], ++d[v];
	iota(all(id), 0); sort(all(id), [&](int x, int y) { return d[x] < d[y]; });
	for (i = 0; i < n; i++) rk[id[i]] = i;
	i = 0;
	for (auto [u, v] : edges)
	{
		if (rk[u] > rk[v]) swap(u, v);
		e[u].push_back({v, i});
		lk[u].push_back({v, i});
		lk[v].push_back({u, i});
		++i;
	}
	int m = edges.size();
	vector<ll> ans(m);
	for (i = 0; i < n; i++)
	{
		for (auto [u, w1] : lk[i]) for (auto [v, w2] : e[u]) if (rk[v] > rk[i])
		{
			cnt[v] += c[w1] * c[w2];
		}
		for (auto [u, w1] : lk[i]) for (auto [v, w2] : e[u]) if (rk[v] > rk[i])
		{
			ans[w1] += (cnt[v] - c[w1] * c[w2]) * c[w2];
			ans[w2] += (cnt[v] - c[w1] * c[w2]) * c[w1];
		}
		for (auto [u, w1] : lk[i]) for (auto [v, w2] : e[u]) if (rk[v] > rk[i]) cnt[v] = 0;
	}
	map<pair<int, int>, ll> mp;
	for (i = 0;i < m;i++) mp[edges[i]] = ans[i];
	return mp;
}
int main()
{
	ios::sync_with_stdio(0); cin.tie(0);
	cout << fixed << setprecision(15);
	int n, m, i;
	cin >> n >> m;
	vector<pair<int, int>> eg(m);
	cin >> eg;
	auto mp = quadruple(eg);
	for (i = 0;i < m;i++)
	{
		auto [u, v] = eg[i];
		if (u > v) swap(u, v);
		cout << mp[{u, v}] << " \n"[i + 1 == m];
	}
}
\end{lstlisting}

\subsection{最短路系列}

Johnson 不适用于图中存在负环的情况,因为负环不一定是可以经过的。

$O(nm\log m)$,$O(n+m)$。

\begin{lstlisting}
vector<ll> spfa(const vector<vector<pair<int, ll>>> &e, int s)
{
	int n=e.size(), i;
	assert(n);
	queue<int> q;
	vector<int> len(n), ed(n);
	vector<ll> dis(n, inf);
	q.push(s); dis[s]=0;
	while (q.size())
	{
		int u=q.front(); q.pop();
		ed[u]=0;
		for (auto [v, w]:e[u]) if (cmin(dis[v], dis[u]+w))
		{
			len[v]=len[u]+1;
			if (len[v]>n) return { };
			if (!ed[v])
			{
				ed[v]=1;
				q.push(v);
			}
		}
	}
	return dis;
}
vector<ll> spfa(const vector<vector<pair<int, ll>>> &e)
{
	int n=e.size(), i;
	assert(n);
	queue<int> q;
	vector<int> len(n), ed(n, 1);
	vector<ll> dis(n);
	for (i=0; i<n; i++) q.push(i);
	while (q.size())
	{
		int u=q.front(); q.pop();
		ed[u]=0;
		for (auto [v, w]:e[u]) if (cmin(dis[v], dis[u]+w))
		{
			len[v]=len[u]+1;
			if (len[v]>n) return { };
			if (!ed[v])
			{
				ed[v]=1;
				q.push(v);
			}
		}
	}
	return dis;
}
vector<ll> dijk(const vector<vector<pair<int, ll>>> &e, int s)
{
	int n=e.size();
	using pa=pair<ll, int>;
	vector<ll> d(n, inf);
	vector<int> ed(n);
	priority_queue<pa, vector<pa>, greater<pa>> q;
	d[s]=0; q.push({0, s});
	while (q.size())
	{
		int u=q.top().second; q.pop();
		ed[u]=1;
		for (auto [v, w]:e[u]) if (cmin(d[v], d[u]+w)) q.push({d[v], v});
		while (q.size()&&ed[q.top().second]) q.pop();
	}
	return d;
}
vector<vector<ll>> dijk(const vector<vector<pair<int, ll>>> &e)
{
	vector<vector<ll>> r;
	for (int i=0; i<e.size(); i++) r.push_back(dijk(e, i));
	return r;
}
vector<vector<ll>> john(vector<vector<pair<int, ll>>> e)
{
	int n=e.size(), i, j;
	assert(n);
	auto h=spfa(e);
	if (!h.size()) return { };
	for (i=0; i<n; i++) for (auto &[v, w]:e[i]) w+=h[i]-h[v];
	auto r=dijk(e);
	for (i=0; i<n; i++) for (j=0; j<n; j++) if (r[i][j]!=inf) r[i][j]-=h[i]-h[j];
	return r;
}
\end{lstlisting}

\subsection{弦图}

单纯点:$v$ 和 $v$ 邻点构成团。

完美消除序列:$v_i$ 在 $\{v_i,v_{i+1},\cdots,v_n\}$ 为单纯点。

$N(v_i)=\{v_j|j>i\land (v_i,v_j)\in E\}$,$next(v_i)$ 为 $N(v_i)$ 最靠前的点。

极大团一定是 $\{v\}\cup N(v)$ 。

最大团大小等于色数。

弦图判定:等价于是否存在完美消除序列。首先求出一个完美消除序列,然后判定是否合法。

判定方法:设 $v_{i+1},\cdots,v_n$ 中与 $v_i$ 相邻的依次为 $v'_1,\cdots,v'_m$。只需判断是否 $v_1'$ 与 $v'_2,\cdots,v'_m$ 相邻。

LexBFS 算法(我不会写)

每个点有一个字符串 label,初始为 $0$。从 $i=n$ 到 $i=1$ 确定,选 label 字典序最大的 $u$,再把 $u$ 邻点的 label 后面接一个 $i$。

最大势算法:从 $v_n$ 求到 $v_1$,设 $label_i$ 表示 $i$ 与多少个已选点相邻,每次选 $label_i$ 最大的点。

弦图极大团:$\{v|\forall next(w)=v,|N(v)|\ge |N(w)|\}$。选出的集合为基本点,按上述极大团构造。

弦图染色:从 $v_n$ 到 $v_1$ 依次选最小可染的色。

最大独立集:从 $v_1$ 到 $v_n$ 能选就选。

最小团覆盖:设最大独立集为 $\{p_m\}$,最小团覆盖为 $\{\{p_i\}\cup N(p_i)\}$。

区间图:两个区间有边当且仅当交集非空。

区间图是弦图。

\subsubsection{代码}

\begin{lstlisting}
namespace chordal_graph//下标从 1 开始
{
	const int N=1e5+2;//点数
	bool ed[N];
	vector<int> e[N];
	int n;
	void init(const vector<pair<int,int>> &edges)
	{
		n=0;
		for (auto [u,v]:edges) n=max({n,u,v});
		for (int i=1;i<=n;i++) e[i].clear();
		for (auto [u,v]:edges) e[u].push_back(v),e[v].push_back(u);
	}
	vector<int> perfect_seq(const vector<pair<int,int>> &edges)//MCS
	{
		init(edges);
		static int d[N];
		static vector<int> buc[N];
		int i,mx=0;
		memset(d+1,0,n*sizeof d[0]);
		memset(ed+1,0,n*sizeof ed[0]);
		for (i=1;i<=n;i++) buc[i].clear();
		buc[0].resize(n);
		iota(all(buc[0]),1);
		vector<int> r(n);
		for (i=n-1;i>=0;i--)
		{
			int u=0;
			while (!u)
			{
				while (buc[mx].size()) if (ed[buc[mx].back()]) buc[mx].pop_back();
				else
				{
					ed[u=buc[mx].back()]=1;
					buc[mx].pop_back();
					goto yes;
				}
				--mx;
			}
			yes:;
			r[i]=u;
			for (int v:e[u]) if (!ed[v]) buc[++d[v]].push_back(v),mx=max(mx,d[v]);
		}
		return r;
	}
	bool check_perfect_seq(vector<int> a)
	{
		static bool ee[N];
		memset(ed+1,0,n*sizeof ed[0]);
		memset(ee+1,0,n*sizeof ee[0]);
		reverse(all(a));
		for (int u:a)
		{
			ed[u]=1;
			int w=0;
			for (int v:e[u]) if (ed[v]) {w=v;break;}
			if (!w) continue;
			ee[w]=1;
			for (int v:e[w]) ee[v]=1;
			for (int v:e[u]) if (ed[v]&&!ee[v]) return 0;
			ee[w]=0;
			for (int v:e[w]) ee[v]=0;
		}
		return 1;
	}
	bool check_chordal(const vector<pair<int,int>> &edges) {return check_perfect_seq(perfect_seq(edges));}
	vector<int> color(int _n,const vector<pair<int,int>> &edges)//返回长度为 _n+1。其中 0 无意义
	{
		auto a=perfect_seq(edges);
		reverse(all(a));
		memset(ed+1,0,n*sizeof ed[0]);
		vector<int> r(_n+1);
		for (int u:a)
		{
			for (int v:e[u]) ed[r[v]]=1;
			int x=1;
			while (ed[x]) ++x;
			r[u]=x;
			for (int v:e[u]) ed[r[v]]=0;
		}
        for (int i=n+1;i<=_n;i++) r[i]=1;
		return r;
	}
	vector<int> max_independent(int _n,const vector<pair<int,int>> &edges)//注意有孤立点这种奇怪东西
	{
		auto a=perfect_seq(edges);
		memset(ed+1,0,n*sizeof ed[0]);
		vector<int> r;
		for (int u:a) if (!ed[u])
		{
			r.push_back(u);
			for (int v:e[u]) ed[v]=1;
		}
		for (int i=n+1;i<=_n;i++) r.push_back(i);
		return r;
	}
}
using chordal_graph::check_chordal,chordal_graph::color,chordal_graph::max_independent;
\end{lstlisting}

\subsection{最小割树}

结论:两个点之间的最小割等于最小割树上两点间最小边权。

直接返回任意两点最小割。

\begin{lstlisting}
template<class T> vector<vector<T>> min_cut(int n, const vector<tuple<int, int, T>> &edges)//[0,n)
{
	int m=edges.size(), i, s, t, cnt=0;
	vector<int> fir(n, -1), nxt(m*2, -1), fc(n), q(n);
	vector<pair<int, T>> e(m*2);
	vector<tuple<T, int, int>> eg;
	auto add=[&](int u, int v, T w)
		{
			e[cnt]={v, w};
			nxt[cnt]=fir[u];
			fir[u]=cnt++;
		};
	for (auto [u, v, w]:edges) add(u, v, w), add(v, u, w);
	auto E=e;
	auto bfs=[&]()
		{
			fill(all(fc), 0);
			int ql=0, qr=0, u, i;
			fc[q[0]=s]=1;
			while (ql<=qr)
			{
				u=q[ql++];
				for (int i=fir[u]; i!=-1; i=nxt[i])
					if (auto &[v, w]=e[i]; w&&!fc[v]) fc[q[++qr]=v]=fc[u]+1;
			}
			return fc[t];
		};
	function<T(int, T)> dfs=[&](int u, T maxf)
		{
			if (u==t) return maxf;
			T j=0, k;
			for (int i=fir[u]; i!=-1; i=nxt[i])
				if (auto &[v, w]=e[i]; w&&fc[v]==fc[u]+1&&(k=dfs(v, min(maxf-j, w))))
				{
					j+=k;
					w-=k;
					e[i^1].second+=k;
					if (j==maxf) return j;
				}
			fc[u]=0;
			return j;
		};
	function<void(vector<int>)> solve=[&](vector<int> id)
		{
			static mt19937 rnd(chrono::steady_clock::now().time_since_epoch().count());
			if (id.size()<=1) return;
			vector<int> u(2);
			sample(all(id), u.begin(), 2, rnd);
			s=u[0], t=u[1], e=E;
			T ans=0;
			while (bfs()) ans+=dfs(s, numeric_limits<T>::max());
			auto it=partition(all(id), [&](int u) { return fc[u]; });
			eg.emplace_back(ans, s, t);
			solve(vector(id.begin(), it));
			solve(vector(it, id.end()));
		};
	solve(range(0, n));
	sort(all(eg), greater<>());
	vector<basic_string<int>> ver(n);
	vector ans(n, vector<T>(n));
	vector<int> f(n);
	for (i=0; i<n; i++) ver[i]={f[i]=i};
	function<int(int)> getf=[&](int u) { return f[u]==u?u:f[u]=getf(f[u]); };
	for (auto [w, u, v]:eg)
	{
		u=getf(u);
		v=getf(v);
		for (int w1:ver[u]) for (int w2:ver[v]) ans[w1][w2]=ans[w2][w1]=w;
		ver[u]+=ver[v];
		f[v]=u;
	}
	return ans;
}
\end{lstlisting}

\subsection{二分图与网络流建图}

以下约定,若为二分图则 $n,m$ 表示两侧点数,否则仅 $n$ 表示全图点数。

\subsubsection{二分图边染色}

留坑待填。

结论:$\Delta(G)\le \chi'(G) \le \Delta(G)+1$,二分图时 $\chi'(G)=\Delta(G)$。$\Delta(G)$ 为图的最大度。

\subsubsection{二分图最小点集覆盖}

$ans=\text{maxmatch}$,方案如下。

\begin{lstlisting}
#include "bits/stdc++.h"
using namespace std;
const int N=5e3+2;
vector<int> e[N];
int ed[N],lk[N],kl[N],flg[N],now;
bool dfs(int u)
{
	for (int v:e[u]) if (ed[v]!=now)
	{
		ed[v]=now;
		if (!lk[v]||dfs(lk[v])) return lk[v]=u;
	}
	return 0;
}
void dfs2(int u)
{
	for (int v:e[u]) if (!flg[v]) flg[v]=1,dfs2(lk[v]);
}
int main()
{
	int n,m,i,r=0;
	cin>>n>>m;
	while (m--)
	{
		int u,v;
		cin>>u>>v;
		e[u].push_back(v);
	}
	for (i=1;i<=n;i++) dfs(now=i);
	for (i=1;i<=n;i++) kl[lk[i]]=i;
	for (i=1;i<=n;i++) if (!kl[i]) dfs2(i);
	vector<int> A[2];
	for (i=1;i<=n;i++) if (lk[i])
	{
		if (flg[i]) A[1].push_back(i); else A[0].push_back(lk[i]);
	}
	for (int j=0;j<2;j++)
	{
		cout<<A[j].size();
		for (int x:A[j]) cout<<' '<<x;cout<<'\n';
	}
}
\end{lstlisting}

\subsubsection{二分图最大独立集}

$ans=n+m-\text{maxmatch}$,方案是最小点集覆盖的补集。

\subsubsection{二分图最小边覆盖}

$ans=n+m-\text{maxmatch}$,方案是最大匹配加随便一些边(用于覆盖失配点)。无解当且仅当有孤立点,算法会视为单选孤立点(无边)。这个定理对一般图也成立。

\subsubsection{有向无环图最小不相交链覆盖}

$ans=n-\text{maxmatch}$,其中二分图建图方法是拆入点和出点(实现时直接跑一次二分图就行,不用额外处理),注意\textbf{不}需要传递闭包。方案如下。

\begin{lstlisting}
#include "bits/stdc++.h"
using namespace std;
const int N=152;
vector<int> e[N];
int lk[N],kl[N],ed[N],now;
bool dfs(int u)
{
	for (int v:e[u]) if (ed[v]!=now)
	{
		ed[v]=now;
		if (!lk[v]||dfs(lk[v])) return lk[v]=u;
	}
	return 0;
}
int main()
{
	int n,m,i;
	ios::sync_with_stdio(0);cin.tie(0);
	cin>>n>>m;
	while (m--)
	{
		int u,v;
		cin>>u>>v;
		e[u].push_back(v);
	}
	int r=0;
	for (i=1;i<=n;i++) r+=dfs(now=i);
	for (i=1;i<=n;i++) kl[lk[i]]=i;
	for (i=1;i<=n;i++) if (ed[i]!=-1&&!lk[i])
	{
		vector<int> ans;
		int u=i;
		while (u)
		{
			ed[u]=-1;
			ans.push_back(u);
			u=kl[u];
		}
		for (int j=0;j<ans.size();j++) cout<<ans[j]<<" \n"[j+1==ans.size()];
	}
	cout<<n-r<<endl;
}
\end{lstlisting}

\subsubsection{有向无环图最大互不可达集}

$ans=n-\text{maxmatch}$,其中二分图建图方法是拆入点和出点(实现时直接跑一次二分图就行,不用额外处理),注意\textbf{需要}传递闭包。方案?

\subsubsection{最大权闭合子图}

若 $v_i>0$,$s\to i$ 流量 $v_i$;若 $v_i<0$,$i\to t$ 流量 $-v_i$。若原图 $u\to v$ 可花费 $w$ 代价违抗,流量 $w$,否则 $+\infty$ 。答案为 $\sum\limits_{v_i>0} v_i-\text{maxflow}$。方案?

\subsection{二分图最大权匹配}

\begin{lstlisting}
namespace KM
{
	const int N=405;//点数
	typedef long long ll;//答案范围
	const ll inf=1e16;
	int lk[N],kl[N],pre[N],q[N],n,h,t;
	ll sl[N],e[N][N],lx[N],ly[N];
	bool edx[N],edy[N];
	bool ck(int v)
	{
		if (edy[v]=1,kl[v]) return edx[q[++t]=kl[v]]=1;
		while (v) swap(v,lk[kl[v]=pre[v]]);
		return 0;
	}
	void bfs(int u)
	{
		fill_n(sl+1,n,inf);
		memset(edx+1,0,n*sizeof edx[0]);
		memset(edy+1,0,n*sizeof edy[0]);
		q[h=t=1]=u;edx[u]=1;
		while (1)
		{
			while (h<=t)
			{
				int u=q[h++],v;
				ll d;
				for (v=1;v<=n;v++) if (!edy[v]&&sl[v]>=(d=lx[u]+ly[v]-e[u][v])) if (pre[v]=u,d) sl[v]=d; else if (!ck(v)) return;
			}
			int i;
			ll m=inf;
			for (i=1;i<=n;i++) if (!edy[i]) m=min(m,sl[i]);
			for (i=1;i<=n;i++)
			{
				if (edx[i]) lx[i]-=m;
				if (edy[i]) ly[i]+=m; else sl[i]-=m;
			}
			for (i=1;i<=n;i++) if (!edy[i]&&!sl[i]&&!ck(i)) return;
		}
	}
	template<class TT> ll max_weighted_match(int N,const vector<tuple<int,int,TT>> &edges)//lk[[1,n]]->[1,n]
	{
		int i;n=N;
		memset(lk+1,0,n*sizeof lk[0]);
		memset(kl+1,0,n*sizeof kl[0]);
		memset(ly+1,0,n*sizeof ly[0]);
		for (i=1;i<=n;i++) fill_n(e[i]+1,n,0);//若不需保证匹配边最多,置 0 即可,否则 -inf/N
		for (auto [u,v,w]:edges) e[u][v]=max(e[u][v],(ll)w);
		for (i=1;i<=n;i++) lx[i]=*max_element(e[i]+1,e[i]+n+1);
		for (i=1;i<=n;i++) bfs(i);
		ll r=0;
		for (i=1;i<=n;i++) r+=e[i][lk[i]];
		return r;
	}
}
using KM::max_weighted_match,KM::lk,KM::kl,KM::e;
\end{lstlisting}

\subsection{一般图最大匹配}

\begin{lstlisting}
namespace blossom_tree
{
	const int N=1005;
	vector<int> e[N];
	int lk[N],rt[N],f[N],dfn[N],typ[N],q[N];
	int id,h,t,n;
	int lca(int u,int v)
	{
		++id;
		while (1)
		{
			if (u)
			{
				if (dfn[u]==id) return u;
				dfn[u]=id;u=rt[f[lk[u]]];
			}
			swap(u,v);
		}
	}
	void blm(int u,int v,int a)
	{
		while (rt[u]!=a)
		{
			f[u]=v;
			v=lk[u];
			if (typ[v]==1) typ[q[++t]=v]=0;
			rt[u]=rt[v]=a;
			u=f[v];
		}
	}
	void aug(int u)
	{
		while (u)
		{
			int v=lk[f[u]];
			lk[lk[u]=f[u]]=u;
			u=v;
		}
	}
	void bfs(int root)
	{
		memset(typ+1,-1,n*sizeof typ[0]);
		iota(rt+1,rt+n+1,1);
		typ[q[h=t=1]=root]=0;
		while (h<=t)
		{
			int u=q[h++];
			for (int v:e[u])
			{
				if (typ[v]==-1)
				{
					typ[v]=1;f[v]=u;
					if (!lk[v]) return aug(v);
					typ[q[++t]=lk[v]]=0;
				} else if (!typ[v]&&rt[u]!=rt[v])
				{
					int a=lca(rt[u],rt[v]);
					blm(v,u,a);blm(u,v,a);
				}
			} 
		}
	}
	int max_general_match(int N,vector<pair<int,int>> edges)//[1,n]
	{
		n=N;id=0;
		memset(f+1,0,n*sizeof f[0]);
		memset(dfn+1,0,n*sizeof dfn[0]);
		memset(lk+1,0,n*sizeof lk[0]);
		int i;
		for (i=1;i<=n;i++) e[i].clear();
		mt19937 rnd(114);
		shuffle(all(edges),rnd);
		for (auto [u,v]:edges)
		{
			e[u].push_back(v),e[v].push_back(u);
			if (!(lk[u]||lk[v])) lk[u]=v,lk[v]=u;
		}
		int r=0;
		for (i=1;i<=n;i++) if (!lk[i]) bfs(i);
		for (i=1;i<=n;i++) r+=!!lk[i];
		return r/2;
	}
}
using blossom_tree::max_general_match,blossom_tree::lk;
\end{lstlisting}

\subsection{一般图最大权匹配}

$n=400$:UOJ 600ms, Luogu 135ms

\begin{lstlisting}
#include"bits/stdc++.h"
using namespace std;
#define all(x) (x).begin(),(x).end()
namespace weighted_blossom_tree
{
	#define d(x) (lab[x.u]+lab[x.v]-e[x.u][x.v].w*2)
	const int N=403*2;//两倍点数
	typedef long long ll;//总和大小
	typedef int T;//权值大小
	//均不允许无符号
	const T inf=numeric_limits<int>::max()>>1;
	struct Q
	{
		int u,v;
		T w;
	} e[N][N];
	T lab[N];
	int n,m=0,id,h,t,lk[N],sl[N],st[N],f[N],b[N][N],s[N],ed[N],q[N];
	vector<int> p[N];
	void upd(int u,int v) {if (!sl[v]||d(e[u][v])<d(e[sl[v]][v])) sl[v]=u;}
	void ss(int v)
	{
		sl[v]=0;
		for (int u=1;u<=n;u++) if (e[u][v].w>0&&st[u]!=v&&!s[st[u]]) upd(u,v);
	}
	void ins(int u) {if (u<=n) q[++t]=u; else for (int v:p[u]) ins(v);}
	void mdf(int u,int w)
	{
		st[u]=w;
		if (u>n) for (int v:p[u]) mdf(v,w);
	}
	int gr(int u,int v)
	{
		if ((v=find(all(p[u]),v)-p[u].begin())&1)
		{
			reverse(1+all(p[u]));
			return (int)p[u].size()-v;
		}
		return v;
	}
	void stm(int u,int v)
	{
		lk[u]=e[u][v].v;
		if (u<=n) return;
		Q w=e[u][v];
		int x=b[u][w.u],y=gr(u,x),i;
		for (i=0;i<y;i++) stm(p[u][i],p[u][i^1]);
		stm(x,v);
		rotate(p[u].begin(),y+all(p[u]));
	}
	void aug(int u,int v)
	{
		int w=st[lk[u]];
		stm(u,v);
		if (!w) return;
		stm(w,st[f[w]]);
		aug(st[f[w]],w);
	}
	int lca(int u,int v)
	{
		for (++id;u|v;swap(u,v))
		{
			if (!u) continue;
			if (ed[u]==id) return u;
			ed[u]=id;//??????????v?? 这是原作者的注释,我也不知道是啥
			if (u=st[lk[u]]) u=st[f[u]];
		}
		return 0;
	}
	void add(int u,int a,int v)
	{
		int x=n+1,i,j;
		while (x<=m&&st[x]) ++x;
		if (x>m) ++m;
		lab[x]=s[x]=st[x]=0;lk[x]=lk[a];
		p[x].clear();p[x].push_back(a);
		for (i=u;i!=a;i=st[f[j]]) p[x].push_back(i),p[x].push_back(j=st[lk[i]]),ins(j);//复制,改一处
		reverse(1+all(p[x]));
		for (i=v;i!=a;i=st[f[j]]) p[x].push_back(i),p[x].push_back(j=st[lk[i]]),ins(j);
		mdf(x,x);
		for (i=1;i<=m;i++) e[x][i].w=e[i][x].w=0;
		memset(b[x]+1,0,n*sizeof b[0][0]);
		for (int u:p[x])
		{
			for (v=1;v<=m;v++) if (!e[x][v].w||d(e[u][v])<d(e[x][v])) e[x][v]=e[u][v],e[v][x]=e[v][u];
			for (v=1;v<=n;v++) if (b[u][v]) b[x][v]=u;
		}
		ss(x);
	}
	void ex(int u)  // s[u] == 1
	{
		for (int x:p[u]) mdf(x,x);
		int a=b[u][e[u][f[u]].u],r=gr(u,a),i;
		for (i=0;i<r;i+=2)
		{
			int x=p[u][i],y=p[u][i+1];
			f[x]=e[y][x].u;
			s[x]=1;s[y]=0;
			sl[x]=0;ss(y);
			ins(y);
		}
		s[a]=1;f[a]=f[u];
		for (i=r+1;i<p[u].size();i++) s[p[u][i]]=-1,ss(p[u][i]);
		st[u]=0;
	}
	bool on(const Q &e)
	{
		int u=st[e.u],v=st[e.v],a;
		if(s[v]==-1)
		{
			f[v]=e.u;s[v]=1;
			a=st[lk[v]];
			sl[v]=sl[a]=s[a]=0;
			ins(a);
		}
		else if(!s[v])
		{
			a=lca(u,v);
			if (!a) return aug(u,v),aug(v,u),1;
			else add(u,a,v);
		}
		return 0;
	}
	bool bfs()
	{
		memset(s+1,-1,m*sizeof s[0]);
		memset(sl+1,0,m*sizeof sl[0]);
		h=1;t=0;
		int i,j;
		for (i=1;i<=m;i++) if (st[i]==i&&!lk[i]) f[i]=s[i]=0,ins(i);
		if (h>t) return 0;
		while (1)
		{
			while (h<=t)
			{
				int u=q[h++],v;
				if (s[st[u]]!=1) for (v=1; v<=n;v++) if (e[u][v].w>0&&st[u]!=st[v])
				{
					if (d(e[u][v])) upd(u,st[v]); else if (on(e[u][v])) return 1;
				}
			}
			T x=inf;
			for (i=n+1;i<=m;i++) if (st[i]==i&&s[i]==1) x=min(x,lab[i]>>1);
			for (i=1;i<=m;i++) if (st[i]==i&&sl[i]&&s[i]!=1) x=min(x,d(e[sl[i]][i])>>s[i]+1);
			for (i=1;i<=n;i++) if (~s[st[i]]) if ((lab[i]+=(s[st[i]]*2-1)*x)<=0) return 0;
			for (i=n+1;i<=m;i++) if (st[i]==i&&~s[st[i]]) lab[i]+=(2-s[st[i]]*4)*x;
			h=1;t=0;
			for (i=1;i<=m;i++) if (st[i]==i&&sl[i]&&st[sl[i]]!=i&&!d(e[sl[i]][i])&&on(e[sl[i]][i])) return 1;
			for (i=n+1;i<=m;i++) if (st[i]==i&&s[i]==1&&!lab[i]) ex(i);
		}
		return 0;
	}
	template<class TT> ll max_weighted_general_match(int N,const vector<tuple<int,int,TT>> &edges)//[1,n],返回权值
	{
		memset(ed+1,0,m*sizeof ed[0]);
		memset(lk+1,0,m*sizeof lk[0]);
		n=m=N;id=0;
		iota(st+1,st+n+1,1);
		int i,j;
		T wm=0;
		ll r=0;
		for (i=1;i<=n;i++) for (j=1;j<=n;j++) e[i][j]={i,j,0};
		for (auto [u,v,w]:edges) wm=max(wm,e[v][u].w=e[u][v].w=max(e[u][v].w,(T)w));
		for (i=1;i<=n;i++) p[i].clear();
		for (i=1;i<=n;i++) for (j=1;j<=n;j++) b[i][j]=i*(i==j);
		fill_n(lab+1,n,wm);
		while (bfs());
		for (i=1;i<=n;i++) if (lk[i]) r+=e[i][lk[i]].w;
		return r/2;
	}
	#undef d
}
using weighted_blossom_tree::max_weighted_general_match,weighted_blossom_tree::lk;
int main()
{
	ios::sync_with_stdio(0);cin.tie(0);
	int n,m;
	cin>>n>>m;
	vector<tuple<int,int,long long>> edges(m);
	for (auto &[u,v,w]:edges) cin>>u>>v>>w;
	cout<<max_weighted_general_match(n,edges)<<'\n';
	for (int i=1;i<=n;i++) cout<<lk[i]<<" \n"[i==n];
}
\end{lstlisting}

\subsection{网络流代码}

\begin{lstlisting}
namespace net
{
	const int N = 4e5 + 50;//number of nodes
	namespace flow
	{
		const ll inf = 4e18;
		struct Q
		{
			int v;
			ll w;
			int id;
		};
		vector<Q> e[N];
		vector<Q>::iterator fir[N];
		int fc[N], q[N];
		int n, s, t;
		int bfs()
		{
			for (int i = 0;i < n;i++)
			{
				fir[i] = e[i].begin();
				fc[i] = 0;
			}
			int p1 = 0, p2 = 0, u;
			fc[s] = 1; q[0] = s;
			while (p1 <= p2)
			{
				int u = q[p1++];
				for (auto [v, w, id] : e[u]) if (w && !fc[v])
				{
					q[++p2] = v;
					fc[v] = fc[u] + 1;
				}
			}
			return fc[t];
		}
		ll dfs(int u, ll maxf)
		{
			if (u == t) return maxf;
			ll j = 0, k;
			for (auto& it = fir[u];it != e[u].end();++it)
			{
				auto& [v, w, id] = *it;
				if (w && fc[v] == fc[u] + 1 && (k = dfs(v, min(maxf - j, w))))
				{
					j += k;
					w -= k;
					e[v][id].w += k;
					if (j == maxf) return j;
				}
			}
			fc[u] = 0;
			return j;
		}
		ll max_flow(int _n, const vector<tuple<int, int, ll>>& edges, int _s, int _t)//[0,n]
		{
			s = _s; t = _t; n = _n + 1;
			for (int i = 0; i < n; i++) e[i].clear();
			for (auto [u, v, w] : edges) if (u != v)
			{
				e[u].push_back({v, w, (int)e[v].size()});
				e[v].push_back({u, 0, (int)e[u].size() - 1});
			}
			ll r = 0;
			while (bfs()) r += dfs(s, inf);
			return r;
		}
	}
	using flow::max_flow, flow::fc;
	namespace match
	{
		int lk[N], kl[N], ed[N];
		vector<int> e[N];
		int max_match(int n, int m, const vector<pair<int, int>>& edges)//lk[[0,n]]->[0,m]
		{
			++n; ++m;
			int s = n + m, t = n + m + 1, i;
			vector<tuple<int, int, ll>> eg;
			eg.reserve(n + m + edges.size());
			for (i = 0; i < n; i++) eg.push_back({s, i, 1});
			for (i = 0; i < m; i++) eg.push_back({i + n, t, 1});
			for (auto [u, v] : edges) eg.push_back({u, v + n, 1});
			int r = max_flow(t, eg, s, t);
			fill_n(lk, n, -1);
			for (i = 0; i < n; i++) for (auto [v, w, id] : flow::e[i]) if (v < s && !w)
			{
				lk[i] = v - n;
				break;
			}
			return r;
		}
		void dfs(int u)
		{
			for (int v : e[u]) if (!ed[v]) ed[v] = 1, dfs(kl[v]);
		}
		pair<vector<int>, vector<int>> min_cover(int n, int m, const vector<pair<int, int>>& edges)//[0,n]-[0,m]
		{
			max_match(n, m, edges);
			++n; ++m;
			fill_n(kl, m, -1); fill_n(ed, m, 0);
			int i;
			for (i = 0; i < n; i++)
			{
				e[i].clear();
				if (lk[i] != -1) kl[lk[i]] = i;
			}
			for (auto [u, v] : edges) e[u].push_back(v);
			for (i = 0; i < n; i++) if (lk[i] == -1) dfs(i);
			vector<int> r[2];
			for (i = 0; i < m; i++) if (kl[i] != -1)
			{
				if (ed[i]) r[1].push_back(i); else r[0].push_back(kl[i]);
			}
			sort(all(r[0]));
			return {r[0], r[1]};
		}
	}
	using match::max_match, match::min_cover, match::lk, match::kl;
	namespace cost_flow
	{
		const ll inf = 4e18;
		struct Q
		{
			int v;
			ll w, c;
			int id;
		};
		vector<Q> e[N];
		ll dis[N];
		int pre[N], pid[N], ipd[N];
		bool ed[N];
		int n, s, t;
		pair<ll, ll> spfa()
		{
			queue<int> q;
			fill_n(dis, n, inf);
			memset(ed, 0, n * sizeof ed[0]);
			q.push(s); dis[s] = 0;
			while (q.size())
			{
				int u = q.front(); q.pop(); ed[u] = 0;
				for (auto [v, w, c, id] : e[u]) if (w && dis[v] > dis[u] + c)
				{
					dis[v] = dis[u] + c;
					pre[v] = u;
					pid[v] = e[v][id].id;
					ipd[v] = id;
					if (!ed[v]) q.push(v), ed[v] = 1;
				}
			}
			if (dis[t] == inf) return {0, 0};
			ll mw = 9e18;
			for (int i = t; i != s; i = pre[i]) mw = min(mw, e[pre[i]][pid[i]].w);
			for (int i = t; i != s; i = pre[i]) e[pre[i]][pid[i]].w -= mw, e[i][ipd[i]].w += mw;
			return {mw, mw * dis[t]};
		}
		pair<ll, ll> mcmf_spfa(int _n, const vector<tuple<int, int, ll, ll>>& edges, int _s, int _t)//[0,n]
		{
			s = _s; t = _t; n = _n + 1;
			for (int i = 0; i < n; i++) e[i].clear();
			for (auto [u, v, w, c] : edges) if (u != v)
			{
				e[u].push_back({v, w, c, (int)e[v].size()});
				e[v].push_back({u, 0, -c, (int)e[u].size() - 1});
			}
			pair<ll, ll> r{0, 0}, rr;
			while ((rr = spfa()).first) r = {r.first + rr.first, r.second + rr.second};
			return r;
		}
		pair<ll, ll> mcmf_dijk(int _n, const vector<tuple<int, int, ll, ll>>& edges, int _s, int _t)//[0,n]
		{
			s = _s; t = _t; n = _n + 1;
			for (int i = 0; i < n; i++) e[i].clear();
			for (auto [u, v, w, c] : edges) if (u != v)
			{
				e[u].push_back({v, w, c, (int)e[v].size()});
				e[v].push_back({u, 0, -c, (int)e[u].size() - 1});
			}
			static ll h[N];
			auto get_h = [&]()
				{
					fill_n(h, n, inf);
					memset(ed, 0, n * sizeof ed[0]);
					queue<int> q;
					q.push(s); h[s] = 0;
					while (q.size())
					{
						int u = q.front(); q.pop(); ed[u] = 0;
						for (auto [v, w, c, id] : e[u]) if (w && h[v] > h[u] + c)
						{
							h[v] = h[u] + c;
							if (!ed[v]) q.push(v), ed[v] = 1;
						}
					}
					return;
				};
			auto dijkstra = [&]() -> pair<ll, ll>
				{
					static int fl[N], zl[N];
					int i;
					memset(ed, 0, n * sizeof ed[0]);
					fill_n(dis, n, inf);
					typedef pair<ll, int> pa;
					priority_queue<pa, vector<pa>, greater<pa>> q;
					dis[s] = 0; q.push({0, s});
					while (q.size())
					{
						int u = q.top().second;
						q.pop(); ed[u] = 1;
						i = 0;
						for (auto [v, w, c, id] : e[u])
						{
							if (w && dis[v] > dis[u] + c) fl[v] = id, zl[v] = i, q.push({dis[v] = dis[pre[v] = u] + c, v});
							++i;
						}
						while (q.size() && ed[q.top().second]) q.pop();
					}
					if (dis[t] == inf) return {0, 0};
					ll tf = numeric_limits<ll>::max();
					for (i = t; i != s; i = pre[i]) tf = min(tf, e[pre[i]][zl[i]].w);
					for (i = t; i != s; i = pre[i]) e[pre[i]][zl[i]].w -= tf, e[i][fl[i]].w += tf;
					for (int u = 0; u < n; u++) for (auto& [v, w, c, id] : e[u]) c += dis[u] - dis[v];
					return {tf, tf * (h[t] += dis[t])};
				};
			get_h();
			for (int u = 0; u < n; u++) for (auto& [v, w, c, id] : e[u]) c += h[u] - h[v];
			pair<ll, ll> r{0, 0}, rr;
			while ((rr = dijkstra()).first) r = {r.first + rr.first, r.second + rr.second};
			return r;
		}
	}
	using cost_flow::mcmf_spfa, cost_flow::mcmf_dijk;
	namespace bounded_flow
	{
		bool valid_flow(int n, const vector<tuple<int, int, ll, ll>>& edges)//方案需加上 l
		{
			if (!edges.size()) return 1;
			++n;
			int i;
			ll tot = 0;
			static ll cd[N];
			memset(cd, 0, n * sizeof cd[0]);
			for (auto [u, v, l, r] : edges) cd[u] += l, cd[v] -= l;
			vector<tuple<int, int, ll>> eg;
			eg.reserve(n + edges.size());
			for (i = 0; i < n; i++) if (cd[i] > 0) eg.push_back({i, n + 1, cd[i]}), tot += cd[i];
			else if (cd[i] < 0) eg.push_back({n, i, -cd[i]});
			for (auto [u, v, l, r] : edges) eg.push_back({u, v, r - l});
			return tot == flow::max_flow(n + 1, eg, n, n + 1);
		}
		ll valid_flow_st(int n, vector<tuple<int, int, ll, ll>> edges, int s, int t)//-1 invalid, ll=ll
		{
			ll tot = 0;
			for (auto [u, v, l, r] : edges) tot += (u == s) * r;
			edges.push_back({t, s, 0, tot});
			if (!valid_flow(n, edges)) return -1;
			assert(flow::e[s].back().v == t);
			assert(flow::e[t].back().v == s);
			return tot - flow::e[t].back().w;
		}
		ll valid_max_flow(int n, const vector<tuple<int, int, ll, ll>>& edges, int s, int t)//-1 invalid, ll=ll
		{
			ll r = valid_flow_st(n, edges, s, t);
			if (r < 0) return r;
			flow::s = s; flow::t = t;
			flow::e[s].pop_back(); flow::e[t].pop_back();
			while (flow::bfs()) r += flow::dfs(s, flow::inf);
			return r;
		}
		ll valid_min_flow(int n, const vector<tuple<int, int, ll, ll>>& edges, int s, int t)//-1 invalid, ll=ll
		{
			ll r = valid_flow_st(n, edges, s, t);
			if (r < 0) return r;
			flow::s = t; flow::t = s;
			flow::e[s].pop_back(); flow::e[t].pop_back();
			while (flow::bfs()) r -= flow::dfs(t, flow::inf);
			return r;
		}//not check
	}
	using bounded_flow::valid_flow, bounded_flow::valid_flow_st, bounded_flow::valid_max_flow, bounded_flow::valid_min_flow;
	namespace bounded_cost_flow
	{
		pair<ll, ll> valid_mcf(int n, const vector<tuple<int, int, ll, ll, ll>>& edges, int s, int t)//[u,v,l,r,c],mincost flow
		{
			++n;
			int ss = n, tt = n + 1;
			static ll cd[N];
			memset(cd, 0, n * sizeof cd[0]);
			for (auto [u, v, l, r, c] : edges) cd[u] += l, cd[v] -= l;
			vector<tuple<int, int, ll, ll>> e;
			ll t1 = 0, t2 = 0;
			for (int i = 0; i < n; i++) if (cd[i] > 0) e.push_back({i, tt, cd[i], 0}), t2 += cd[i];
			else if (cd[i] < 0) e.push_back({ss, i, -cd[i], 0});
			for (auto [u, v, l, r, c] : edges) e.push_back({u, v, r - l, c});
			for (auto [u, v, w, c] : e) t1 += (u == s) * w;
			e.push_back({t, s, t1, 0});
			auto res = mcmf_spfa(tt, e, ss, tt);//checked dijk
			if (res.first != t2) return {-1, -1};
			res.first = cost_flow::e[s].back().w;
			for (auto [u, v, l, r, c] : edges) res.second += l * c;
			return res;
		}
		pair<ll, ll> valid_mcmf(int n, const vector<tuple<int, int, ll, ll, ll>>& edges, int s, int t)//[u,v,l,r,c],mincost max_flow
		{
			auto r = valid_mcf(n, edges, s, t);
			if (r.first < 0) return {-1, -1};
			cost_flow::e[s].pop_back();
			cost_flow::e[t].pop_back();
			cost_flow::s = s; cost_flow::t = t;
			pair<ll, ll> rr;
			while ((rr = cost_flow::spfa()).first) r = {r.first + rr.first, r.second + rr.second};//spfa ver. not checked dijk
			return r;
		}
	}
	using bounded_cost_flow::valid_mcf, bounded_cost_flow::valid_mcmf;
	namespace ne_cost_flow
	{
		pair<ll, ll> ne_mcmf(int n, const vector<tuple<int, int, ll, ll>>& edges, int s, int t)
		{
			vector<tuple<int, int, ll, ll, ll>> e;
			for (auto [u, v, w, c] : edges) if (c >= 0) e.push_back({u, v, 0, w, c}); else
			{
				e.push_back({u, v, w, w, c});
				e.push_back({v, u, 0, w, -c});
			}
			return valid_mcmf(n, e, s, t);
		}
	}
	using ne_cost_flow::ne_mcmf;
}
\end{lstlisting}


\subsection{假带花树}

一种错误的一般图最大匹配算法,但较难卡掉。推荐在时间不足时作为乱搞使用。

\begin{lstlisting}
mt19937 rnd(3214);
vector<int> lj[N];
int lk[N],ed[N];
int n,m,cnt,i,t,x,y,ans,la;
bool dfs(int x)
{
	ed[x]=cnt;int v;
	shuffle(lj[x].begin(),lj[x].end(),rnd);
	for (auto u:lj[x]) if (ed[v=lk[u]]!=cnt)
	{
		lk[v]=0,lk[u]=x,lk[x]=u;
		if (!v||dfs(v)) return 1;
		lk[v]=u,lk[u]=v,lk[x]=0;
	}
	return 0;
}
int main()
{
	srand(time(0));la=-1;
	cin>>n>>m;
	while (m--) cin>>x>>y,lj[x].push_back(y),lj[y].push_back(x);
	while (la!=ans)
	{
		memset(ed+1,0,n<<2);la=ans;
		for (i=1;i<=n;i++) if (!lk[i]) ans+=dfs(cnt=i);
	}
	cout<<ans<<'\n';
	for (i=1;i<=n;i++) cout<<lk[i]<<" \n"[i==n];
}
\end{lstlisting}

\subsection{Stoer-Wagner 全局最小割}

无向图 $G$ 的最小割为:一个去掉后可以使 $G$ 变成两个连通分量,且边权和最小的边集的边权和。

$O(n^3)$。可优化到 $O(nm\log n)$。

\begin{lstlisting}
#include "bits/stdc++.h"
using namespace std;
namespace StoerWagner
{
	const int N=602;//点数
	typedef int T;//边权和
	T e[N][N],w[N];
	int ed[N],p[N],f[N];//f 仅输出方案用
	int getf(int u){return f[u]==u?u:f[u]=getf(f[u]);}
	template<class TT> pair<T,vector<int>> mincut(int n,const vector<tuple<int,int,TT>> &edges)//[1,n],返回某一集合
	{
		vector<int> ans;ans.reserve(n);
		int i,j,m;
		T r;
		r=numeric_limits<T>::max();
		for (i=1;i<=n;i++) memset(e[i]+1,0,n*sizeof e[0][0]);
		for (auto [u,v,w]:edges) e[u][v]+=w,e[v][u]+=w;
		fill_n(ed+1,n,0);
		iota(f+1,f+n+1,1);
		for (m=n;m>1;m--)
		{
			fill_n(w+1,n,0);
			for (i=1;i<=n;i++) ed[i]&=2;
			for (i=1;i<=m;i++)
			{
				int x=0;
				for (j=1;j<=n;j++) if (!ed[j]) break;x=j;
				for (j++;j<=n;j++) if (!ed[j]*w[j]>w[x]) x=j;
				ed[p[i]=x]=1;
				for (j=1;j<=n;j++) w[j]+=!ed[j]*e[x][j];
			}
			int s=p[m-1],t=p[m];
			if (r>w[t])
			{
				r=w[t];ans.clear();
				for (i=1;i<=n;i++) if (getf(i)==getf(t)) ans.push_back(i);
			}
			for (i=1;i<=n;i++) e[i][s]=e[s][i]+=e[t][i];
			ed[t]=2;
			f[getf(s)]=getf(t);
		}
		return {r,ans};
	}
}
int main()
{
	ios::sync_with_stdio(0);cin.tie(0);
	int n,m;
	cin>>n>>m;
	vector<tuple<int,int,int>> e(m);
	for (auto &[u,v,w]:e) cin>>u>>v>>w;
	auto [_,v]=StoerWagner::mincut(n,e);
	cout<<_<<endl;
	static int ed[602];
	for (int x:v) ed[x]=1;
	for (auto [u,v,w]:e) _-=w*(ed[u]^ed[v]);
	assert(!_);
}
\end{lstlisting}


\subsection{双极分解}

无向图,图点双连通时对任意 $s,t$ 存在。

含义:确定一个拓扑序,使得按这个拓扑序定向后,入度为 $0$ 的只有 $s$,出度为 $0$ 的只有 $t$。

\begin{lstlisting}
vector<int> bipolar_orientation(const vector<pair<int, int>> &edges, int n, int s, int t)//[0,n)
{
	assert(s!=t);
	vector e(n, vector<int>());
	for (auto [u, v]:edges)
	{
		e[u].push_back(v);
		e[v].push_back(u);
	}
	int cur=1, i;
	vector<int> pre(n), low(n), p(n);
	pre[s]=1;
	vector<int> id;
	bool flg=0;
	function<void(int)> dfs=[&](int x)
		{
			pre[x]=++cur;
			low[x]=x;
			for (int y:e[x])
			{
				flg|=y==s;
				if (pre[y]==0)
				{
					id.push_back(y);
					dfs(y);
					p[y]=x;
					if (pre[low[y]]<pre[low[x]]) low[x]=low[y];
				}
				else if (pre[y]!=0&&pre[y]<pre[low[x]]) low[x]=y;
			}
		};
	dfs(t);
	if (!flg) return { };
	vector<int> sign(n, -1);
	vector<int> l(n), r(n);
	r[s]=t;
	l[t]=s;
	for (int v:id)
	{
		if (sign[low[v]]==-1)
		{
			l[v]=l[p[v]];
			r[l[v]]=v;
			l[p[v]]=v;
			r[v]=p[v];
			sign[p[v]]=1;
		}
		else
		{
			r[v]=r[p[v]];
			l[r[v]]=v;
			r[p[v]]=v;
			l[v]=p[v];
			sign[p[v]]=-1;
		}
	}
	vector<int> a(n);
	int x;
	for (i=0, x=s; i<n; x=r[x], i++) a[i]=x;
	vector<int> ia(n, -1), rd(n), cd(n);
	for (i=0; i<n; i++) ia[a[i]]=i;
	if (count(all(ia), -1)) return { };
	for (auto [u, v]:edges)
	{
		if (ia[u]>ia[v]) swap(u, v);
		++cd[u]; ++rd[v];
	}
	for (i=0; i<n; i++) if (i!=s&&i!=t&&(!cd[i]||!rd[i])) return { };
	return a;
}

\end{lstlisting}

\subsection{点双}

一些结论:

判定一个图里是否有(点不重复)偶环:看其所有点双,若存在点数为偶数的或边数多于点数的点双,则存在偶环。

(无自环时)点双的边不交,边双的点不交。点双内的总点数 $O(n)$,总边数为 $m$,边双内的总点数为 $n$,总边数不超过 $m$。

构造函数传入邻接表和边数,其中 \verb|pair| 的 \verb|second| 是边的标号。

所有标号从 $0$ 开始。

不能处理有自环的情况,因为此时点双内的总边数不是线性的。

\verb|bcc_node|:每个点双包含的点(已验证);
\verb|bcc_edge|:每个点双包含的边;
\verb|bcc_n|:新图点数;
\verb|ct|:是否割点(已验证);
\verb|blk|:边所属点双标号。

\begin{lstlisting}
struct node_bcc
{
	int n, id, tp, bcc_n;
	vector<vector<pair<int, int>>> e;
	vector<vector<int>> bcc_node, bcc_edge;
	vector<int> dfn, low, st, ed, blk, ct;
	node_bcc(const vector<vector<pair<int, int>>> &e, int m) :
		n(e.size()), id(0), tp(0), bcc_n(0), e(e), dfn(n, -1), low(n, -1), st(m), ed(m), blk(m), ct(n)
	{
		for (int i = 0; i < n; i++) if (dfn[i] == -1) dfs(i, 1);
		bcc_node.resize(bcc_n);
		for (int i = 0; i < n; i++) for (auto [v, w] : e[i]) bcc_node[blk[w]].push_back(i);
		vector<int> flg(n);
		for (auto &v : bcc_node)
		{
			vector<int> t;
			for (int x : v) if (!exchange(flg[x], 1)) t.push_back(x);
			swap(t, v);
			for (int x : v) flg[x] = 0;
		}
		for (int i = 0; i < n; i++) if (e[i].size() == 0)
		{
			bcc_node.push_back({i});
			bcc_edge.push_back({ });
			++bcc_n;
		}
	}
	void dfs(int u, bool rt)
	{
		dfn[u] = low[u] = id++;
		int cnt = 0;
		for (auto [v, w] : e[u]) if (!ed[w])
		{
			st[tp++] = w;
			ed[w] = 1;
			if (dfn[v] == -1)
			{
				dfs(v, 0);
				++cnt;
				cmin(low[u], low[v]);
				if (dfn[u] <= low[v])
				{
					ct[u] = cnt > rt;
					bcc_edge.push_back({ });
					do
					{
						bcc_edge[bcc_n].push_back(st[--tp]);
						blk[st[tp]] = bcc_n;
					} while (st[tp] != w);
					++bcc_n;
				}
			}
			else cmin(low[u], dfn[v]);
		}
	}
};
\end{lstlisting}

\subsection{边双}

$O(n+m)$,$O(n+m)$。

构造函数传入邻接表和边数,其中 \verb|pair| 的 \verb|second| 是边的标号。

所有标号从 $0$ 开始。

\verb|bcc_node|:每个边双包含的点(已验证);
\verb|bcc_edge|:每个边双包含的边;
\verb|bcc_n|:新图点数;
\verb|cur_e|:新图边表;
\verb|ct|:是否割边;
\verb|blk|:点所属边双标号。

\begin{lstlisting}
struct edge_bcc
{
	int n, id, tp, bcc_n;
	vector<vector<pair<int, int>>> e, cur_e;
	vector<vector<int>> bcc_node, bcc_edge;
	vector<int> dfn, low, st, blk, ct;
	edge_bcc(const vector<vector<pair<int, int>>> &e, int m) :
		n(e.size()), id(0), tp(0), bcc_n(0), e(e), dfn(n, -1), low(n, -1), st(n), blk(n), ct(m)
	{
		for (int i = 0; i < n; i++) if (dfn[i] == -1) dfs(i, -1);
		cur_e.resize(bcc_n);
		for (int i = 0; i < n; i++) for (auto [v, w] : e[i]) if (ct[w]) cur_e[blk[i]].push_back({blk[v], w});
		else bcc_edge[blk[i]].push_back(w);
		vector<int> flg(m);
		for (auto &v : bcc_edge)
		{
			vector<int> t;
			for (int x : v) if (!exchange(flg[x], 1)) t.push_back(x);
			swap(t, v);
		}
	}
	void dfs(int u, int fw)
	{
		dfn[u] = low[u] = id++;
		st[tp++] = u;
		for (auto [v, w] : e[u]) if (w != fw)
		{
			if (dfn[v] == -1)
			{
				dfs(v, w);
				cmin(low[u], low[v]);
				ct[w] = (dfn[u] < low[v]);
			}
			else cmin(low[u], dfn[v]);
		}
		if (dfn[u] == low[u])
		{
			bcc_node.push_back({ });
			bcc_edge.push_back({ });
			do
			{
				bcc_node[bcc_n].push_back(st[--tp]);
				blk[st[tp]] = bcc_n;
			} while (st[tp] != u);
			++bcc_n;
		}
	}
};
int main()
{
	ios::sync_with_stdio(0); cin.tie(0);
	int n, m, i;
	cin >> n >> m;
	vector<vector<pair<int, int>>> e(n);
	for (i = 0; i < m; i++)
	{
		int u, v;
		cin >> u >> v;
		--u, --v;
		e[u].push_back({v, i});
		e[v].push_back({u, i});
	}
	edge_bcc s(e, m);
	cout << s.bcc_n << '\n';
	for (auto &v : s.bcc_node)
	{
		for (int &x : v) ++x;
		cout << v.size() << ' ' << v << '\n';
	}
}
\end{lstlisting}

\subsection{输出负环}

\begin{lstlisting}
#include "bits/stdc++.h"
using namespace std;
const int N=34;
struct Q
{
	int v,w,c;
	Q(){}
	Q(int x,int y,int z):v(x),w(y),c(z){}
};
vector<Q> lj[N];
int dis[N],cnt[N],pt[N],S;
Q pre[N],st[N];
int n,m,ans,tp;
bool ed[N];
int main()
{
	freopen("arbitrage.in","r",stdin);
	freopen("arbitrage.out","w",stdout);
	ios::sync_with_stdio(0);cin.tie(0);
	cin>>n>>m;
	while (m--)
	{
		int x,y,z,w;
		cin>>x>>y>>z>>w;
		lj[x].emplace_back(y,w,z);
		lj[y].emplace_back(x,0,-z);
	}
	for (int i=1;i<=n;i++) lj[0].emplace_back(i,1,0);
	while (1)
	{
		memset(dis,-0x3f,sizeof dis);dis[0]=0;
		for (int i=0;i<=n;i++) ed[i]=cnt[i]=0;S=-1;
		queue<int> q;q.push(0);
		while (!q.empty())
		{
			int u=q.front();q.pop();ed[u]=0;
			for (auto &[v,w,c]:lj[u]) if (w&&dis[v]<dis[u]+c)
			{
				dis[v]=dis[u]+c;pre[v]=Q(u,w,c);
				if (!ed[v])
				{
					if (++cnt[v]>n+1) {S=v;goto aa;}
					ed[v]=1;q.push(v);
				}
			}
		}
		aa:;
		if (S==-1) break;
		{
			static bool ed[N];
			memset(ed,0,sizeof ed);
			while (!ed[S]) ed[S]=1,S=pre[S].v;
		}
		st[tp=1]=pre[S];pt[1]=S;
		int x=pre[S].v;
		while (x!=S)
		{
			st[++tp]=pre[x];pt[tp]=x;
			x=pre[x].v;
			assert(tp<=n+5);
		}
		int fl=1e9;
		for (int j=1;j<=tp;j++) fl=min(fl,st[j].w);
		assert(fl);
		for (int j=1;j<=tp;j++)
		{
			ans+=fl*st[j].c;
			int nn=0;
			for (auto &[v,w,c]:lj[st[j].v]) if (v==pt[j]&&st[j].c==c&&st[j].w==w) {++nn;w-=fl;break;}
			for (auto &[v,w,c]:lj[pt[j]]) if (v==st[j].v&&st[j].c+c==0) {++nn;w+=fl;break;}assert(nn==2);
		}
	}
	cout<<ans<<endl;
}
\end{lstlisting}

\subsection{(基环)树哈希}

有根树返回每个子树的哈希值,无根树返回树的哈希值(长度至多为 $2$ 的 \verb|vector|),基环树返回图的哈希值(长度等于环长的 \verb|vector|)。

\begin{lstlisting}
vector<int> tree_hash(const vector<vector<int>>& e, int root)//[0,n)
{
	int n = e.size();
	static map<vector<int>, int> mp;
	static int id = 0;
	vector<int> h(n), ed(n);
	function<void(int)> dfs = [&](int u)
		{
			ed[u] = 1;
			vector<int> c;
			for (int v : e[u]) if (!ed[v])
			{
				dfs(v);
				c.push_back(h[v]);
			}
			sort(all(c));
			if (!mp.count(c)) mp[c] = id++;
			h[u] = mp[c];
		};
	dfs(root);
	return h;
}
vector<int> tree_hash(const vector<vector<int>>& e)//[0,n)
{
	int n = e.size();
	if (n == 0) return { };
	vector<int> sz(n), mx(n);
	function<void(int)> dfs = [&](int u)
		{
			sz[u] = 1;
			for (int v : e[u]) if (!sz[v])
			{
				dfs(v);
				sz[u] += sz[v];
				cmax(mx[u], sz[v]);
			}
			cmax(mx[u], n - sz[u]);
		};
	dfs(0);
	int m = *min_element(all(mx)), i;
	vector<int> rt;
	for (i = 0;i < n;i++) if (mx[i] == m) rt.push_back(i);
	for (int& u : rt) u = tree_hash(e, u)[u];
	sort(all(rt));
	return rt;
}
template<class T> void min_order(vector<T>& a)
{
	int n = a.size(), i, j, k;
	a.resize(n * 2);
	for (i = 0;i < n;i++) a[i + n] = a[i];
	i = k = 0;j = 1;
	while (i < n && j < n && k < n)
	{
		T x = a[i + k], y = a[j + k];
		if (x == y) ++k; else
		{
			(x > y ? i : j) += k + 1;
			j += (i == j);
			k = 0;
		}
	}
	a.resize(n);
	//[min(i,j),n)+[0,min(i,j))
	rotate(a.begin(), min(i, j) + all(a));
}
vector<int> pseudotree_hash(const vector<vector<int>>& e)//[0,n)
{
	int n = e.size();
	static map<vector<int>, int> mp;
	static int id = 0;
	vector<int> f(n), ed(n), h(n);
	pair lp{-1, -1};
	function<void(int)> dfs = [&](int u)
		{
			ed[u] = 1;
			for (int v : e[u]) if (!ed[v])
			{
				f[v] = u;
				dfs(v);
			}
			else if (v != f[u]) lp = {u, v};
		};
	dfs(0);
	auto [x, y] = lp;
	vector<int> node = {y};
	do node.push_back(y = f[y]); while (y != x);
	fill(all(ed), 0);
	for (int u : node) ed[u] = 1;
	dfs = [&](int u)
		{
			ed[u] = 1;
			vector<int> c;
			for (int v : e[u]) if (!ed[v])
			{
				dfs(v);
				c.push_back(h[v]);
			}
			sort(all(c));
			if (!mp.count(c)) mp[c] = id++;
			h[u] = mp[c];
		};
	vector<int> r0;
	for (int u : node)
	{
		dfs(u);
		r0.push_back(h[u]);
	}
	auto r1 = r0;
	reverse(all(r1));
	min_order(r0);
	min_order(r1);
	return min(r0, r1);
}
\end{lstlisting}

\subsection{无向图最小环}

原理:floyd 外层循环本质是计算只经过 $\le k$ 的点的最短路。因此枚举环上标号最大的,在做这一轮转移之前正好是不经过它的最短路。

$O(n^3)$,$O(n^2)$。

\begin{lstlisting}
int f[N][N],jl[N][N];
int n,m,c,ans=inf,i,j,k,x,y,z;
int main()
{
	cin>>n>>m;
	memset(f,0x3f,sizeof(f));
	memset(jl,0x3f,sizeof(jl));
	while (m--)
	{
		cin>>x>>y>>z;
		jl[x][y]=jl[y][x]=f[x][y]=f[y][x]=min(f[y][x],z);
	}
	for (k=1;k<=n;k++)
	{
		for (i=1;i<k;i++) if (jl[k][i]!=jl[0][0]) for (j=1;j<i;j++)
			if (jl[k][j]!=jl[0][0]) ans=min(ans,jl[k][i]+jl[k][j]+f[i][j]);
		for (i=1;i<=n;i++) if (i!=k) for (j=1;j<=n;j++)
			if ((j!=i)&&(j!=k)) f[i][j]=min(f[i][j],f[i][k]+f[k][j]);
	}
	if (ans==inf) cout<<"No solution.\n"; else cout<<ans<<endl;
}
\end{lstlisting}

\subsection{切比雪夫距离最小生成树}

原理:先转曼哈顿距离,再用曼哈顿的板子。

$O(n\log n)$,$O(n)$。

\begin{lstlisting}
const int N=3e5+2,M=N<<2;
struct P
{
	int u,v,w;
	P(int a=0,int b=0,int c=0):u(a),v(b),w(c){}
	bool operator<(const P &o) const {return w<o.w;}
};
struct Q
{
	int x,y,id;
	Q(int a=0,int b=0,int c=0):x(a),y(b),id(c){}
	bool operator<(const Q &o) const {return x!=o.x?x>o.x:y>o.y;}
};
ll ans;
P lb[M];
Q a[N],b[N];
int f[N],c[N];
int n,m,i,x,y;
struct bit
{
	int a[N],pos[N],n;
	void init(int &nn)
	{
		memset(a+1,0x7f,(n=nn)*sizeof a[0]);
		memset(pos+1,0,n*sizeof pos[0]);
	}
	void mdf(int x,const int y,const int z)
	{
		if (a[x]>y) a[x]=y,pos[x]=z;
		while (x-=x&-x) if (a[x]>y) a[x]=y,pos[x]=z;
	}
	int sum(int x)
	{
		int r=a[x],rr=pos[x];
		while ((x+=x&-x)<=n) if (a[x]<r) r=a[x],rr=pos[x];
		return rr;
	}
};
bit s;
void cal()
{
	int i,x,y;
	s.init(n);
	memcpy(b+1,a+1,sizeof(Q)*n);
	sort(a+1,a+n+1);
	for (i=1;i<=n;i++) c[i]=a[i].y-a[i].x;
	sort(c+1,c+n+1);
	for (i=1;i<=n;i++)
	{
		if (x=s.sum(y=lower_bound(c+1,c+n+1,a[i].y-a[i].x)-c))
			lb[++m]=P(a[x].id,a[i].id,a[x].x+a[x].y-a[i].x-a[i].y);//谨防 int 爆
		s.mdf(y,a[i].y+a[i].x,i);
	}
	memcpy(a+1,b+1,sizeof(Q)*n);
}
int getf(int x) {return f[x]==x?x:f[x]=getf(f[x]);}
int main()
{
	cin>>n;
	for (i=1;i<=n;i++) {cin>>a[f[i]=a[i].id=i].x>>a[i].y;
		swap(a[i].x,a[i].y);a[i]=Q(a[i].x+a[i].y,a[i].x-a[i].y,i);}
	cal();for (i=1;i<=n;i++) swap(a[i].x,a[i].y);
	cal();for (i=1;i<=n;i++) a[i].y=-a[i].y;
	cal();for (i=1;i<=n;i++) swap(a[i].x,a[i].y);
	cal();sort(lb+1,lb+m+1);
	for (i=1;i<=m;i++) if ((x=getf(lb[i].u))!=(y=getf(lb[i].v))) f[x]=y,ans+=lb[i].w;
	cout<<ans/2<<endl;
}
\end{lstlisting}

\subsection{点分治}

点分治板子的参考意义不大。

$O(n\log n)$,$O(n)$。

\begin{lstlisting}
int siz[N], dep[N];
int n, ksiz, md, rt, mn;
bool ed[N];
void find(int u)
{
	ed[u] = 1; siz[u] = 1;
	int mx = 0;
	for (int v : e[u]) if (!ed[v])
	{
		find(v);
		siz[u] += siz[v];
		mx = max(mx, siz[v]);
	}
	mx = max(mx, ksiz - siz[u]);
	if (mn > mx) mn = mx, rt = u;
	ed[u] = 0;
}
void cal(int u)
{
	md = max(md, dep[u]);
	ed[u] = 1; ++cnt[dep[u]];
	for (int v : e[u]) if (!ed[v])
	{
		dep[v] = dep[u] + 1;
		cal(v);
	}
	ed[u] = 0;
}
void solve(int u)
{
	mn = 1e9;
	find(u);
	ed[rt] = 1;
	vector<int> c;
	for (int v : e[rt]) if (!ed[v])
	{
		c.push_back(v);
		if (siz[v] >= siz[rt]) siz[v] = siz[u] - siz[rt];
	}
	sort(all(c), [&](const int &a, const int &b) {return siz[a] < siz[b]; });
	NTT::Q a(vector<ui>{1});
	NT::Q b(vector<ui>{1});
	for (int v : c)
	{
		md = 0; dep[v] = 1;
		cal(v); ++md;
		vector<ui> d(cnt, cnt + md);
		NTT::Q e(d);
		NT::Q f(d);
		auto g = e & a;
		auto h = f & b;
		for (int i = 0; i < g.a.size(); i++) r1[i] = (r1[i] + g.a[i]) % NTT::p;
		for (int i = 0; i < h.a.size(); i++) r2[i] = (r2[i] + h.a[i]) % NT::p;
		a += e; b += f;
		fill_n(cnt, md, 0);
	}
	for (int v : c)
	{
		ksiz = siz[v];
		solve(v);
	}
}
\end{lstlisting}

\subsection{点分树}

核心结论:点分树上 lca 出现在原树路径上。

$O(n\log^2 n)$,$O(n\log n)$。

\begin{lstlisting}
template<typename typC> struct bit
{
	vector<typC> a;
	int n;
	bit() { }
	bit(int nn) :n(nn), a(nn + 1) { }
	template<typename T> bit(int nn, T *b) : n(nn), a(nn + 1)
	{
		for (int i = 1; i <= n; i++) a[i] = b[i - 1];
		for (int i = 1; i <= n; i++) if (i + (i & -i) <= n) a[i + (i & -i)] += a[i];
	}
	void add(int x, typC y)
	{
		//cerr<<"add "<<x<<" by "<<y<<endl;
		++x;
		x = clamp(x, 1, n + 1);
		if (x > n) return;
		assert(1 <= x && x <= n);
		a[x] += y;
		while ((x += x & -x) <= n) a[x] += y;
	}
	typC sum(int x)
	{
		//cerr<<"sum "<<x;
		++x;
		x = clamp(x, 0, n);
		assert(0 <= x && x <= n);
		typC r = a[x];
		while (x ^= x & -x) r += a[x];
		//cerr<<"= "<<r<<endl;
		return r;
	}
	typC sum(int x, int y)
	{
		return sum(y) - sum(x - 1);
	}
	int lower_bound(typC x)
	{
		if (n == 0) return 0;
		int i = __lg(n), j = 0;
		for (; i >= 0; i--) if ((1 << i | j) <= n && a[1 << i | j] < x) j |= 1 << i, x -= a[j];
		return j + 1;
	}
};
namespace DFS
{
	typedef long long ll;
	const int N = 1e5 + 5, M = 18;
	ll a[N];
	int st[M][N * 2], lg[N * 2];
	int dep[N], dfn[N], siz[N], f[N], szp[N], szn[N];
	vector<int> e[N], c[N], rg[N];
	bool ed[N];
	int n, ksiz, rt, mn, id;
	int lca(int u, int v)
	{
		u = dfn[u]; v = dfn[v];
		if (u > v) swap(u, v);
		int z = lg[v - u + 1];
		return dep[st[z][u]] < dep[st[z][v - (1 << z) + 1]] ? st[z][u] : st[z][v - (1 << z) + 1];
	}
	int dis(int u, int v)
	{
		return dep[u] + dep[v] - dep[lca(u, v)] * 2;
	}
	void findroot(int u)
	{
		ed[u] = siz[u] = 1;
		int mx = 0;
		for (int v : e[u]) if (!ed[v])
		{
			findroot(v);
			siz[u] += siz[v];
			mx = max(mx, siz[v]);
		}
		mx = max(mx, ksiz - siz[u]);
		ed[u] = 0;
		if (mn > mx) mn = mx, rt = u;
	}
	int dfs(int u)
	{
		mn = 1e9;
		findroot(u);
		u = rt;
		ed[u] = 1;
		for (int v : e[u]) if (!ed[v] && siz[v] > siz[u]) siz[v] = ksiz - siz[u];
		for (int v : e[u]) if (!ed[v])
		{
			ksiz = siz[v];
			c[u].push_back(dfs(v));
			f[c[u].back()] = u;
		}
		return u;
	}
	void pre_dfs(int u)
	{
		st[0][dfn[u] = ++id] = u;
		ed[u] = 1;
		for (int v : e[u]) if (!ed[v])
		{
			dep[v] = dep[u] + 1;
			pre_dfs(v);
			st[0][++id] = u;
		}
		ed[u] = 0;
	}
	void init(int _n)
	{
		n = _n; id = 0;
		int i;
		for (int i = 1; i <= n; i++)
		{
			e[i].clear();
			a[i] = f[i] = ed[i] = 0;
		}
	}
	void new_dfs(int u)
	{
		siz[u] = 1;
		for (int v : c[u]) new_dfs(v), siz[u] += siz[v];
		vector<int> &q = rg[u];
		q = {u};
		int ql = 0;
		while (ql < q.size())
		{
			int x = q[ql++];
			for (int v : c[x]) q.push_back(v);
		}
	}
	void fun()
	{
		pre_dfs(1);
		int i, j;
		for (i = 2; i <= id; i++) lg[i] = lg[i >> 1] + 1;
		for (j = 0; j < lg[id]; j++)
		{
			int R = id - (2 << j) + 1;
			for (i = 1; i <= R; i++) st[j + 1][i] = dep[st[j][i]] < dep[st[j][i + (1 << j)]] ? st[j][i] : st[j][i + (1 << j)];
		}
		ksiz = n;
		rt = dfs(1);
		new_dfs(rt);
	}
	vector<int> get(int u)
	{
		vector<int> st = {u};
		while (u = f[u]) st.push_back(u);
		return st;
	}
}
using DFS::init, DFS::fun, DFS::e, DFS::dis, DFS::rg, DFS::get;

\end{lstlisting}

圆环修改和单点查询:

\begin{lstlisting}
int main()
{
	ios::sync_with_stdio(0); cin.tie(0);
	cout << fixed << setprecision(15);
	int n, m, i;
	cin >> n >> m;
	vector<int> a(n + 1);
	for (i = 1; i <= n; i++) cin >> a[i];
	DFS::init(n);
	for (i = 1; i < n; i++)
	{
		int u, v;
		cin >> u >> v;
		++u; ++v;
		e[u].push_back(v);
		e[v].push_back(u);
	}
	DFS::fun();
	vector<bit<ll>> inc(n + 1), dec(n + 1);
	for (i = 1; i <= n; i++)
	{
		int mx = 0;
		for (int v : rg[i]) cmax(mx, dis(i, v));
		inc[i] = bit<ll>(mx + 1);
		if (i != DFS::rt)
		{
			mx = 0;
			for (int v : rg[i]) cmax(mx, dis(DFS::f[i], v));
			dec[i] = bit<ll>(mx + 1);
		}
	}
	while (m--)
	{
		int op, u;
		cin >> op >> u; ++u;
		if (op == 0)
		{
			int l, r, x;
			cin >> l >> r >> x;
			auto v = get(u);
			int m = v.size();
			for (i = 0; i < m; i++)
			{
				inc[v[i]].add(l - dis(v[i], u), x);
				inc[v[i]].add(r - dis(v[i], u), -x);
			}
			for (i = 0; i + 1 < m; i++)
			{
				dec[v[i]].add(l - dis(v[i + 1], u), x);
				dec[v[i]].add(r - dis(v[i + 1], u), -x);
			}
		}
		else
		{
			ll res = a[u];
			auto v = get(u);
			int m = v.size();
			for (i = 0; i < m; i++) res += inc[v[i]].sum(dis(v[i], u));
			for (i = 0; i + 1 < m; i++) res -= dec[v[i]].sum(dis(v[i + 1], u));
			cout << res << '\n';
		}
	}
}
\end{lstlisting}

单点修改和圆环查询:

\begin{lstlisting}
int main()
{
	ios::sync_with_stdio(0); cin.tie(0);
	cout << fixed << setprecision(15);
	int n, m, i;
	cin >> n >> m;
	vector<int> a(n + 1);
	for (i = 1; i <= n; i++) cin >> a[i];
	DFS::init(n);
	for (i = 1; i < n; i++)
	{
		int u, v;
		cin >> u >> v;
		++u; ++v;
		e[u].push_back(v);
		e[v].push_back(u);
	}
	DFS::fun();
	vector<bit<ll>> inc(n + 1), dec(n + 1);
	vector<ll> tmp(n + 1);
	for (i = 1; i <= n; i++)
	{
		int mx = 0;
		for (int v : rg[i])
		{
			int d = dis(i, v);
			cmax(mx, d);
			tmp[d] += a[v];
		}
		inc[i] = bit<ll>(mx + 1, tmp.data());
		fill_n(tmp.begin(), mx + 1, 0);
		if (i != DFS::rt)
		{
			mx = 0;
			for (int v : rg[i])
			{
				int d = dis(DFS::f[i], v);
				cmax(mx, d);
				tmp[d] += a[v];
			}
			dec[i] = bit<ll>(mx + 1, tmp.data());
			fill_n(tmp.begin(), mx + 1, 0);
		}
	}
	while (m--)
	{
		int op, u;
		cin >> op >> u; ++u;
		if (op == 0)
		{
			int x;
			cin >> x;
			auto v = get(u);
			int m = v.size();
			for (i = 0; i < m; i++) inc[v[i]].add(dis(v[i], u), x);
			for (i = 0; i + 1 < m; i++) dec[v[i]].add(dis(v[i + 1], u), x);
		}
		else
		{
			int l, r;
			cin >> l >> r;
			--r;
			ll res = 0;
			auto v = get(u);
			int m = v.size();
			for (i = 0; i < m; i++) res += inc[v[i]].sum(l - dis(v[i], u), r - dis(v[i], u));
			for (i = 0; i + 1 < m; i++) res -= dec[v[i]].sum(l - dis(v[i + 1], u), r - dis(v[i + 1], u));
			cout << res << '\n';
		}
	}
}
\end{lstlisting}

\subsection{prufer 与树的互相转化}

$O(n)$,$O(n)$。

\begin{lstlisting}
vector<int> edges_to_prufer(const vector<pair<int,int>> &eg)//[1,n],定根为 n
{
	int n=eg.size()+1,i,j,k;
	vector<int> fir(n+1),nxt(n*2+1),e(n*2+1),rd(n+1);
	int cnt=0;
	for (auto [u,v]:eg)
	{
		e[++cnt]=v;nxt[cnt]=fir[u];fir[u]=cnt;++rd[v];
		e[++cnt]=u;nxt[cnt]=fir[v];fir[v]=cnt;++rd[u];
	}
	for (i=1;i<=n;i++) if (rd[i]==1) break;
	int u=i;
	vector<int> r;r.reserve(n-2);
	for (j=1;j<n-1;j++)
	{	
		for (k=fir[u],u=rd[u]=0;k;k=nxt[k]) if (rd[e[k]]) 
		{
			r.push_back(e[k]);
			if ((--rd[e[k]]==1)&&(e[k]<i)) u=e[k];
		}
		if (!u) { while (rd[i]!=1) ++i;u=i;}
	}
	return r;
}
vector<pair<int,int>> prufer_to_edges(const vector<int> &p)//[1,n],定根为 n
{
	int n=p.size(),i,j,k;
	int m=n+3;
	vector<int> cs(m);
	for (i=0;i<n;i++) ++cs[p[i]];
	i=0;
	while (cs[++i]);
	int u=i,v;
	vector<pair<int,int>> r;
	r.reserve(n-2);
	for (j=0;j<n;j++)
	{
		cs[u]=1e9;
		r.push_back({u,v=p[j]});
		if ((--cs[v]==0)&&(v<i)) u=v;
		if (v!=u) {while (cs[i]) ++i;u=i;}
	}
	r.push_back({u,n+2});
	return r;
}

\end{lstlisting}



\subsection{LCT}

$O(n\log n)$,$O(n)$。

\verb|makeroot| 会变根,\verb|split| 会把 $y$ 变根,\verb|findroot| 会把根变根,\verb|link| 会把 $x,y$ 变根($y$ 是新的),\verb|cut| 会把 $x,y$ 变根($x$ 是新的),注意 \verb|swap| 子节点可能要 \verb|pushup|。

\begin{lstlisting}
template<int N,class Q> struct LCT
{
	int f[N],c[N][2],siz[N],st[N];
	Q s[N],v[N];
	#ifdef Rev
	Q rs[N];
	#endif
	//heap g[N]; //虚子树
	bool lz[N];
	void init(int n)
	{
		++n;
		for (int i=0;i<n;i++)
		{
			f[i]=c[i][0]=c[i][1]=lz[i]=0;
			s[i]=v[i]=Q();
			#ifdef Rev
			rs[i]=Q();
			#endif
			siz[i]=!!i;
		}
	}
	void modify(int x,const Q &o)
	{
		makeroot(x);
		v[x]=o;
		pushup(x);
	}
	bool nroot(int x) const
	{
		return c[f[x]][0]==x||c[f[x]][1]==x;
	}
	void pushup(int x)
	{
		int lc=c[x][0],rc=c[x][1];
		s[x]=v[x];siz[x]=1;
		#ifdef Rev
		rs[x]=v[x];
		#endif
		if (lc)
		{
			s[x]=s[lc]+s[x];
			siz[x]+=siz[lc];
			#ifdef Rev
			rs[x]=rs[x]+rs[lc];
			#endif
		}
		if (rc)
		{
			s[x]=s[x]+s[rc];
			siz[x]+=siz[rc];
			#ifdef Rev
			rs[x]=rs[rc]+rs[x];
			#endif
		}
	}
	void swp(int x)
	{
		swap(c[x][0],c[x][1]);
		#ifdef Rev
		swap(s[x],rs[x]);
		#endif
		lz[x]^=1;
	}
	void pushdown(int x)
	{
		int lc=c[x][0],rc=c[x][1];
		if (lz[x])
		{
			if (lc) swp(lc);
			if (rc) swp(rc);
			lz[x]=0;
		}
	}
	void zigzag(int x)
	{
		int y=f[x],z=f[y],typ=(c[y][0]==x);
		if (nroot(y)) c[z][c[z][1]==y]=x;
		f[x]=z;f[y]=x;
		if (c[x][typ]) f[c[x][typ]]=y;
		c[y][typ^1]=c[x][typ];c[x][typ]=y;
		pushup(y);
	}
	void splay(int x)
	{
		int y,tp=0;
		st[tp=1]=y=x;
		while (nroot(y)) st[++tp]=y=f[y];
		while (tp) pushdown(st[tp--]);
		for (;nroot(x);zigzag(x)) if (!nroot(f[x])) continue; else zigzag((c[f[x]][0]==x)^(c[f[f[x]]][0]==f[x]) ? x:f[x]);
		pushup(x);
	}
	void access(int x)
	{
		for (int y=0;x;x=f[y=x])
		{
			splay(x);
			//g[x].ins(s[c[x][1]]);g[x].del(s[y]);虚子树变化
			c[x][1]=y;pushup(x);
		}
	}
	int findroot(int x)
	{
		access(x);splay(x);pushdown(x);
		while (c[x][0]) pushdown(x=c[x][0]);
		splay(x);
		return x;
	}
	void split(int x,int y)//x 为树新根,y 为 splay 新根
	{
		makeroot(x);
		access(y);
		splay(y);
	}
	void makeroot(int x)
	{
		access(x);splay(x);
		swp(x);
	}
	void link(int x,int y)//y 为新根
	{
		makeroot(x);
		if (x!=findroot(y))//可能已经连通
		{
			makeroot(y);f[x]=y;//虚子树变化
		}
	}
	void cut(int x,int y)
	{
		makeroot(x);
		if (x==findroot(y))//可能本不连通
		{
			pushdown(x);
			if (c[x][1]==y&&!c[y][0]&&!c[y][1])//可能连通但无边
			{
				c[x][1]=f[y]=0;//可能需要修改
				pushup(x);
			}
		}
	}
};
\end{lstlisting}

\subsection{LCT(重构,代码为动态割边割点)}

\begin{lstlisting}
#include "bits/stdc++.h"
using namespace std;
template<int N,class info,class tag> struct LCT
{
	int f[N],c[N][2];
	info s[N],v[N];
#ifdef Rev
	info rs[N];
#endif
	tag tg[N];
	bool rev[N],lz[N];
	void init(int n,info *a)
	{
		for (int i=0; i<=n; i++)
		{
			rev[i]=lz[i]=0;
			f[i]=c[i][0]=c[i][1]=0;
			s[i]=v[i]=a[i];
#ifdef Rev
			rs[i]=a[i];
#endif
		}
	}
	bool nroot(int x) const
	{
		return c[f[x]][0]==x||c[f[x]][1]==x;
	}
	void pushup(int x)
	{
		int lc=c[x][0],rc=c[x][1];
		s[x]=v[x];
#ifdef Rev
		rs[x]=v[x];
#endif
		if (lc)
		{
			s[x]=s[lc]+s[x];
#ifdef Rev
			rs[x]=rs[x]+rs[lc];
#endif
		}
		if (rc)
		{
			s[x]=s[x]+s[rc];
#ifdef Rev
			rs[x]=rs[rc]+rs[x];
#endif
		}
	}
	void swp(int x)
	{
		swap(c[x][0],c[x][1]);
#ifdef Rev
		swap(s[x],rs[x]);
#endif
		rev[x]^=1;
	}
	void pushdown(int x)
	{
		if (rev[x])
		{
			for (int y:c[x]) if (y) swp(y);
			rev[x]=0;
		}
		if (lz[x])
		{
			for (int y:c[x]) if (y)
			{
				if (lz[y]) tg[y]+=tg[x]; else tg[y]=tg[x],lz[y]=1;
				s[y]+=tg[x];
			}
			lz[x]=0;
		}
	}
	void zigzag(int x)
	{
		int y=f[x],z=f[y],typ=(c[y][0]==x);
		if (nroot(y)) c[z][c[z][1]==y]=x;
		f[x]=z; f[y]=x;
		if (c[x][typ]) f[c[x][typ]]=y;
		c[y][typ^1]=c[x][typ]; c[x][typ]=y;
		pushup(y);
	}
	void splay(int x)
	{
		static int st[N];
		int y,tp;
		st[tp=1]=y=x;
		while (nroot(y)) st[++tp]=y=f[y];
		while (tp) pushdown(st[tp--]);
		for (; nroot(x); zigzag(x)) if (nroot(y=f[x])) zigzag((c[y][0]==x)^(c[f[y]][0]==y)?x:f[x]);
		pushup(x);
	}
	int access(int x)
	{
		int y=0;
		for (; x; x=f[y=x]) splay(x),c[x][1]=y,pushup(x);
		return y;
	}
	int findroot(int x)//splay 根为树根,splay 维护树根到 x 的链
	{
		access(x); splay(x); pushdown(x);
		while (c[x][0]) pushdown(x=c[x][0]);
		splay(x); return x;
	}
	void split(int x,int y)//x 为树新根,y 为 splay 新根
	{ makeroot(x); access(y); splay(y); }
	void makeroot(int x)//x 为树、splay 新根
	{ access(x); splay(x); swp(x); }
	void modify(int x,const info &o)
	{ makeroot(x); v[x]=o; pushup(x); }
	void modify(int x,int y,const tag &o)
	{
		split(x,y); s[y]+=o;
		if (lz[y]) tg[y]+=o; else tg[y]=o,lz[y]=1;
	}
	info ask(int x,int y) { split(x,y); return s[y]; }
	bool connected(int x,int y)//注意会改变形态
	{ makeroot(x); return findroot(y)==x; }
	void link(int x,int y)//y 为新根
	{ if (!connected(x,y)) makeroot(f[x]=y); }
	void cut(int x,int y)
	{
		if (connected(x,y))//可能本不连通
		{
			pushdown(x);
			if (c[x][1]==y&&!c[y][0]&&!c[y][1])//可能连通但无边
			{
				c[x][1]=f[y]=0;
				pushup(x);
			}
		}
	}
	int lca(int x,int y) { access(x); return access(y); }
	vector<int> res;
	void dfs(int x)
	{
		if (!x) return;
		pushdown(x);
		dfs(c[x][0]); res.push_back(x); dfs(c[x][1]);
	}
	vector<int> get_path(int x,int y)
	{
		res.clear(); split(x,y); dfs(y);
		if (res[0]!=x) return {};
		return res;
	}
};
const int N=2e5+5,M=4e5+5;
struct Q
{
	void operator+=(const Q &o) const {}
};
void operator+=(int &x,const Q &o) { x=0; }
LCT<N,int,Q> s;
LCT<M,int,Q> t;
int a[N],b[M];
int main()
{
	ios::sync_with_stdio(0); cin.tie(0);
	int n,m,i,r=0;
	cin>>n>>m;
	fill_n(a+n+1,n,1);
	fill_n(b+1,n,1);
	s.init(n*2,a);
	t.init(n+m,b);
	int bs=n,ds=n;
	while (m--)
	{
		int op,u,v;
		cin>>op>>u>>v;
		u^=r; v^=r;
		// dbg(op,u,v);
		if (u<1||u>n||v<1||v>n) return 0;
		if (op==1)
		{
			if (s.connected(u,v))
			{
				s.modify(u,v,{});
				auto c=t.get_path(u,v);
				for (i=1; i<c.size(); i++) t.cut(c[i-1],c[i]);
				++ds;
				for (int x:c) t.link(ds,x);
			}
			else
			{
				s.link(++bs,u);
				s.link(bs,v);
				t.link(++ds,u);
				t.link(ds,v);
			}
		}
		else
		{
			if (!s.connected(u,v))
			{
				cout<<"-1\n";
				continue;
			}
			r=op==2?s.ask(u,v):t.ask(u,v);
			cout<<r<<'\n';
		}
	}
}
\end{lstlisting}


\subsection{带子树的 LCT}

$O(n\log n)$,$O(n)$。

\begin{lstlisting}
#include "bits/stdc++.h"
using namespace std;
typedef long long ll;
template<int N> struct LCT
{
	ll s[N],v[N],sg[N];
	int f[N],c[N][2],siz[N],st[N];
	//heap g[N]; //虚子树
	bool lz[N];
	void init(int n)
	{
		memset(f,0,n+1<<2);
		memset(c,0,n+1<<3);
		memset(s,0,n+1<<3);
		memset(v,0,n+1<<3);
		memset(lz,0,n+1);
	}
	bool nroot(int x)
	{
		return c[f[x]][0]==x||c[f[x]][1]==x;
	}
	void pushup(int x)
	{
		s[x]=s[c[x][0]]+s[c[x][1]]+v[x]+sg[x];
		siz[x]=siz[c[x][0]]+siz[c[x][1]]+1;
	}
	void pushdown(int x)
	{
		if (lz[x])
		{
			swap(c[c[x][0]][0],c[c[x][0]][1]);
			swap(c[c[x][1]][0],c[c[x][1]][1]);
			lz[c[x][0]]^=1;
			lz[c[x][1]]^=1;
			lz[x]=0;
		}
	}
	void zigzag(int x)
	{
		int y=f[x],z=f[y],typ=(c[y][0]==x);
		if (nroot(y)) c[z][c[z][1]==y]=x;
		f[x]=z;f[y]=x;
		if (c[x][typ]) f[c[x][typ]]=y;
		c[y][typ^1]=c[x][typ];c[x][typ]=y;
		pushup(y);
	}
	void splay(int x)
	{
		int y,tp=0;
		st[tp=1]=y=x;
		while (nroot(y)) st[++tp]=y=f[y];
		while (tp) pushdown(st[tp--]);
		for (;nroot(x);zigzag(x)) if (!nroot(f[x])) continue; else zigzag((c[f[x]][0]==x)^(c[f[f[x]]][0]==f[x]) ? x:f[x]);
		pushup(x);
	}
	void access(int x)
	{
		for (int y=0;x;x=f[y=x])
		{
			splay(x);sg[x]-=s[y];s[x]-=s[y];
			sg[x]+=s[c[x][1]];s[x]+=s[c[x][1]];
			//g[x].ins(s[c[x][1]]);g[x].del(s[y]);虚子树变化
			c[x][1]=y;pushup(x);
		}
	}
	int findroot(int x)
	{
		access(x);splay(x);pushdown(x);
		while (c[x][0]) pushdown(x=c[x][0]);
		splay(x);
		return x;
	}
	void split(int x,int y)
	{
		makeroot(x);
		access(y);
		splay(y);
	}
	void makeroot(int x)
	{
		access(x);splay(x);lz[x]^=1;swap(c[x][0],c[x][1]); pushup(x);
	}
	void link(int x,int y)
	{
		makeroot(x);
		if (x!=findroot(y))//可能已经连通
		{
			makeroot(y);f[x]=y;//虚子树变化
			sg[y]+=s[x];s[y]+=s[x];
		}
	}
	void cut(int x,int y)
	{
		makeroot(x);
		if (x==findroot(y))//可能本不连通
		{
			pushdown(x);
			if (c[x][1]==y&&!c[y][0]&&!c[y][1])//可能连通但无边
			{
				c[x][1]=f[y]=0;//可能需要修改
				pushup(x);
			}
		}
	}
};
const int N=2e5+2;
LCT<N> s;
int n,q,i,x,y,z,w;
int main()
{
	cin>>n>>q;s.init(n);
	for (i=1;i<=n;i++) cin>>x,s.s[i]=s.v[i]=x;
	for (i=1;i<n;i++)
	{
		cin>>x>>y;++x;++y;
		s.link(x,y);
	}
	while (q--)
	{
		cin>>x>>y>>z;++y;
		if (x==0)
		{
			cin>>x>>w;
			++z;++x;++w;
			s.cut(y,z);s.link(x,w);
			continue;
		}
		if (x==1)
		{
			s.split(y,y);
			s.s[y]=(s.v[y]+=z);
		}
		else
		{
			++z;
			s.split(y,z);
			printf("%lld\n",s.s[y]);
		}
	}
}
\end{lstlisting}

\subsection{轻重链剖分/DFS 序 LCA}

首先 \verb|init(n)|,然后正常存边($[1,n]$),然后 \verb|fun(root)|。

\verb|get_path| 会返回这条路径上的 \verb|dfn| 区间。

\begin{lstlisting}
namespace HLD
{
	const int N = 5e5 + 2;
	vector<int> e[N];
	int dfn[N], nfd[N], dep[N], f[N], siz[N], hc[N], top[N];
	int id, n;
	void dfs1(int u)
	{
		siz[u] = 1;
		for (int v : e[u]) if (v != f[u])
		{
			dep[v] = dep[f[v] = u] + 1;
			dfs1(v);
			siz[u] += siz[v];
			if (siz[v] > siz[hc[u]]) hc[u] = v;
		}
	}
	void dfs2(int u)
	{
		dfn[u] = ++id;
		nfd[id] = u;
		if (hc[u])
		{
			top[hc[u]] = top[u];
			dfs2(hc[u]);
			for (int v : e[u]) if (v != hc[u] && v != f[u]) dfs2(top[v] = v);
		}
	}
	int lca(int u, int v)
	{
		while (top[u] != top[v])
		{
			if (dep[top[u]] < dep[top[v]]) swap(u, v);
			u = f[top[u]];
		}
		if (dep[u] > dep[v]) swap(u, v);
		return u;
	}
	int dis(int u, int v)
	{
		return dep[u] + dep[v] - (dep[lca(u, v)] << 1);
	}
	void init(int _n)
	{
		n = _n;
		for (int i = 1; i <= n; i++)
		{
			e[i].clear();
			f[i] = hc[i] = 0;
		}
		id = 0;
	}
	void fun(int root)
	{
		dep[root] = 1; dfs1(root); dfs2(top[root] = root);
	}
	vector<pair<int, int>> get_path(int u, int v)//u->v,注意可能出现 [r>l](表示反过来走)
	{
		//cerr<<"path from "<<u<<" to "<<v<<": ";
		vector<pair<int, int>> v1, v2;
		while (top[u] != top[v])
		{
			if (dep[top[u]] > dep[top[v]]) v1.push_back({dfn[u], dfn[top[u]]}), u = f[top[u]];
			else v2.push_back({dfn[top[v]], dfn[v]}), v = f[top[v]];
		}
		v1.reserve(v1.size() + v2.size() + 1);
		v1.push_back({dfn[u], dfn[v]});
		reverse(v2.begin(), v2.end());
		for (auto v : v2) v1.push_back(v);
		//for (auto [x,y]:v1) cerr<<"["<<x<<','<<y<<"] ";cerr<<endl;
		return v1;
	}
}
using HLD::e, HLD::dfn, HLD::nfd, HLD::dep, HLD::f, HLD::siz, HLD::get_path;
using HLD::init;//5e5
namespace LCA
{
	using HLD::N, HLD::n;
	int st[__lg(N) + 1][N];
	int cmp(const int &x, const int &y) { return dep[x] < dep[y] ? x : y; }
	void fun(int rt)
	{
		HLD::fun(rt);
		assert(f[rt] == 0);
		for (int i = 1; i <= n; i++) st[0][dfn[i] - 1] = f[i];
		for (int j = 0; j < __lg(n); j++)
			for (int i = 1, k = n - (1 << j + 1); i <= k; i++) st[j + 1][i] = cmp(st[j][i], st[j][i + (1 << j)]);
	}
	int lca(int u, int v)
	{
		if (u == v) return u;
		u = dfn[u], v = dfn[v];
		if (u > v) swap(u, v);
		int g = __lg(v - u);
		return cmp(st[g][u], st[g][v - (1 << g)]);
	}
	int dis(int u, int v)
	{
		return dep[u] + dep[v] - (dep[lca(u, v)] << 1);
	}
}
using LCA::lca, LCA::fun, LCA::dis;
\end{lstlisting}

\subsection{换根树剖}

本质是对普通树剖在换根后的子树进行分类讨论。

设预处理的根是 $u$,当前根是 $v$,那么 $w$ 的子树如下:

\begin{enumerate}
	\item $w=v$,\verb|dfn| 区间为 $[1,n]$。
	\item $w$ 在 $u,v$ 之间,\verb|dfn| 区间为 $[1,n]$ 去掉 $w$ 前往 $v$ 方向的子树。找到这个子树的方法见 \verb|find| 函数。
	\item 其余情况,\verb|dfn| 区间和原来一致。
\end{enumerate}

$O(n+q\log n)$,$O(n)$。

\subsection{树上启发式合并,DSU on tree}

一种过时的、基于两次 \verb|dfs| 的写法,在复杂度要求不严时不如直接存储 \verb|set|。

流程:

\begin{enumerate}
	\item \verb|dfs| 轻子树计算答案,并清空全局统计信息。
	\item \verb|dfs| 重子树统计答案和全局信息。
	\item \verb|dfs| 轻子树统计全局信息。
\end{enumerate}

\begin{lstlisting}
void dfs1(int x)
{
	siz[x]=zdep[x]=1;
	int i;
	for (i=fir[x];i;i=nxt[i]) if (lj[i]!=f[x])
	{
		dep[lj[i]]=dep[f[lj[i]]=x]+1;
		dfs1(lj[i]);
		siz[x]+=siz[lj[i]];
		if (siz[hc[x]]<siz[lj[i]]) hc[x]=lj[i];
		zdep[x]=max(zdep[x],zdep[lj[i]]+1);
	}
}
void cal(int x)
{
	int i;
	dl[tou=wei=1]=x;
	while (tou<=wei)
	{
		++dp[dep[x=dl[tou++]]];
		if ((dp[dep[x]]>dp[zd])||(dp[dep[x]]==dp[zd])&&(dep[x]<zd)) zd=dep[x];
		for (i=fir[x];i;i=nxt[i]) if (lj[i]!=f[x]) dl[++wei]=lj[i];
	}
}
void dfs2(int x)
{
	if (!hc[x])
	{
		if (++dp[dep[x]]>dp[zd]) zd=dep[x];
		return;
	}
	int i;
	for (i=fir[x];i;i=nxt[i]) if ((lj[i]!=f[x])&&(lj[i]!=hc[x]))
	{
		dfs2(lj[i]);
		memset(dp+dep[lj[i]],0,zdep[lj[i]]<<2);
	}
	dfs2(hc[x]);
	dp[dep[x]]=1;
	if (dp[zd]<=1) zd=dep[x];
	for (i=fir[x];i;i=nxt[i]) if ((lj[i]!=f[x])&&(lj[i]!=hc[x])) cal(lj[i]);
	ans[x]=zd-dep[x];
}
\end{lstlisting}

\subsection{长链剖分($k$ 级祖先)}

$O(n\log n+q)$,$O(n)$。

\begin{lstlisting}
void dfs1(int x)
{
	int i;
	for (i = 1; i <= er[dep[x] - 1]; i++) f[x][i] = f[f[x][i - 1]][i - 1]; md[x] = dep[x];
	for (i = fir[x]; i; i = nxt[i]) { dep[lj[i]] = dep[x] + 1; dfs1(lj[i]); if (md[lj[i]] > md[dc[x]]) dc[x] = lj[i]; }
	if (dc[x]) md[x] = md[dc[x]];
}
void dfs2(int x)
{
	int i;
	if (dc[x])
	{
		top[dc[x]] = top[x];
		dfs2(dc[x]);
		for (i = fir[x]; i; i = nxt[i]) if (lj[i] != dc[x]) dfs2(top[lj[i]] = lj[i]);
	}
	if (x == top[x])
	{
		c = md[x] - dep[x]; y = x; up[x].push_back(x); down[x].push_back(x);
		for (i = 1; (i <= c) && (y = f[y][0]); i++) up[x].push_back(y); y = x;
		for (i = 1; i <= c; i++) down[x].push_back(y = dc[y]);
	}
}
int main()
{
	int n, q, ans = 0, x, y, c, i;
	ll ta = 0;
	cin >> n >> q >> s;
	for (i = 1; i <= n; i++) { cin >> f[i][0]; if (f[i][0] == 0) rt = i; else add(f[i][0], i); }
	for (i = 2; i <= n; i++) er[i] = er[i >> 1] + 1; dep[rt] = 1;
	dfs1(rt); dfs2(top[rt] = rt);
	for (i = 1; i <= q; i++)
	{
		x = (get(s) ^ ans) % n + 1; y = (get(s) ^ ans) % dep[x];
		//此时计算 x 的 y 级祖先。结果在 ans 中。
		if (y == 0) { ans = x; ta ^= (ll)i * ans; continue; }
		c = dep[x] - y; x = top[f[x][er[y]]];
		if (dep[x] > c) ans = up[x][dep[x] - c]; else ans = down[x][c - dep[x]];
		ta ^= (ll)i * ans;
	}
	cout << ta << endl;
}
\end{lstlisting}

\subsection{长链剖分(dp 合并)}

一种常见的实现方法是用指针指向同一片数组区域,使得从链头到链尾正好指向连续的一段数组,就不需要计算偏移量了。

$O(n)$,$O(n)$。

\begin{lstlisting}
void dfs1(int x)
{
	top[x]=1;
	for (int i=fir[x];i;i=nxt[i]) if (!top[lj[i]])
	{
		dfs1(lj[i]);
		if (len[lj[i]]>len[hc[x]]) hc[x]=lj[i];
	}
	len[x]=len[hc[x]]+1;top[hc[x]]=0;
}
void dfs2(int x)
{
	*f[x]=1;gs[x]=1;
	if (!hc[x]) return;
	ed[x]=1;f[hc[x]]=f[x]+1;
	for (int i=fir[x];i;i=nxt[i]) if (!ed[lj[i]]) dfs2(lj[i]);
	ans[x]=ans[hc[x]]+1;gs[x]=gs[hc[x]];
	if (gs[x]==1) ans[x]=0;
	for (int i=fir[x];i;i=nxt[i]) if ((!ed[lj[i]])&&(lj[i]!=hc[x]))
	{
		int v=lj[i],*p;
		for (int j=0;j<len[v];j++)
		{
			*(p=f[x]+j+1)+=*(f[v]+j);
			if (j+1==ans[x]) {gs[x]=*p;continue;}
			if ((*p>gs[x])||(*p==gs[x])&&(j+1<ans[x])) {gs[x]=*p;ans[x]=j+1;}
		}
	}
	gs[x]=*(f[x]+ans[x]);
	ed[x]=0;
}
\end{lstlisting}

\subsection{动态 dp(全局平衡二叉树)}

意义不大。

$O((n+q)\log n)$,$O(n)$。

\begin{lstlisting}
#include <stdio.h>
#include <string.h>
#include <algorithm>
#include <fstream>
using namespace std;
const int N=1e6+2,M=6e7+2,INF=-1e9;
struct matrix
{
	int a[2][2];
};
matrix s[N],js;
matrix operator *(matrix x,matrix y)
{
	js.a[0][0]=max(x.a[0][0]+y.a[0][0],x.a[0][1]+y.a[1][0]);
	js.a[0][1]=max(x.a[0][0]+y.a[0][1],x.a[0][1]+y.a[1][1]);
	js.a[1][0]=max(x.a[1][0]+y.a[0][0],x.a[1][1]+y.a[1][0]);
	js.a[1][1]=max(x.a[1][0]+y.a[0][1],x.a[1][1]+y.a[1][1]);
	return js;
}
int st[N],c[N][2],hc[N],lj[N<<1],nxt[N<<1],fir[N],siz[N],v[N],g[N][2],fa[N],f[N],val[N];
int n,m,i,j,x,y,z,dtp,stp,tp,fh,bs,rt,aaa,la;
char dr[M+5],sc[M];
void pushup(int x)
{
	s[x].a[0][0]=s[x].a[0][1]=g[x][0];
	s[x].a[1][0]=g[x][1];s[x].a[1][1]=INF;
	if (c[x][0]) s[x]=s[c[x][0]]*s[x];
	if (c[x][1]) s[x]=s[x]*s[c[x][1]];
}
void add(int x,int y)
{
	lj[++bs]=y;
	nxt[bs]=fir[x];
	fir[x]=bs;
	lj[++bs]=x;
	nxt[bs]=fir[y];
	fir[y]=bs;
}
bool nroot(int x)
{
	return ((c[f[x]][0]==x)||(c[f[x]][1]==x));
}
void dfs1(int x)
{
	siz[x]=1;
	int i;
	for (i=fir[x];i;i=nxt[i]) if (lj[i]!=fa[x])
	{
		fa[lj[i]]=x;
		dfs1(lj[i]);
		siz[x]+=siz[lj[i]];
		if (siz[hc[x]]<siz[lj[i]]) hc[x]=lj[i];
	}
}
int build(int l,int r)
{
	if (l>r) return 0;
	int i,tot=0,upn=0;
	for (i=l;i<=r;i++) tot+=val[i];tot>>=1;
	for (i=l;i<=r;i++)
	{
		upn+=val[i];
		if (upn>=tot)
		{
			f[c[st[i]][0]=build(l,i-1)]=st[i];
			f[c[st[i]][1]=build(i+1,r)]=st[i];
			pushup(st[i]);
			++aaa;
			return st[i];
		}
	}
}
int dfs2(int x)
{
	int i,j;
	for (i=x;i;i=hc[i]) for (j=fir[i];j;j=nxt[j]) if ((lj[j]!=fa[i])&&(lj[j]!=hc[i]))
	{
		f[y=dfs2(lj[j])]=i;
		g[i][0]+=max(s[y].a[0][0],s[y].a[1][0]);
		g[i][1]+=s[y].a[0][0];
	}
	tp=0;
	for (i=x;i;i=hc[i]) st[++tp]=i;
	for (i=1;i<tp;i++) val[i]=siz[st[i]]-siz[st[i+1]];
	val[tp]=siz[st[tp]];
	return build(1,tp);
}
void change(int x,int y)
{
	g[x][1]+=y-v[x];v[x]=y;
	while (f[x])
	{
		if (nroot(x)) pushup(x);
		else
		{
			g[f[x]][0]-=max(s[x].a[0][0],s[x].a[1][0]);
			g[f[x]][1]-=s[x].a[0][0];
			pushup(x);
			g[f[x]][0]+=max(s[x].a[0][0],s[x].a[1][0]);
			g[f[x]][1]+=s[x].a[0][0];
		}
		x=f[x];
	}
	pushup(x);
}
int main()
{
	scanf("%d%d",&n,&m);
	fread(dr+1,1,min(M,n*20+m*20),stdin);
	for (i=1;i<=n;i++)
	{
		read(g[i][1]);
		v[i]=g[i][1];
	}
	for (i=1;i<n;i++)
	{
		read(x);read(y);
		add(x,y);
	}
	dfs1(1);
	rt=dfs2(1);tp=0;
	while (m--)
	{
		read(x);read(y);
		change(x^la,y);
		x=la=max(s[rt].a[0][0],s[rt].a[1][0]);
		while (x)
		{
			st[++tp]=x%10;
			x/=10;
		}
		while (tp) sc[++stp]=st[tp--]|48;
		sc[++stp]=10;
	}
	fwrite(sc+1,1,stp,stdout);
}
\end{lstlisting}

\subsection{全局平衡二叉树(修改版)}

$O((n+q)\log n)$,$O(n)$。

\begin{lstlisting}
#include "bits/stdc++.h"
using namespace std;
typedef long long ll;
typedef pair<int, int> pa;
const int N = 1e6 + 2, M = 1e6 + 2;
ll ans;
pa w[N];
int c[N][2], f[N], fa[N], v[N], s[N], lz[N], lj[M], nxt[M], siz[N], hc[N], fir[N], st[N];
int a[N], top[N];
int n, i, x, y, z, bs, tp, rt;
void add()
{
	lj[++bs] = y; nxt[bs] = fir[x]; fir[x] = bs;
	lj[++bs] = x; nxt[bs] = fir[y]; fir[y] = bs;
}
void pushup(int &x)
{
	s[x] = min(v[x], min(s[c[x][0]], s[c[x][1]]));
}
void pushdown(int &x)
{
	if (lz[x] < 0)
	{
		int cc = c[x][0];
		if (cc)
		{
			lz[cc] += lz[x]; s[cc] += lz[x]; v[cc] += lz[x];
		}
		cc = c[x][1];
		if (cc)
		{
			v[cc] += lz[x]; lz[cc] += lz[x]; s[cc] += lz[x];
		}lz[x] = 0;
		return;
	}
}
bool nroot(int &x) { return c[f[x]][0] == x || c[f[x]][1] == x; }
bool cmp(pa &o, pa &p) { return o > p; }
void dfs1(int x)
{
	siz[x] = 1;
	for (int i = fir[x]; i; i = nxt[i]) if (lj[i] != fa[x])
	{
		fa[lj[i]] = x; dfs1(lj[i]); siz[x] += siz[lj[i]];
		if (siz[hc[x]] < siz[lj[i]]) hc[x] = lj[i];
	}
}
int build(int l, int r)
{
	if (l > r) return 0;
	if (l == r)
	{
		l = st[l]; s[l] = v[l] = siz[l] >> 1;
		return l;
	}
	int x = lower_bound(a + l, a + r + 1, a[l] + a[r] >> 1) - a, y = st[x];
	c[y][0] = build(l, x - 1);
	c[y][1] = build(x + 1, r);
	v[y] = siz[y] >> 1;
	if (c[y][0]) f[c[y][0]] = y;
	if (c[y][1]) f[c[y][1]] = y;
	pushup(y);
	return y;
}
void dfs2(int x)
{
	if (!hc[x]) return;
	int i;
	top[hc[x]] = top[x];
	if (top[x] == x)
	{
		st[tp = 1] = x;
		for (i = hc[x]; i; i = hc[i]) st[++tp] = i;
		for (i = 1; i <= tp; i++) a[i] = siz[st[i]] - siz[hc[st[i]]] + a[i - 1];
		f[build(1, tp)] = fa[x];
	}
	dfs2(hc[x]);
	for (i = fir[x]; i; i = nxt[i]) if (lj[i] != fa[x] && lj[i] != hc[x]) dfs2(top[lj[i]] = lj[i]);
}
void mdf(int x)
{
	int y = x;
	st[tp = 1] = x;
	while (y = f[y]) st[++tp] = y; y = x;
	while (tp) pushdown(st[tp--]);
	while (x)
	{
		--v[x]; --lz[c[x][0]]; --v[c[x][0]]; --s[c[x][0]];
		while (c[f[x]][0] == x) x = f[x]; x = f[x];
	}
	pushup(y);
	while (y = f[y]) pushup(y);
}
int ask(int x)
{
	int y = x;
	st[tp = 1] = x;
	while (y = f[y]) st[++tp] = y;
	while (tp) pushdown(st[tp--]);
	int r = v[x];
	while (x)
	{
		r = min(r, min(v[x], s[c[x][0]]));
		while (c[f[x]][0] == x) x = f[x]; x = f[x];
	}
	return r;
}
signed main()
{
	cin >> n; s[0] = 1e9;
	for (i = 1; i <= n; i++) cin >> w[w[i].second = i].first;
	for (i = 1; i < n; i++) cin >> x >> y, add();
	sort(w + 1, w + n + 1, cmp); dfs1(1); dfs2(top[1] = 1); rt = 1; while (f[rt]) rt = f[rt];
	for (i = 1; i <= n && v[rt]; i++) if (ask(w[i].second)) mdf(w[i].second), ans += w[i].first;
	cout << ans << endl;
}
\end{lstlisting}

\subsection{虚树}

传入点标号列表,返回虚树边表。自动认为 $1$ 是根,标号从 $1$ 开始。

需要注意的是:在清空的时候需要同时考虑点列表和边表,都清空一下。

你需要提供的是:\verb|dep|,\verb|lca|,\verb|dfn|。

$O(n+\sum k\log n)$,$O(n)$。

\begin{lstlisting}
vector<pair<int, int>> get_tree(vector<int> a)
{
	vector<pair<int, int>> edges;
	sort(all(a), [&](int u, int v) { return dfn[u]<dfn[v]; });
	vector<int> st(a.size()+2);
	int tp=0;
	auto ins=[&](int u)
		{
			if (tp==0)
			{
				st[tp=1]=u;
				return;
			}
			int v=lca(st[tp], u);
			while (tp>1&&dep[v]<dep[st[tp-1]])
			{
				edges.emplace_back(st[tp-1], st[tp]);
				--tp;
			}
			if (dep[v]<dep[st[tp]]) edges.emplace_back(v, st[tp--]);
			if (!tp||st[tp]!=v) st[++tp]=v;
			st[++tp]=u;
		};
	if (a[0]!=1) st[tp=1]=1;//先行添加根节点
	for (int u:a) ins(u);
	if (tp) while (--tp) edges.emplace_back(st[tp], st[tp+1]);//回溯
	return edges;
}
\end{lstlisting}

\subsection{圆方树}

题意:求仙人掌上两点最短路。

$O(n+m)$,$O(n+m)$。

\begin{lstlisting}
#include "bits/stdc++.h"
using namespace std;
#if !defined(ONLINE_JUDGE)&&defined(LOCAL)
#include "my_header\debug.h"
#else
#define dbg(...); 1;
#endif
typedef unsigned int ui;
typedef long long ll;
#define all(x) (x).begin(),(x).end()
const int N=3e4+2,M=3e4+2;//M 包括方点
struct P
{
	int v,w,id;
	P(int a,int b,int c):v(a),w(b),id(c){}
};
struct Q
{
	int v,w;
	Q(int a,int b):v(a),w(b){}
};
vector<P> e[N];
vector<Q> fe[M];
int dfn[M],low[N],st[N],len[M],top[M],siz[M],hc[M],dep[M],f[M],rb[N];
bool ed[M];//ed,dfn,loop,sum,fe,hc,tp,id,cnt,dep[1] 需初始化(注意倍率),ed 大小为边数
int tp,id,cnt,n;
void dfs1(int u)
{
	dfn[u]=low[u]=++id;
	st[++tp]=u;
	for (auto [v,w,id]:e[u]) if (!ed[id])
	{
		if (dfn[v]) low[u]=min(low[u],dfn[v]),rb[v]=w; else
		{
			ed[id]=1;
			dfs1(v);
			if (dfn[u]>low[v]) low[u]=min(low[u],low[v]),rb[v]=w; else
			{
				int ntp=tp;
				while (st[ntp]!=v) --ntp;
				if (ntp==tp)//圆圆边
				{
					--tp;
					fe[u].emplace_back(v,w);
					f[v]=u;
					continue;
				}
				++cnt;f[cnt]=u;
				for (int i=ntp;i<=tp;i++) f[st[i]]=cnt;
				len[st[ntp]]=w;
				for (int i=ntp+1;i<=tp;i++) len[st[i]]=len[st[i-1]]+rb[st[i]];
				len[cnt]=len[st[tp]]+rb[u];
				fe[u].emplace_back(cnt,0);
				for (int i=ntp;i<=tp;i++) fe[cnt].emplace_back(st[i],min(len[st[i]],len[cnt]-len[st[i]]));
				tp=ntp-1;
			}
		}
	}
}
void dfs2(int u)
{
	siz[u]=1;
	for (auto [v,w]:fe[u])
	{
		dep[v]=dep[u]+w;
		dfs2(v);
		siz[u]+=siz[v];
		if (siz[v]>siz[hc[u]]) hc[u]=v;
	}
}
void dfs3(int u)
{
	dfn[u]=++id;
	if (hc[u])
	{
		top[hc[u]]=top[u];
		dfs3(hc[u]);
		for (auto [v,w]:fe[u]) if (v!=hc[u]) dfs3(top[v]=v);
	}
}
int lca(int u,int v)
{
	while (top[u]!=top[v]) if (dfn[top[u]]>dfn[top[v]]) u=f[top[u]]; else v=f[top[v]];//注意不能用 dep
	return dfn[u]<dfn[v]?u:v;
}
int find(int u,int v)//u 是根
{
	if (dfn[hc[u]]+siz[hc[u]]>dfn[v]) return hc[u];
	while (f[top[v]]!=u) v=f[top[v]];
	return top[v];
}
int dis(int u,int v)
{
	int o=lca(u,v),r=dep[u]+dep[v];
	if (o<=n) return r-(dep[o]<<1);
	u=find(o,u);v=find(o,v);
	if (len[u]>len[v]) swap(u,v);
	return r+min(len[v]-len[u],len[o]-(len[v]-len[u]))-dep[u]-dep[v];
}
int main()
{
	ios::sync_with_stdio(0);cin.tie(0);
	int m,q,i;
	cin>>n>>m>>q;cnt=n;
	for (i=1;i<=m;i++)
	{
		int u,v,w;
		cin>>u>>v>>w;
		e[u].emplace_back(v,w,i);
		e[v].emplace_back(u,w,i);
	}
	mt19937 rnd(time(0));
	for (i=1;i<=n;i++) shuffle(all(e[i]),rnd);
	dfs1(1);id=0;
	dfs2(1);
	dfs3(top[1]=1);
	while (q--)
	{
		int u,v;
		cin>>u>>v;
		cout<<dis(u,v)<<'\n';
	}
}
\end{lstlisting}

\subsection{广义圆方树}

$O(n+m)$,$O(n+m)$。

\begin{lstlisting}
void dfs(int u)
{
	dfn[u]=low[u]=++id;
	st[++tp]=u;
	for (int v:e[u]) if (dfn[v]) low[u]=min(low[u],dfn[v]); else
	{
		dfs(v);
		low[u]=min(low[u],low[v]);
		if (dfn[u]<=low[v])
		{
			vector cur={u};
			do
			{
				cur.push_back(st[tp]);
			} while (st[tp--]!=v);
			ans.push_back(cur);
		}
	}
}
\end{lstlisting}

\subsection{支配树(DAG 版)}

其定义见一般图版。

$O(m\log n)$,$O(n\log n)$。

\begin{lstlisting}
int lca(int x, int y)
{
	int i;
	if (dep[x] < dep[y]) swap(x, y);
	for (i = lm[x]; dep[x] != dep[y]; i--) if (dep[f[x][i]] >= dep[y]) x = f[x][i];
	if (x == y) return x;
	for (i = lm[x]; f[x][0] != f[y][0]; i--) if (f[x][i] != f[y][i])
	{
		x = f[x][i]; y = f[y][i];
	}
	return f[x][0];
}
void dfs(int x)
{
	s[x] = 1;
	int i;
	for (i = sfir[x]; i; i = snxt[i])
	{
		dfs(slj[i]);
		s[x] += s[slj[i]];
	}
}
int main()
{
	dep[0] = -1;
	cin >> n;
	for (i = 1; i <= n; i++)
	{
		cin >> x;
		while (x)
		{
			add(x, i);
			cin >> x;
		}
	}
	dl[tou = wei = 1] = ++n;
	for (i = 1; i < n; i++) if (!rd[i]) add(n, i);
	while (tou <= wei)
	{
		for (i = fir[x = dl[tou++]]; i; i = nxt[i]) if (--rd[lj[i]] == 0) dl[++wei] = lj[i];
		if (i = ffir[x])
		{
			y = flj[i];
			while (i = fnxt[i]) y = lca(y, flj[i]);
			f[x][0] = y;
		}
		else y = 0;
		sadd(y, x);
		f[x][0] = y;
		for (i = 1; i <= 16; i++) if (0 == (f[x][i] = f[f[x][i - 1]][i - 1]))
		{
			lm[x] = i;
			break;
		}
		dep[x] = dep[y] + 1;
	}
	dfs(n);
	for (i = 1; i < n; i++) printf("%d\n", s[i] - 1);
}
\end{lstlisting}

\subsection{支配树(一般图)}

$u$ 支配 $v$ 指的是从 $S$ 到 $v$ 的路径必然经过 $u$。支配树是保持支配关系不变的树,其中 $s$ 是根,$idom[u]$ 是 $u$ 的父节点。

\begin{lstlisting}
vector<int> dom_tree(vector<vector<int>> e, int s)//[1,n]
{
	int n = e.size() - 1, i, id = 0;
	vector<vector<int>> c(n + 1), buc(c), ie(c);
	vector<int> mn(n + 1), f(n + 1), sdom(n + 1), idom(n + 1), dfn(n + 1), nfd(n + 1), pv(n + 1), ed(n + 1);
	auto cmp = [&](int x, int y) {return dfn[x] < dfn[y] ? x : y; };
	auto cmp2 = [&](int x, int y) {return dfn[sdom[x]] < dfn[sdom[y]] ? x : y; };
	function<void(int)> getf = [&](int u) {
		if (f[u] == u) return;
		getf(f[u]);
		mn[u] = cmp2(mn[u], mn[f[u]]);
		f[u] = f[f[u]];
	};
	for (i = 1; i <= n; i++) mn[i] = f[i] = i;
	function<void(int)> dfs = [&](int u) {
		ed[u] = 1;
		for (int v : e[u]) if (!ed[v]) dfs(v);
	};
	dfs(s);
	for (i = 1; i <= n; i++) if (ed[i]) erase_if(e[i], [&](int v) { return !ed[v]; });
	else e[i].clear();
	for (i = 1; i <= n; i++) for (int v : e[i]) ie[v].push_back(i);
	dfs = [&](int u) {
		nfd[dfn[u] = ++id] = u;
		for (int v : e[u]) if (!dfn[v]) dfs(v), c[u].push_back(v);
	};
	dfs(s); dfn[0] = 1e9;
	for (i = id; i; i--)
	{
		int u = nfd[i], w = 0;
		for (int v : ie[u])
		{
			sdom[u] = cmp(sdom[u], v);
			if (dfn[v] > dfn[u])
			{
				getf(v);
				w = cmp2(w, mn[v]);
			}
		}
		sdom[u] = cmp(sdom[u], sdom[w]);
		buc[sdom[u]].push_back(u);
		for (int v : buc[u]) getf(v), pv[v] = mn[v];
		for (int v : c[u]) f[v] = u, mn[v] = cmp2(mn[v], mn[u]);
	}
	for (i = 1; i <= n; i++) idom[nfd[i]] = (sdom[pv[nfd[i]]] == sdom[nfd[i]]) ? sdom[nfd[i]] : idom[pv[nfd[i]]];
	idom[s] = s;
	return idom;
}
int main()
{
	int n, m, s;
	cin >> n >> m >> s; ++s;
	vector<vector<int>> e(n + 1);
	for (int i = 1; i <= m; i++)
	{
		int u, v;
		cin >> u >> v; ++u; ++v;
		e[u].push_back(v);
	}
	auto r = dom_tree(e, s);
	for (int i = 1; i <= n; i++) cout << r[i] - 1 << " \n"[i == n];
}

\end{lstlisting}

\subsection{最小乘积生成树}

题意:每条边有两个属性 $x_i,y_i$,你需要最小化 $(\sum x_i)(\sum y_i)$。

你需要实现的是 \verb|sol1|,即按照 $val$ 为权值的答案。$val_i$ 是根据 $x_i,y_i$ 计算的。

\begin{lstlisting}
#include "bits/stdc++.h"
using namespace std;
typedef long long ll;
const int N = 202, M = 10002;
struct P
{
	int x, y;
	P(int a = 0, int b = 0) :x(a), y(b) { }
	bool operator<(const P &o) const { return (ll)x * y < (ll)o.x * o.y || (ll)x * y == (ll)o.x * o.y && x < o.x; }
};
struct Q
{
	int u, v, x, y, val;
	bool operator<(const Q &o) const { return val < o.val; }
};
P ans = P(1e9, 1e9), l, r;
Q a[M];
int f[N];
int n, m, i;
int getf(int x)
{
	if (f[x] == x) return x;
	return f[x] = getf(f[x]);
}
P sol1()
{
	P r = P(0, 0);
	for (i = 1; i <= n; i++) f[i] = i;
	sort(a + 1, a + m + 1);
	for (i = 1; i <= m; i++) if (getf(a[i].u) != getf(a[i].v))
	{
		f[f[a[i].u]] = f[a[i].v];
		r.x += a[i].x, r.y += a[i].y;
	}
	return r;
}
void sol2(P l, P r)
{
	for (i = 1; i <= m; i++) a[i].val = (r.x - l.x) * a[i].y + (l.y - r.y) * a[i].x;
	P np = sol1();
	ans = min(ans, np);
	if ((ll)(r.x - l.x) * (np.y - l.y) - (ll)(r.y - l.y) * (np.x - l.x) >= 0) return;
	sol2(l, np); sol2(np, r);
}
int main()
{
	cin >> n >> m;
	for (i = 1; i <= m; i++) cin >> a[i].u >> a[i].v >> a[i].x >> a[i].y, ++a[i].u, ++a[i].v;
	for (i = 1; i <= m; i++) a[i].val = a[i].x; l = sol1();
	for (i = 1; i <= m; i++) a[i].val = a[i].y; r = sol1();
	ans = min(ans, min(l, r)); sol2(l, r);
	cout<<ans.x<<' '<<ans.y<<endl;
}
\end{lstlisting}

\subsection{最小斯坦纳树}

题意:让给定点集连通的最小生成树(不要求全图连通)

$O(3^kn+2^km\log m)$。

\begin{lstlisting}
const int N = 102, M = 1002, K = 1024;
typedef long long ll;
typedef pair<ll, int> pa;
priority_queue<pa, vector<pa>, greater<pa> > heap;
pa cr;
ll f[K][N], inf;
int lj[M], len[M], nxt[M], fir[N];
int n, m, q, i, j, k, x, y, z, bs, c;
void add()
{
	lj[++bs] = y;
	len[bs] = z;
	nxt[bs] = fir[x];
	fir[x] = bs;
	lj[++bs] = x;
	len[bs] = z;
	nxt[bs] = fir[y];
	fir[y] = bs;
}
void dijk(int s)
{
	int i;
	while (!heap.empty())
	{
		x = heap.top().second; heap.pop();
		for (i = fir[x]; i; i = nxt[i]) if (f[s][lj[i]] > f[s][x] + len[i])
		{
			cr.first = f[s][cr.second = lj[i]] = f[s][x] + len[i];
			heap.push(cr);
		}
		while ((!heap.empty()) && (heap.top().first != f[s][heap.top().second])) heap.pop();
	}
}
int main()
{
	memset(f, 0x3f, sizeof(f)); inf = f[0][0];
	cin >> n >> m >> q;
	while (m--)
	{
		cin >> x >> y >> z;
		add();
	}
	for (i = 1; i <= q; i++)
	{
		cin >> x;
		f[1 << i - 1][x] = 0;
	}
	q = (1 << q) - 1;
	for (i = 1; i <= q; i++)
	{
		for (k = 1; k <= n; k++)
		{
			for (j = i & (i - 1); j; j = i & (j - 1)) f[i][k] = min(f[i][k], f[j][k] + f[i ^ j][k]);
			if (f[i][k] < inf) heap.push(pa(f[i][k], k));
		}
		dijk(i);
	}
	for (i = 1; i <= n; i++) inf = min(inf, f[q][i]);
	cout << inf << endl;
}
\end{lstlisting}

\subsection{$2$-sat}

支持添加一个条件 \verb|add(u,x,v,y)|,表示 $a_u=x\Rightarrow a_v=y$。支持设定一个变量的值。

$O(n+m)$,$O(n+m)$。

\begin{lstlisting}
struct sat
{
	vector<vector<int>> e;
	vector<int> dfn,low,st,f,ed;
	int fs,tp,id,n;
	sat(int n):n(n),e(n*2),dfn(n*2,-1),low(n*2),st(n*2),f(n*2,-1),ed(n*2),fs(0),tp(-1),id(0){}
	void dfs(int u)
	{
		dfn[u]=low[u]=id++;
		ed[u]=1;st[++tp]=u;
		for (int v:e[u]) if (dfn[v]!=-1)
		{
			if (ed[v]) low[u]=min(low[u],dfn[v]);
		} else dfs(v),low[u]=min(low[u],low[v]);
		if (dfn[u]==low[u])
		{
			do
			{
				f[st[tp]]=fs;
				ed[st[tp]]=0;
			} while (st[tp--]!=u);
			++fs;
		}
	}
	void add(int u,bool x,int v,bool y)
	{
		assert(u>=0&&u<n&&v>=0&&v<n);
		e[u+x*n].push_back(v+y*n);
		e[v+(y^1)*n].push_back(u+(x^1)*n);
	}
	void set(int u,bool x)
	{
		assert(u>=0&&u<n);
		e[u+(x^1)*n].push_back(u+x*n);
	}
	vector<int> getans()
	{
		int i;
		for (i=0;i<n*2;i++) if (dfn[i]==-1) dfs(i);
		vector<int> r(n);
		for (i=0;i<n;i++)
		{
			if (f[i]==f[i+n]) return {};
			r[i]=f[i]>f[i+n];
		}
		return r;
	}
};
\end{lstlisting}

\subsection{Kosaraju 强连通分量(bitset 优化)}

实用意义不大。

$O(\frac{n^2}w)$,$O(\frac {n^2}w)$。

\begin{lstlisting}
void dfs1(int x)
{
	int i; ed[x] = 0;
	for (i = (lj[x] & ed)._Find_first(); i <= n; i = (lj[x] & ed)._Find_next(i)) dfs1(i);
	sx[--tp] = x;
}
void dfs2(int x)
{
	int i; ed[x] = 0; tv[f[x] = f[0]] += v[x];
	for (i = (fj[x] & ed)._Find_first(); i <= n; i = (fj[x] & ed)._Find_next(i)) dfs2(i);
}
int main()
{
	cin >> n >> m;
	tp = n + 1;
	for (i = 1; i <= n; i++) { ed[i] = 1; cin >> v[i]; }
	for (i = 1; i <= m; i++)
	{
		cin >> x >> y;
		lj[x][y] = 1; fj[y][x] = 1; lb[i][0] = x; lb[i][1] = y;
	}
	for (i = 1; i <= n; i++) if (ed[i]) dfs1(i);
	ed.set();
	for (i = 1; i <= n; i++) if (ed[sx[i]]) { ++f[0]; dfs2(sx[i]); }
	for (i = 1; i <= m; i++) if (f[lb[i][0]] != f[lb[i][1]])
	{
		flj[f[lb[i][0]]].push_back(f[lb[i][1]]); ++rd[f[lb[i][1]]];
	}
	for (i = 1; i <= f[0]; i++) if (!rd[i]) dl[++wei] = i;
	while (tou <= wei)
	{
		x = dl[tou++]; g[x] += tv[x];
		for (i = 0; i < flj[x].size(); i++)
		{
			g[flj[x][i]] = max(g[flj[x][i]], g[x]);
			if (--rd[flj[x][i]] == 0) dl[++wei] = flj[x][i];
		}
	}
	for (i = 1; i <= f[0]; i++) ans = max(ans, g[i]); printf("%d", ans);
}
\end{lstlisting}

\subsection{Tarjan 强连通分量}

$O(n+m)$,$O(n+m)$。

\begin{lstlisting}
int dfn[N],low[N],st[N],f[N],fs,tp,id;
bool ed[N];
void tarjan(int u)
{
	dfn[u]=low[u]=++id;
	ed[u]=1;st[++tp]=u;
	for (int v:e[u]) if (dfn[v])
	{
		if (ed[v]) low[u]=min(low[u],dfn[v]);
	} else tarjan(v),low[u]=min(low[u],low[v]);
	if (dfn[u]==low[u])
	{
		++fs;
		do
		{
			f[st[tp]]=fs;
			ed[st[tp]]=0;
		} while (st[tp--]!=u);
	}
}

\end{lstlisting}

\subsection{动态强连通分量}

给出一个加边序列,\verb|solve| 会返回每个时间进入强连通分量的边。点标号范围是 $[0,n)$

\begin{lstlisting}

struct union_set
{
	vector<int> f;
	int n;
	union_set() { }
	union_set(int nn) :n(nn), f(nn+1)
	{
		iota(all(f), 0);
	}
	int getf(int u) { return f[u]==u ? u : f[u] = getf(f[u]); }
	bool merge(int u, int v)
	{
		u = getf(u); v = getf(v);
		if (u==v) return 0;
		f[u] = v;
		return 1;
	}
	bool connected(int u, int v) { return getf(u)==getf(v); }
};
struct edge
{
	int u, v, t;
};
vector<vector<edge>> solve(int n, const auto& eg)//[0,n)
{
	int m = eg.size(), tp = -1, id = 0, fs = 0;
	vector<vector<edge>> res(m);
	vector e(n, vector<int>());
	vector<int> dfn(n, -1), low(n, -1), st(n), ed(n), blk(n), node;
	union_set s(n-1);
	function<void(int)> dfs = [&](int u)
		{
			dfn[u] = low[u] = id++;
			ed[st[++tp] = u] = 1;
			for (int v : e[u]) if (dfn[v]!=-1)
			{
				if (ed[v]) cmin(low[u], dfn[v]);
			}
			else dfs(v), cmin(low[u], low[v]);
			if (dfn[u]==low[u])
			{
				do
				{
					ed[st[tp]] = 0;
					blk[st[tp]] = fs;
				} while (st[tp--]!=u);
				++fs;
			}
		};
	auto ztef = [&](auto ztef, int l, int r, const vector<edge>& q)
		{
			if (eg.size()==0) return;
			if (l+1==r)
			{
				if (l<m)
				{
					res[l].insert(res[l].end(), all(q));
					for (auto [u, v, t]:q) s.merge(u, v);
				}
				return;
			}
			int m = (l+r)/2;
			node.clear();
			for (auto [u, v, t]:q) if (t<m)
			{
				u = s.getf(u);
				v = s.getf(v);
				e[u].push_back(v);
				node.push_back(u);
				node.push_back(v);
			}
			else break;
			for (int u : node) if (dfn[u]==-1) dfs(u);
			vector<vector<edge>> g(2);
			for (auto [u, v, t]:q) g[t<m&&blk[s.f[u]]==blk[s.f[v]]].push_back({u, v, t});
			for (int u : node)
			{
				e[u].clear();
				dfn[u] = low[u] = -1;
			}
			id = fs = 0;
			ztef(ztef, l, m, g[1]);
			ztef(ztef, m, r, g[0]);
		};
	ztef(ztef, 0, m+1, eg);
	return res;
}
int main()
{
	ios::sync_with_stdio(0); cin.tie(0);
	cout<<fixed<<setprecision(15);
	int n, m, i, j;
	cin>>n>>m;
	vector<ll> x(n);
	cin>>x;
	vector<edge> edges(m);
	for (i = 0;i<m;i++)
	{
		auto& [u, v, t] = edges[i];
		cin>>u>>v;
		t = i;
	}
	auto event = solve(n, edges);
	union_set s(n-1);
	ll ans = 0;
	for (auto e:event)
	{
		for (auto [u, v, t]:e)
		{
			u = s.getf(u);
			v = s.getf(v);
			if (u==v) continue;
			s.f[v] = u;
			(ans += x[u]*x[v]) %= p;
			(x[u] += x[v]) %= p;
		}
		cout<<ans<<'\n';
	}
}
\end{lstlisting}
\subsection{欧拉路径(字典序最小)}

\begin{lstlisting}
#include "bits/stdc++.h"
using namespace std;
#if !defined(ONLINE_JUDGE)&&defined(LOCAL)
#include "my_header\debug.h"
#else
#define dbg(...); 1;
#endif
typedef unsigned int ui;
typedef long long ll;
#define all(x) (x).begin(),(x).end()
const int N=1e5+2;
vector<int> e[N];
int rd[N],cd[N];
vector<int> ans;
void dfs(int u)
{
	while (e[u].size())
	{
		int v=e[u].back();
		e[u].pop_back();
		dfs(v);
		ans.push_back(v);
	}
}
int main()
{
	ios::sync_with_stdio(0);cin.tie(0);
	int n,m,i,x=0;
	cin>>n>>m;ans.reserve(m);
	while (m--)
	{
		int u,v;
		cin>>u>>v;
		e[u].push_back(v);
		++cd[u];++rd[v];
	}
	for (i=1;i<=n;i++) if (cd[i]!=rd[i])
	{
		if (abs(cd[i]-rd[i])>1) goto no;
		++x;
	}
	if (x>2) goto no;x=1;
	for (i=1;i<=n;i++) if (cd[i]>rd[i]) {x=i;break;}
	for (i=1;i<=n;i++) sort(all(e[i])),reverse(all(e[i]));
	dfs(x);ans.push_back(x);reverse(all(ans));
	for (i=0;i<ans.size();i++) cout<<ans[i]<<" \n"[i+1==ans.size()];
	return 0;
	no:cout<<"No"<<endl;
}
\end{lstlisting}

\subsection{欧拉回/通路构造}

$O(n+m)$,$O(n+m)$。

\begin{lstlisting}
optional<vector<int>> undirected_euler_cycle(int n,const vector<pair<int,int>> &edges)//[1,n]/[1,m], 正数表示正向,负数表示反向
{
	int i=0;
	vector<int> rd(n+1),ed(edges.size()+1),r;
	vector<vector<pair<int,int>>> e(n+1);
	for (auto [u,v]:edges)
	{
		++rd[u],++rd[v];
		e[u].push_back({v,++i});
		e[v].push_back({u,-i});
	}
	for (i=1;i<=n;i++) if (rd[i]&1) return {};
	function<void(int)> dfs=[&](int u)
	{
		while (e[u].size())
		{
			auto [v,w]=e[u].back();
			e[u].pop_back();
			if (ed[abs(w)]) continue;
			ed[abs(w)]=1;
			dfs(v);
			r.push_back(w);
		}
	};
	for (i=1;i<=n;i++) if (rd[i]) {dfs(i);break;}
	reverse(all(r));
	if (r.size()!=edges.size()) return {};
	return {r};
}
optional<vector<int>> directed_euler_cycle(int n,const vector<pair<int,int>> &edges)//[1,n]/[1,m]
{
	int i=0;
	vector<int> rd(n+1),cd(n+1),r;
	vector<vector<pair<int,int>>> e(n+1);
	for (auto [u,v]:edges)
	{
		++cd[u],++rd[v];
		e[u].push_back({v,++i});
	}
	for (i=1;i<=n;i++) if (rd[i]!=cd[i]) return {};
	function<void(int)> dfs=[&](int u)
	{
		while (e[u].size())
		{
			auto [v,w]=e[u].back();
			e[u].pop_back();
			dfs(v);
			r.push_back(w);
		}
	};
	for (i=1;i<=n;i++) if (cd[i]) {dfs(i);break;}
	reverse(all(r));
	if (r.size()!=edges.size()) return {};
	return {r};
}
optional<vector<int>> undirected_euler_trail(int n,const vector<pair<int,int>> &edges)//[1,n]/[1,m], 正数表示正向,负数表示反向
{
	int i=0;
	vector<int> rd(n+1),ed(edges.size()+1),r;
	vector<vector<pair<int,int>>> e(n+1);
	for (auto [u,v]:edges)
	{
		++rd[u],++rd[v];
		e[u].push_back({v,++i});
		e[v].push_back({u,-i});
	}
	int odd=0;
	for (i=1; i<=n; i++) odd+=rd[i]&1;
	if (odd>2) return { };
	function<void(int)> dfs=[&](int u)
		{
			while (e[u].size())
			{
				auto [v,w]=e[u].back();
				e[u].pop_back();
				if (ed[abs(w)]) continue;
				ed[abs(w)]=1;
				dfs(v);
				r.push_back(w);
			}
		};
	for (i=1; i<=n; i++) if (rd[i]&1) { dfs(i); break; }
	if (i>n)
	{
		for (i=1; i<=n; i++) if (rd[i]) { dfs(i); break; }
	}
	reverse(all(r));
	if (r.size()!=edges.size()) return { };
	return {r};
}
optional<vector<int>> directed_euler_trail(int n,const vector<pair<int,int>> &edges)//[1,n]/[1,m]
{
	int i=0;
	vector<int> rd(n+1),cd(n+1),r;
	vector<vector<pair<int,int>>> e(n+1);
	for (auto [u,v]:edges)
	{
		++cd[u],++rd[v];
		e[u].push_back({v,++i});
	}
	int diff=0;
	for (i=1; i<=n; i++)
	{
		if (abs(rd[i]-cd[i])>1) return { };
		if (rd[i]!=cd[i]) ++diff;
	}
	if (diff>2) return { };
	function<void(int)> dfs=[&](int u)
		{
			while (e[u].size())
			{
				auto [v,w]=e[u].back();
				e[u].pop_back();
				dfs(v);
				r.push_back(w);
			}
		};
	for (i=1; i<=n; i++) if (cd[i]>rd[i]) { dfs(i); break; }
	if (i>n)
	{
		for (i=1; i<=n; i++) if (cd[i]) { dfs(i); break; }
	}
	reverse(all(r));
	if (r.size()!=edges.size()) return { };
	return {r};
}
\end{lstlisting}

\subsection{有向图欧拉回路计数(BEST 定理)/生成树计数}

$O(n^3)$,$O(n^2)$。

以 $u$ 为起点的欧拉回路个数 $sum=T(u)\times \prod\limits_{v=1}^n(out(v)-1)!$,其中 $T(u)$ 是以 $u$ 为根的内向树个数(出度矩阵-邻接矩阵),$out(v)$ 是 $v$ 的出度。若允许循环同构(如 $1\to 2\to 1\to 3\to 1$ 与 $1\to 3\to 1\to 2\to 1$),还需多乘 $out(u)$。

这里的部分代码是未经验证的。

\begin{lstlisting}
ll det(vector<vector<ll>> b)
{
	ll r=1;
	int n=b.size(), i, j, k;
	for (i=0; i<n; i++)
	{
		for (j=i; j<n; j++) if (b[j][i]) break;
		if (j==n) return 0;
		swap(b[j], b[i]);
		if (j!=i) r=(p-r)%p;
		r=r*b[i][i]%p;
		b[i][i]=ksm(b[i][i], p-2);
		for (j=n-1; j>=i; j--) b[i][j]=b[i][j]*b[i][i]%p;
		for (j=i+1; j<n; j++) for (k=n-1; k>=i; k--) b[j][k]=(b[j][k]+(p-b[j][i])*b[i][k])%p;
	}
	return r;
}
ll eular_path_count(vector<vector<int>> a, int s, int t)
{
	int n=a.size(), i, j, k;
	++a[t][s]; s=t;
	vector<int> rd(n), cd(n);
	for (i=0; i<n; i++) for (j=0; j<n; j++) cd[i]+=a[i][j], rd[j]+=a[i][j];
	for (i=0; i<n; i++) if (cd[i]!=rd[i]) return 0;
	vector<int> f(n);
	iota(all(f), 0);
	function<int(int)> getf=[&](int u) { return f[u]==u?u:f[u]=getf(f[u]); };
	for (i=0; i<n; i++) for (j=0; j<n; j++) if (a[i][j]) f[getf(i)]=getf(j);
	ll r=1;
	vector<int> id;
	for (i=0; i<n; i++) if (cd[i])
	{
		if (getf(i)!=getf(s)) return 0;
		r=r*fac[cd[i]-1]%p;
		if (i!=s) id.push_back(i);
	}
	n=id.size();
	vector b(n, vector<ll>(n));
	for (i=0; i<n; i++)
	{
		b[i][i]=cd[id[i]]-a[id[i]][id[i]];
		for (j=0; j<n; j++) if (i!=j) b[i][j]=(p-a[id[i]][id[j]])%p;
	}
	return r*det(b)%p;
}
ll eular_path_count(vector<vector<int>> a)
{
	int n=a.size(), i, j, s=-1, t=-1;
	vector<int> rd(n), cd(n), d(n);
	for (i=0; i<n; i++) for (j=0; j<n; j++) cd[i]+=a[i][j], rd[j]+=a[i][j];
	if (count(all(cd), 0)==n) return 1;
	for (i=0; i<n; i++) d[i]=cd[i]-rd[i];
	s=max_element(all(d))-d.begin();
	t=min_element(all(d))-d.begin();
	ll r=0;
	if (s==t)
	{
		for (i=0; i<n; i++) if (cd[i]) r+=eular_path_count(a, i, i);
	}
	else r=eular_path_count(a, s, t);
	return r%p;
}
ll eular_circuit_count(vector<vector<int>> a)
{
	int n=a.size(), i, j;
	for (i=0; i<n; i++) for (j=0; j<n; j++) if (a[i][j]) return eular_path_count(a, i, i)*ksm(accumulate(all(a[i]), 0llu)%p, p-2)%p;
	return 1;
}
ll directed_spanning_tree_count(vector<vector<int>> a, int s)
{
	int n=a.size(), i, j;
	vector b(n-1, vector<ll>(n-1));
	for (i=0; i<n; i++) a[i][i]=0;
	for (i=0; i<n; i++) if (i!=s) for (j=0; j<n; j++) if (j!=s&&i!=j) b[i-(i>s)][j-(j>s)]=(p-a[i][j])%p;
	for (i=0; i<n; i++) if (i!=s) for (j=0; j<n; j++) (b[i-(i>s)][i-(i>s)]+=a[j][i])%=p;
	return det(b);
}//外向
ll undirected_spanning_tree_count(vector<vector<int>> a)
{
	int n=a.size(), i, j;
	--n;
	vector b(n, vector<ll>(n));
	for (i=0; i<n; i++) a[i][i]=0;
	for (i=0; i<n; i++) for (j=0; j<n; j++) if (i!=j) b[i][j]=(p-a[i][j])%p;
	for (i=0; i<n; i++) b[i][i]=reduce(all(a[i]), 0llu)%p;
	return det(b);
}
\end{lstlisting}

\subsection{点染色}

结论:$\chi(G)\le \Delta(G)+1$,其中 $\Delta(G)$ 是图的最大度。只有奇圈和完全图取等。

构造方案只能爆搜。

\begin{lstlisting}
vector<int> chromatic_number(int n,const vector<pair<int,int>> &edges)//[0,n)
{
	vector r(n,-1),cur(n,-1);
	vector<vector<int>> e(n);
	int ans=0,i;
	for (auto [u,v]:edges) e[u].push_back(v),e[v].push_back(u);
	for (i=0;i<n;i++) ans=max(ans,(int)e[i].size());
	ans+=2;
	vector p(n,vector(ans,0));
	function<void(int)> dfs=[&](int u)
	{
		int col=u?*max_element(cur.begin(),cur.begin()+u)+1:0;
		if (col>=ans) return;
		if (u==n)
		{
			r=cur;
			ans=col;
			return;
		}
		int i;
		for (int i=0;i<=col;i++) if (!p[u][i])
		{
			cur[u]=i;
			for (int v:e[u]) ++p[v][i];
			dfs(u+1);
			for (int v:e[u]) --p[v][i];
		}
	};
	dfs(0);
	return r;
}
\end{lstlisting}

\subsection{最大独立集}

爆搜。

\begin{lstlisting}
vector<int> indep_set(int n,const vector<pair<int,int>> &edges)//[0,n)
{
	vector<vector<int>> e(n);
	mt19937 rnd(998);
	vector<int> p(n),q(n),ed(n);
	iota(all(p),0);
	shuffle(all(p),rnd);
	for (int i=0;i<n;i++) q[p[i]]=i;
	for (auto [u,v]:edges)
	{
		e[p[u]].push_back(p[v]);
		e[p[v]].push_back(p[u]);
	}
	vector<int> r,cur;
	function<void(int)> dfs=[&](int u)
	{
		if (cur.size()+n-u<=r.size()) return;
		if (u==n)
		{
			r=cur;
			return;
		}
		if (!ed[u])
		{
			cur.push_back(u);
			for (int v:e[u]) ++ed[v];
			dfs(u+1);
			for (int v:e[u]) --ed[v];
			cur.pop_back();
		}
		if (ed[u]||e[u].size()) dfs(u+1);
	};dfs(0);
	for (int &x:r) x=q[x];
	sort(all(r));
	return r;
}
\end{lstlisting}


\newpage

\section{计算几何}

\subsection{自适应 simpson 法}

\verb|sim(l,r)| 计算 $\int_l^r f(x)\dif x$

\begin{lstlisting}
const db eps=1e-7;
db sl,sr,sm,a;
db f(db x)
{
	return pow(x,a/x-x);
}
db g(db l,db r)
{
	db mid=(l+r)*0.5;
	return (f(l)+f(r)+f(mid)*4)/6*(r-l);
}
db sim(db l,db r)
{
	db mid=(l+r)*0.5;
	sl=g(l,mid);sr=g(mid,r);sm=g(l,r);
	if (abs(sl+sr-sm)<eps) return sl+sr;
	return sim(l,mid)+sim(mid,r);
}
\end{lstlisting}

\subsection{计算几何全}

功能其实比较少,因为实际遇到的几何题不多。最有用的可能是闵可夫斯基和合并凸包,和常规的线段判交之类的。其余功能最好直接使用 HDU 板。

\begin{lstlisting}
namespace geometry//不要用 int!
{
#define tmpl template<class T>
	typedef long long ll;
	typedef long double db;
	const db eps = 1e-6;
#define all(x) (x).begin(),(x).end()
	inline int sgn(const ll &x)
	{
		if (x < 0) return -1;
		return x > 0;
	}
	inline int sgn(const db &x)
	{
		if (fabs(x) < eps) return 0;
		return x > 0 ? 1 : -1;
	}
	tmpl struct point//* 为叉乘,& 为点乘,只允许使用 (long )double 和 ll
	{
		T x, y;
		point() { }
		point(T a, T b) :x(a), y(b) { }
		operator point<ll>() const { return point<ll>(x, y); }
		operator point<db>() const { return point<db>(x, y); }
		point<T> operator+(const point<T> &o) const { return point(x + o.x, y + o.y); }
		point<T> operator-(const point<T> &o) const { return point(x - o.x, y - o.y); }
		point<T> operator*(const T &k) const { return point(x * k, y * k); }
		point<T> operator/(const T &k) const { return point(x / k, y / k); }
		T operator*(const point<T> &o) const { return x * o.y - y * o.x; }
		T operator&(const point<T> &o) const { return x * o.x + y * o.y; }
		void operator+=(const point<T> &o) { x += o.x; y += o.y; }
		void operator-=(const point<T> &o) { x -= o.x; y -= o.y; }
		void operator*=(const T &k) { x *= k; y *= k; }
		void operator/=(const T &k) { x /= k; y /= k; }
		bool operator==(const point<T> &o) const { return x == o.x && y == o.y; }
		bool operator!=(const point<T> &o) const { return x != o.x || y != o.y; }
		db len() const { return sqrt(len2()); }//模长
		T len2() const { return x * x + y * y; }
	};
	const point<db> npos = point<db>(514e194, 9810e191), apos = point<db>(145e174, 999e180);
	const int DS[4] = {1, 2, 4, 3};
	tmpl int quad(const point<T> &o)//坐标轴归右上象限,返回值 [1,4]
	{
		return DS[(sgn(o.y) < 0) * 2 + (sgn(o.x) < 0)];
	}
	tmpl bool angle_cmp(const point<T> &a, const point<T> &b)
	{
		int c = quad(a), d = quad(b);
		if (c != d) return c < d;
		return a * b > 0;
	}
	tmpl db dis(const point<T> &a, const point<T> &b) { return (a - b).len(); }
	tmpl T dis2(const point<T> &a, const point<T> &b) { return (a - b).len2(); }
	tmpl point<T> operator*(const T &k, const point<T> &o) { return point<T>(k * o.x, k * o.y); }
	tmpl bool operator<(const point<T> &a, const point<T> &b)
	{
		int s = sgn(a * b);
		return s > 0 || s == 0 && sgn(a.len2() - b.len2()) < 0;
	}
	istream &operator>>(istream &cin, point<ll> &o) { return cin >> o.x >> o.y; }
	istream &operator>>(istream &cin, point<db> &o)
	{
		string s;
		cin >> s;
		o.x = stod(s);
		cin >> s;
		o.y = stod(s);
		return cin;
	}
	tmpl ostream &operator<<(ostream &cout, const point<T> &o)
	{
		if ((point<db>)o == apos) return cout << "all position";
		if ((point<db>)o == npos) return cout << "no position";
		return cout << '(' << o.x << ',' << o.y << ')';
	}
	tmpl struct line
	{
		point<T> o, d;
		line() { }
		line(const point<T> &a, const point<T> &b, int twopoint);
		bool operator!=(const line<T> &m) { return !(*this == m); }
	};
	template<> line<ll>::line(const point<ll> &a, const point<ll> &b, int twopoint)
	{
		o = a;
		d = twopoint ? b - a : b;
		ll tmp = gcd(d.x, d.y);
		assert(tmp);
		if (d.x < 0 || d.x == 0 && d.y < 0) tmp = -tmp;
		d.x /= tmp; d.y /= tmp;
	}
	template<> line<db>::line(const point<db> &a, const point<db> &b, int twopoint)
	{
		o = a;
		d = twopoint ? b - a : b;
		int s = sgn(d.x);
		if (s < 0 || !s && d.y < 0) d.x = -d.x, d.y = -d.y;
	}
	tmpl line<T> rotate_90(const line<T> &m) { return line(m.o, point(m.d.y, -m.d.x), 0); }
	tmpl line<db> rotate(const line<T> &m, db angle)
	{
		return {(point<db>)m.o, {m.d.x * cos(angle) - m.d.y * sin(angle), m.d.x * sin(angle) + m.d.y * cos(angle)}, 0};
	}
	tmpl db get_angle(const line<T> &m, const line<T> &n) { return asin((m.d * n.d) / (m.d.len() * n.d.len())); }
	tmpl bool operator<(const line<T> &m, const line<T> &n)
	{
		int s = sgn(m.d * n.d);
		return s ? s > 0:m.d * m.o < n.d * n.o;
	}
	bool operator==(const line<ll> &m, const line<ll> &n) { return m.d == n.d && (m.o - n.o) * m.d == 0; }
	bool operator==(const line<db> &m, const line<db> &n) { return fabs(m.d * n.d) < eps && fabs((n.o - m.o) * m.d) < eps; }
	tmpl ostream &operator<<(ostream &cout, const line<T> &o) { return cout << '(' << o.d.x << " k + " << o.o.x << " , " << o.d.y << " k + " << o.o.y << ")"; }
	tmpl point<db> intersect(const line<T> &m, const line<T> &n)
	{
		if (!sgn(m.d * n.d))
		{
			if (!sgn(m.d * (n.o - m.o))) return apos;
			return npos;
		}
		return (point<db>)m.o + (n.o - m.o) * n.d / (db)(m.d * n.d) * (point<db>)m.d;
	}
	tmpl db dis(const line<T> &m, const point<T> &o) { return abs(m.d * (o - m.o) / m.d.len()); }
	tmpl db dis(const point<T> &o, const line<T> &m) { return abs(m.d * (o - m.o) / m.d.len()); }
	struct circle
	{
		point<db> o;
		db r;
		circle() { }
		circle(const point<db> &O, const db &R = 0) :o(point<db>((db)O.x, (db)O.y)), r(R) { }//圆心半径构造
		circle(const point<db> &a, const point<db> &b)//直径构造
		{
			o = (a + b) * 0.5;
			r = dis(b, o);
		}
		circle(const point<db> &a, const point<db> &b, const point<db> &c)//三点构造外接圆(非最小圆)
		{
			auto A = (b + c) * 0.5, B = (a + c) * 0.5;
			o = intersect(rotate_90(line(A, c, 1)), rotate_90(line(B, c, 1)));
			r = dis(o, c);
		}
		circle(vector<point<db>> a)
		{
			int n = a.size(), i, j, k;
			mt19937 rnd(75643);
			shuffle(all(a), rnd);
			*this = circle(a[0]);
			for (i = 1; i < n; i++) if (!cover(a[i]))
			{
				*this = circle(a[i]);
				for (j = 0; j < i; j++) if (!cover(a[j]))
				{
					*this = circle(a[i], a[j]);
					for (k = 0; k < j; k++) if (!cover(a[k])) *this = circle(a[i], a[j], a[k]);
				}
			}
		}
		circle(const vector<point<ll>> &b)
		{
			vector<point<db>> a(b.size());
			int n = a.size(), i, j, k;
			for (i = 0; i < a.size(); i++) a[i] = (point<db>)b[i];
			*this = circle(a);
		}
		tmpl bool cover(const point<T> &a) { return sgn(dis((point<db>)a, o) - r) <= 0; }
	};
	tmpl struct segment
	{
		point<T> a, b;
		segment() { }
		segment(point<T> o, point<T> p)
		{
			int s = sgn(o.x - p.x);
			if (s > 0 || !s && o.y > p.y) swap(o, p);
			a = o; b = p;
		}
	};
	tmpl bool intersect(const segment<T> &m, const segment<T> &n)
	{
		auto a = n.b - n.a, b = m.b - m.a;
		auto d = n.a - m.a;
		if (sgn(n.b.x - m.a.x) < 0 || sgn(m.b.x - n.a.x) < 0) return 0;
		if (sgn(max(n.a.y, n.b.y) - min(m.a.y, m.b.y)) < 0 || sgn(max(m.a.y, m.b.y) - min(n.a.y, n.b.y)) < 0) return 0;
		return sgn(b * d) * sgn((n.b - m.a) * b) >= 0 && sgn(a * d) * sgn((m.b - n.a) * a) <= 0;
	}
	tmpl struct convex
	{
		vector<point<T>> p;
		convex(vector<point<T>> a);
		db peri()//周长
		{
			int i, n = p.size();
			db C = (p[n - 1] - p[0]).len();
			for (i = 1; i < n; i++) C += (p[i - 1] - p[i]).len();
			return C;
		}
		db area() { return area2() * 0.5; }//面积
		T area2()//两倍面积
		{
			int i, n = p.size();
			T S = p[n - 1] * p[0];
			for (i = 1; i < n; i++) S += p[i - 1] * p[i];
			return abs(S);
		}
		db diam() { return sqrt(diam2()); }
		T diam2()//直径平方
		{
			T r = 0;
			int n = p.size(), i, j;
			if (n <= 2)
			{
				for (i = 0; i < n; i++) for (j = i + 1; j < n; j++) r = max(r, dis2(p[i], p[j]));
				return r;
			}
			p.push_back(p[0]);
			for (i = 0, j = 1; i < n; i++)
			{
				while ((p[i + 1] - p[i]) * (p[j] - p[i]) <= (p[i + 1] - p[i]) * (p[j + 1] - p[i])) if (++j == n) j = 0;
				r = max({r, dis2(p[i], p[j]), dis2(p[i + 1], p[j])});
			}
			p.pop_back();
			return r;
		}
		bool cover(const point<T> &o) const//点是否在凸包内
		{
			if (o.x < p[0].x || o.x == p[0].x && o.y < p[0].y) return 0;
			if (o == p[0]) return 1;
			if (p.size() == 1) return 0;
			ll tmp = (o - p[0]) * (p.back() - p[0]);
			if (tmp == 0) return dis2(o, p[0]) <= dis2(p.back(), p[0]);
			if (tmp < 0 || p.size() == 2) return 0;
			int x = upper_bound(1 + all(p), o, [&](const point<T> &a, const point<T> &b) { return (a - p[0]) * (b - p[0]) > 0; }) - p.begin() - 1;
			return (o - p[x]) * (p[x + 1] - p[x]) <= 0;
		}
		convex<T> operator+(const convex<T> &A) const
		{
			int n = p.size(), m = A.p.size(), i, j;
			vector<point<T>> c;
			if (min(n, m) <= 2)
			{
				c.reserve(n * m);
				for (i = 0; i < n; i++) for (j = 0; j < m; j++) c.push_back(p[i] + A.p[j]);
				return convex<T>(c);
			}
			point<T> a[n], b[m];
			for (i = 0; i + 1 < n; i++) a[i] = p[i + 1] - p[i];
			a[n - 1] = p[0] - p[n - 1];
			for (i = 0; i + 1 < m; i++) b[i] = A.p[i + 1] - A.p[i];
			b[m - 1] = A.p[0] - A.p[m - 1];
			c.reserve(n + m);
			c.push_back(p[0] + A.p[0]);
			for (i = j = 0; i < n && j < m;) c.push_back(c.back() + (a[i] * b[j] > 0 ? a[i++] : b[j++]));
			while (i < n - 1) c.push_back(c.back() + a[i++]);
			while (j < m - 1) c.push_back(c.back() + b[j++]);
			return convex<T>(c);
		}
		void operator+=(const convex &a) { *this = *this + a; }
	};
	tmpl convex<T>::convex(vector<point<T>> a)
	{
		int n = a.size(), i;
		if (!n) return;
		p = a;
		for (i = 1; i < n; i++) if (p[i].x < p[0].x || p[i].x == p[0].x && p[i].y < p[0].y) swap(p[0], p[i]);
		a.resize(0); a.reserve(n);
		for (i = 1; i < n; i++) if (p[i] != p[0]) a.push_back(p[i] - p[0]);
		sort(all(a));
		for (i = 0; i < a.size(); i++) a[i] += p[0];
		point<T> *st = p.data() - 1;
		int tp = 1;
		for (auto &v : a)
		{
			while (tp > 1 && sgn((st[tp] - st[tp - 1]) * (v - st[tp - 1])) <= 0) --tp;
			st[++tp] = v;
		}
		p.resize(tp);
	}
	template<> bool convex<db>::cover(const point<db> &o) const//点是否在凸包内
	{
		if (o.x < p[0].x || o.x == p[0].x && o.y < p[0].y) return 0;
		if (o == p[0]) return 1;
		if (p.size() == 1) return 0;
		ll tmp = (o - p[0]) * (p.back() - p[0]);
		if (tmp == 0) return dis2(o, p[0]) <= dis2(p.back(), p[0]);
		if (tmp < 0 || p.size() == 2) return 0;
		int x = upper_bound(1 + all(p), o, [&](const point<db> &a, const point<db> &b) { return (a - p[0]) * (b - p[0]) > eps; }) - p.begin() - 1;
		return (o - p[x]) * (p[x + 1] - p[x]) <= 0;
	}
	tmpl struct half_plane//默认左侧
	{
		point<T> o, d;
		operator half_plane<ll>() const { return {(point<ll>)o, (point<ll>)d, 0}; }
		operator half_plane<db>() const { return {(point<db>)o, (point<db>)d, 0}; }
		half_plane() { }
		half_plane(const point<T> &a, const point<T> &b, bool twopoint)
		{
			o = a;
			d = twopoint ? b - a : b;
		}
		bool operator<(const half_plane<T> &a) const
		{
			int p = quad(d), q = quad(a.d);
			if (p != q) return p < q;
			p = sgn(d * a.d);
			if (p) return p > 0;
			return sgn(d * (a.o - o)) > 0;
		}
	};
	tmpl ostream &operator<<(ostream &cout, half_plane<T> &m) { return cout << m.o << " | " << m.d; }
	tmpl point<db> intersect(const half_plane<T> &m, const half_plane<T> &n)
	{
		if (!sgn(m.d * n.d))
		{
			if (!sgn(m.d * (n.o - m.o))) return apos;
			return npos;
		}
		return (point<db>)m.o + (n.o - m.o) * n.d / (db)(m.d * n.d) * (point<db>)m.d;
	}
	const db inf = 1e9;
	tmpl convex<db> intersect(vector<half_plane<T>> a)
	{
		T I = inf;
		a.push_back({{-I, -I}, {I, -I}, 1});
		a.push_back({{I, -I}, {I, I}, 1});
		a.push_back({{I, I}, {-I, I}, 1});
		a.push_back({{-I, I}, {-I, -I}, 1});
		sort(all(a));
		int n = a.size(), i, h = 0, t = -1;
		half_plane<db> q[n];
		point<db> p[n];
		vector<point<db>> r;
		for (i = 0; i < n; i++) if (i == n - 1 || sgn(a[i].d * a[i + 1].d))
		{
			auto x = (half_plane<db>)a[i];
			while (h < t && sgn((p[t - 1] - x.o) * x.d) >= 0) --t;
			while (h < t && sgn((p[h] - x.o) * x.d) >= 0) ++h;
			q[++t] = x;
			if (h < t) p[t - 1] = intersect(q[t - 1], q[t]);
		}
		while (h < t && sgn((p[t - 1] - q[h].o) * q[h].d) >= 0) --t;
		if (h == t) return convex<db>(vector<point<db>>(0));
		p[t] = intersect(q[h], q[t]);
		return convex<db>(vector<point<db>>(p + h, p + t + 1));
	}
	tmpl db dis(const point<db> &o, const segment<T> &l)
	{
		if ((l.b - l.a & o - l.a) < 0 || (l.a - l.b & o - l.b) < 0) return min(dis(o, l.a), dis(o, l.b));
		return dis(o, line(l.a, l.b, 1));
	}
	tmpl db dis(const segment<T> &l, const point<db> &o)
	{
		if ((l.b - l.a & o - l.a) < 0 || (l.a - l.b & o - l.b) < 0) return min(dis(o, l.a), dis(o, l.b));
		return dis(o, line(l.a, l.b, 1));
	}
	pair<ll, ll> __sqrt(ll x)
	{
		ll y = sqrtl(x);
		while (y * y > x) --y;
		while ((y + 1) * (y + 1) <= x) ++y;
		return {y, y + (y * y < x)};
	}
	pair<int, int> closest_pair(const vector<point<ll>> &a)
	{
		int n = a.size(), i, j;
		assert(n >= 2);
		auto b = a;
		sort(all(b), [&](auto p, auto q) {
			return p.x == q.x ? p.y < q.y : p.x < q.x;
		});
		tuple<ll, int, int> ans = {dis2(b[0], b[1]), 0, 1};
		set<pair<ll, int>> s;
		for (i = j = 0; i < n; i++)
		{
			auto [x, y] = b[i];
			ll d = __sqrt(get<0>(ans)).first;
			if (d == 0) break;
			for (auto it = s.lower_bound({y - d, 0}); it != s.end(); ++it)
			{
				auto [q, k] = *it;
				cmin(ans, tuple{dis2(b[k], b[i]), i, k});
			}
			s.emplace(y, i);
			while (b[j].x < x - d) s.erase({b[j].y, j}), ++j;
		}
		auto [_, j1, j2] = ans;
		int i1, i2;
		for (i1 = 0; i1 < n; i1++) if (a[i1] == b[j1]) break;
		for (i2 = 0; i2 < n; i2++) if (i2 != i1 && a[i2] == b[j2]) break;
		return {i1, i2};
	}
	pair<int, int> furthest_pair(const vector<point<ll>> &a)
	{
		int n = a.size(), i, j;
		assert(n >= 2);
		auto b = convex(a).p;
		int m = b.size();
		if (m == 1) return {0, 1};
		b.push_back(b[0]);
		tuple<ll, int, int> ans{dis2(b[0], b[1]), 0, 1};
		for (i = 0, j = 1; i < m; i++)
		{
			while (abs((b[i + 1] - b[i]) * (b[j] - b[i])) < abs((b[i + 1] - b[i]) * (b[(j + 1) % m] - b[i]))) j = (j + 1) % m;
			cmax(ans, tuple{dis2(b[i], b[j]), i, j});
			cmax(ans, tuple{dis2(b[i + 1], b[j]), i + 1, j});
		}
		auto [_, j1, j2] = ans;
		int i1, i2;
		for (i1 = 0; i1 < n; i1++) if (a[i1] == b[j1]) break;
		for (i2 = 0; i2 < n; i2++) if (i2 != i1 && a[i2] == b[j2]) break;
		return {i1, i2};

	}
#undef tmpl
}
using geometry::point, geometry::line, geometry::circle, geometry::convex, geometry::half_plane;
using geometry::db, geometry::sgn, geometry::eps, geometry::segment;
using geometry::intersect, geometry::dis;
\end{lstlisting}


\newpage

\section{公式与杂项}

\subsection{枚举大小为 $k$ 的集合}

思路:通过进位创造 $1$,再把一串 $1$ 移到最后。

\begin{lstlisting}
for (int s=(1<<k)-1,t;s<1<<n;t=s+(s&-s),s=(s&~t)>>__lg(s&-s)+1|t)
{}
\end{lstlisting}
\subsection{min plus 卷积}

计算 $c_i=\min\limits_{j=0}^i a_j+b_{i-j}$。

要求 $b$ 是凸的,即 $b_{i+1}-b_i$ 不降。

\begin{lstlisting}
template <class T> vector<T> min_plus_convolution(const vector<T> &a, const vector<T> &b)
{
	int n = a.size(), m = b.size(), i;
	vector<T> c(n + m - 1);
	function<void(int, int, int, int)> dfs = [&](int l, int r, int ql, int qr) {
		if (l > r) return;
		int mid = l + r >> 1;
		while (ql + m <= l) ++ql;
		while (qr > r) --qr;
		int qmid = -1;
		c[mid] = inf;
		for (int i = ql; i <= qr; i++) if (mid - i >= 0 && mid - i < m && cmin(c[mid], a[i] + b[mid - i])) qmid = i;
		dfs(l, mid - 1, ql, qmid);
		dfs(mid + 1, r, qmid, qr);
	};
	dfs(0, n + m - 2, 0, n - 1);
	return c;
}

\end{lstlisting}

\subsection{所有区间 GCD}

需要自定义 \verb|fun|,如 $\gcd$,$\operatorname{and}$,$\operatorname{or}$。

\begin{lstlisting}
template<class T> struct GCD
{
	vector<pair<int, T>> res;
	GCD(const vector<T> &a) :res(n)
	{
		int n = a.size(), i, j;
		vector<ll> v(n);
		vector<int> l(n);
		for (i = 0; i < n; i++)
		{
			for (v[i] = a[i], j = l[i] = i; j >= 0; j = l[j] - 1)
			{
				v[j] = fun(v[j], a[i]);
				while (l[j] && fun(a[i], v[l[j] - 1]) == fun(a[i], v[j])) l[j] = l[l[j] - 1];
				//[l[j]..j,i]区间内的值求 fun 均为 v[j]
			}
		}
	}
};
\end{lstlisting}
\subsection{整体二分(区间 $k$-th)}

$O((n+q)\log a)$,$O(n+q)$。

\begin{lstlisting}
struct cz
{
	int x, y, kth, pos, typ;
};
cz q[M], st1[M], st2[M];
int a[N], b[N], d[N], ans[N], s[N];
int n, m, t1, t2, i, j, c, gs;
int lb(int x)
{
	return x & (-x);
}
void add(int x, int y)
{
	for (; x <= n; x += lb(x)) s[x] += y;
}
int sum(int x)
{
	int ans = 0;
	for (; x; x -= lb(x)) ans += s[x];
	return ans;
}
void ztef(int ql, int qr, int l, int r)
{
	if (ql > qr) return;
	int mid = l + r >> 1, i, midd;
	t1 = t2 = 0;
	if (l == r)
	{
		for (i = ql; i <= qr; i++) if (q[i].typ) ans[q[i].pos] = d[l];
		return;
	}
	for (i = ql; i <= qr; i++) if (q[i].typ)
	{
		midd = sum(q[i].y) - sum(q[i].x - 1);
		if (midd >= q[i].kth) st1[++t1] = q[i]; else
		{
			st2[++t2] = q[i];
			st2[t2].kth -= midd;
		}
	}
	else if (q[i].pos <= mid)
	{
		add(q[i].x, 1);
		st1[++t1] = q[i];
	}
	else st2[++t2] = q[i];
	for (i = 1; i <= t1; i++) if (!st1[i].typ) add(st1[i].x, -1);
	for (i = 1; i <= t1; i++) q[i + ql - 1] = st1[i];
	midd = ql + t1 - 1;
	for (i = 1; i <= t2; i++) q[i + midd] = st2[i];
	ztef(ql, midd, l, mid); ztef(midd + 1, qr, mid + 1, r);
}
int main()
{
	cin >> n >> m;
	for (i = 1; i <= n; i++)
	{
		cin >> a[i];
		b[i] = a[i];
	}
	sort(b + 1, b + n + 1);
	d[gs = 1] = b[1];
	for (i = 2; i <= n; i++) if (b[i] != b[i - 1]) d[++gs] = b[i];
	for (i = 1; i <= n; i++) a[i] = lower_bound(d + 1, d + gs + 1, a[i]) - d;
	for (i = 1; i <= n; i++)
	{
		q[i].x = i; q[i].pos = a[i]; q[i].typ = 0;
	}
	for (i = 1; i <= m; i++)
	{
		cin >> q[i + n].x >> q[i + n].y >> q[i + n].kth;
		q[i + n].pos = i; q[i + n].typ = 1;
	}
	ztef(1, n + m, 1, gs);
	for (i = 1; i <= m; i++) printf("%d\n", ans[i]);
}
\end{lstlisting}

\subsection{高精度}

除法和取模有点问题,但 gcd 是对的。

\begin{lstlisting}
struct bigint;
int cmp(const bigint &a, const bigint &b);
struct bigint
{
	using ll = unsigned long long;
	using lll = unsigned __int128;
	const static ll sign = 1llu << 63;
	const static lll p = 4'179'340'454'199'820'289;
	const static lll g = 3;
	const static ll base = 1e6;
	const static int output_base = 10;
	const static int length = round(log(bigint::base) / log(output_base));
	static_assert(output_base == 10 || output_base == 16, "output_base must be 10 or 16");
	static_assert(round(pow(output_base, length)) == base);
	const static int N = 1 << 23;
	static int r[N];
	static lll w[N];
	bool neg;
	vector<ll> a;
private:
	static lll ksm(lll x, ll y)
	{
		lll r = 1;
		while (y)
		{
			if (y & 1) r = r * x % p;
			x = x * x % p; y >>= 1;
		}
		return r;
	}
	static void init(int n)
	{
		static int pr = 0, pw = 0;
		if (pr == n) return;
		int b = __lg(n) - 1, i, j, k;
		for (i = 1; i < n; i++) r[i] = r[i >> 1] >> 1 | (i & 1) << b;
		if (pw < n)
		{
			for (j = 1; j < n; j = k)
			{
				k = j * 2;
				ll wn = ksm(g, (p - 1) / k);
				w[j] = 1;
				for (i = j + 1; i < k; i++) w[i] = w[i - 1] * wn % p;
			}
			pw = n;
		}
		pr = n;
	}
	static void dft(vector<lll> &a, int o = 0)
	{
		int n = a.size(), i, j, k;
		lll y, *f, *g, *wn, *A = a.data();
		init(n);
		for (i = 1; i < n; i++) if (i < r[i]) swap(A[i], A[r[i]]);
		static const int T = 12;
		static_assert(T + 2 <= numeric_limits<lll>::max() / (p * p));
		for (k = 1; k < n; k *= 2)
		{
			wn = w + k;
			for (i = 0; i < n; i += k * 2)
			{
				f = A + i; g = A + i + k;
				for (j = 0; j < k; j++)
				{
					y = g[j] * wn[j] % p;
					g[j] = f[j] + p - y;
					f[j] += y;
				}
			}
			if (__lg(n / k) % T == 1) for (lll &x : a) x %= p;
		}
		if (o)
		{
			y = ksm(n, p - 2);
			for (lll &x : a) x = x * y % p;
			reverse(1 + all(a));
		}
	}
	ll &operator[](const int &x) { return a[x]; }
	const ll &operator[](const int &x) const { return a[x]; }
	static void plus_by(vector<ll> &a, const vector<ll> &b)
	{
		int n = a.size(), m = b.size(), i, j;
		cmax(n, m);
		a.resize(++n);
		for (i = 0; i < m; i++) if ((a[i] += b[i]) >= base) a[i] -= base, ++a[i + 1];
		for (i = m; i < n && a[i] >= base; i++) a[i] -= base, ++a[i + 1];
		if (a[n - 1] == 0) a.pop_back();
	}
	static void minus_by(vector<ll> &a, const vector<ll> &b)
	{
		int n = a.size(), m = b.size(), i, j;
		for (i = 0; i < m; i++) if (!(a[i] & sign) && a[i] >= b[i]) a[i] -= b[i];
		else --a[i + 1], a[i] += base - b[i];
		for (; i < n && (a[i] & sign); i++) a[i] += base, --a[i + 1];
		while (a.size() > 1 && !a.back()) a.pop_back();
	}
	static bool less(const vector<ll> &a, const vector<ll> &b)
	{
		if (a.size() != b.size()) return a.size() < b.size();
		for (int i = a.size() - 1; i >= 0; i--) if (a[i] != b[i]) return a[i] < b[i];
		return 0;
	}
	static int cal(int x) { return 1 << __lg(max(x, 1) * 2 - 1); }
public:
	bigint &operator+=(const bigint &o)
	{
		if (neg == o.neg) plus_by(a, o.a);
		else if (neg)
		{
			if (less(o.a, a)) minus_by(a, o.a);
			else
			{
				neg = 0;
				auto t = o.a;
				swap(a, t);
				minus_by(a, t);
			}
		}
		else
		{
			if (less(a, o.a))
			{
				neg = 1;
				auto t = o.a;
				swap(a, t);
				minus_by(a, t);
			}
			else minus_by(a, o.a);
		}
		return *this;
	}
	bigint &operator-=(const bigint &o)
	{
		neg ^= 1;
		*this += o;
		neg ^= 1;
		if (a == vector<ll>{0}) neg = 0;
		return *this;
	}
	bigint &operator*=(const bigint &o)
	{
		neg ^= o.neg;
		int n = a.size(), m = o.a.size(), i, j;
		assert(min(n, m) <= p / ((base - 1) * (base - 1)));
		if (min(n, m) <= 64 && 0)
		{
			vector<ll> c(n + m);
			for (i = 0; i < n; i++) for (j = 0; j < m; j++) c[i + j] += a[i] * o[j];
			for (i = 0; i < n + m - 1; i++)
			{
				c[i + 1] += c[i] / base;
				c[i] %= base;
			}
			swap(a, c);
			while (a.size() > 1 && !a.back()) a.pop_back();
			if (a == vector<ll>{0}) neg = 0;
			return *this;
		}
		int len = cal(n + m);
		vector<lll> f(len), g(len);
		copy_n(a.begin(), n, f.begin());
		copy_n(o.a.begin(), m, g.begin());
		dft(f); dft(g);
		for (i = 0; i < len; i++) f[i] = f[i] * g[i] % p;
		dft(f, 1);
		a.resize(n + m);
		copy_n(f.begin(), n + m - 1, a.begin());
		for (i = n + m - 2; i >= 0; i--)
		{
			a[i + 1] += a[i] / base;
			a[i] %= base;
		}
		for (i = 0; i < n + m - 1; i++)
		{
			a[i + 1] += a[i] / base;
			a[i] %= base;
		}
		while (a.size() > 1 && !a.back()) a.pop_back();
		if (a == vector<ll>{0}) neg = 0;
		return *this;
	}
	bigint &operator/=(long long x)//to zero
	{
		if (x < 0) x = -x, neg ^= 1;
		for (int i = a.size() - 1; i; i--)
		{
			a[i - 1] += a[i] % x * base;
			a[i] /= x;
		}
		a[0] /= x;
		while (a.size() > 1 && !a.back()) a.pop_back();
		if (a == vector<ll>{0}) neg = 0;
		return *this;
	}
	bigint operator+(bigint o) const { return o += *this; }
	bigint operator-(bigint o) const { o -= *this; if (o.a != vector<ll>{0}) o.neg ^= 1; return o; }
	bigint operator*(bigint o) const { return o *= *this; }
	bigint operator/(long long x) const { auto res = *this; return res /= x; }
	long long operator%(long long x) const
	{
		bool flg = neg;
		if (x < 0) flg ^= 1, x = -x;
		ll res = 0;
		for (int i = (base % x == 0 ? 0 : a.size() - 1); i >= 0; i--) res = (res * base + a[i]) % x;
		return (long long)res * (flg ? -1 : 1);
	}
	bigint(long long x = 0) :neg(0)
	{
		if (x < 0) x = -x, neg = 1;
		a.push_back(x % base);
		while (x /= base) a.push_back(x % base);
	}
	bool operator<(const bigint &o) const { return cmp(*this, o) < 0; }
	bool operator>(const bigint &o) const { return cmp(*this, o) > 0; }
	bool operator<=(const bigint &o) const { return cmp(*this, o) <= 0; }
	bool operator>=(const bigint &o) const { return cmp(*this, o) >= 0; }
	bool operator==(const bigint &o) const { return cmp(*this, o) == 0; }
	bool operator!=(const bigint &o) const { return cmp(*this, o) != 0; }
};
int cmp(const bigint &a, const bigint &b)
{
	if (a.neg != b.neg) return a.neg ? -1 : 1;
	if (a.neg) return -cmp(b, a);
	if (a.a.size() != b.a.size()) return a.a.size() < b.a.size() ? -1 : 1;
	for (int i = a.a.size() - 1; i >= 0; i--) if (a.a[i] != b.a[i]) return a.a[i] < b.a[i] ? -1 : 1;
	return 0;
}
istream &operator>>(istream &cin, bigint &x)
{
	x.neg = 0;
	x.a.clear();
	string s;
	cin >> s;
	const static int length = bigint::length;
	static int mp[128], _ = [&]() {
		for (int i = '0'; i <= '9'; i++) mp[i] = i - '0';
		for (int i = 'a'; i <= 'z'; i++) mp[i] = i - 'a' + 10;
		for (int i = 'A'; i <= 'Z'; i++) mp[i] = i - 'A' + 10;
		return 0;
	}();
	reverse(all(s));
	if (s.back() == '-') x.neg = 1, s.pop_back();
	ll base = 1;
	for (int i = 0; i < s.size(); i++)
	{
		if (i % length == 0) x.a.push_back(0), base = 1;
		x.a.back() += mp[s[i]] * base;
		base *= bigint::output_base;
	}
	return cin;
}
ostream &operator<<(ostream &cout, const bigint &x)
{
	if (x.neg) cout << "-";
	const static int length = bigint::length;
	if (bigint::output_base == 10)
	{
		cout << setfill('0') << x.a.back();
		for (int i = (int)x.a.size() - 2; i >= 0; i--) cout << setw(length) << x.a[i];
	}
	else if (bigint::output_base == 16)
	{
		cout << hex << uppercase << setfill('0') << x.a.back();
		for (int i = (int)x.a.size() - 2; i >= 0; i--) cout << setw(length) << x.a[i];
		cout << dec;
	}
	else assert(0);
	return cout;
}
bigint abs(bigint x)
{
	x.neg = 0;
	return x;
}
bigint gcd(bigint x, bigint y)
{
	x.neg = y.neg = 0;
	if (x == bigint(0)) return y;
	if (y == bigint(0)) return x;
	int c1 = 0, c2 = 0;
	while (x % 2 == 0) x /= 2, ++c1;
	while (y % 2 == 0) y /= 2, ++c2;
	cmin(c1, c2);
	if (x > y) swap(x, y);
	while (x != y)
	{
		y -= x;
		y /= 2;
		while (y % 2 == 0) y /= 2;
		if (x > y) swap(x, y);
	}
	while (c1--) y *= bigint(2);
	return y;
}
bigint::lll bigint::w[bigint::N];
int bigint::r[bigint::N];
int main()
{
	ios::sync_with_stdio(0); cin.tie(0);
	cout << fixed << setprecision(0);
	int T; cin >> T;
	while (T--)
	{
		bigint a, b;
		cin >> a >> b;
		cout << (a *= b) << '\n';
	}
}
\end{lstlisting}

\subsection{分散层叠算法(Fractional Cascading)}

$O(n+q(k+\log n))$,$O(n)$。

给出 $k$ 个长度为 $n$ 的有序数组。

现在有 $q$ 个查询 : 给出数 $x$,分别求出每个数组中大于等于 $x$ 的最小的数(非严格后继)。

若后继不存在,则定义为 $0$。你需要在线地回答这些询问。

\begin{lstlisting}
int a[M][N], b[M][N << 1], c[M][N << 1][2], len[M], ans[M];
int n, m, qs, p, q, d, i, j, x, y, la;
int main()
{
	cin >> n >> m >> qs >> d;
	for (j = 1; j <= m; j++) for (i = 0; i < n; i++) cin >> a[j][i];
	for (j = 1; j <= m; j++) a[j][n] = inf + j; ++n;
	for (i = 0; i < n; i++) b[m][i] = a[m][i], c[m][i][0] = i;
	len[m] = n;
	for (j = m - 1; j; j--)
	{
		p = 0, q = 1;
		while (p < n && q < len[j + 1])
			if (a[j][p] < b[j + 1][q]) b[j][len[j]] = a[j][p], c[j][len[j]][0] = p++, c[j][len[j]++][1] = q;
			else b[j][len[j]] = b[j + 1][q], c[j][len[j]][0] = p, c[j][len[j]++][1] = q, q += 2;
		while (p < n) b[j][len[j]] = a[j][p], c[j][len[j]][0] = p++, c[j][len[j]++][1] = q;
		while (q < len[j + 1]) b[j][len[j]] = b[j + 1][q], c[j][len[j]][0] = p, c[j][len[j]++][1] = q, q += 2;
	}
	for (int ii = 1; ii <= qs; ii++)
	{
		cin >> x; x ^= la;
		y = lower_bound(b[1], b[1] + len[1], x) - b[1];
		ans[1] = a[1][c[1][y][0]]; y = c[1][y][1];//下标是c[1][y][0]
		for (j = 2; j <= m; j++)
		{
			if (y && b[j][y - 1] >= x) --y;
			ans[j] = a[j][c[j][y][0]];//下标是c[j][y][0]
			y = c[j][y][1];
		}
		la = 0;
		for (i = 1; i <= m; i++) la ^= ans[i] > inf ? 0 : ans[i];
		if (ii % d == 0) printf("%d\n", la);
	}
}

\end{lstlisting}

\subsection{圆上整点(二平方和定理)}

$x^2+y^2=n$ 的整数解的数目的四分之一 $f(n)$ 是积性数论函数,且对于{\bf{素数幂}}有:
$f(p^k)=\begin{cases}
		1            & p=2               \\
		k+1          & p\equiv 1 \pmod 4 \\
		(k+1)\bmod 2 & p\equiv 3 \pmod 4
	\end{cases}$

以下代码给出所有的非负整数解。注意非负整数解个数不等于 $f(n)$。

时间复杂度为 $O(n^{\frac{1}{4}}+f(n))$,其中 $O(n^{\frac{1}{4}})$ 是 pollard-rho 的复杂度。

$f(n)$ 的量级不好分析,但不会超过约数个数 $O(d(n))\approx O(n^{\frac{1}{3}})$,且可以推测不能达到。实践上 $10^{18}$ 以内 $f(n)\le 3072$。

\begin{lstlisting}
namespace pr
{
	typedef long long ll;
	typedef __int128 lll;
	typedef pair<ll, int> pa;
	ll ksm(ll x, ll y, const ll p)
	{
		ll r=1;
		while (y)
		{
			if (y&1) r=(lll)r*x%p;
			x=(lll)x*x%p; y>>=1;
		}
		return r;
	}
	namespace miller
	{
		const int p[7]={2, 3, 5, 7, 11, 61, 24251};
		ll s, t;
		bool test(ll n, int p)
		{
			if (p>=n) return 1;
			ll r=ksm(p, t, n), w;
			for (int j=0; j<s&&r!=1; j++)
			{
				w=(lll)r*r%n;
				if (w==1&&r!=n-1) return 0;
				r=w;
			}
			return r==1;
		}
		bool prime(ll n)
		{
			if (n<2||n==46'856'248'255'981ll) return 0;
			for (int i=0; i<7; ++i) if (n%p[i]==0) return n==p[i];
			s=__builtin_ctz(n-1); t=n-1>>s;
			for (int i=0; i<7; ++i) if (!test(n, p[i])) return 0;
			return 1;
		}
	}
	using miller::prime;
	mt19937_64 rnd(chrono::steady_clock::now().time_since_epoch().count());
	namespace rho
	{
		void nxt(ll &x, ll &y, ll &p) { x=((lll)x*x+y)%p; }
		ll find(ll n, ll C)
		{
			ll l, r, d, p=1;
			l=rnd()%(n-2)+2, r=l;
			nxt(r, C, n);
			int cnt=0;
			while (l^r)
			{
				p=(lll)p*llabs(l-r)%n;
				if (!p) return gcd(n, llabs(l-r));
				++cnt;
				if (cnt==127)
				{
					cnt=0;
					d=gcd(llabs(l-r), n);
					if (d>1) return d;
				}
				nxt(l, C, n); nxt(r, C, n); nxt(r, C, n);
			}
			return gcd(n, p);
		}
		vector<pa> w;
		vector<ll> d;
		void dfs(ll n, int cnt)
		{
			if (n==1) return;
			if (prime(n)) return w.emplace_back(n, cnt), void();
			ll p=n, C=rnd()%(n-1)+1;
			while (p==1||p==n) p=find(n, C++);
			int r=1; n/=p;
			while (n%p==0) n/=p, ++r;
			dfs(p, r*cnt); dfs(n, cnt);
		}
		vector<pa> getw(ll n)
		{
			w=vector<pa>(0); dfs(n, 1);
			if (n==1) return w;
			sort(w.begin(), w.end());
			int i, j;
			for (i=1, j=0; i<w.size(); i++) if (w[i].first==w[j].first) w[j].second+=w[i].second; else w[++j]=w[i];
			w.resize(j+1);
			return w;
		}
		void dfss(int x, ll n)
		{
			if (x==w.size()) return d.push_back(n), void();
			dfss(x+1, n);
			for (int i=1; i<=w[x].second; i++) dfss(x+1, n*=w[x].first);
		}
		vector<ll> getd(ll n)
		{
			getw(n); d=vector<ll>(0); dfss(0, 1);
			sort(d.begin(), d.end());
			return d;
		}
	}
	using rho::getw, rho::getd;
	using miller::prime;
}
using pr::getw, pr::getd, pr::prime;
lll roundiv(lll x, lll y)
{
	return x>=0?(x+y/2)/y:(x-y/2)/y;
}
struct G
{
	lll x, y;
	G operator~() const { return {x, -y}; }
	lll len2() const { return x*x+y*y; }
	G operator+(const G &o) const { return {x+o.x, y+o.y}; }
	G operator-(const G &o) const { return {x-o.x, y-o.y}; }
	G operator*(const G &o) const { return {x*o.x-y*o.y, x*o.y+y*o.x}; }
	G operator/(const G &o) const
	{
		G t=*this*~o;
		lll l=o.len2();
		return {roundiv(t.x, l), roundiv(t.y, l)};
	}
	G operator%(const G &o) const { return *this-*this/o*o; }
};
G gcd(G a, G b)
{
	if (a.len2()>b.len2()) swap(a, b);
	while (a.len2())
	{
		b=b%a;
		swap(a, b);
	}
	return b;
}
namespace cipolla
{
	typedef unsigned long long ui;
	typedef __uint128_t ll;
	ui p, w;
	struct Q
	{
		ll x, y;
		Q operator*(const Q &o) const { return {(x*o.x+y*o.y%p*w)%p, (x*o.y+y*o.x)%p}; }
	};
	ui ksm(ll x, ui y)
	{
		ll r=1;
		while (y)
		{
			if (y&1) r=r*x%p;
			x=x*x%p; y>>=1;
		}
		return r;
	}
	Q ksm(Q x, ui y)
	{
		Q r={1, 0};
		while (y)
		{
			if (y&1) r=r*x;
			x=x*x; y>>=1;
		}
		return r;
	}
	ui mosqrt(ui x, ui P)//0<=x<P
	{
		if (x==0||P==2) return x;
		p=P;
		if (ksm(x, p-1>>1)!=1) return -1;
		ui y;
		mt19937_64 rnd(chrono::steady_clock::now().time_since_epoch().count());
		do y=rnd()%p, w=((ll)y*y+p-x)%p; while (ksm(w, p-1>>1)<=1);//not for p=2
		y=ksm({y, 1}, p+1>>1).x;
		if (y*2>p) y=p-y;//两解取小
		return y;
	}
}
using cipolla::mosqrt;
vector<pair<ll, ll>> two_sqr_sum(ll n)//只会返回非负解,按照字典序排序
{
	if (n<0) return { };
	if (n==0) return {{0, 0}};
	ll m=__lg(n&-n), d=1<<m/2, i;
	n>>=m;
	auto w=getw(n);
	vector<G> r((m&1)?vector{G{1, 1}}:vector{G{0, 1}, G{1, 0}});
	for (auto [p, k]:w) if (p%4==1)
	{
		vector<G> pw(k+1);
		pw[0]={1, 0};
		pw[1]=gcd(G(p, 0), G(mosqrt(p-1, p), 1));
		assert(pw[1].len2()==p);
		for (i=2; i<=k; i++) pw[i]=pw[i-1]*pw[1];
		vector<G> rr; rr.reserve(r.size()*(k+1));
		for (i=0; i<=k; i++)
		{
			G x=pw[i]*~pw[k-i];
			for (G y:r) rr.push_back(x*y);
		}
		swap(r, rr);
	}
	else
	{
		if (k%2) return { };
		k/=2;
		while (k--) d*=p;
	}
	vector<pair<ll, ll>> ans;
	ans.reserve(r.size());
	for (auto [x, y]:r) ans.push_back({abs((ll)x*d), abs((ll)y*d)});
	sort(all(ans));
	ans.resize(unique(all(ans))-ans.begin());
	return ans;
}
\end{lstlisting}


\subsection{快速取模}

\begin{lstlisting}
__uint128_t brt=((__uint128_t)1<<64)/mod;
for(int i=1;i<=n;i++)
{
	ans*=i;
	ans=ans-mod*(brt*ans>>64);
	while(ans>=mod) ans-=mod;//可以替换为 if,但据说会变慢。如果循环展开则需要替换
}

struct barret{
    ll p,m; //p 表示上面的模数, m 为取模参数
    int c=0;
    inline void init(ll t){
    	c=48+log2(t),p=t;
		m=(ll((ulll(1)<<c)/t));
	}
    friend inline ll operator % (ll n,const barret &d) { // get n % d
        return n-((ulll(n)*d.m)>>d.c)*d.p;
    }
}modp;
\end{lstlisting}

\subsection{IO 优化}

\begin{lstlisting}
class fast_iostream{
private:
	const int MAXBF = 1 << 20; FILE *inf, *ouf;
	char *inbuf, *inst, *ined;
	char *oubuf, *oust, *oued;
	inline void _flush(){fwrite(oubuf, 1, oued - oust, ouf);}
	inline char _getchar(){
		if(inst == ined) inst = inbuf, ined = inbuf + fread(inbuf, 1, MAXBF, inf);
		return inst == ined ? EOF : *inst++;
	}
	inline void _putchar(char c){
		if(oued == oust + MAXBF) _flush(), oued = oubuf;
		*oued++ = c;
	}
public:
	 fast_iostream(FILE *_inf = stdin, FILE * _ouf = stdout)
	:inbuf(new char[MAXBF]), inf(_inf), inst(inbuf), ined(inbuf),
	 oubuf(new char[MAXBF]), ouf(_ouf), oust(oubuf), oued(oubuf){}
	~fast_iostream(){_flush(); delete inbuf; delete oubuf;}
	template <class Int>
	fast_iostream& operator >> (Int  &n){
		static char c;
		while((c = _getchar()) < '0' || c > '9');n = c - '0';
		while((c = _getchar()) >='0' && c <='9') n = n * 10 + c - '0';
		return *this;
	}
	template <class Int>
	fast_iostream& operator << (Int   n){
		if(n < 0) _putchar('-'), n = -n; static char S[20]; int t = 0;
		do{S[t++] = '0' + n % 10, n /= 10;} while(n);
		for(int i = 0;i < t;++i) _putchar(S[t - i - 1]);
		return *this;
	}
	fast_iostream& operator << (char  c){_putchar(c);    return *this;}
	fast_iostream& operator << (const char *s){
		for(int i = 0;s[i];++i) _putchar(s[i]); return *this;
	}
}fio;//unsigned
\end{lstlisting}
\subsection{手动开栈}

一种写法是文件开头放,但部分 OJ 会失效。

\begin{lstlisting}
#pragma comment(linker, "/STACK:102400000,102400000")
\end{lstlisting}

另一种写法是在 \verb|main| 开头写,但必须以 \verb|exit(0)| 结束程序。

以下两个应该有一个是对的,不对会 CE。

\begin{lstlisting}
{
	static int OP = 0;
	if (OP++ == 0)
	{
		int size = 256 << 20; // 256MB
		char *p = (char *)malloc(size) + size;
		__asm__("movl %0, %%esp\n" :: "r"(p));
	}
}
{
	static int OP=0;
	if (OP++==0)
	{
		int size=128<<20;//128MB
		char* p=new char[size]+size;
		__asm__ __volatile__("movq %0, %%rsp\n""pushq $exit\n""jmp main\n"::"r"(p));
	}
}
\end{lstlisting}

\subsection{德扑}

\verb|solve| 返回按照出现次数排序的 \verb|vector<int>|($0$ 下标处为牌型),这样就可以字典序比较了。

\begin{lstlisting}
struct Q
{
	int suit, rank;
	bool operator<(const Q &o) const { return pair{rank, suit}<pair{o.rank, o.suit}; }
	bool operator==(const Q &o) const { return pair{rank, suit}==pair{o.rank, o.suit}; }
};
auto solve=[&](vector<Q> a)
{
	vector<int> res;
	vector<int> cnt(15);
	for (auto [s, r]:a) ++cnt[r];
	sort(all(a));
	int i;
	bool is_flush=1, is_str=0;
	for (i=1; i<5; i++) is_flush&=a[i].suit==a[0].suit;
	is_str=*max_element(all(cnt))==1&&a[0].rank+4==a[4].rank;
	vector<int> b(6);
	for (i=1; i<6; i++) b[i]=a[i-1].rank;
	sort(1+all(b), [&](int x, int y)
		{
			return pair{cnt[x], x}>pair{cnt[y], y};
		});
	if (b==vector{0, 12, 3, 2, 1, 0}) is_str=1, b[1]=0;
	if (is_flush&&is_str) return b[0]=9, b;
	if (cnt[b[1]]==4) return b[0]=8, b;
	if (cnt[b[1]]==3&&cnt[b[4]]==2) return b[0]=7, b;
	if (is_flush) return b[0]=6, b;
	if (is_str) return b[0]=5, b;
	if (cnt[b[1]]==3) return b[0]=4, b;
	if (cnt[b[1]]==2&&cnt[b[3]]==2) return b[0]=3, b;
	if (cnt[b[1]]==2) return b[0]=2, b;
	return b;
};
auto turn=[&](string s)
{
	Q res=Q{"SHDC"s.find(s[0]), "23456789TJQKA"s.find(s[1])};
	return res;
};
\end{lstlisting}


\subsection{约数个数表}
\begin{table}[H]
	\centering
	\begin{tabular}{c|c|c|c|c|c}
		\toprule
		$n$       & $n$ 前第一个质数   & $n$ 后第一个质数   & $\max\{\omega(n)\}$ & $\max\{d(n)\}$                    & $\pi(n)$            \\
		\midrule
		$10^{1}$  & $10^{1}-3$   & $10^{1}+1$   & $2$                 & $d(6) = 4$                        & $4$                 \\
		$10^{2}$  & $10^{2}-3$   & $10^{2}+1$   & $3$                 & $d(60) = 12$                      & $25$                \\
		$10^{3}$  & $10^{3}-3$   & $10^{3}+13$  & $4$                 & $d(840) = 32$                     & $168$               \\
		$10^{4}$  & $10^{4}-27$  & $10^{4}+7$   & $5$                 & $d(7560) = 64$                    & $1229$              \\
		$10^{5}$  & $10^{5}-9$   & $10^{5}+3$   & $6$                 & $d(83160) = 128$                  & $9592$              \\
		$10^{6}$  & $10^{6}-17$  & $10^{6}+3$   & $7$                 & $d(720720) = 240$                 & $7.9\times 10^4$    \\
		$10^{7}$  & $10^{7}-9$   & $10^{7}+19$  & $8$                 & $d(8648640) = 448$                & $6.7\times 10^5$    \\
		$10^{8}$  & $10^{8}-11$  & $10^{8}+7$   & $8$                 & $d(73513440) = 768$               & $5.8\times 10^6$    \\
		$10^{9}$  & $10^{9}-63$  & $10^{9}+7$   & $9$                 & $d(735134400) = 1344$             & $5.1\times 10^7$    \\
		$10^{10}$ & $10^{10}-33$ & $10^{10}+19$ & $10$                & $d(6983776800) = 2304$            & $4.6\times 10^8$    \\
		$10^{11}$ & $10^{11}-23$ & $10^{11}+3$  & $10$                & $d(97772875200) = 4032$           & $4.2\times 10^8$    \\
		$10^{12}$ & $10^{12}-11$ & $10^{12}+39$ & $11$                & $d(963761198400) = 6720$          & $3.8\times 10^9$    \\
		$10^{13}$ & $10^{13}-29$ & $10^{13}+37$ & $12$                & $d(9316358251200) = 10752$        & $3.5\times 10^{10}$ \\
		$10^{14}$ & $10^{14}-27$ & $10^{14}+31$ & $12$                & $d(97821761637600) = 17280$       & $3.3\times 10^{11}$ \\
		$10^{15}$ & $10^{15}-11$ & $10^{15}+37$ & $13$                & $d(866421317361600) = 26880$      & $3\times 10^{12}$   \\
		$10^{16}$ & $10^{16}-63$ & $10^{16}+61$ & $13$                & $d(8086598962041600) = 41472$     & $2.8\times 10^{13}$ \\
		$10^{17}$ & $10^{17}-3$  & $10^{17}+3$  & $14$                & $d(74801040398884800) = 64512$    &                     \\
		$10^{18}$ & $10^{18}-11$ & $10^{18}+3$  & $15$                & $d(897612484786617600) = 103680$  &                     \\
		$10^{19}$ & $10^{19}-39$ & $10^{19}+51$ & $16$                & $d(9200527969062830400) = 161280$ &                     \\
		\bottomrule
	\end{tabular}
\end{table}
\subsection{NTT 质数}
\begin{table}[H]
	\centering
	\begin{tabular}{c|c|c|c}

		\toprule
		$p=r\times 2^k+1$     & $r$   & $k$  & $g$(最小原根) \\
		\midrule
		$17$                  & $1$   & $4$  & $3$       \\
		$97$                  & $3$   & $5$  & $5$       \\
		$193$                 & $3$   & $6$  & $5$       \\
		$257$                 & $1$   & $8$  & $3$       \\
		$7681$                & $15$  & $9$  & $17$      \\
		$12289$               & $3$   & $12$ & $11$      \\
		$40961$               & $5$   & $13$ & $3$       \\
		$65537$               & $1$   & $16$ & $3$       \\
		$786433$              & $3$   & $18$ & $10$      \\
		$5767169$             & $11$  & $19$ & $3$       \\
		$7340033$             & $7$   & $20$ & $3$       \\
		$23068673$            & $11$  & $21$ & $3$       \\
		$104857601$           & $25$  & $22$ & $3$       \\
		$167772161$           & $5$   & $25$ & $3$       \\
		$469762049$           & $7$   & $26$ & $3$       \\
		$998244353$           & $119$ & $23$ & $3$       \\
		$1004535809$          & $479$ & $21$ & $3$       \\
		$2013265921$          & $15$  & $27$ & $31$      \\
		$2281701377$          & $17$  & $27$ & $3$       \\
		$3221225473$          & $3$   & $30$ & $5$       \\
		$75161927681$         & $35$  & $31$ & $3$       \\
		$77309411329$         & $9$   & $33$ & $7$       \\
		$206158430209$        & $3$   & $36$ & $22$      \\
		$2061584302081$       & $15$  & $37$ & $7$       \\
		$2748779069441$       & $5$   & $39$ & $3$       \\
		$6597069766657$       & $3$   & $41$ & $5$       \\
		$39582418599937$      & $9$   & $42$ & $5$       \\
		$79164837199873$      & $9$   & $43$ & $5$       \\
		$263882790666241$     & $15$  & $44$ & $7$       \\
		$1231453023109121$    & $35$  & $45$ & $3$       \\
		$1337006139375617$    & $19$  & $46$ & $3$       \\
		$3799912185593857$    & $27$  & $47$ & $5$       \\
		$4222124650659841$    & $15$  & $48$ & $19$      \\
		$7881299347898369$    & $7$   & $50$ & $6$       \\
		$31525197391593473$   & $7$   & $52$ & $3$       \\
		$180143985094819841$  & $5$   & $55$ & $6$       \\
		$1945555039024054273$ & $27$  & $56$ & $5$       \\
		$4179340454199820289$ & $29$  & $57$ & $3$       \\
		\bottomrule
	\end{tabular}

\end{table}
\subsection{公式}

向上取整的整除分块 $[i,\lfloor\dfrac{n-1}{\lceil\dfrac ni \rceil-1}\rfloor]$

$n$ 个点 $k$ 个连通块的生成树方案 $n^{k-2}\prod\limits_{i=1}^k siz_i$

$(x,y)$ 曼哈顿距离 $\to$ $(x+y,x-y)$ 切比雪夫距离

$(x,y)$ 切比雪夫距离 $\to$ $(\dfrac{x+y}{2},\dfrac{x-y}{2})$ 曼哈顿距离

Kummer's Theorem: $\tbinom{n+m}{n}$ 含 $p~(p\in \text {prime})$ 的次数是 $n+m$ 在 $p$ 进制下的进位数

$\ln (1-x^V)=-\sum\limits_{i\ge1}\dfrac{x^{Vi}}{i}$

$x^{\bar n}=\sum\limits_i S_1(n,i)x^i$

$\begin{cases}x\equiv a_1\pmod {m_1}\\x\equiv a_2\pmod {m_2}\\\cdots\\x\equiv a_n\pmod {m_n}\end{cases}$

$m_i$ 为不同的质数。设 $M=\prod\limits_{i=1}^nm_i$,$t_i\times \dfrac {M}{m_i}\equiv 1\pmod {m_i}$,则 $x\equiv \sum\limits_{i=1}^na_it_i\dfrac {M}{m_i}$。

$V-E+F=2$,$S=n+\frac s2-1$。($n$ 为内部,$s$ 为边上)

用途:对于相邻的不相等的值,在中间画一条线(最外也画),$\text{连通块个数}=1+E-V+\text{内部框个数}$

注意全都是不含矩形边界上的。

贝尔数(划分集合方案数)EGF:$\exp(e^x-1)$,$B_n=\sum\limits_{i=0}^n S_2(n,i)$,伯努利数 EGF:$\dfrac{x}{e^x-1}$

$S_1(i,m)$ EGF:$\dfrac{(\sum\limits_{i\ge 0}\dfrac{x^i}i)^m}{m!}$,$S_2(i,m)$ EGF:$\dfrac{(e^x-1)^m}{m!}$

多项式牛顿迭代:如果已知 $G(F(x))\equiv0\pmod{x^{2n}}$,$G(F_*(x))\equiv0\pmod {x^n}$,则有 $F(x)\equiv F_*(x)-\frac{G(F_*(x))}{G'(F_*(x))}\pmod{x^{2n}}$。求导时孤立的多项式视为常数。

$\int_0^1 t^a(1-t)^b\mathrm{d}t=\dfrac{a!b!}{(a+b+1)!}$,$\sum\limits_{i=0}^{n-1}i^{\underline{k}}=\dfrac{n^{\underline{k+1}}}{k+1}$

Burnside 引理:等价类数量为 $\sum\limits_{g\in G}\dfrac{X^g}{|G|}$,$X^g$ 表示 $g$ 变换下不动点的数量。

Polya 定理:染色方案数为 $\sum\limits_{g\in G}\dfrac{m^{c(g)}}{|G|}$,其中 $c(g)$ 表示 $g$ 变换下环的数量。

假设已经只保留了一个牛人酋长,其名字为 $A=a_1a_2\cdots a_l$。

假设王国旁边开了一座赌场,每单位时间(就称为“秒”吧)会有一个赌徒带着 $1$ 铜币进入赌场。

赌场规则很简单:支付 $x$ 铜币赌下一秒会唱出 $y$,如果猜对了就返还 $nx$ 铜币,否则钱就没了。

每个赌徒会如下行动:支付 $1$ 铜币赌下一秒会唱出 $a_1$,如果赌对了就支付得到的 $n$ 铜币赌下一秒会唱出 $a_2$,如果还对了就支付得到的 $n^2$ 铜币赌下一秒会唱出 $a_3$,等等,以此类推,最后支付 $n^{l-1}$ 铜币赌下一秒会唱出 $a_l$。

一旦连续唱出了 $a_1a_2\cdots a_l$,赌场老板就会认为自己亏大了而关门,并驱散所有赌徒。

那么关门前发生了什么呢?以 $A=\{1,4,1,5,1,1,4,1\},n=5$ 为例:

- 最后一位赌徒拿着 $5$ 铜币离开;
- 倒数第三位赌徒拿着 $5^3$ 铜币离开;
- 倒数第八位赌徒拿着 $5^8$ 铜币离开;
- 其他所有赌徒空手而归。

我们可以发现 $1,3$ 恰好是原序列的所有 border 的长度,而且对于其他的名字也有这样的规律。

这时候最神奇的一步来了:由于这个赌博游戏是公平的,因此赌场应该期望下不赚不赔,因此关门时期望来了 $5+5^3+5^8$ 个赌徒,因此期望需要 $5+5^3+5^8$ 单位时间唱出这个名字。

同理,即可知道对于一般的 $A$,答案为:

$$\sum\limits_{a_1a_2\cdots a_c=a_{l-c+1}a_{l-c+2}\cdots a_l} n^c$$


\newpage

\section{语言基础}

\subsection{Makefile}

\begin{lstlisting}
%:%.cpp %.in
	g++ $< -o $@ -std=c++17 -D_GLIBCXX_DEBUG -D_GLIBCXX_DEBUG_PEDANTIC
	./$@ < $@.in
\end{lstlisting}
\subsection{初始代码}

\begin{lstlisting}
#include "bits/stdc++.h"
using namespace std;
typedef long long ll;
#define all(x) (x).begin(),(x).end()
int main()
{
	ios::sync_with_stdio(0); cin.tie(0);
	int T; cin>>T;
	while (T--)
	{

	}
}
\end{lstlisting}


\subsection{bitset}

\begin{lstlisting}
#include "bits/stdc++.h"
using namespace std;
bitset<10> f(12);
char s2[]="100101";
bitset<10> g(s2);
string s="100101";//reverse 了
bitset<10> h(s);
int main()
{
	for (int i=0;i<=9;i++) if (f[i]) printf("1"); else printf("0");puts("");
	for (int i=0;i<=9;i++) if (g[i]) printf("1"); else printf("0");puts("");
	for (int i=0;i<=9;i++) if (h[i]) printf("1"); else printf("0");puts("");
	cout<<h<<endl;
    foo.count();//1的个数
	foo.flip();//全部翻转
	foo.set();//变1
	foo.reset();//变0
	foo.to_string();
	foo.to_ulong();
	foo.to_ullong();
	foo._Find_first();
	foo._Find_next();
    //位运算:<< 变大,>> 变小
}
\end{lstlisting}

输出:

\begin{verbatim}
0011000000
1010010000
1010010000
0000100101
\end{verbatim}

\subsection{pb\_ds 和一些奇怪的用法}

\begin{lstlisting}
#pragma GCC optimize("Ofast")
#pragma GCC target("popcnt","sse3","sse2","sse","avx","sse4","sse4.1","sse4.2","ssse3","f16c","fma","avx2","xop","fma4")
#pragma GCC optimize("inline","fast-math","unroll-loops","no-stack-protector")
#include "bits/stdc++.h"
#include "ext/pb_ds/assoc_container.hpp"
#include "ext/pb_ds/tree_policy.hpp" //balanced tree
#include "ext/pb_ds/hash_policy.hpp" //hash table
#include "ext/pb_ds/priority_queue.hpp" //priority_queue
using namespace __gnu_pbds;
using namespace std;
typedef tree<int,null_type,less<int>,rb_tree_tag,tree_order_statistics_node_update> rbtree;
cc_hash_table<string,int>mp1;//拉链法
gp_hash_table<string,int>mp2;//查探法
rbtree s1,s2;//注意是不可重的
//null_type无映射(低版本g++为null_mapped_type)
//less<int>从小到大排序
//插入t.insert();
//删除t.erase();
//求有多少个数比 k 小:t.order_of_key(k);
//求树中第 k+1 小:t.find_by_order(k);
//a.join(b) b并入a,前提是两棵树的 key 的取值范围不相交,b 会清空但迭代器没事,如不满足会抛出异常。我听说复杂度是线性???
//a.split(v,b) key 小于等于 v 的元素属于 a,其余的属于 b
//T.lower_bound(x)  >=x 的 min 的迭代器
//T.upper_bound(x)  >x 的 min 的迭代器
__gnu_pbds::priority_queue<int,greater<int>,pairing_heap_tag> pq;
//join(priority_queue &other)  //合并两个堆,other会被清空
//split(Pred prd,priority_queue &other)  //分离出两个堆
//modify(point_iterator it,const key)  //修改一个节点的值
int main()
{
    __builtin_clz();//前导 0
    __builtin_ctz();//后面的 0
	ios::sync_with_stdio(0);cin.tie(0);
	mt19937 rnd(chrono::steady_clock::now().time_since_epoch().count());
    cout<<fixed<<setprecision(15);
	rbtree::iterator it;
	string s="abc",t="dabce";
	boyer_moore_horspool_searcher S(all(s));
	if (search(all(t),S)!=t.end())
	{
		cout<<"find\n";
	}
    uniform_real_distribution<> a(1,2);
	numeric_limits<int>::max();
}
\end{lstlisting}

\subsection{python 使用方法}

注意事项:python 容易爆栈,且引用与赋值较为混乱。注意局部变量的 global 怎么写(如果需要修改全局内容)。

文件操作

\begin{lstlisting}
fi = open("discuss.in", "r")
fo = open("discuss.out", "w")
n=int(fi.readline())
fo.write(str(ans))
\end{lstlisting}


类的构造,重载运算符
\begin{lstlisting}
class Q:
	def __init__(self,x,y):
		self.x=x
		self.y=y
	def __add__(self,o):
		r=Q(self.x+o.x,self.y+o.y)
		return r
	def __sub__(self,o):
		r=Q(self.x-o.x,self.y-o.y)
		return r
	def __mul__(self,o):
		return self.x*o.y-self.y*o.x
	def __lt__(self,o):
		if self.x!=o.x:
			return self.x<o.x
		return self.y<o.y
n,m=map(int,input().split())
c=list(map(int,input().split()))
print(*c)
a=Q(0,0)
b=Q(1,1)
if a<b-a:
	pass
\end{lstlisting}

\newpage

\section{其他人的板子(补充)}

\subsection{MTT+exp}

\begin{lstlisting}
#include"bits/stdc++.h"
using namespace std;
typedef long long ll;
typedef double db;
int read(){
	int res=0;
	char c=getchar(),f=1;
	while(c<48||c>57){if(c=='-')f=0;c=getchar();}
	while(c>=48&&c<=57)res=(res<<3)+(res<<1)+(c&15),c=getchar();
	return f?res:-res;
}

const int L=1<<19,mod=1e9+7;
const db pi2=3.141592653589793*2;
int inc(int x,int y){return x+y>=mod?x+y-mod:x+y;}
int dec(int x,int y){return x-y<0?x-y+mod:x-y;}
int mul(int x,int y){return (ll)x*y%mod;}
int qpow(int x,int y){
	int res=1;
	for(;y;y>>=1)res=y&1?mul(res,x):res,x=mul(x,x);
	return res;
}
int inv(int x){return qpow(x,mod-2);}

struct cp{
	db x,y;
	cp(){}
	cp(db a,db b){x=a,y=b;}
	cp operator+(const cp& p)const{return cp(x+p.x,y+p.y);}
	cp operator-(const cp& p)const{return cp(x-p.x,y-p.y);}
	cp operator*(const cp& p)const{return cp(x*p.x-y*p.y,x*p.y+y*p.x);}
	cp conj(){return cp(x,-y);}
}w[L];
int re[L];
int getre(int n){
	int len=1,bit=0;
	while(len<n)++bit,len<<=1;
	for(int i=1;i<len;++i)re[i]=(re[i>>1]>>1)|((i&1)<<(bit-1));
	return len;
}
void getw(){
	for(int i=0;i<L;++i)w[i]=cp(cos(pi2/L*i),sin(pi2/L*i));
}
void fft(cp* a,int len,int m){
	for(int i=1;i<len;++i)if(i<re[i])swap(a[i],a[re[i]]);
	for(int k=1,r=L>>1;k<len;k<<=1,r>>=1)
		for(int i=0;i<len;i+=k<<1)
			for(int j=0;j<k;++j){
				cp &L=a[i+j],&R=a[i+j+k],t=w[r*j]*R;
				R=L-t,L=L+t;
			}
	if(!~m){
		reverse(a+1,a+len);
		cp tmp=cp(1.0/len,0);
		for(int i=0;i<len;++i)a[i]=a[i]*tmp;
	}
}
void mul(int* a,int* b,int* c,int n1,int n2,int n){
	static cp f1[L],f2[L],f3[L],f4[L];
	int len=getre(n1+n2-1);
	for(int i=0;i<len;++i){
		f1[i]=i<n1?cp(a[i]>>15,a[i]&32767):cp(0,0);
		f2[i]=i<n2?cp(b[i]>>15,b[i]&32767):cp(0,0);
	}
	fft(f1,len,1),fft(f2,len,1);
	cp t1=cp(0.5,0),t2=cp(0,-0.5),r=cp(0,1);
	cp x1,x2,x3,x4;
	for(int i=0;i<len;++i){
		int j=(len-i)&(len-1);
		x1=(f1[i]+f1[j].conj())*t1;
		x2=(f1[i]-f1[j].conj())*t2;
		x3=(f2[i]+f2[j].conj())*t1;
		x4=(f2[i]-f2[j].conj())*t2;
		f3[i]=x1*(x3+x4*r);
		f4[i]=x2*(x3+x4*r);
	}
	fft(f3,len,-1),fft(f4,len,-1);
	ll c1,c2,c3,c4;
	for(int i=0;i<n;++i){
		c1=(ll)(f3[i].x+0.5)%mod,c2=(ll)(f3[i].y+0.5)%mod;
		c3=(ll)(f4[i].x+0.5)%mod,c4=(ll)(f4[i].y+0.5)%mod;
		c[i]=((((c1<<15)+c2+c3)<<15)+c4)%mod;
	}
}
void inv(int* a,int* b,int n){
	if(n==1){b[0]=1;return;}
	static int c[L];
	int l=(n+1)>>1;
	inv(a,b,l);
	mul(a,b,c,n,l,n);
	for(int i=0;i<n;++i)c[i]=mod-c[i];
	c[0]+=2;
	mul(b,c,b,n,n,n);
}
void der(int* a,int n){
	for(int i=1;i<n;++i)a[i-1]=mul(a[i],i);
	a[n-1]=0;
}
void its(int* a,int n){
	for(int i=n-1;i;--i)a[i]=mul(a[i-1],inv(i));
	a[0]=0;
}
void ln(int* a,int* b,int n){
	static int c[L];
	for(int i=0;i<n;++i)c[i]=a[i];
	der(c,n);
	inv(a,b,n);
	mul(b,c,b,n,n,n);
	its(b,n);
}
void exp(int* a,int* b,int n){
	if(n==1){b[0]=1;return;}
	static int c[L];
	int l=(n+1)>>1;
	exp(a,b,l);
	ln(b,c,n);
	for(int i=0;i<n;++i)c[i]=dec(a[i],c[i]);
	++c[0];
	mul(b,c,b,l,n,n);
	for(int i=0;i<n;++i)c[i]=0;
}

int n,k,a[L],f[L],g[L];
int main(){
	getw();
	n=read(),k=read();
	for(int i=1;i<=k;++i)a[i]=inv(i);
	for(int i=2;i<=n;++i)
		for(int j=1;i*j<=k;++j)
			f[i*j]=inc(f[i*j],a[j]);
	for(int i=1;i<=k;++i)f[i]=mod-f[i];
	for(int i=1;i<=k;++i)f[i]=inc(f[i],mul(n-1,a[i]));
	exp(f,g,k+1);
	printf("%d\n",g[k]);
}
\end{lstlisting}

\subsection{半平面交}

\begin{lstlisting}
const int N=305;
const db inf=1e15,eps=1e-10;
int sign(db x){
	if(fabs(x)<eps)return 0;
	return x>0?1:-1;
}

struct vec{
	db x,y;
	vec(){}
	vec(db a,db b){x=a,y=b;}
	vec operator+(const vec& p)const{
		return vec(x+p.x,y+p.y);
	}
	vec operator-(const vec& p)const{
		return vec(x-p.x,y-p.y);
	}
	db operator*(const vec& p)const{
		return x*p.y-y*p.x;
	}
	vec operator*(const db& p)const{
		return vec(x*p,y*p);
	}
}p1[N],p2[N];

struct line{
	vec s,t;
	line(){}
	line(vec a,vec b){s=a,t=b;}
}a[N],q[N];
db ang(vec v){
	return atan2(v.y,v.x);
}
db ang(line l){
	return ang(l.t-l.s);
}
bool cmp(line x,line y){
	int s=sign(ang(x)-ang(y));
	return s?s<0:sign((x.t-x.s)*(y.t-x.s))>0;
}

vec inter(line x,line y){
	vec a=y.s-x.s,b=x.t-x.s,c=y.t-y.s;
	return y.s+c*((a*b)/(b*c));
}
bool out(line l,vec p){
	return sign((l.t-l.s)*(p-l.s))<0;
}

int n,tot=0;
db ans=inf;
int main(){
	scanf("%d",&n);
	for(int i=1;i<=n;++i)scanf("%lf",&p1[i].x);
	for(int i=1;i<=n;++i)scanf("%lf",&p1[i].y);
	for(int i=1;i<n;++i)a[i]=line(p1[i],p1[i+1]);
	a[n]=line(vec(p1[1].x,inf),vec(p1[1].x,p1[1].y));
	a[n+1]=line(vec(p1[n].x,p1[n].y),vec(p1[n].x,inf));
	
	sort(a+1,a+n+2,cmp);
	for(int i=1;i<=n;++i){
		if(!sign(ang(a[i])-ang(a[i+1])))continue;
		a[++tot]=a[i];
	}a[++tot]=a[n+1];
	
	int l=1,r=0;
	q[++r]=a[1],q[++r]=a[2];
	for(int i=3;i<=tot;++i){
		while(l<r&&out(a[i],inter(q[r],q[r-1])))--r;
		while(l<r&&out(a[i],inter(q[l],q[l+1])))++l;
		q[++r]=a[i];
	}
	while(l<r&&out(q[l],inter(q[r],q[r-1])))--r;
	while(l<r&&out(q[r],inter(q[l],q[l+1])))++l;
//......
}
\end{lstlisting}

\subsection{多项式复合 (yurzhang)}

$O(n\log n\sqrt{n\log n})$,奇慢无比,慎用

\begin{lstlisting}
#pragma GCC optimize("Ofast,inline")
#pragma GCC target("sse,sse2,sse3,ssse3,sse4,sse4.1,sse4.2,popcnt,abm,mmx,avx,avx2,tune=native")
#include <cstdio>
#include <cstring>
#include <cmath>
#include <algorithm>

#define MOD 998244353
#define G 332748118
#define N 262210
#define re register
#define gc pa==pb&&(pb=(pa=buf)+fread(buf,1,100000,stdin),pa==pb)?EOF:*pa++
typedef long long ll;
static char buf[100000],*pa(buf),*pb(buf);
static char pbuf[3000000],*pp(pbuf),st[15];
int read() {
    re int x(0);re char c(gc);
    while(c<'0'||c>'9')c=gc;
    while(c>='0'&&c<='9')
        x=x*10+c-48,c=gc;
    return x;
}
void write(re int v) {
    if(v==0)
        *pp++=48;
    else {
        re int tp(0);
        while(v)
            st[++tp]=v%10+48,v/=10;
        while(tp)
            *pp++=st[tp--];
    }
    *pp++=32;
}

int pow(re int a,re int b) {
    re int ans(1);
    while(b)
        ans=b&1?(ll)ans*a%MOD:ans,a=(ll)a*a%MOD,b>>=1;
    return ans;
}

int inv[N],ifac[N];
void pre(re int n) {
    inv[1]=ifac[0]=1;
    for(re int i(2);i<=n;++i)
        inv[i]=(ll)(MOD-MOD/i)*inv[MOD%i]%MOD;
    for(re int i(1);i<=n;++i)
        ifac[i]=(ll)ifac[i-1]*inv[i]%MOD;
}

int getLen(re int t) {
	return 1<<(32-__builtin_clz(t));
}

int lmt(1),r[N],w[N];
void init(re int n) {
	re int l(0);
	while(lmt<=n)
		lmt<<=1,++l;
	for(re int i(1);i<lmt;++i)
		r[i]=(r[i>>1]>>1)|((i&1)<<(l-1));
	re int wn(pow(3,(MOD-1)/lmt));
	w[lmt>>1]=1;
	for(re int i((lmt>>1)+1);i<lmt;++i)
		w[i]=(ll)w[i-1]*wn%MOD;
	for(re int i((lmt>>1)-1);i;--i)
		w[i]=w[i<<1];
}

void DFT(int*a,re int l) {
	static unsigned long long tmp[N];
	re int u(__builtin_ctz(lmt)-__builtin_ctz(l)),t;
	for(re int i(0);i<l;++i)
		tmp[i]=(a[r[i]>>u])%MOD;
	for(re int i(1);i<l;i<<=1)
		for(re int j(0),step(i<<1);j<l;j+=step)
			for(re int k(0);k<i;++k)
				t=(ll)w[i+k]*tmp[i+j+k]%MOD,
				tmp[i+j+k]=tmp[j+k]+MOD-t,
				tmp[j+k]+=t;
	for(re int i(0);i<l;++i)
		a[i]=tmp[i]%MOD;
}

void IDFT(int*a,re int l) {
	std::reverse(a+1,a+l);DFT(a,l);
	re int bk(MOD-(MOD-1)/l);
	for(re int i(0);i<l;++i)
		a[i]=(ll)a[i]*bk%MOD;
}

int n,m;
int a[N],b[N],c[N];

void getInv(int*a,int*b,int deg) {
    if(deg==1)
        b[0]=pow(a[0],MOD-2);
    else {
        static int tmp[N];
        getInv(a,b,(deg+1)>>1);
        re int l(getLen(deg<<1));
        for(re int i(0);i<l;++i)
            tmp[i]=i<deg?a[i]:0;
        DFT(tmp,l),DFT(b,l);
        for(re int i(0);i<l;++i)
            b[i]=(2ll-(ll)tmp[i]*b[i]%MOD+MOD)%MOD*b[i]%MOD;
        IDFT(b,l);
        for(re int i(deg);i<l;++i)
            b[i]=0;
    }
}

void getDer(int*a,int*b,int deg) {
    for(re int i(0);i+1<deg;++i)
        b[i]=(ll)a[i+1]*(i+1)%MOD;
    b[deg-1]=0;
}

void getComp(int*a,int*b,int k,int m,int&n,int*c,int*d) {
    if(k==1) {
        for(re int i(0);i<m;++i)
            c[i]=0,d[i]=b[i];
        n=m,c[0]=a[0];
    } else {
        static int t1[N],t2[N];
        int nl(n),nr(n),*cl,*cr,*dl,*dr;
        getComp(a,b,k>>1,m,nl,cl=c,dl=d);
        getComp(a+(k>>1),b,(k+1)>>1,m,nr,cr=c+nl,dr=d+nl);
        n=std::min(n,nl+nr-1);
        re int _l(getLen(nl+nr));
        for(re int i(0);i<_l;++i)
            t1[i]=i<nl?dl[i]:0;
        for(re int i(0);i<_l;++i)
            t2[i]=i<nr?cr[i]:0;
        DFT(t1,_l),DFT(t2,_l);
        for(re int i(0);i<_l;++i)
            t2[i]=(ll)t1[i]*t2[i]%MOD;
        IDFT(t2,_l);
        for(re int i(0);i<n;++i)
            c[i]=((i<nl?cl[i]:0)+t2[i])%MOD;
        for(re int i(0);i<_l;++i)
            t2[i]=i<nr?dr[i]:0;
        DFT(t2,_l);
        for(re int i(0);i<_l;++i)
            t2[i]=(ll)t1[i]*t2[i]%MOD;
        IDFT(t2,_l);
        for(re int i(0);i<n;++i)
            d[i]=t2[i];
    }
}

void getComp(int*a,int*b,int*c,int deg) {
    static int ts[N],ps[N],c0[N],_t1[N],idM[N];
    int M(std::max((int)ceil(sqrt(deg/log2(deg))*2.5),2)),_n(deg+deg/M);
    getComp(a,b,deg,M,_n,c0,_t1);
    re int _l(getLen(_n+deg));
    for(re int i(_n);i<_l;++i)
        c0[i]=0;
    for(re int i(0);i<_l;++i)
        ps[i]=i==0;
    for(re int i(0);i<_l;++i)
        ts[i]=M<=i&&i<deg?b[i]:0;
    getDer(b,_t1,M);
    for(re int i(M-1);i<deg;++i)
        _t1[i]=0; /// Important!!!
    getInv(_t1,idM,deg);
    for(int i=deg;i<_l;++i)
    	idM[i]=0;
    DFT(ts,_l),DFT(idM,_l);
    for(re int t(0);t*M<deg;++t) {
        for(re int i(0);i<_l;++i)
            _t1[i]=i<deg?c0[i]:0;
        DFT(ps,_l),DFT(_t1,_l);
        for(re int i(0);i<_l;++i)
            _t1[i]=(ll)_t1[i]*ps[i]%MOD,
            ps[i]=(ll)ps[i]*ts[i]%MOD;
        IDFT(ps,_l),IDFT(_t1,_l);
        for(re int i(deg);i<_l;++i)
            ps[i]=0;
        for(re int i(0);i<deg;++i)
            c[i]=((ll)_t1[i]*ifac[t]+c[i])%MOD;
        getDer(c0,c0,_n);
        for(re int i(_n-1);i<_l;++i)
            c0[i]=0;
        DFT(c0,_l);
        for(re int i(0);i<_l;++i)
            c0[i]=(ll)c0[i]*idM[i]%MOD;
        IDFT(c0,_l);
        for(re int i(_n-1);i<_l;++i)
            c0[i]=0;
    }
}

int main() {
    n=read(),m=read();
    for(re int i(0);i<=n;++i)
        a[i]=read();
    for(re int i(0);i<=m;++i)
        b[i]=read();
    
    m=(n>m?n:m)+1;
    pre(m);init(m*5);
    getComp(a,b,c,m);
    
    for(re int i(0);i<=n;++i)
        write(c[i]);
    fwrite(pbuf,1,pp-pbuf,stdout);
    return 0;
}
\end{lstlisting}

\subsection{下降幂多项式乘法}

$O(n\log n)$。

\begin{lstlisting}
#include<cstdio>
#include<algorithm>
const int N=524288,md=998244353,g3=(md+1)/3;
typedef long long LL;
int n,m,A[N],B[N],fac[N],iv[N],rev[N],C[N],g[20][N],lim,M;
int pow(int a,int b){
    int ret=1;
    for(;b;b>>=1,a=(LL)a*a%md)if(b&1)ret=(LL)ret*a%md;
    return ret;
}
void upd(int&a){a+=a>>31&md;}
void init(int n){
    int l=-1;
    for(lim=1;lim<n;lim<<=1)++l;M=l+1;
    for(int i=1;i<lim;++i)
    rev[i]=((rev[i>>1])>>1)|((i&1)<<l);
}
void NTT(int*a,int f){
    for(int i=1;i<lim;++i)if(i<rev[i])std::swap(a[i],a[rev[i]]);
    for(int i=0;i<M;++i){
        const int*G=g[i],c=1<<i;
        for(int j=0;j<lim;j+=c<<1)
        for(int k=0;k<c;++k){
            const int x=a[j+k],y=a[j+k+c]*(LL)G[k]%md;
            upd(a[j+k]+=y-md),upd(a[j+k+c]=x-y);
        }
    }
    if(!f){
        const int iv=pow(lim,md-2);
        for(int i=0;i<lim;++i)a[i]=(LL)a[i]*iv%md;
        std::reverse(a+1,a+lim);
    }
}
int main(){
    scanf("%d%d",&n,&m);++n,++m;
    for(int i=0;i<20;++i){
        int*G=g[i];
        G[0]=1;
        const int gi=G[1]=pow(3,(md-1)/(1<<i+1));
        for(int j=2;j<1<<i;++j)G[j]=(LL)G[j-1]*gi%md;
    }
    for(int i=0;i<n;++i)scanf("%d",A+i);
    for(int i=0;i<m;++i)scanf("%d",B+i);
    for(int i=*fac=1;i<N;++i)
    fac[i]=fac[i-1]*(LL)i%md;
    iv[N-1]=pow(fac[N-1],md-2);
    for(int i=N-2;~i;--i)iv[i]=(i+1LL)*iv[i+1]%md;
    init(n+m<<1);
    for(int i=0;i<n+m-1;++i)C[i]=iv[i];
    NTT(A,1),NTT(B,1),NTT(C,1);
    for(int i=0;i<lim;++i)A[i]=(LL)A[i]*C[i]%md,B[i]=(LL)B[i]*C[i]%md;
    NTT(A,0),NTT(B,0);
    for(int i=0;i<lim;++i)C[i]=0;
    for(int i=0;i<n+m-1;++i)
    C[i]=(i&1)?md-iv[i]:iv[i];
    for(int i=0;i<lim;++i)A[i]=(LL)A[i]*B[i]%md*fac[i]%md;
    for(int i=n+m-1;i<lim;++i)A[i]=0;
    NTT(A,1),NTT(C,1);
    for(int i=0;i<lim;++i)A[i]=(LL)A[i]*C[i]%md;
    NTT(A,0);
    for(int i=0;i<n+m-1;++i)printf("%d%c",A[i]," \n"[i==n+m-2]);
    return 0;
}
\end{lstlisting}

\subsection{弦图找错}

\begin{lstlisting}
#include "bits/stdc++.h"
using namespace std;
const int MAXN = 200005;
using lint = long long;
using pi = pair<int, int>;
// the algorithm may be wrong. if you have any ideas for proving / disproving this, please contact me.
vector<int> gph[MAXN];
int n, m, cnt[MAXN], idx[MAXN];
int mark[MAXN], vis[MAXN], par[MAXN];
void report(int x, int y){
	gph[x].erase(find(gph[x].begin(), gph[x].end(), y));
	gph[y].erase(find(gph[y].begin(), gph[y].end(), x));
	for(int i=1; i<=n; i++){
		if(binary_search(gph[i].begin(), gph[i].end(), x) && 
			binary_search(gph[i].begin(), gph[i].end(), y)){
			mark[i] = 1;
		}
	}
	queue<int> que;
	vis[x] = 1;
	que.push(x);
	while(!que.empty()){
		int x = que.front(); que.pop();
		for(auto &i : gph[x]){
			if(!mark[i] && !vis[i]){
				par[i] = x;
				vis[i] = 1;
				que.push(i);
			}
		}
	}
	assert(vis[y]);
	vector<int> v;
	while(y){
		v.push_back(y);
		y = par[y];
	}
	printf("NO\n%d\n", v.size());
	for(auto &i : v) printf("%d ", i-1);
}

int main(){
	scanf("%d %d",&n,&m);
	for(int i=0; i<m; i++){
		int s, e; scanf("%d %d",&s,&e);
		s++, e++;
		gph[s].push_back(e);
		gph[e].push_back(s);
	}
	for(int i=1; i<=n; i++) sort(gph[i].begin(), gph[i].end());
	priority_queue<pi> pq;
	for(int i=1; i<=n; i++) pq.emplace(cnt[i], i);
	vector<int> ord;
	while(!pq.empty()){
		int x = pq.top().second, y = pq.top().first;
		pq.pop();
		if(cnt[x] != y || idx[x]) continue;
		ord.push_back(x);
		idx[x] = n + 1 - ord.size();
		for(auto &i : gph[x]){
			if(!idx[i]){
				cnt[i]++;
				pq.emplace(cnt[i], i);
			}
		}
	}
	reverse(ord.begin(), ord.end());
	for(auto &i : ord){
		int minBef = 1e9;
		for(auto &j : gph[i]){
			if(idx[j] > idx[i]) minBef = min(minBef, idx[j]);
		}
		minBef--;
		if(minBef < n){
			minBef = ord[minBef];
			for(auto &j : gph[i]){
				if(idx[j] > idx[minBef] && !binary_search(gph[minBef].begin(), gph[minBef].end(), j)){
					report(minBef, i);
					return 0;
				}
			}
		}
	}
	puts("YES");
	for(auto &i : ord) printf("%d ", i-1);
}
\end{lstlisting}

\subsection{最长公共子序列}

复杂度 $O(\frac{nm}{\omega})$。

\begin{lstlisting}
/*
 * Author : _Wallace_
 * Source : https://www.cnblogs.com/-Wallace-/
 * Problem : LOJ #6564. 最长公共子序列
 * Standard : GNU C++ 03
 * Optimal : -Ofast
 */
#include <algorithm>
#include <cstddef>
#include <cstdio>
#include <cstring>

typedef unsigned long long ULL;

const int N = 7e4 + 5;
int n, m, u;

struct bitset {
  ULL t[N / 64 + 5];

  bitset() {
    memset(t, 0, sizeof(t));
  }
  bitset(const bitset &rhs) {
    memcpy(t, rhs.t, sizeof(t));
  }

  bitset& set(int p) {
    t[p >> 6] |= 1llu << (p & 63);
    return *this;
  }
  bitset& shift() {
    ULL last = 0llu;
    for (int i = 0; i < u; i++) {
      ULL cur = t[i] >> 63;
      (t[i] <<= 1) |= last, last = cur;
    }
    return *this;
  }
  int count() {
    int ret = 0;
    for (int i = 0; i < u; i++)
      ret += __builtin_popcountll(t[i]);
    return ret;
  }

  bitset& operator = (const bitset &rhs) {
    memcpy(t, rhs.t, sizeof(t));
    return *this;
  }
  bitset& operator &= (const bitset &rhs) {
    for (int i = 0; i < u; i++) t[i] &= rhs.t[i];
    return *this;
  }
  bitset& operator |= (const bitset &rhs) {
    for (int i = 0; i < u; i++) t[i] |= rhs.t[i];
    return *this;
  }
  bitset& operator ^= (const bitset &rhs) {
    for (int i = 0; i < u; i++) t[i] ^= rhs.t[i];
    return *this;
  }

  friend bitset operator - (const bitset &lhs, const bitset &rhs) {
    ULL last = 0llu; bitset ret;
    for (int i = 0; i < u; i++){
      ULL cur = (lhs.t[i] < rhs.t[i] + last);
      ret.t[i] = lhs.t[i] - rhs.t[i] - last;
      last = cur;
    }
    return ret;
  }
} p[N], f, g;

signed main() {
  scanf("%d%d", &n, &m), u = n / 64 + 1;
  for (int i = 1, c; i <= n; i++)
    scanf("%d", &c), p[c].set(i);
  for (int i = 1, c; i <= m; i++) {
    scanf("%d", &c), (g = f) |= p[c];
    f.shift(), f.set(0);
    ((f = g - f) ^= g) &= g;
  }
  printf("%d\n", f.count());
  return 0;
}
\end{lstlisting}

另一个实现

\begin{lstlisting}
#include "bits/stdc++.h"
#pragma GCC target("popcnt,bmi")

using namespace std;
using ull = uint64_t;

const int N = 70005, M = 1136;

int n, m;
ull g[N][M], f[M];

int read() {
    const int M = 1e6;
    static streambuf *in = cin.rdbuf();
#define gc (p1 == p2 && (p2 = (p1 = buf) + in -> sgetn(buf, M), p1 == p2) ? -1 : *p1++)
    static char buf[M], *p1, *p2;
    int c = gc, r = 0;

    while (c < 48)
        c = gc;

    while (c > 47)
        r = r * 10 + (c & 15), c = gc;

    return r;
}
int main() {
    cin.tie(0)->sync_with_stdio(0);
    cin >> n >> m;

    for (int i = 0; i < n; i++)
        g[read()][i / 62] |= 1ULL << (i % 62);

    int lim = (n - 1) / 62;

    for (int i = 0; i < m; i++) {
        int c = 1;
        auto can = g[read()];

        for (int j = 0; j <= lim; j++) {
            ull x = f[j], y = x | can[j];
            x += x + c + (~y & (1ULL << 62) - 1);
            f[j] = x & y, c = x >> 62;
        }
    }

    int ans = 0;

    for (int i = 0; i <= lim; i++)
        ans += __builtin_popcountll(f[i]);

    cout << ans;
}
\end{lstlisting}

\subsection{区间 LIS(排列)}

\begin{lstlisting}
#include"bits/stdc++.h"
using namespace std;
//dengyaotriangle!

const int maxn=100005;

int pool[(int)5e7];int ps;
inline int *aloc(int x){
    ps+=x;return pool+ps-x;
}
void unit_monge_mult(int *a,int *b,int *r,int n){
    if(n==2){
        if(a[0]==0&&b[0]==0)r[0]=0,r[1]=1;
        else r[0]=1,r[1]=0;
        return;
    }
    if(n==1){r[0]=0;return;}
    int lps=ps;
    int d=n/2;
    int *a1=aloc(d),*a2=aloc(n-d),*b1=aloc(d),*b2=aloc(n-d);
    int *mpa1=aloc(d),*mpa2=aloc(n-d),*mpb1=aloc(d),*mpb2=aloc(n-d);
    int p[2]={0,0};
    for(int i=0;i<n;i++){
        if(a[i]<d)a1[p[0]]=a[i],mpa1[p[0]]=i,p[0]++;
        else a2[p[1]]=a[i]-d,mpa2[p[1]]=i,p[1]++;
    }
    p[0]=p[1]=0;
    for(int i=0;i<n;i++){
        if(b[i]<d)b1[p[0]]=b[i],mpb1[p[0]]=i,p[0]++;
        else b2[p[1]]=b[i]-d,mpb2[p[1]]=i,p[1]++;
    }
    int *c1=aloc(d),*c2=aloc(n-d);
    unit_monge_mult(a1,b1,c1,d),unit_monge_mult(a2,b2,c2,n-d);
    int *cpx=aloc(n),*cpy=aloc(n),*cqx=aloc(n),*cqy=aloc(n);
    for(int i=0;i<d;i++)cpx[mpa1[i]]=mpb1[c1[i]],cpy[mpa1[i]]=0;
    for(int i=0;i<n-d;i++)cpx[mpa2[i]]=mpb2[c2[i]],cpy[mpa2[i]]=1;
    for(int i=0;i<n;i++)r[i]=cpx[i];
    for(int i=0;i<n;i++)cqx[cpx[i]]=i,cqy[cpx[i]]=cpy[i];
    int hi=n,lo=n,his=0,los=0;
    for(int i=0;i<n;i++){
        if(cqy[i]^(cqx[i]>=hi))his--;
        while(hi>0&&his<0){
            hi--;
            if(cpy[hi]^(cpx[hi]>i))his++;
        }
        while(lo>0&&los<=0){
            lo--;
            if(cpy[lo]^(cpx[lo]>=i))los++;
        }
        if(los>0&&hi==lo)r[lo]=i;
        if(cqy[i]^(cqx[i]>=lo))los--;
    }
    ps=lps;
}
void subunit_monge_mult(int*a,int*b,int*c,int n){
    int lps=ps;
    int *za=aloc(n),*zb=aloc(n),*res=aloc(n),*vis=aloc(n),*mpa=aloc(n),*mpb=aloc(n),*rb=aloc(n);
    memset(vis,0,sizeof(int)*n);
    memset(mpa,-1,sizeof(int)*n);
    memset(mpb,-1,sizeof(int)*n);
    memset(rb,-1,sizeof(int)*n);
    int ca=n;
    for(int i=n-1;i>=0;i--)if(a[i]!=-1){
        vis[a[i]]=1;ca--;za[ca]=a[i];mpa[ca]=i;
    }
    for(int i=n-1;i>=0;i--)if(!vis[i])za[--ca]=i;
    memset(vis,-1,sizeof(int)*n);
    for(int i=0;i<n;i++)if(b[i]!=-1)vis[b[i]]=i;
    ca=0;
    for(int i=0;i<n;i++)if(vis[i]!=-1){
        mpb[ca]=i;rb[vis[i]]=ca++;
    }
    for(int i=0;i<n;i++)if(rb[i]==-1)rb[i]=ca++;
    for(int i=0;i<n;i++)zb[rb[i]]=i;
    unit_monge_mult(za,zb,res,n);
    memset(c,-1,sizeof(int)*n);
    for(int i=0;i<n;i++)if(mpa[i]!=-1&&mpb[res[i]]!=-1)c[mpa[i]]=mpb[res[i]];
    ps=lps;
}


void solve(int *p,int *ret,int n){
    if(n==1){ret[0]=-1;return;}
    int lps=ps,d=n/2;
    int *pl=aloc(d),*pr=aloc(n-d);
    for(int i=0;i<d;i++)pl[i]=p[i];
    for(int i=0;i<n-d;i++)pr[i]=p[i+d];
    int *vis=aloc(n);memset(vis,-1,sizeof(int)*n);
    for(int i=0;i<d;i++)vis[pl[i]]=i;
    int *tl=aloc(d),*tr=aloc(n-d),*mpl=aloc(d),*mpr=aloc(n-d);
    int ca=0;
    for(int i=0;i<n;i++)if(vis[i]!=-1)mpl[ca]=i,tl[vis[i]]=ca++;
    ca=0;memset(vis,-1,sizeof(int)*n);
    for(int i=0;i<n-d;i++)vis[pr[i]]=i;
    for(int i=0;i<n;i++)if(vis[i]!=-1)mpr[ca]=i,tr[vis[i]]=ca++;
    int *vl=aloc(d),*vr=aloc(n-d);
    solve(tl,vl,d),solve(tr,vr,n-d);
    int *sl=aloc(n),*sr=aloc(n);
    iota(sl,sl+n,0);iota(sr,sr+n,0);
    for(int i=0;i<d;i++)sl[mpl[i]]=(vl[i]==-1?-1:mpl[vl[i]]);
    for(int i=0;i<n-d;i++)sr[mpr[i]]=(vr[i]==-1?-1:mpr[vr[i]]);
    subunit_monge_mult(sl,sr,ret,n);
    ps=lps;
}
int invp[maxn],res_monge[maxn];
int main(){
    ios::sync_with_stdio(0);cin.tie(0);
    int n,q;
    cin>>n>>q;
    vector<int> a(n);
    for(int i=0;i<n;i++)cin>>a[i],invp[a[i]]=i;
    solve(invp,res_monge,n);
    vector<int> fwk(n+1),ans(q);
    vector<vector<pair<int,int> > > qry(n+1);
    for(int i=0;i<q;i++){
        int l,r;
        cin>>l>>r;
        qry[l].push_back({r,i});
        ans[i]=r-l;
    }
    for(int i=n-1;i>=0;i--){
        if(res_monge[i]!=-1){
            for(int p=res_monge[i]+1;p<=n;p+=p&-p)fwk[p]++;
        }
        for(auto& z:qry[i]){
            int id,c;tie(id,c)=z;
            for(int p=id;p;p-=p&-p)ans[c]-=fwk[p];
        }
    }
    for(int i=0;i<q;i++)cout<<ans[i]<<'\n';
    return 0;
}
\end{lstlisting}

\subsection{区间 LCS}

$s_{[0,a)}$ 和 $t_{[b,c)}$ 的 LCS

\begin{lstlisting}
#include"bits/stdc++.h"
using namespace std;
//dengyaotriangle!

const int maxn=1005;
const int maxq=500005;
int n,m,q;
char a[maxn],b[maxn];
struct qryt{
    int x,nxt;
}z[maxq];
int qry[maxn][maxn];
int ans[maxq];
int r[maxn];
int bit[maxn];

int main(){
    ios::sync_with_stdio(0);cin.tie(0);
    cin>>q>>b>>a;n=strlen(a);m=strlen(b);
	//q,s,t
    for(int i=1;i<=q;i++){
        int a,b,c;
        cin>>a>>b>>c;
        if(a){
            ans[i]=c-b;
            z[i].x=b;z[i].nxt=qry[a][c];
            qry[a][c]=i;
        }
    }    
    for(int i=0;i<n;i++)r[i]=i;
    for(int i=0;i<m;i++){
        int lp=-1;
        for(int j=0;j<n;j++)if(a[j]==b[i]){lp=j;break;}
        if(lp!=-1){
            for(int j=lp+1;j<n;j++){
                if(a[j]!=b[i]){
                    if(r[j-1]<r[j])swap(r[j-1],r[j]);
                }
            }
            for(int i=n-1;i>lp;i--)r[i]=r[i-1];
            r[lp]=-1;
        }
        for(int i=0;i<=n;i++)bit[i]=0;
        for(int j=0;j<n;j++){
            if(r[j]!=-1){
                for(int p=n-r[j];p<=n;p+=p&-p)bit[p]++;
            }
            for(int y=qry[i+1][j+1];y;y=z[y].nxt){
                for(int p=n-z[y].x;p;p-=p&-p)ans[y]-=bit[p];
            }
        }
    }
    for(int i=1;i<=q;i++)cout<<ans[i]<<'\n';
    return 0;
}
\end{lstlisting}

\subsection{毛毛虫剖分}
毛毛虫剖分,一种由轻重链剖分(HLD)推广而成的树上结点重标号方法,支持修改 / 查询一只毛毛虫的信息,并且可以对毛毛虫的身体和足分别修改 / 查询不同信息 .

严格强于树剖,而且复杂度和树剖一样哦!

一些定义(默认在一棵树上):

毛毛虫:一条链和与这条链邻接的所有结点构成的集合 .
虫身(身体):毛毛虫的链部分 .
虫足(足):毛毛虫除虫身的部分 .
重标号方法
首先重剖求出重链 .
DFS,若现在处理到结点 u:
若 u 还未被标号,则为其标号 .
若 u 是重链头,遍历这条重链,将邻接这条链的结点依次标号 .
先递归重儿子,再递归轻儿子 .
重标号性质
对于重链,除链头外的结点标号连续 .
对于任意结点,其轻儿子标号连续 .
对于以重链头为根的子树,与这条重链邻接的所有结点标号连续 .
这样就可以随便维护毛毛虫信息了,顺便还能维护链信息,子树信息等 .

时间复杂度同轻重链剖分 .

以 SAM 为例,若我们只保留所有的转移边 $(u,v)$ ,满足到达 $u$ 的路径数目大于到达 $v$ 的路径数目一半,且从 $v$ 出发的路径数目大于从 $u$ 出发的路径数目一半,这样剩余的子图显然会形成若干条链,且每个点恰好在一条链上。这样,我们容易证明,从根结点出发的任何一条路径,至多经过 $O(\log n)$ 条不在链上的转移边(也意味着至多经过 $O(\log n)$ 条链) 。

以下是一段示例代码,展示了将一条链对应区间取出来的过程

\begin{lstlisting}

vector<int> e[N];
vector<pair<int, int>> seg[N], qu[N];
int ans[Q];
int dfn[N], dep[N], nfd[N], top[N], f[N], sz[N], hc[N], pre[N], fir[N], lst2[N], rt[N];
int
void insert()
void dfs1(int u)
{
    sz[u] = 1;
    for (int v : e[u]) if (v != f[u])
    {
        dep[v] = dep[u] + 1;
        f[v] = u;
        dfs1(v);
        sz[u] += sz[v];
        if (sz[v] > sz[hc[u]]) hc[u] = v;
    }
    if (f[u]) erase(e[u], f[u]);
}
void dfs2(int u)
{
    static int id = 0;
    //dbg(u);
    if (!dfn[u])
    {
        dfn[u] = ++id;
        nfd[id] = u;
    }
    if (top[u] == u)
    {
        vector<int> stk;
        for (int v = u;v;v = hc[v])
        {
            for (int w : e[v]) if (w != hc[v])
            {
                dfn[w] = ++id;
                nfd[id] = w;
                pre[v] = id;
                cmin(fir[v], id);
                lst2[v] = id;
            }
            stk.push_back(v);
        }
        for (int i = (int)stk.size() - 2;i >= 0;i--)
        {
            cmin(fir[stk[i]], fir[stk[i + 1]]);
            cmax(lst2[stk[i]], lst2[stk[i + 1]]);
        }
        for (int i = 1;i < stk.size();i++)
        {
            cmax(pre[stk[i]], pre[stk[i - 1]]);
        }
    }
    //dbg(u);
    top[hc[u]] = top[u];
    if (hc[u]) dfs2(hc[u]);
    for (int v : e[u]) if (v != hc[u]) dfs2(top[v] = v);
}
mt19937 rnd(245);
int main()
{
    memset(fir, 0x3f, sizeof fir);
    ios::sync_with_stdio(0); cin.tie(0);
    cout << fixed << setprecision(15);
    int n, m, q, i, j;
    cin >> n >> m >> q;
    for (i = 1;i < n;i++)
    {
        int u, v;
        //cin >> u >> v;
        u = i + 1;
        v = rnd() % i + 1;
        //v = (i + 1) / 2;
        //v = i / 2 + 1;
        //dbg(u, v);
        e[u].push_back(v);
        e[v].push_back(u);
    }
    dfs1(dep[1] = 1);
    //dbg("??");
    dfs2(top[1] = 1);
    //for (i = 1;i <= n;i++) cerr << i << ": " << dfn[i] << endl;
    for (i = 1;i <= m;i++)
    {
        int u, v;
        //cin >> u >> v;
        u = rnd() % n + 1;
        v = rnd() % n + 1;
        int uu = u, vv = v;
        //dbg(uu, vv);
        auto& w = seg[i];
        while (top[u] != top[v])
        {
            if (dep[top[u]] < dep[top[v]]) swap(u, v);
            w.push_back({fir[top[u]], pre[u]});
            //else w.push_back({fir[top[u]], lst2[top[u]]});
            if (hc[u]) w.push_back({dfn[hc[top[u]]], dfn[hc[u]]});
            else if (top[u] != u) w.push_back({dfn[hc[top[u]]], dfn[u]});
            //dbg(u, v, w);
            //[fir[top[u]],lst[u]]
            u = f[top[u]];
        }
        if (dep[u] < dep[v]) swap(u, v);
        w.push_back({fir[v], pre[u]});
        //else if (!hc[u]) w.push_back({fir[v], lst2[v]});
        //dbg(v, lst2[v], fir[v]);
        if (hc[u]) w.push_back({dfn[hc[v]], dfn[hc[u]]});
        else if (u != v) w.push_back({dfn[hc[v]], dfn[u]});
        //dbg(w);
        w.push_back({dfn[v], dfn[v]});
        if (f[v]) w.push_back({dfn[f[v]], dfn[f[v]]});
        erase_if(w, [&](const auto& x) {return x.first > x.second;});
        //int len = 0;
        //for (auto [l, r] : w) len += r - l + 1;
        //for (auto [l, r] : w)
        //{
        //    for (int j = l;j <= r;j++) cerr << nfd[j] << ' ';cerr << " | ";
        //}
        //cerr << endl;
        //int tl = 0;
        //set<int> s = {uu, vv};
        //while (uu != vv)
        //{
        //    if (dep[uu] < dep[vv]) swap(uu, vv);
        //    s.insert(all(e[uu]));s.insert(f[uu]);uu = f[uu];
        //}
        //s.insert(all(e[uu]));
        //if (f[uu]) s.insert(f[uu]);
        ////dbg(s);
        //assert(len == s.size());
    }
    for (i = 1;i <= q;i++)
    {
        int l, r;
        cin >> l >> r;
        qu[l].push_back({r, i});
    }
    for (i = m;i;i--)
    {

    }
    for (i = 1;i <= q;i++) cout << ans[i] << '\n';
    //cerr << "??\n";
}

\end{lstlisting}

\end{document}
