\documentclass[12pt]{ctexart}
\usepackage{amsfonts,amssymb}
\usepackage{graphicx}
\usepackage{amsmath}
\usepackage{listings}
\usepackage{geometry}
\usepackage[usenames,dvipsnames]{xcolor}
\usepackage{setspace}
\setstretch{1.1} 
\geometry{a4paper}
\geometry{top=2cm} 
\geometry{bottom=1cm} 
\definecolor{mygreen}{rgb}{0,0.6,0}
\definecolor{mygray}{rgb}{0.5,0.5,0.5}
\definecolor{mymauve}{rgb}{0.58,0,0.82}
\lstset{ %
backgroundcolor=\color{white},   % choose the background color
basicstyle=\footnotesize\ttfamily,        % size of fonts used for the code
columns=fullflexible,
breaklines=true,                 % automatic line breaking only at whitespace
captionpos=b,                    % sets the caption-position to bottom
tabsize=4,
commentstyle=\color{mygreen},    % comment style
escapeinside={\%*}{*)},          % if you want to add LaTeX within your code
keywordstyle=\color{blue},       % keyword style
stringstyle=\color{mymauve}\ttfamily,     % string literal style
frame=single,
rulesepcolor=\color{red!20!green!20!blue!20},
% identifierstyle=\color{red},
language=c++,
}
\title{SSerxhs 的 ICPC 板子}

\begin{document}

\maketitle

\centerline{ver:3.3.5}

\tableofcontents

\newpage

\section{数据结构}

\subsection{哈希表}

\begin{lstlisting}
template<int N,typename T,typename TT> struct ht//个数,定义域,值域
{
	const static int p=1e6+7,M=p+2;
	TT a[N];
	T v[N];
	int fir[p+2],nxt[N],st[p+2];//和模数相适应
	int tp,ds;//自定义模数
	ht(){memset(fir,0,sizeof fir);tp=ds=0;}
	void mdf(T x,TT z)//位置,值
	{
		int y=x%p;
		if (y<0) y+=p;
		for (int i=fir[y];i;i=nxt[i]) if (v[i]==x) return a[i]=z,void();//若不可能重复不需要 for
		v[++ds]=x;a[ds]=z;
		if (!fir[y]) st[++tp]=y;
		nxt[ds]=fir[y];fir[y]=ds;
	}
	TT find(T x)
	{
		int y=x%p;
		if (y<0) y+=p;
		int i;
		for (i=fir[y];i;i=nxt[i]) if (v[i]==x) return a[i];
		return 0;//返回值和是否判断依据要求决定
	}
	void clear()
	{
		++tp;
		while (--tp) fir[st[tp]]=0;
		ds=0;
	}
};
\end{lstlisting}

\subsection{珂朵莉树}

\begin{lstlisting}
namespace chtholly_tree
{
	typedef int TT;
	struct Q
	{
		int l;
		mutable int r;
		mutable TT v;
		int len() const { return r-l+1; }
		bool operator<(const Q &x) const { return l<x.l; }
	};
	void add(const Q &a) {}
	void del(const Q &a) {}
	class odt: public set<Q>
	{
	public:
		typedef odt::iterator iter;
		iter split(int x)
		{
			iter it=lower_bound({x});
			if (it!=end()&&it->l==x) return it;
			Q t=*--it,a={t.l,x-1,t.v},b={x,t.r,t.v};
			del(*it); add(a); add(b);
			erase(it); insert(a);
			return insert(b).first;
		}
		void modify(int l,int r,TT v)//[l,r]
		{
			iter lt,rt,it;
			rt=r==rbegin()->r?end():split(r+1); lt=split(l);//[lt,rt)
			while (lt!=begin()&&(it=prev(lt))->v==v) l=(lt=it)->l;
			while (rt!=end()&&rt->v==v) r=(rt++)->r;
			for (it=lt; it!=rt; it++) del(*it);
			add({l,r,v});
			erase(lt,rt); insert({l,r,v});
		}
		TT operator[](const int x) const { return prev(upper_bound({x}))->v; }
	};
}
using chtholly_tree::Q,chtholly_tree::odt;
typedef odt::iterator iter;
\end{lstlisting}

\subsection{带删堆}

\begin{lstlisting}
template<typename T> struct heap//大根堆
{
	priority_queue<T> p,q;
	void push(const T &x)
	{
		if (!q.empty()&&q.top()==x)
		{
			q.pop();
			while (!q.empty()&&q.top()==p.top()) p.pop(),q.pop();
		} else p.push(x);
	}
	void pop()
	{
		p.pop();
		while (!q.empty()&&p.top()==q.top()) p.pop(),q.pop(); 
	}
	void pop(const T &x)
	{
		if (p.top()==x)
		{
			p.pop();
			while (!q.empty()&&p.top()==q.top()) p.pop(),q.pop(); 
		} else q.push(x);
	}
	T top() {return p.top();}
	bool empty() {return p.empty();}
};
\end{lstlisting}



\subsection{可持久化数组}

$O((n+q)\log(n))$,$O((n+q)\log (n))$。

\begin{lstlisting}
struct arr
{
	int c[M][2],rt[O],s[M],b[N];
	int ds,n,ver,v,p,i;
	void build(int &x,int l,int r)
	{
		x=++ds;
		if (l==r) {s[x]=b[l];return;}
		build(c[x][0],l,l+r>>1);
		build(c[x][1],(l+r>>1)+1,r);
	}
	void rebuild(int &x,int pre)
	{
		x=++ds;int l=1,r=n,mid,now=x;
		while (l<r)
		{
			mid=l+r>>1;
			if (mid>=p){c[now][1]=c[pre][1];now=c[now][0]=++ds;r=mid;pre=c[pre][0];} else {c[now][0]=c[pre][0];now=c[now][1]=++ds;l=mid+1;pre=c[pre][1];}
		}
		s[now]=v;
	}
	void init(int *a,int nn)
	{
		n=nn;
		for (i=1;i<=n;i++) b[i]=a[i];
		build(rt[0],1,n);
	}
	int mdf(int pv,int pos,int val)
	{
		p=pos,v=val;
		rebuild(rt[++ver],rt[pv]);
		return ver;
	}
	int ask(int ve,int pos)
	{
		int l=1,r=n,x=rt[ve],mid;
		rt[++ver]=rt[ve];
		while (l<r)
		{
			mid=l+r>>1;
			if (mid>=pos) {x=c[x][0];r=mid;} else {x=c[x][1];l=mid+1;}
		}
		return s[x];
	}
};
\end{lstlisting}

\subsection{左偏树/可并堆}

$O((n+q)\log n)$,$O(n)$。

\begin{lstlisting}
struct left_tree//小根堆,大根堆需要改的地方注释了
{
	int jl[N],v[N],f[N],c[N][2],tf[N],n;//tf只有删非堆顶才用
	bool ed[N];
	void init(const int nn,const int *a)
	{
		jl[0]=-1;n=nn;
		memset(jl+1,0,n<<2);
		memset(tf+1,0,n<<2);//同上
		memset(c+1,0,n<<3);
		memset(ed+1,0,n);
		for (int i=1;i<=n;i++) v[f[i]=i]=a[i];
	}
	int mg(int x,int y)
	{
		if (!(x&&y)) return x|y;
		if (v[x]>v[y]||v[x]==v[y]&&x>y) swap(x,y);//改
		tf[c[x][1]=mg(c[x][1],y)]=x;//同上
		if (jl[c[x][0]]<jl[c[x][1]]) swap(c[x][0],c[x][1]);
		jl[x]=jl[c[x][1]]+1;
		return x;
	}
	int getf(int x)
	{
		if (f[x]==x) return x;
		return f[x]=getf(f[x]);
	}
	int merge(int x,int y)
	{
		if (ed[x]||ed[y]||(x=getf(x))==(y=getf(y))) return x;
		int z=mg(x,y);return f[x]=f[y]=z;
	}
	int getv(int x)//需要自行判断是否存在
	{
		return v[getf(x)];
	}
	int del(int x)//删除堆内最值
	{
		tf[c[x][0]]=tf[c[x][1]]=0;
		f[c[x][0]]=f[c[x][1]]=f[x]=mg(c[x][0],c[x][1]);
		ed[x]=1;c[x][0]=c[x][1]=tf[x]=0;return f[x];
	}
	int del_all(int x)//删除堆内非最值(没验证过)
	{
		int fa=tf[x];
		if (f[c[x][0]]==x) f[c[x][0]]=getf(tf[x]);
		if (f[c[x][1]]==x) f[c[x][1]]=f[tf[x]];
		tf[x]=tf[c[x][0]]=tf[c[x][1]]=0;
		tf[c[fa][c[fa][1]==x]=mg(c[x][0],c[x][1])]=fa;
		c[x][0]=c[x][1]=0;
		while (jl[c[fa][0]]<jl[c[fa][1]])
		{
			swap(c[fa][0],c[fa][1]);
			jl[fa]=jl[c[fa][1]]+1;
			fa=tf[fa];
		}
	}
	void out(int n)
	{
		for (int i=1;i<=n;i++) printf("%d: c%d&%d f%d v%d\n",i,c[i][0],c[i][1],f[i],v[i]);
	}
};
\end{lstlisting}

\subsection{树状数组区间修改区间求和}

$O(n)\sim O(q\log n)$,$O(n)$。

\begin{lstlisting}
struct bit
{
	ll a[N],b[N],s[N];//有初始值
	int n;
	void init(int nn,int *a)//初始值
	{
		n=nn;s[0]=0;
		for (int i=1;i<=n;i++) s[i]=s[i-1]+a[i];
	}
	void mdf(int l,int r,ll dt)
	{
		int i;++r;
		ll j=dt*l;
		a[l]+=dt;b[l]+=j;
		while ((l+=l&-l)<=n)
		{
			a[l]+=dt;
			b[l]+=j;
		}
		if (r<=n)
		{
			j=dt*r;
			a[r]-=dt;b[r]-=j;
			while ((r+=r&-r)<=n)
			{
				a[r]-=dt;
				b[r]-=j;
			}
		}
	}
	ll presum(int x)
	{
		ll r=a[x],rr=b[x];
		int y=x;
		while (x^=x&-x)
		{
			r+=a[x];
			rr+=b[x];
		}
		return r*(y+1)-rr+s[y];
	}
	ll sum(int l,int r)
	{
		return presum(r)-presum(l-1);
	}
};
\end{lstlisting}

\subsection{二维树状数组矩形修改矩形求和}

$O(n^2)\sim O(q\log^2n)$,$O(n^2)$

\begin{lstlisting}
struct bit2
{
	ll a[2050][2050],b[2050][2050],c[2050][2050],d[2050][2050];
	int n,m;
	private:
	void cha(ll a[][2050],int x,int y,int z)
	{
		int i,j;
		for (i=x;i<=n;i+=(i&(-i))) for (j=y;j<=m;j+=(j&(-j))) a[i][j]+=z;
	}
	ll he(int x,int y)
	{
		if ((x<=0)||(y<=0)) return 0;
		int i,j;
		ll z=0,w=0;
		for (i=x;i;i-=(i&(-i))) for (j=y;j;j-=(j&(-j))) z+=a[i][j];
		z*=(x+1)*(y+1);
		w=0;
		for (i=x;i;i-=(i&(-i))) for (j=y;j;j-=(j&(-j))) w+=b[i][j];
		z-=w*(y+1);
		w=0;
		for (i=x;i;i-=(i&(-i))) for (j=y;j;j-=(j&(-j))) w+=c[i][j];
		z-=w*(x+1);
		for (i=x;i;i-=(i&(-i))) for (j=y;j;j-=(j&(-j))) z+=d[i][j];
		return z;
	}
	public:
	void init(int x,int y)
	{
		n=x;m=y;
	}
	void add(int u,int v,int x,int y,int z)//(x1,y1,x2,y2,dt)
	{
		cha(a,u,v,z);
		cha(b,u,v,u*z);//小心乘爆
		cha(c,u,v,v*z);
		cha(d,u,v,u*v*z);
		++x;++y;
		if (x<=n)
		{
			cha(a,x,v,-z);
			cha(b,x,v,-z*x);
			cha(c,x,v,-z*v);
			cha(d,x,v,-z*x*v);
		}
		if (y<=m)
		{
			cha(a,u,y,-z);
			cha(b,u,y,-z*u);
			cha(c,u,y,-z*y);
			cha(d,u,y,-z*u*y);
			if (x<=n)
			{
				cha(a,x,y,z);
				cha(b,x,y,z*x);
				cha(c,x,y,z*y);
				cha(d,x,y,z*x*y);
			}
		}
	}
	ll sum(int u,int v,int x,int y)//(x1,y1,x2,y2)
	{
		--u;--v;
		return (he(x,y)+he(u,v)-he(u,y)-he(x,v));
	}
};
\end{lstlisting}

\subsection{带修莫队(功能:区间数有多少种不同的数字)}

$O(n^{\frac {5}{3}})$,$O(n)$。

\begin{lstlisting}
#include <bits/stdc++.h>
using namespace std;
typedef long long ll;
#define all(x) (x).begin(),(x).end()
const int N=1.4e5,M=1e6+2;
int a[N],ans[N],bel[N],cnt[M],sum,z,y,cur;
struct P
{
	int p,v;
};
struct Q
{
	int l,r,t,p;
	bool operator<(const Q &o) const
	{
		if (bel[l]!=bel[o.l]) return bel[l]<bel[o.l];
		if (bel[r]!=bel[o.r]) return (bel[l]&1)^bel[r]<bel[o.r];
		return (bel[r]&1)?t<o.t:t>o.t;
	}
};
Q b[N];
P d[N];
inline void add(const int &x) {sum+=!(cnt[a[x]]++);}
inline void del(const int &x) {sum-=!(--cnt[a[x]]);}
inline void mdf(const int &x)
{
	auto &[p,v]=d[x];
	if (z<=p&&p<=y) del(p);
	swap(a[p],v);
	if (z<=p&&p<=y) add(p);
}
int main()
{
	ios::sync_with_stdio(0);cin.tie(0);
	int n,m,q1=0,q2=0,i,ksiz;
	cin>>n>>m;
	for (i=1;i<=n;i++) cin>>a[i];
	for (i=1;i<=m;i++)
	{
		char c;
		int l,r;
		cin>>c>>l>>r;
		if (c=='Q') ++q1,b[q1]={l,r,q2,q1};
		else d[++q2]={l,r};
	}
	ksiz=max(1.0,round(cbrt((ll)n*n)));
	for (i=1;i<=n;i++) bel[i]=i/ksiz;
	sort(b+1,b+q1+1);
	z=b[1].l;y=z-1;cur=0;
	for (i=1;i<=q1;i++)
	{
		auto [l,r,t,p]=b[i];
		while (z>l) add(--z);
		while (y<r) add(++y);
		while (z<l) del(z++);
		while (y>r) del(y--);
		while (cur<t) mdf(++cur);
		while (cur>t) mdf(cur--);
		ans[p]=sum;
	}
	for (i=1;i<=q1;i++) cout<<ans[i]<<'\n';
}

\end{lstlisting}

\subsection{二次离线莫队}

$O(n\sqrt n)$,$O(n)$。

珂朵莉给了你一个序列 $a$,每次查询给一个区间 $[l,r]$,查询 $l \leq i< j \leq r$,且 $a_i \oplus a_j$ 的二进制表示下有 $k$ 个 $1$ 的二元组 $(i,j)$ 的个数。$\oplus$ 是指按位异或。

二次离线莫队,通过扫描线,再次将更新答案的过程离线处理,降低时间复杂度。假设更新答案的复杂度为 $O(k)$,它将莫队的复杂度从 $O(nk\sqrt n)$ 降到了 $O(nk + n\sqrt n)$,大大简化了计算。
设 $x$ 对区间 $[l,r]$ 的贡献为 $f(x,[l,r])$,我们考虑区间端点变化对答案的影响:以 $[l..r]$ 变成 $[l..(r+k)]$ 为例,$\forall x \in [r+1,r+k]$ 求 $f(x,[l,x-1])$。
我们可以进行差分:$f(x,[l,x-1])=f(x,[1,x-1])-f(x,[1,l-1])$,这样转化为了一个数对一个前缀的贡献。保存下来所有这样的询问,从左到右扫描数组计算就可以了。  
但是这样做,空间是 $O(n\sqrt n)$ 的,不太优秀,而且时间常数巨大。。
这样的贡献分为两类:

1. 减号左边的贡献永远是一个前缀 和它后面一个数的贡献。这可以预处理出来。
2. 减号右边的贡献对于一次移动中所有的 $x$ 来说,都是不变的。我们打标记的时候,可以只标记左右端点。

这样,减小时间常数的同时,空间降为了 $O(n)$ 级别。是一个很优秀的算法了。处理前缀询问的时候,我们利用异或运算的交换律,即 $a~\mathrm{xor}~b=c \Longleftrightarrow a~\mathrm{xor}~c=b$
开一个桶 $t$,$t[i]$ 表示当前前缀中与 $i$ 异或有 $k$ 个数位为 $1$ 的数有多少个。
则每加入一个数 $a[i]$,对于所有 $\mathrm{popcount}(x)=k$ 的 $x$,$t[a[i]\operatorname{xor}x]\gets t[a[i]\operatorname{xor}x]+1$ 即可。

\begin{lstlisting}
typedef long long ll;
const int N=1e5+2,M=1<<14;
ll f[N],ans[N],ta[N];
int a[N],cnt[M],bel[N],pc[M],st[N];
int n,m,ksiz;
struct Q
{
	int z,y,wz;
	bool operator<(const Q& x) const {return (bel[z]<bel[x.z])||(bel[z]==bel[x.z])&&((y<x.y)&&(bel[z]&1)||(y>x.y)&&(1^bel[z]&1));}
};
Q mq(const int x,const int y,const int z)
{
	Q a;
	a.z=x;a.y=y;a.wz=z;
	return a; 
}
Q q[N];
vector<Q> b[N];
void read(int &x)
{
	int c=getchar();
	while ((c<48)||(c>57)) c=getchar();
	x=c^48;c=getchar();
	while ((c>=48)&&(c<=57))
	{
		x=x*10+(c^48);
		c=getchar();
	}
}
int main()
{
	int i,j,k,l=1,r=0,tp=0,x,na;
	read(n);read(m);read(k);ksiz=sqrt(n);
	for (i=1;i<=n;i++) {read(a[i]);bel[i]=(i-1)/ksiz+1;}
	if (k==0) st[++tp]=0;
	for (i=1;i<16384;i++)
	{
		if (i&1) pc[i]=pc[i>>1]+1; else pc[i]=pc[i>>1];
		if (pc[i]==k) st[++tp]=i;
	}
	for (i=1;i<=n;i++)
	{
		j=tp+1;f[i]=f[i-1];
		while (--j) f[i]+=cnt[st[j]^a[i]];
		++cnt[a[i]];
	}
	for (i=1;i<=m;i++) {read(q[i].z);read(q[q[i].wz=i].y);}
	sort(q+1,q+m+1);
	for (i=1;i<=m;i++)
	{
		ans[i]=f[q[i].y]-f[r]+f[q[i].z-1]-f[l-1];
		if (k==0) ans[i]+=q[i].z-l;
		if (r<q[i].y)
		{
			b[l-1].push_back(mq(r+1,q[i].y,-i));
			r=q[i].y;
		}
		if (l>q[i].z)
		{
			b[r].push_back(mq(q[i].z,l-1,i));
			l=q[i].z;
		}
		if (r>q[i].y)
		{
			b[l-1].push_back(mq(q[i].y+1,r,i));
			r=q[i].y;
		}
		if (l<q[i].z)
		{
			b[r].push_back(mq(l,q[i].z-1,-i));
			l=q[i].z;
		}
	}
	memset(cnt,0,sizeof(cnt));
	for (i=1;i<=n;i++)
	{
		j=tp+1;x=a[i];
		while (--j) ++cnt[x^st[j]];
		for (j=0;j<b[i].size();j++)
		{
			na=0;l=b[i][j].z;r=b[i][j].y;
			for (k=l;k<=r;k++) na+=cnt[a[k]];
			if (b[i][j].wz>0) ans[b[i][j].wz]+=na; else ans[-b[i][j].wz]-=na;
		}
	}
	for (i=2;i<=m;i++) ans[i]+=ans[i-1];
	for (i=1;i<=m;i++) ta[q[i].wz]=ans[i];
	for (i=1;i<=m;i++) printf("%lld\n",ta[i]);
}
\end{lstlisting}

\subsection{回滚莫队}

$O(n\sqrt n)$,$O(n)$。

\begin{lstlisting}
#include <bits/stdc++.h>
using namespace std;
const int N=2e5+2;
int a[N],z[N],y[N],wz[N],b[N],d[N],bel[N],ans[N],st[N][2],pos[N][2];
int n,m,i,j,x,c,ksiz,gs,l=1,r,tp,na,ca;
void read(int &x)
{
	c=getchar();
	while ((c<48)||(c>57)) c=getchar();
	x=c^48;c=getchar();
	while ((c>=48)&&(c<=57))
	{
		x=x*10+(c^48);
		c=getchar();
	}
}
void qs(int l,int r)
{
	int i=l,j=r,m=bel[z[l+r>>1]],mm=y[l+r>>1];
	while (i<=j)
	{
		while ((bel[z[i]]<m)||(bel[z[i]]==m)&&(y[i]<mm)) ++i;
		while ((bel[z[j]]>m)||(bel[z[j]]==m)&&(y[j]>mm)) --j;
		if (i<=j)
		{
			swap(wz[i],wz[j]);
			swap(z[i],z[j]);
			swap(y[i++],y[j--]);
		}
	}
	if (i<r) qs(i,r);
	if (l<j) qs(l,j);
}
int main()
{
	read(n);ksiz=sqrt(n);
	for (i=1;i<=n;i++) {read(a[i]);b[i]=a[i];bel[i]=(i-1)/ksiz+1;}
	sort(b+1,b+n+1);
	d[gs=1]=b[1];
	for (i=2;i<=n;i++) if (b[i]!=b[i-1]) d[++gs]=b[i];
	for (i=1;i<=n;i++) a[i]=lower_bound(d+1,d+gs+1,a[i])-d;
	read(m);assert(int(n/sqrt(m)));
	for (i=1;i<=m;i++) {read(z[i]);read(y[wz[i]=i]);}
	qs(1,m);
	for (i=1;i<=m;i++)
	{
		if (bel[z[i]]>bel[z[i-1]])
		{
			while (l<=r) {pos[a[l]][0]=pos[a[l]][1]=0;++l;}na=0;
			if (bel[z[i]]==bel[y[i]])
			{
				for (j=z[i];j<=y[i];j++) if (pos[a[j]][0]) na=max(na,j-pos[a[j]][0]); else pos[a[j]][0]=j;
				ans[wz[i]]=na;for (j=z[i];j<=y[i];j++) pos[a[j]][0]=0;na=0;l=ksiz*bel[z[i]];r=l-1;
				continue;
			}
			l=ksiz*bel[z[i]];r=l-1;na=0;
		}
		if (bel[z[i]]==bel[y[i]])
		{
			while (l<=r) {pos[a[l]][0]=pos[a[l]][1]=0;++l;}na=0;
			for (j=z[i];j<=y[i];j++) if (pos[a[j]][0]) na=max(na,j-pos[a[j]][0]); else pos[a[j]][0]=j;
			ans[wz[i]]=na;for (j=z[i];j<=y[i];j++) pos[a[j]][0]=0;
			l=ksiz*bel[z[i]];r=l-1;na=0;
			continue;
		}
		while (r<y[i])
		{
			x=a[++r];pos[x][1]=r;
			if (!pos[x][0]) pos[x][0]=r; else na=max(na,r-pos[x][0]);
		}c=na;
		while (l>z[i])
		{
			x=a[--l];st[++tp][0]=x;st[tp][1]=pos[x][0];
			pos[x][0]=l;
			if (!pos[x][1])
			{
				st[++tp][0]=x+n;st[tp][1]=0;
				pos[x][1]=l;
			} else na=max(na,pos[x][1]-l);
		}
		ans[wz[i]]=na;na=c;++tp;l=ksiz*bel[z[i]];
		while (--tp) if (st[tp][0]<=n) pos[st[tp][0]][0]=st[tp][1]; else pos[st[tp][0]-n][1]=st[tp][1];
	}
	for (i=1;i<=m;i++) printf("%d\n",ans[i]);
}
\end{lstlisting}

\subsection{李超树}

题意:插入线段,查询某个 x 的最大 y(输出最小编号)

算法核心:seg 每个点维护在中点取值最大的线段,显然只会向一边递归

\begin{lstlisting}
struct Q
{
	int x0,y0,dx,dy,id;
	Q():x0(0),y0(-1),dx(1),dy(0),id(-1){}//y>=0
	Q(int a,int b,int c,int d,int e):x0(a),y0(b),dx(c),dy(d),id(e){}
	bool contains(const int &x) const {return x0<=x&&x<=x0+dx;}
};
bool cmp(const Q &a,const Q &b,int x)//小心数值爆炸
{
	ll A=((ll)a.y0*a.dx+(ll)(x-a.x0)*a.dy)*b.dx,B=((ll)b.y0*b.dx+(ll)(x-b.x0)*b.dy)*a.dx;
	if (A!=B) return A<B;
	return a.id>b.id;
}
bool cmp2(const Q &a,const Q &b)
{
	if (a.y0+a.dy!=b.y0+b.dy) return a.y0+a.dy<b.y0+b.dy;
	return a.id>b.id;
}
const int inf=1e9;
int ans;
namespace seg
{
	const int N=4e4+2,M=N*4;
	Q s[M],X[N];
	int n,z,y;
	void init(int nn) {n=nn;for (int i=1;i<=n*4;i++) s[i]=Q();}
	void insert(int x,int l,int r,Q dt)
	{
		int c=x*2,m=l+r>>1;
		if (z<=l&&r<=y)
		{
			if (cmp(s[x],dt,m)) swap(s[x],dt);
			if (l==r) return;
			if (cmp(s[x],dt,l)) insert(c,l,m,dt);
			else if (cmp(s[x],dt,r)) insert(c+1,m+1,r,dt);
			return;
		}
		if (z<=m) insert(c,l,m,dt);
		if (y>m) insert(c+1,m+1,r,dt);
	}
	void insert(const Q &o)
	{
		z=o.x0;y=z+o.dx;
		assert(1<=z&&z<=y&&y<=n);
		if (z==y)
		{
			if (cmp2(X[z],o)) X[z]=o;
			return;
		}
		insert(1,1,n,o);
	}
	Q askmax(int p)
	{
		Q ans=s[1].contains(p)?s[1]:Q();
		int x=1,l=1,r=n,c,m;
		while (l<r)
		{
			c=x*2,m=l+r>>1;
			if (p<=m) x=c,r=m; else x=c+1,l=m+1;
			if (s[x].contains(p)&&cmp(ans,s[x],p)) ans=s[x];
		}
		Q o(X[p].x0,X[p].y0+X[p].dy,1,0,0);
		return cmp(ans,o,p)?X[p]:ans;
	}
}
int main()
{
	ios::sync_with_stdio(0);cin.tie(0);
	cout<<setiosflags(ios::fixed)<<setprecision(15);
	int n=4e4,m,i;
	seg::init(n);
	cin>>m;
	while (m--)
	{
		int op;
		cin>>op;
		if (op)
		{
			int x[2],y[2];
			cin>>x[0]>>y[0]>>x[1]>>y[1];
			for (int &v:x) v=(v+ans-1)%39989+1;
			for (int &v:y) v=(v+ans-1)%inf+1;
			if (x[0]>x[1]||x[0]==x[1]&&y[0]>y[1]) swap(x[0],x[1]),swap(y[0],y[1]);
			static int id;
			seg::insert({x[0],y[0],x[1]-x[0],y[1]-y[0],++id});
		}
		else
		{
			int x;
			cin>>x;
			x=(x+ans-1)%39989+1;
			cout<<(ans=max(0,seg::askmax(x).id))<<'\n';
		}
	}
}
\end{lstlisting}

\subsection{李超树(动态开点)}

\begin{lstlisting}
struct Q
{
	int k;
	ll b;
	ll y(const int &x) const {return (ll)k*x+b;}
};
const int inf=1e9;
const ll INF=1e18;
struct seg//可以析构,不能并行
{
	const static int N=4e5+2,M=N*8*8+(1<<23);
	const static ll npos=9e18;
	static Q s[M];
	static int c[M][2],id;
	int z,y,L,R;
	seg(int l,int r)
	{
		L=l;R=r;id=1;
		s[1]={0,npos};
		assert(L<=R&&(ll)R-L<1ll<<32);
	}
private:
	void insert(int &x,int l,int r,Q o)
	{
		if (!x)
		{
			x=++id;
			assert(id<M);
			s[x]={0,npos};
		}
		int m=l+(r-l>>1);
		if (z<=l&&r<=y)
		{
			if (s[x].y(m)>o.y(m)) swap(s[x],o);
			if (s[x].y(l)>o.y(l)) insert(c[x][0],l,m,o);
			else if (s[x].y(r)>o.y(r)) insert(c[x][1],m+1,r,o);
			return;
		}
		if (z<=m) insert(c[x][0],l,m,o);
		if (y>m) insert(c[x][1],m+1,r,o);
	}
public:
	void insert(const Q &x,const int &l,const int &r)//[l,r]
	{
		z=l;y=r;int tmp=1;
		insert(tmp,L,R,x);
		assert(tmp==1);
	}
	ll askmin(const int &p)
	{
		ll res=s[1].y(p);
		int l=L,r=R,m,x=1;
		while (l<r)
		{
			m=l+(r-l>>1);
			if (p<=m) x=c[x][0],r=m; else x=c[x][1],l=m+1;
			if (!x) return res;
			res=min(res,s[x].y(p));
		}
		return res;
	}
	~seg()
	{
		++id;
		while (--id) c[id][0]=c[id][1]=0;
	}
};
Q seg::s[seg::M];
int seg::c[seg::M][2],seg::id;
\end{lstlisting}

\subsection{splay}

$O(n)$,$O((n+q)\log n)$。

\begin{lstlisting}
#include <bits/stdc++.h>
using namespace std;
typedef long long ll;
typedef unsigned int ui;
const int N=1e6+20,p=998244353;
void inc(int &x,const int y){if ((x+=y)>=p) x-=p;}
void dec(int &x,const int y){if ((x-=y)<0) x+=p;}
void mul(int &x,const int y){x=(ll)x*y%p;}
template<int N> struct _splay
{
	int c[N][2],plz[N],clz[N],st[N],siz[N],s[N],v[N],f[N];
	bool fg[N],flz[N];
	int tp,rt;
	void allout(int x)
	{
		if (!x) return;
		pushdown(x);
		allout(c[x][0]);
		if (x>2) printf("%d ",v[x]);
		allout(c[x][1]);
	}
	void out(int x)
	{
		printf("%d: c %d & %d f %d s %d v %d siz %d\n",x,c[x][0],c[x][1],f[x],s[x],v[x],siz[x]);
		if (c[x][0]) out(c[x][0]);
		if (c[x][1]) out(c[x][1]);
		if (x==rt) puts("-------------------");
	}
	void iinit()
	{
		for (int i=1;i<N;i++) st[N-i]=i;
		tp=N-1;
	}
	void init()
	{
		tp=N-3;
		c[1][0]=c[1][1]=flz[1]=plz[1]=fg[1]=v[1]=f[1]=s[1]=0;clz[1]=1;
		c[2][0]=c[2][1]=flz[2]=plz[2]=fg[2]=v[2]=f[2]=s[2]=0;clz[2]=1;
		c[1][1]=2;f[2]=1;rt=1;siz[2]=1;siz[1]=2;
	}
	void pushup(int x)
	{
		s[x]=((ui)s[c[x][0]]+s[c[x][1]]+v[x])%p;
		siz[x]=siz[c[x][0]]+siz[c[x][1]]+1;
	}
	void pushdown(int x)
	{
		int lc=c[x][0],rc=c[x][1];
		if (flz[x])
		{
			if (lc) flz[lc]^=1,swap(c[lc][0],c[lc][1]);
			if (rc) flz[rc]^=1,swap(c[rc][0],c[rc][1]);
			flz[x]=0;
		}
		if (fg[x])
		{
			clz[x]=1;plz[x]=0;
			if (lc) fg[lc]=1,v[lc]=v[x],s[lc]=(ll)v[x]*siz[lc]%p;
			if (rc) fg[rc]=1,v[rc]=v[x],s[rc]=(ll)v[x]*siz[rc]%p;
			fg[x]=0;
		}
		else
		{
			if (clz[x]!=1)
			{
				if (lc) mul(clz[lc],clz[x]),mul(s[lc],clz[x]),mul(plz[lc],clz[x]),mul(v[lc],clz[x]);
				if (rc) mul(clz[rc],clz[x]),mul(s[rc],clz[x]),mul(plz[rc],clz[x]),mul(v[rc],clz[x]);
				clz[x]=1;
			}
			if (plz[x])
			{
				if (lc) inc(plz[lc],plz[x]),inc(v[lc],plz[x]),s[lc]=(s[lc]+(ll)siz[lc]*plz[x])%p;
				if (rc) inc(plz[rc],plz[x]),inc(v[rc],plz[x]),s[rc]=(s[rc]+(ll)siz[rc]*plz[x])%p;
				plz[x]=0;
			}
		}
	}
	void zigzag(int x)
	{
		int y=f[x],z=f[y],typ=(c[y][0]==x);
		if (z) c[z][c[z][1]==y]=x;
		f[x]=z;f[y]=x;c[y][typ^1]=c[x][typ];
		if (c[x][typ]) f[c[x][typ]]=y;
		c[x][typ]=y;
		pushup(y);
	}
	void allpd(int x)
	{
		static int st[N],tp;
		st[tp=1]=x;
		while (x=f[x]) st[++tp]=x;
		while (tp) pushdown(st[tp--]);
	}
	void splay(int x,int tar)
	{
		if (!tar) rt=x;
		int y;
		while ((y=f[x])!=tar)
		{
			if (f[y]!=tar) zigzag(c[f[y]][0]==y^c[y][0]==x?x:y);
			zigzag(x);
		}
		pushup(x);
	}
	void find(int kth,int tar)
	{
		int x=rt;
		while (siz[c[x][0]]+1!=kth)
		{
			pushdown(x);
			if (siz[c[x][0]]>=kth) x=c[x][0]; else
			{
				kth-=siz[c[x][0]]+1;
				x=c[x][1];
			}
		}
		pushdown(x);
		splay(x,tar);
	}
	int rk(int x)
	{
		allpd(x);
		splay(x,0);
		return siz[c[x][0]];
	}
	void split(int x,int y)
	{
		find(x,0);find(y+2,rt);
	}
	int npt()
	{
		int x=st[tp--];
		c[x][0]=c[x][1]=plz[x]=siz[x]=s[x]=v[x]=fg[x]=flz[x]=0;
		clz[x]=1;
		return x;
	}
	int build(int *a,int l,int r)
	{
		if (l>r) return 0;
		int m=l+r>>1,x;
		v[x=npt()]=a[m];
		//printf("build %d %d %d\n",l,r,x);
		if (l==r)
		{
			siz[x]=1;
			s[x]=v[x];
			return x;
		}
		c[x][0]=build(a,l,m-1);
		c[x][1]=build(a,m+1,r);
		if (c[x][0]) f[c[x][0]]=x;
		if (c[x][1]) f[c[x][1]]=x;
		pushup(x);
		return x;
	}
	void ins(int pos,int *a,int n)//在pos后插入
	{
		if (!n) return;
		split(pos+1,pos);
		// out(rt);
		int x=c[rt][1];
		c[x][0]=build(a,1,n);
		// printf("%d %d\n",x,c[x][0]);
		f[c[x][0]]=x;
		pushup(x);pushup(rt);
	}
	void del(int l,int r)//删除[l,r]
	{
		split(l,r);
		c[c[rt][1]][0]=0;
		pushup(c[rt][1]);
		pushup(rt);
	}
	void rev(int l,int r)
	{
		split(l,r);
		int x=c[c[rt][1]][0];
		swap(c[x][0],c[x][1]);
		flz[x]^=1;
	}
	void add(int l,int r,int val)
	{
		split(l,r);
		int x=c[c[rt][1]][0];
		inc(v[x],val);inc(plz[x],val);
		s[x]=(s[x]+(ll)val*siz[x])%p;
		pushup(f[x]);pushup(rt);
	}
	void multi(int l,int r,int val)
	{
		split(l,r);
		int x=c[c[rt][1]][0];
		mul(v[x],val);mul(plz[x],val);
		mul(s[x],val);mul(clz[x],val);
		pushup(f[x]);pushup(rt);
	}
	void mov(int l1,int r1,int l2)//都是原下标
	{
		if (l2>l1) l2-=r1-l1+1;
		split(l1,r1);int x=c[c[rt][1]][0];
		allpd(x);c[f[x]][0]=0;
		pushup(f[x]);pushup(rt);
		split(l2+1,l2);
		allpd(c[rt][1]);
		c[c[rt][1]][0]=x;f[x]=c[rt][1];
		pushup(f[x]);pushup(rt);
	}
	int sum(int l,int r)
	{
		split(l,r);//puts("spe ");out(rt);
		return s[c[c[rt][1]][0]];
	}
};
_splay<N> s;
int a[N];
int n,q,i,x,y,z;
void read(int &x)
{
	int c=getchar();
	while (c<48||c>57) c=getchar();
	x=c^48;c=getchar();
	while (c>=48&&c<=57) x=x*10+(c^48),c=getchar();
}
int main()
{
	read(n);read(q);s.iinit();
	for (i=1;i<=n;i++) a[i]=i;
	s.init();s.ins(0,a,n);//s.out(s.rt);
	while (q--)
	{
		read(x);read(y);s.rev(x,y);
	}
	s.allout(s.rt);
}
\end{lstlisting}

\subsection{区间线性基}

$O((n+q)\log a)$,$O(n\log a)$。

\begin{lstlisting}
int v[N][M][2];//N是序列总长,M是位数
void ins(int x,int p)//在p插入x
{
	memcpy(v[p],v[p-1],sizeof(v[p]));int n=p,y;
	for (int i=M-1;x;i--) if (x&1<<i)
	{
		if (v[n][i][0])
		{
			if (p>v[n][i][1])
			{
				y=v[n][i][0];v[n][i][0]=x;
				x^=y;swap(p,v[n][i][1]);
			} else x^=v[n][i][0];
		} else v[n][i][0]=x,v[n][i][1]=p,x=0;
	}
}

ans=0;//[z,y]
for (i=M-1;~i;i--) if ((1^ans>>i&1)&&(v[y][i][1]>=z)) ans^=v[y][i][0];
\end{lstlisting}

\subsection{splay 重构}

$O(n)$,$O((n+q)\log n)$。

\begin{lstlisting}

const ui p=998244353;
template<typename inf,typename tag> struct splay
{
	#define _rev
	struct node
	{
		node *c[2],*f;
		int siz;
		inf s,v;
		tag t;
		node():c{},f(0),siz(1),s(),v(),t(){}
		node(inf x):c{},f(0),siz(1),s(x),v(x),t(){}
		void operator+=(const tag &o)
		{
			s+=o;v+=o;t+=o;
		#ifdef _rev
			if (o.rev) swap(c[0],c[1]);
		#endif
		}
		void pushup()
		{
			if (c[0]) s=c[0]->s+v,siz=c[0]->siz+1; else s=v,siz=1;
			if (c[1]) s=s+c[1]->s,siz+=c[1]->siz;
		}
		void pushdown()
		{
			for (auto x:c) if (x) *x+=t;
			t={};
		}
		void zigzag()
		{
			node *y=f,*z=y->f;
			int typ=y->c[0]==this;
			if (z) z->c[z->c[1]==y]=this;
			f=z; y->f=this;
			y->c[typ^1]=c[typ];
			if (c[typ]) c[typ]->f=y;
			c[typ]=y;
			y->pushup();
		}
		void splay(node *tar)//不要在 makeroot 以外调用
		{
			for (node *y=f;y!=tar;zigzag(),y=f) if (node *z=y->f;z!=tar) (z->c[1]==y^y->c[1]==this?this:y)->zigzag();
			pushup();
		}
		void clear()
		{
			for (node *x:c) if (x) x->clear();
			delete this;
		}
	};
	node *rt;
	void debug()
	{
		map<node*,int> id;
		id[0]=0;id[rt]=1;
		int cnt=1;
		function<void(node*)> out=[&](node *x)
		{
			if (!x) return;
			for (auto y:x->c) if (!id.count(y)) id[y]=++cnt;
			cerr<<id[x]<<' '<<id[x->c[0]]<<' '<<id[x->c[1]]<<' '<<id[x->f]<<' '<<x->siz<<'\n';
			for (auto y:x->c) out(y);
		};
		out(rt);
	}
	node *build(inf *a,int n)
	{
		if (n==0) return 0;
		int m=n-1>>1;
		node *x=new node(a[m]);
		x->c[0]=build(a,m);
		x->c[1]=build(a+m+1,n-1-m);
		for (node *y:x->c) if (y) y->f=x;
		x->pushup();
		return x;
	}
	splay()
	{
		rt=new node;
		rt->c[1]=new node;
		rt->c[1]->f=rt;
		rt->siz=2;
	}
	splay(inf *a,int n)//[1,n]
	{
		rt=new node;
		rt->c[1]=new node;
		rt->c[1]->f=rt;
		rt->c[1]->c[0]=build(a+1,n);
		rt->c[1]->c[0]->f=rt->c[1];
		rt->c[1]->pushup();
		rt->pushup();
	}
	void makeroot(node *u,node *tar)
	{
		if (!tar) rt=u;
		u->splay();
	}
	void findnth(int k,node *tar)
	{
		node *x=rt;
		while (1)
		{
			x->pushdown();
			int v=x->c[0]?x->c[0]->siz:0;
			if (v+1==k) {x->splay(tar);if (!tar) rt=x;return;}
			if (v>=k) x=x->c[0]; else x=x->c[1],k-=v+1;
		}
	}
	void split(int l,int r)
	{
		assert(1<=l&&r<=rt->siz-2&&l-1<=r);
		findnth(l,0);
		findnth(r+2,rt);
	}
	void insert(int pos,inf x)//insert before pos
	{
		assert(1<=pos&&pos<=rt->siz-1);
		split(pos,pos-1);
		rt->c[1]->c[0]=new node(x);
		rt->c[1]->c[0]->f=rt->c[1];
		rt->c[1]->pushup();
		rt->pushup();
	}
	void insert(int pos,inf *a,int n)//insert before pos, [1,n]
	{
		assert(1<=pos&&pos<=rt->siz-1);
		split(pos,pos-1);
		rt->c[1]->c[0]=build(a,n);
		rt->c[1]->c[0]->f=rt->c[1];
		rt->c[1]->pushup();
		rt->pushup();
	}
	void erase(int pos)
	{
		assert(1<=pos&&pos<=rt->siz-2);
		split(pos,pos);
		delete rt->c[1]->c[0];
		rt->c[1]->c[0]=0;
		rt->c[1]->pushup();
		rt->pushup();
	}
	void erase(int l,int r)
	{
		assert(1<=l&&l<=r&&r<=rt->siz-2);
		split(l,r);
		rt->c[1]->c[0]->clear();
		rt->c[1]->c[0]=0;
		rt->c[1]->pushup();
		rt->pushup();
	}
	void modify(int pos,inf x)//not checked
	{
		assert(1<=pos&&pos<=rt->siz-2);
		findnth(pos+1,0);
		rt->v=x;rt->pushup();
	}
	void modify(int l,int r,tag w)
	{
		split(l,r);
		node *x=rt->c[1]->c[0];
		*x+=w;
		rt->c[1]->pushup();
		rt->pushup();
	}
	inf ask(int l,int r)
	{
		split(l,r);
		return rt->c[1]->c[0]->s;
	}
	~splay(){rt->clear();}
	#undef _rev
};
struct Q
{
	ll pl,ti;
	bool rev;
	Q():pl(0),ti(1),rev(0){}
	Q(ll a,ll b,bool c):pl(a),ti(b),rev(c){}
	void operator+=(const Q &o)
	{
		pl=(pl*o.ti+o.pl)%p;
		ti=ti*o.ti%p;
		rev^=o.rev;
	}
};
struct P
{
	ll s,len;
	void operator+=(const Q &o)
	{
		s=(s*o.ti+o.pl*len)%p;
	}
	P operator+(const P &o) {return {s+o.s>=p?s+o.s-p:s+o.s,len+o.len};}
};

\end{lstlisting}

\subsection{第 $k$ 大线性基}

$O((n+q)\log a)$,$O(\log a)$。

\begin{lstlisting}
void ins(ll x)
{
	if (x==0) return con=1,void();//con=1:有0
	int i;
	for (i=50;x;i--) if (x>>i&1)
	{
		if (!ji[i]) {ji[i]=x;i=-1;break;}x^=ji[i];
	}
	if (!x) con=1;
}
ll kmax(ll x)//若有初始值改 r 即可
{
	ll r=0;
	int m=0,i;
	for (i=50;~i;i--) if (ji[i]) a[++m]=i;
	if (1ll<<m<=x-con) return -1;//个数少于k
	x=(1ll<<m)-x;
	for (i=1;i<=m;i++) if ((x>>m-i^r>>a[i])&1) r^=ji[a[i]];
	return r;
}
ll kmin(ll x)//若有初始值改 r 即可
{
	ll r=0;
	int m=0,i;
	for (i=50;~i;i--) if (ji[i]) a[++m]=i;
	x-=con;
	if (1ll<<m<=x) return -1;//个数少于k
	for (i=1;i<=m;i++) if ((x>>m-i^r>>a[i])&1) r^=ji[a[i]];
	return r;
}

 \end{lstlisting}

\subsection{fhq-treap}

$O((n+q)\log n)$,$O(n)$。

\begin{lstlisting}
const int N=1.1e6+2;
int c[N][2],v[N],w[N],s[N];
int n,i,x,y,ds,val,kth,p,q,z,rt,la,m,ans;
void pushup(const int x)
{
	s[x]=s[c[x][0]]+s[c[x][1]]+1;
}
void split_val(int now,int &x,int &y)//调用外部val,相等归入y
{
	if (!now) return x=y=0,void();
	if (val<=v[now]) split_val(c[y=now][0],x,c[now][0]);
	else split_val(c[x=now][1],c[now][1],y);
	pushup(now);
}
void split_kth(int now,int &x,int &y)//调用外部kth,左子树大小为 kth
{
	if (!now) return x=y=0,void();
	if (kth<=s[c[now][0]]) split_kth(c[y=now][0],x,c[now][0]);
	else kth-=s[c[now][0]]+1,split_kth(c[x=now][1],c[now][1],y);
	pushup(now);
}
int merge(int x,int y)//小根ver.
{
	if (!(x&&y)) return x|y;
	if (w[x]<w[y]) {c[x][1]=merge(c[x][1],y);pushup(x);return x;}
	else {c[y][0]=merge(x,c[y][0]);pushup(y);return y;}
}
int main()
{
	read(n);read(m);srand(998244353);
	for (i=1;i<=n;i++)
	{
		read(x);val=v[++ds]=x;w[ds]=rand();s[ds]=1;split_val(rt,p,q);rt=merge(merge(p,ds),q);
	}
	while (m--)
	{
		read(y);read(x);x^=la;
		if (y==4)
		{
			kth=x;split_kth(rt,p,q);x=p;
			while (c[x][1]) x=c[x][1];
			ans^=(la=v[x]);rt=merge(p,q);
			continue;
		}
		val=x;
		if (y==1)
		{
			v[++ds]=x;w[ds]=rand();s[ds]=1;
			split_val(rt,p,q);rt=merge(merge(p,ds),q);
			continue;
		}
		if (y==2)
		{
			split_val(rt,p,q);kth=1;split_kth(q,i,z);
			rt=merge(p,z);continue;
		}
		if (y==3)
		{
			split_val(rt,p,q);ans^=(la=s[p]+1);
			rt=merge(p,q);continue;
		}
		if (y==5)
		{
			split_val(rt,p,q);x=p;
			while (c[x][1]) x=c[x][1];ans^=(la=v[x]);
			rt=merge(p,q);continue;
		}
		++val;split_val(rt,p,q);x=q;
		while (c[x][0]) x=c[x][0];
		ans^=(la=v[x]);rt=merge(p,q);
	}printf("%d",ans);
}
\end{lstlisting}

\subsection{笛卡尔树}

$O(n)$,$O(n)$。

\begin{lstlisting}
int c[N][2],p[N],st[N];
int main()
{
	int i,n,tp=0;
	ll la=0,ra=0;
	read(n);
	for (i=1;i<=n;i++)
	{
		read(p[i]);st[tp+1]=0;
		while ((tp)&&(p[st[tp]]>p[i])) --tp;
		c[c[st[tp]][1]=i][0]=st[tp+1];st[++tp]=i;
	}
	for (i=1;i<=n;i++) la^=(ll)i*(c[i][0]+1);
	for (i=1;i<=n;i++) ra^=(ll)i*(c[i][1]+1);
	printf("%lld %lld",la,ra);
}
\end{lstlisting}

\subsection{扫描线}

$O((n+q)\log n)$,$O(n+q)$。

\begin{lstlisting}
const int N=2e5+2,M=8e5+2;//2倍N
struct Q
{
	int l,r,h,typ;
	Q(int a=0,int b=0,int c=0,int d=0):l(a),r(b),h(c),typ(d){}
	bool operator<(const Q &o) const {return h<o.h;}
};
ll ans;
Q q[N];
int l[M],r[M],s[M][2],lz[M];
int xx[N>>1][2],yy[N>>1][2],a[N];
int n,i,j,x,y,z,dt,m,len;
void pushup(int x)
{
	int c=x<<1;
	if (s[c][0]==s[c|1][0]) s[x][0]=s[c][0],s[x][1]=s[c][1]+s[c|1][1];
	else
	{
		if (s[c][0]>s[c|1][0]) c|=1; 
		s[x][0]=s[c][0];s[x][1]=s[c][1];
	}
}
void pushdown(int x)
{
	if (lz[x])
	{
		int c=x<<1;
		lz[c]+=lz[x];s[c][0]+=lz[x];c|=1;
		lz[c]+=lz[x];s[c][0]+=lz[x];lz[x]=0;
	}
}
void build(int x)
{
	if (l[x]==r[x]) return s[x][1]=a[l[x]+1]-a[l[x]],void();
	int c=x<<1;
	l[c]=l[x];r[c]=l[x]+r[x]>>1;
	l[c|1]=r[c]+1;r[c|1]=r[x];
	build(c);build(c|1);
	pushup(x);
}
void mdf(int x)
{
	if (z<=l[x]&&r[x]<=y)
	{
		lz[x]+=dt;s[x][0]+=dt;return;
	}
	pushdown(x);
	int c=x<<1;
	if (z<=r[c]) mdf(c);
	if (y>r[c]) mdf(c|1);
	pushup(x);
}
int main()
{
	read(n);
	for (i=1;i<=n;i++)
	{
		read(xx[i][0]);read(yy[i][0]);
		read(xx[i][1]);read(yy[i][1]);
		a[++m]=xx[i][0],a[++m]=xx[i][1];
	}
sort(a+1,a+m+1);r[l[1]=1]=(m=unique(a+1,a+m+1)-a-1)-1;build(1);
	for (i=1;i<=n;i++)
	{
		xx[i][0]=lower_bound(a+1,a+m+1,xx[i][0])-a;
		xx[i][1]=lower_bound(a+1,a+m+1,xx[i][1])-a;
		q[i]=Q(xx[i][0],xx[i][1],yy[i][0],1);
		q[i+n]=Q(xx[i][0],xx[i][1],yy[i][1],-1);
	}m=n<<1;len=a[m]-a[1];
	sort(q+1,q+m+1);
	for (i=1;i<=m;i++)
	{
		z=q[i].l;y=q[i].r-1;dt=q[i].typ;mdf(1);
		ans+=(ll)(s[1][0]?len:len-s[1][1])*(q[i+1].h-q[i].h);
	}
	printf("%lld\n",ans);
}
\end{lstlisting}

\subsection{Segmenttree Beats!}

$O((n+q)\log n)\sim O(n+q\log^2 n)$,$O(n)$。

\begin{itemize}

\item $\texttt{1 l r k}$:对于所有的 $i\in[l,r]$,将 $A_i$ 加上 $k$($k$ 可以为负数)。
\item $\texttt{2 l r v}$:对于所有的 $i\in[l,r]$,将 $A_i$ 变成 $\min(A_i,v)$。
\item $\texttt{3 l r}$:求 $\sum_{i=l}^{r}A_i$。
\item $\texttt{4 l r}$:对于所有的 $i\in[l,r]$,求 $A_i$ 的最大值。
\item $\texttt{5 l r}$:对于所有的 $i\in[l,r]$,求 $B_i$ 的最大值。
\end{itemize}

\begin{lstlisting}
typedef long long ll;d
struct Q
{
	ll mxp,mx,vp,v;
	Q(ll a=0,ll b=0,ll c=0,ll d=0):mxp(a),mx(b),vp(c),v(d){}
};
const int N=5e5+2,M=2e6+2;
const ll inf=-1e18;
Q lz[M];
ll mx[M],cnt[M],se[M],pmx[M],s[M],ans;
int l[M],r[M],cd[M],a[N];
int n,m,i,x,y,z,typ,dt;
void pushup(int x)
{
	int lc=x<<1,rc=lc|1;
	s[x]=s[lc]+s[rc];
	pmx[x]=max(pmx[lc],pmx[rc]);
	if (mx[lc]==mx[rc])
	{
		mx[x]=mx[lc];cnt[x]=cnt[lc]+cnt[rc];
		se[x]=max(se[lc],se[rc]);
	}
	else if (mx[lc]<mx[rc]) mx[x]=mx[rc],cnt[x]=cnt[rc],se[x]=max(mx[lc],se[rc]);
	else mx[x]=mx[lc],cnt[x]=cnt[lc],se[x]=max(mx[rc],se[lc]);
}
void build(int x)
{
	cd[x]=r[x]-l[x]+1;
	if (l[x]==r[x]) return s[x]=mx[x]=pmx[x]=a[l[x]],se[x]=inf,cnt[x]=1,void();
	int c=x<<1;
	l[c]=l[x];r[c]=l[x]+r[x]>>1;
	l[c|1]=r[c]+1;r[c|1]=r[x];
	build(c);build(c|1);
	pushup(x);
}
void upd(int x,Q o)
{
	lz[x]=Q(max(lz[x].mxp,lz[x].mx+o.mxp),lz[x].mx+o.mx,max(lz[x].vp,lz[x].v+o.vp),lz[x].v+o.v);
	s[x]+=o.mx*cnt[x]+o.v*(cd[x]-cnt[x]);se[x]+=o.v;
	pmx[x]=max(pmx[x],mx[x]+o.mxp);mx[x]+=o.mx;
}
void pushdown(int x)
{
	int c=x<<1;
	ll mxx=max(mx[c],mx[c|1]);
	if (mx[c]==mxx) upd(c,lz[x]); else upd(c,Q(lz[x].vp,lz[x].v,lz[x].vp,lz[x].v));
	c|=1;
	if (mx[c]==mxx) upd(c,lz[x]); else upd(c,Q(lz[x].vp,lz[x].v,lz[x].vp,lz[x].v));
	lz[x]=Q();
}
void mdf1(int x)
{
	if (z<=l[x]&&r[x]<=y)
	{
		upd(x,Q(max(dt,0),dt,max(dt,0),dt));
		return;
	}
	pushdown(x);
	int c=x<<1;
	if (z<=r[c]) mdf1(c);
	if (y>r[c]) mdf1(c|1);
	pushup(x);
}
void mdf2(int x)
{
	if (dt>=mx[x]) return;
	if (z<=l[x]&&r[x]<=y)
	{
		if (dt<=se[x])
		{
			pushdown(x);
			mdf2(x<<1);mdf2(x<<1|1);
			pushup(x);
		}
		else
		{
			upd(x,Q(0,dt-mx[x],0,0));
		}
		return;
	}
	pushdown(x);
	int c=x<<1;
	if (z<=r[c]) mdf2(c);
	if (y>r[c]) mdf2(c|1);
	pushup(x);
}
void sol3(int x)
{
	if (z<=l[x]&&r[x]<=y) return ans+=s[x],void();
	pushdown(x);
	int c=x<<1;
	if (z<=r[c]) sol3(c);
	if (y>r[c]) sol3(c|1);
}
void sol4(int x)
{
	if (z<=l[x]&&r[x]<=y) return ans=max(ans,mx[x]),void();
	pushdown(x);
	int c=x<<1;
	if (z<=r[c]) sol4(c);
	if (y>r[c]) sol4(c|1);
}
void sol5(int x)
{
	if (z<=l[x]&&r[x]<=y) return ans=max(ans,pmx[x]),void();
	pushdown(x);
	int c=x<<1;
	if (z<=r[c]) sol5(c);
	if (y>r[c]) sol5(c|1);
}
int main()
{
	read(n);read(m);
	for (i=1;i<=n;i++) read(a[i]);
	r[l[1]=1]=n;build(1);
	while (m--)
	{
		read(typ);read(z);read(y);
		if (typ>=3)
		{
			ans=(typ==3)?0:inf;
			if (typ==3) sol3(1); else if (typ==4) sol4(1); else sol5(1);
			printf("%lld\n",ans);
		}
		else
		{
			read(dt);
			if (typ==1) mdf1(1); else mdf2(1);
		}
	}
} 
\end{lstlisting}

\subsection{$k$-d 树(二进制分组)}

均摊 $O(\log ^2n)$ 插入,$O(\sqrt n)$ 矩形查询。

\begin{lstlisting}
#define tmpl template<typename T>
typedef long long ll;
tmpl struct P
{
	ll x,y;
	T v;
};
tmpl struct Q
{
	ll x[2],y[2];
	bool t;
	T s;
	Q() {}
	Q(const P<T> &a)
	{
		x[0]=x[1]=a.x;
		y[0]=y[1]=a.y;
		s=a.v;
	}
};
tmpl bool cmp0(const P<T> &a,const P<T> &b) { return a.x<b.x; }
tmpl bool cmp1(const P<T> &a,const P<T> &b) { return a.y<b.y; }
tmpl struct kdt
{
	vector<P<T>> c;
	vector<Q<T>> a;
	ll m,u,d,l,r;
	T ans;
	bool fir;
	void build(int x,P<T> *b,int n)
	{
		if (x==1)
		{
			a.resize(m=n<<1);
			a[x].t=0;
			c.resize(n);
			for (int i=0; i<n; i++) c[i]=b[i];
		}
		if (n==1)
		{
			a[x]=Q<T>(b[0]);
			return;
		}
		int mid=n>>1,c=x<<1;
		nth_element(b,b+mid,b+n,a[x].t?cmp1<T>:cmp0<T>);
		a[c].t=a[c|1].t=a[x].t^1;
		build(c,b,mid);
		build(c|1,b+mid,n-mid);
		a[x].s=a[c].s+a[c|1].s;
		a[x].x[0]=min(a[c].x[0],a[c|1].x[0]);
		a[x].x[1]=max(a[c].x[1],a[c|1].x[1]);
		a[x].y[0]=min(a[c].y[0],a[c|1].y[0]);
		a[x].y[1]=max(a[c].y[1],a[c|1].y[1]);
	}
	void find(int x)
	{
		if (x>=m||a[x].x[1]<u||a[x].x[0]>d||a[x].y[1]<l||a[x].y[0]>r) return;
		if (u<=a[x].x[0]&&a[x].x[1]<=d&&l<=a[x].y[0]&&a[x].y[1]<=r)
		{
			ans=fir?a[x].s:ans+a[x].s;
			fir=0;
			return;
		}
		find(x<<1); find(x<<1|1);
	}
	pair<bool,T> find(ll x1,ll y1,ll x2,ll y2)
	{
		fir=1;
		ans={};
		u=x1; d=x2;
		l=y1; r=y2;
		find(1);
		return {!fir,ans};
	}
};
const int N=2e5+2,M=18;
tmpl struct KDT
{
	kdt<T> s[M];
	P<T> a[N];
	int n,m,i;
	KDT() { n=0; }
	KDT(int N,ll *x,ll *y,T *w)//[0,n)
	{
		n=N;
		int i,j;
		for (i=0; i<n; i++) a[i]={x[i],y[i],w[i]};
		for (i=j=0; n>>i; i++) if (n>>i&1) s[i].build(1,a+j,1<<i),j+=1<<i;
	}
	void insert(ll x,ll y,T w)
	{
		a[0]={x,y,w}; m=1;
		for (i=0; n&1<<i; i++) for (auto u:s[i].c) a[m++]=u;
		s[i].build(1,a,m);
		++n;
	}
	pair<bool,T> ask(ll x,ll y,ll xx,ll yy)
	{
		T ans;
		bool fir=1;
		for (i=0; 1<<i<=n; i++) if (1<<i&n)
		{
			auto [_,tmp]=s[i].find(x,y,xx,yy);
			if (!_) continue;
			ans=fir?tmp:ans+tmp;
			fir=0;
		}
		return {!fir,ans};
	}
};
int x[N],y[N],w[N];
int main()
{
	ios::sync_with_stdio(0); cin.tie(0); cout.tie(0);
	int n,q,i;
	cin>>n>>q;
	for (i=0; i<n; i++) cin>>x[i]>>y[i]>>w[i];
	KDT<ll> s(n,x,y,w);
	while (q--)
	{
		int op,x,y,w;
		cin>>op>>x>>y>>w;
		if (op==0) s.insert(x,y,w); else
		{
			cin>>op;
			cout<<s.ask(x,y,w-1,op-1)<<'\n';
		}
	}
	return 0;
}
\end{lstlisting}

\subsection{双端队列全局查询}

\begin{lstlisting}

template<typename T> struct dq
{
	vector<T> l,sl,r,sr;
	void push_front(const T &o)
	{
		sl.push_back(sl.size()?o+sl.back():o);
		l.push_back(o);
	}
	void push_back(const T &o)
	{
		sr.push_back(sr.size()?sr.back()+o:o);
		r.push_back(o);
	}
	void pop_front()
	{
		if (l.size()) sl.pop_back(),l.pop_back();
		else
		{
			assert(r.size());
			int n=r.size(),m,i;
			if (m=n-1>>1)
			{
				l.resize(m); sl.resize(m);
				for (i=1; i<=m; i++) l[m-i]=r[i];
				sl[0]=l[0];
				for (i=1; i<m; i++) sl[i]=l[i]+sl[i-1];
			}
			for (i=m+1; i<n; i++) r[i-(m+1)]=r[i];
			m=n-(m+1);
			r.resize(m); sr.resize(m);
			if (m)
			{
				sr[0]=r[0];
				for (i=1; i<m; i++) sr[i]=sr[i-1]+r[i];
			}
		}
	}
	void pop_back()
	{
		if (r.size()) sr.pop_back(),r.pop_back();
		else
		{
			assert(l.size());
			int n=l.size(),m,i;
			if (m=n-1>>1)
			{
				r.resize(m); sr.resize(m);
				for (i=1; i<=m; i++) r[m-i]=l[i];
				sr[0]=r[0];
				for (i=1; i<m; i++) sr[i]=sr[i-1]+r[i];
			}
			for (i=m+1; i<n; i++) l[i-(m+1)]=l[i];
			m=n-(m+1);
			l.resize(m); sl.resize(m);
			if (m)
			{
				sl[0]=l[0];
				for (i=1; i<m; i++) sl[i]=l[i]+sl[i-1];
			}
		}
	}
	template<typename TT> TT ask(TT r)
	{
		if (sl.size()) r=r+sl.back();
		if (sr.size()) r=r+sr.back();
		return r;
	}
	T ask()
	{
		assert(sl.size()||sr.size());
		if (sl.size()&&sr.size()) return sl.back()+sr.back();
		return sl.size()?sl.back():sr.back();
	}
};//参数:类型。结合使用 + 运算符

\end{lstlisting}

\subsection{静态矩形加矩形和}

\begin{lstlisting}
const ll p=998244353;
struct Q
{
	int n,m;
	ll w;
	int typ;
	bool operator<(const Q &o) const
	{
		if (n!=o.n) return n<o.n;
		return typ<o.typ;
	}
};
template<typename T> struct tork
{
	vector<T> a;
	int n;
	tork(const vector<T> &b):a(all(b))
	{
		sort(all(a));
		a.resize(unique(all(a))-a.begin());
		n=a.size();
	}
	tork(const T *first,const T *last):a(first,last)
	{
		sort(all(a));
		a.resize(unique(all(a))-a.begin());
		n=a.size();
	}
	void get(T &x) { x=lower_bound(all(a),x)-a.begin()+1; }
	T operator[](const int &x) { return a[x]; }
};
struct bit
{
	vector<ll> a;
	int n;
	bit() {}
	bit(int nn):n(nn),a(nn+1) {}
	template<typename T> bit(int nn,T *b):n(nn),a(nn+1)
	{
		for (int i=1; i<=n; i++) a[i]=b[i];
		for (int i=1; i<=n; i++) if (i+(i&-i)<=n) a[i+(i&-i)]+=a[i];
	}
	void add(int x,ll y)
	{
		// cerr<<"add "<<x<<" by "<<y<<endl;
		assert(1<=x&&x<=n);
		if ((a[x]+=y)>=p) a[x]-=p;
		while ((x+=x&-x)<=n) if ((a[x]+=y)>=p) a[x]-=p;
	}
	ll sum(int x)
	{
		// cerr<<"sum "<<x;
		assert(0<=x&&x<=n);
		ll r=a[x];
		while (x^=x&-x) r+=a[x];
		// cerr<<"= "<<r<<endl;
		return r%p;
	}
	ll sum(int x,int y)
	{
		return (sum(y)+p-sum(x-1))%p;
	}
};
struct matrix
{
	int l,d,r,u;
	ll w;
};
vector<ll> rec_add_rec_sum(const vector<matrix> &op,const vector<matrix> &query)
{
	vector<Q> a[4];
	int n=op.size(),m=query.size(),i;
	for (auto &v:a) v.reserve(n+m<<2);
	for (auto [l,d,r,u,w]:op)//[l,r)*[d,u) += w
	{
		a[0].push_back({l,d,w*l%p*d%p,-1});
		a[1].push_back({l,d,w*l%p,-1});
		a[2].push_back({l,d,w*d%p,-1});
		a[3].push_back({l,d,w,-1});
		w=(p-w)%p;
		a[0].push_back({l,u,w*l%p*u%p,-1});
		a[1].push_back({l,u,w*l%p,-1});
		a[2].push_back({l,u,w*u%p,-1});
		a[3].push_back({l,u,w,-1});
		a[0].push_back({r,d,w*r%p*d%p,-1});
		a[1].push_back({r,d,w*r%p,-1});
		a[2].push_back({r,d,w*d%p,-1});
		a[3].push_back({r,d,w,-1});
		w=(p-w)%p;
		a[0].push_back({r,u,w*r%p*u%p,-1});
		a[1].push_back({r,u,w*r%p,-1});
		a[2].push_back({r,u,w*u%p,-1});
		a[3].push_back({r,u,w,-1});
	}
	i=0;
	for (auto [l,d,r,u,w]:query)//ask sum of [l,r)*[d,u)
	{
		a[0].push_back({l,d,1,i});
		a[1].push_back({l,d,(p*2-d)%p,i});
		a[2].push_back({l,d,(p*2-l)%p,i});
		a[3].push_back({l,d,(ll)l*d%p,i});
		a[0].push_back({l,u,p-1,i});
		a[1].push_back({l,u,u%p,i});
		a[2].push_back({l,u,l%p,i});
		a[3].push_back({l,u,(p*2-l)*u%p,i});
		a[0].push_back({r,u,1,i});
		a[1].push_back({r,u,(p*2-u)%p,i});
		a[2].push_back({r,u,(p*2-r)%p,i});
		a[3].push_back({r,u,(ll)u*r%p,i});
		a[0].push_back({r,d,p-1,i});
		a[1].push_back({r,d,d%p,i});
		a[2].push_back({r,d,r%p,i});
		a[3].push_back({r,d,(p*2-d)*r%p,i});
		++i;
	}
	assert(a[0].size()==n+m<<2);
	vector<ll> ans(m);
	auto cal=[&](vector<Q> a)
	{
		int n=a.size(),i;
		vector<int> b(n);
		for (i=0; i<n; i++) b[i]=(a[i].m-=a[i].typ>=0),a[i].n-=a[i].typ>=0;
		sort(all(a));
		tork t(b);
		for (i=0; i<n; i++) t.get(a[i].m);
		int m=t.a.size();
		bit s(m);
		for (auto [n,m,w,typ]:a) if (typ>=0) ans[typ]=(ans[typ]+s.sum(m)*w)%p; else s.add(m,w);
	};
	for (auto &v:a) cal(v);
	return ans;
}
int main()
{
	ios::sync_with_stdio(0); cin.tie(0);
	cout<<setiosflags(ios::fixed)<<setprecision(15);
	int n,m,i;
	cin>>n>>m;
	vector<matrix> a(n),b(m);
	for (auto &[l,d,r,u,w]:a) cin>>l>>d>>r>>u>>w;
	for (auto &[l,d,r,u,w]:b) cin>>l>>d>>r>>u;
	auto ans=rec_add_rec_sum(a,b);
	for (i=0; i<m; i++) cout<<ans[i]<<'\n';
}

\end{lstlisting}

\subsection{线段树分裂}

\begin{lstlisting}

namespace sgt
{
#define ask_kth
	int L=0,R=1e9;
	void set_bound(int l,int r) { L=l; R=r; }
	typedef ll info;
	const info E=0;//找不到会返回 E
	const int N=8e6+5;
#define lc(x) (a[x].lc)
#define rc(x) (a[x].rc)
#define s(x) (a[x].s)
	struct node
	{
		int lc,rc;
		info s;
	};
	node a[N];
	vector<int> id;
	int ids=0,pos,z,y;
	bool fir;
	info tmp;
	int npt()
	{
		int x;
		if (id.size()) x=id.back(),id.pop_back();
		else x=++ids;
		lc(x)=rc(x)=0;
		return x;
	}
	void pushup(int &x)
	{
		if (lc(x)&&rc(x)) s(x)=s(lc(x))+s(rc(x));
		else if (lc(x)) s(x)=s(lc(x));
		else if (rc(x)) s(x)=s(rc(x));
		else id.push_back(x),x=0;
	}
	void insert(int &x,int l,int r)
	{
		if (l==r)
		{
			if (!x) x=npt(),s(x)=tmp;
			else s(x)=s(x)+tmp;
			return;
		}
		if (!x) x=npt();
		int mid=l+r>>1;
		if (pos<=mid)
		{
			insert(lc(x),l,mid);
			if (rc(x)) s(x)=s(lc(x))+s(rc(x)); else s(x)=s(lc(x));
		}
		else
		{
			insert(rc(x),mid+1,r);
			if (lc(x)) s(x)=s(lc(x))+s(rc(x)); else s(x)=s(rc(x));
		}
	}
	void modify(int &x,int l,int r)
	{
		if (!x) x=npt();
		if (l==r)
		{
			s(x)=tmp;
			return;
		}
		int mid=l+r>>1;
		if (pos<=mid)
		{
			insert(lc(x),l,mid);
			if (rc(x)) s(x)=s(lc(x))+s(rc(x)); else s(x)=s(lc(x));
		}
		else
		{
			insert(rc(x),mid+1,r);
			if (lc(x)) s(x)=s(lc(x))+s(rc(x)); else s(x)=s(rc(x));
		}
	}
	int merge(int x1,int x2,int l,int r)
	{
		if (!(x1&&x2)) return x1|x2;
		if (l==r) { s(x1)=s(x1)+s(x2); return x1; }
		int mid=l+r>>1;
		lc(x1)=merge(lc(x1),lc(x2),l,mid);
		rc(x1)=merge(rc(x1),rc(x2),mid+1,r);
		pushup(x1);
		return x1;
	}
	void ask(int x,int l,int r)
	{
		if (!x) return;
		if (z<=l&&r<=y)
		{
			if (fir) tmp=s(x),fir=0; else tmp=tmp+s(x);
			return;
		}
		int mid=l+r>>1;
		if (z<=mid) ask(lc(x),l,mid);
		if (y>mid) ask(rc(x),mid+1,r);
	}
	void split(int &x1,int &x2,int l,int r)
	{
		assert(!x1);
		if (!x2) return;
		if (z<=l&&r<=y) { x1=x2; x2=0; return; }
		x1=npt();
		int mid=l+r>>1;
		if (z<=mid) split(lc(x1),lc(x2),l,mid);
		if (y>mid) split(rc(x1),rc(x2),mid+1,r);
		pushup(x1); pushup(x2);
	}
	info *b;
	void build(int &x,int l,int r)
	{
		x=npt();
		if (l==r) { s(x)=b[l]; return; }
		int mid=l+r>>1;
		build(lc(x),l,mid); build(rc(x),mid+1,r);
		s(x)=s(lc(x))+s(rc(x));
	}
	struct set
	{
		int rt;
		set():rt(0) {}
		set(info *a):rt(0) { b=a; build(rt,L,R); }
		void modify(int p,const info &o) { pos=p; tmp=o; sgt::modify(rt,L,R); }
		void insert(int p,const info &o) { pos=p; tmp=o; sgt::insert(rt,L,R); }
		void join(const set &o) { rt=merge(rt,o.rt,L,R); }
		info ask(int l,int r)
		{
			z=l; y=r; fir=1;
			sgt::ask(rt,L,R);
			return fir?E:tmp;
		}
		set split(int l,int r)
		{
			z=l; y=r; set p;
			sgt::split(p.rt,rt,L,R);
			return p;
		}
#ifdef ask_kth
		int kth(info k)
		{
			int x=rt,l=L,r=R,mid;
			if (k>s(x)) return -1;
			s(0)=0;
			while (l<r)
			{
				mid=l+r>>1;
				if (s(lc(x))>=k) x=lc(x),r=mid;
				else k-=s(lc(x)),x=rc(x),l=mid+1;
			}
			return l;
		}
#endif
	};
#undef lc
#undef rc
#undef s
}
typedef sgt::set tree;
\end{lstlisting}

\newpage

\section{数学}

\subsection{单情况矩阵 (+)}

\begin{lstlisting}

template<typename T,int n> struct matrix
{
	#define all(x) (x).begin(),(x).end()
	array<pair<int,T>,n> a;
	matrix(char c='E')
	{
		int i;
		if (c=='E') for (i=0;i<n;i++) a[i]={i,0};
		else assert(0);
	}
	matrix(char c,int x)
	{
		
	}
	matrix operator+(const matrix &o) const
	{
		matrix r;
		int i,j,k;
		for (i=0;i<n;i++)
		{
			auto [x,y]=a[i];
			r.a[i]={o.a[x].first,o.a[x].second+y};
		}
		return r;
	}
};

\end{lstlisting}

\subsection{矩阵求逆(要求质数)}

$O(n^3)$,$O(n^2)$。

\begin{lstlisting}
#include <bits/stdc++.h>
using namespace std;
typedef long long ll;
const int N=402,p=1e9+7;
void inv(int &x)
{
	int y=p-2,r=1;
	while (y)
	{
		if (y&1) r=(ll)r*x%p;
		x=(ll)x*x%p;
		y>>=1;
	}
	x=r;
}
int a[N][N],ih[N],jh[N];
int main()
{
	ios::sync_with_stdio(0);cin.tie(0);
	int i,j,k,n;
	cin>>n;
	for (i=1;i<=n;i++) for (j=1;j<=n;j++) cin>>a[i][j];
	for (k=1;k<=n;k++)
	{//ih,jh要清空
		for (i=k;i<=n;i++) if (!ih[k]) for (j=k;j<=n;j++) if (a[i][j])
		{
			ih[k]=i;jh[k]=j;break;
		}
		if (!ih[k]) return cout<<"No Solution"<<endl,0;
		for (j=1;j<=n;j++) swap(a[k][j],a[ih[k]][j]);
		for (i=1;i<=n;i++) swap(a[i][k],a[i][jh[k]]);
		if (!a[k][k]) return cout<<"No Solution"<<endl,0;inv(a[k][k]);
		for (i=1;i<=n;i++) if (i!=k) a[k][i]=(ll)a[k][i]*a[k][k]%p;
		for (i=1;i<=n;i++) if (i!=k) for (j=1;j<=n;j++) if (j!=k) a[i][j]=(a[i][j]+(ll)(p-a[i][k])*a[k][j])%p;
		for (i=1;i<=n;i++) if (i!=k) a[i][k]=(ll)(p-a[i][k])*a[k][k]%p;	
	}
	for (k=n;k;k--)
	{
		for (j=1;j<=n;j++) swap(a[k][j],a[jh[k]][j]);
		for (i=1;i<=n;i++) swap(a[i][k],a[i][ih[k]]);
	}
	for (i=1;i<=n;i++)
	{
		for (j=1;j<=n;j++) cout<<a[i][j]<<" \n"[j==n];
	}
}
/*
输入
3
1 2 8
2 5 6
5 1 2
输出
718750005 718750005 968750007
171875001 671875005 296875002
117187501 867187506 429687503
*/
\end{lstlisting}

\subsection{任意模数矩阵求逆(未验)}

$O(n^3)$,$O(n^2)$。

\begin{lstlisting}
int ksm(int x,int y)
{
	int r=1;
	while (y)
	{
		if (y&1) r=(ll)r*x%p;
		y>>=1;x=(ll)x*x%p;
	}
	return r;
}
int phi(int n)
{
	int r=n;
	for (int i=2;i*i<=n;i++) if (n%i==0)
	{
		r=r/i*(i-1);n/=i;
		while (n%i==0) n/=i;
	}
	if (n>1) r=r/n*(n-1);
	return r;
}
void cal(int a[][N],int b[][N],int n)
{
	int i,j,k,r,ph=phi(p);
	for (i=1;i<=n;i++) memset(b+1,0,n<<2);
	for (i=1;i<=n;i++) b[i][i]=1;
	for (i=1;i<=n;i++)
	{
		k=i;
		for (j=i+1;j<=n;j++) if (a[j][i]&&a[j][i]<a[k][i]) k=j;
		if (!a[k][i]) {puts("No Solution");exit(0);}
		swap(a[i],a[k]);swap(b[i],b[k]);
		for (j=i+1;j<=n;j++) if (a[j][i])
		{
			r=p-a[j][i]/a[i][i];
			for (k=i;k<=n;k++) a[j][k]=(a[j][k]+(ll)r*a[i][k])%p;
			for (k=1;k<=n;k++) b[j][k]=(b[j][k]+(ll)r*b[i][k])%p;
			while (a[j][i])
			{
				swap(a[i],a[j]);swap(b[i],b[j]);
				r=p-a[j][i]/a[i][i];
				for (k=i;k<=n;k++) a[j][k]=(a[j][k]+(ll)r*a[i][k])%p;
				for (k=1;k<=n;k++) b[j][k]=(b[j][k]+(ll)r*b[i][k])%p;
			}
		}
		if (__gcd(a[i][i],p)!=1) {puts("No Solution");exit(0);}
		r=ksm(a[i][i],ph-1);
		for (j=i;j<=n;j++) a[i][j]=(ll)a[i][j]*r%p;
		for (j=1;j<=n;j++) b[i][j]=(ll)b[i][j]*r%p;
		assert(a[i][i]==1);
		for (j=1;j<i;j++)
		{
			r=p-a[j][i];
			for (k=i;k<=n;k++) a[j][k]=(a[j][k]+(ll)r*a[i][k])%p;
			for (k=1;k<=n;k++) b[j][k]=(b[j][k]+(ll)r*b[i][k])%p;
		}
	}
}
\end{lstlisting}

\subsection{矩阵的特征多项式}

$O(n^3)$,$O(n^2)$。

\begin{lstlisting}
#include <bits/stdc++.h>
using namespace std;
typedef long long ll;
const int N=502,p=998244353;
int a[N][N],f[N];
int n,i,j,k,x,y,r;
void inc(int &x,const int y)
{
	if ((x+=y)>=p) x-=p;
}
void dec(int &x,const int y)
{
	if ((x-=y)<0) x+=p;
}
int ksm(int x,int y)
{
	int r=1;
	while (y)
	{
		if (y&1) r=(ll)r*x%p;
		x=(ll)x*x%p;y>>=1;
	}
	return r;
}
void calmatrix(int a[N][N],int n)
{
	int i,j,k,r;
	for (i=2;i<=n;i++)
	{
		for (j=i;j<=n&&!a[j][i-1];j++);
		if (j>n) continue;
		if (j>i)
		{
			swap(a[i],a[j]);
			for (k=1;k<=n;k++) swap(a[k][j],a[k][i]);
		}
		r=a[i][i-1];
		for (j=1;j<=n;j++) a[j][i]=(ll)a[j][i]*r%p;
		r=ksm(r,p-2);
		for (j=i-1;j<=n;j++) a[i][j]=(ll)a[i][j]*r%p;
		for (j=i+1;j<=n;j++)
		{
			r=a[j][i-1];
			for (k=1;k<=n;k++) a[k][i]=(a[k][i]+(ll)a[k][j]*r)%p;
			r=p-r;
			for (k=i-1;k<=n;k++) a[j][k]=(a[j][k]+(ll)a[i][k]*r)%p;
		}
	}
}
void calpoly(int a[N][N],int n,int *f)
{
	static int g[N][N];
	memset(g,0,sizeof(g));
	g[0][0]=1;
	int i,j,k,r,rr;
	for (i=1;i<=n;i++)
	{
		r=p-1;
		for (j=i;j;j--)//第 j 行选第 n 列
		{
			rr=(ll)r*a[j][i]%p;
			for (k=0;k<j;k++) g[i][k]=(g[i][k]+(ll)rr*g[j-1][k])%p;
			r=(ll)r*a[j][j-1]%p;
		}
		for (k=1;k<=i;k++) inc(g[i][k],g[i-1][k-1]);
	}
	memcpy(f,g[n],n+1<<2);
	//if (n&1) for (i=0;i<=n;i++) if (f[i]) f[i]=p-f[i];//若注释掉则为 |kE-A|
}
int main()
{
	ios::sync_with_stdio(0);cin.tie(0);
	cin>>n;
	for (i=1;i<=n;i++) for (j=1;j<=n;j++) cin>>a[i][j];
	calmatrix(a,n);calpoly(a,n,f);
	for (i=0;i<=n;i++) cout<<f[i]<<"x^"<<i<<"+\n"[i==n];
}
/*
3
1 2 3
4 5 6
7 8 9
输出:0x^0+998244335x^1+998244338x^2+1x^3
*/
\end{lstlisting}

\subsection{最短递推式(BM 算法)}

$\sum_{j=0}^{m-1} a_{i-j-1}r_j=a_i$。

\begin{lstlisting}
vector<ui> bm(const vector<ui> &a)
{
	vector<ui> r,lst;
	int n=a.size(),m=0,q=0,i,j,k=-1;
	ui D=0;
	for (i=0;i<n;i++)
	{
		ui cur=0;
		for (j=0;j<m;j++) cur=(cur+(ll)a[i-j-1]*r[j])%p;
		cur=(a[i]+p-cur)%p;
		if (!cur) continue;
		if (k==-1)
		{
			k=i;
			D=cur;
			r.resize(m=i+1);
			continue;
		}
		auto v=r;
		ui x=(ll)cur*ksm(D,p-2)%p;
		if (m<q+i-k) r.resize(m=q+i-k);
		(r[i-k-1]+=x)%=p;
		ui *b=r.data()+i-k;
		x=(p-x)%p;
		for (j=0;j<q;j++) b[j]=(b[j]+(ll)x*lst[j])%p;
		if (v.size()+k<lst.size()+i)
		{
			lst=v;
			q=v.size();
			k=i;
			D=cur;
		}
	}
	return r;
}
\end{lstlisting}

\subsection{在线 $O(1)$ 逆元}

\begin{lstlisting}
namespace online_inv
{
	typedef unsigned int ui;
	typedef unsigned long long ll;
	const ui p=1e9+7;
	const ui n=1010,m=n*n,N=m+2;
	int l[N],r[N];
	ui y[N];
	bool s[N];
	ui _inv[N*2],i,j,k;
	void init_inv()
	{
		assert(n*n*n>p);
		_inv[1]=1;
		for (i=2;i<m*2;i++)
		{
			j=p/i;
			_inv[i]=(ll)(p-j)*_inv[p-i*j]%p;
		}
		s[0]=y[0]=1;
		for (i=1;i<n;i++) for (j=i;j<n;j++) if (!s[k=i*m/j])
		{
			y[k]=j;
			s[k]=1;
		}
		l[0]=1;
		for (i=1;i<=m;i++) l[i]=s[i]?y[i]:l[i-1];
		r[m]=1;
		for (i=m-1;~i;i--) r[i]=s[i]?y[i]:r[i+1];
		for (i=0;i<=m;i++) y[i]=min(l[i],r[i]);
	}
	inline ui inv(const ui &x)
	{
		assert(x&&x<p);
		if (x<m*2) return _inv[x];
		k=(ll)x*m/p;
		j=(ll)y[k]*x%p;
		return (j<m*2?_inv[j]:p-_inv[p-j])*(ll)y[k]%p;
	}
}
using online_inv::init_inv,online_inv::inv,online_inv::p;
\end{lstlisting}

\subsection{Strassen 矩阵乘法}

$O(n^{\log_27})$。

\begin{lstlisting}
#include <bits/stdc++.h>
using namespace std;
typedef unsigned int ui;
typedef unsigned long long ull;
const ui p=998244353;
const ull fh=1ull<<31;
struct Q
{
	ui **a;
	int n;
	Q(){n=0;}
	void clear()
	{
		for (int i=0;i<n;i++) delete a[i];
		if (n) delete a;n=0;
	}
	Q(int nn)//不能传入不是 2 的幂的数!
	{
		n=nn;
		assert(n==(n&-n));
		a=new ui*[n];
		for (int i=0;i<n;i++) a[i]=new ui[n],memset(a[i],0,n*sizeof a[0][0]);
	}
	const Q & operator=(const Q& b)
	{
		clear();n=b.n;
		a=new ui*[n];
		for (int i=0;i<n;i++) a[i]=new ui[n],memcpy(a[i],b.a[i],n*sizeof a[0][0]);
		return *this;
	}
	~Q(){clear();}
	Q operator+(const Q &b)
	{
		Q c(n);
		for (int i=0;i<n;i++) for (int j=0;j<n;j++) if ((c.a[i][j]=a[i][j]+b.a[i][j])>=p) c.a[i][j]-=p;
		return c;
	}
	Q operator-(const Q &b)
	{	
		Q c(n);
		for (int i=0;i<n;i++) for (int j=0;j<n;j++) if ((c.a[i][j]=a[i][j]-b.a[i][j])&fh) c.a[i][j]+=p;
		return c;
	}
	Q operator*(Q &b)
	{
		Q c(n);
		if (n<=128)
		{
			for (int i=0;i<n;i++) for (int k=0;k<n;k++) for (int j=0;j<n;j++) c.a[i][j]=(c.a[i][j]+(ull)a[i][k]*b.a[k][j])%p;
			return c;
		}
		Q A[2][2],B[2][2],s[10],p[5];
		n>>=1;
		int i,j,k,l;
		for (i=0;i<2;i++) for (j=0;j<2;j++)
		{
			A[i][j]=Q(n);
			for (k=0;k<n;k++) memcpy(A[i][j].a[k],a[k+i*n]+j*n,n*sizeof a[0][0]);
			B[i][j]=Q(n);
			for (k=0;k<n;k++) memcpy(B[i][j].a[k],b.a[k+i*n]+j*n,n*sizeof a[0][0]);
		}
		s[0]=B[0][1]-B[1][1];
		s[1]=A[0][0]+A[0][1];
		s[2]=A[1][0]+A[1][1];
		s[3]=B[1][0]-B[0][0];
		s[4]=A[0][0]+A[1][1];
		s[5]=B[0][0]+B[1][1];
		s[6]=A[0][1]-A[1][1];
		s[7]=B[1][0]+B[1][1];
		s[8]=A[0][0]-A[1][0];
		s[9]=B[0][0]+B[0][1];
		p[0]=A[0][0]*s[0];
		p[1]=s[1]*B[1][1];
		p[2]=s[2]*B[0][0];
		p[3]=A[1][1]*s[3];
		p[4]=s[4]*s[5];
		A[0][0]=p[4]+p[3]-p[1]+s[6]*s[7];
		A[0][1]=p[0]+p[1];
		A[1][0]=p[2]+p[3];
		A[1][1]=p[4]+p[0]-p[2]-s[8]*s[9];
		for (i=0;i<2;i++) for (j=0;j<2;j++)	for (k=0;k<n;k++) memcpy(c.a[k+i*n]+j*n,A[i][j].a[k],n*sizeof a[0][0]);
		n<<=1;
		return c;
	}
};
int main()
{
	int i,j,n,m,k;
	ios::sync_with_stdio(0);cin.tie(0);
	cin>>n>>m>>k;
	int N=1<<32-min({__builtin_clz(n-1),__builtin_clz(m-1),__builtin_clz(k-1)});
	Q a(N),b(N);
	for (i=0;i<n;i++) for (j=0;j<m;j++) cin>>a.a[i][j];
	for (i=0;i<m;i++) for (j=0;j<k;j++) cin>>b.a[i][j];
	a=a*b;
	for (i=0;i<n;i++) for (j=0;j<k;j++) cout<<a.a[i][j]<<" \n"[j+1==k]; 
}
\end{lstlisting}

\subsection{扩展欧拉定理}

$a\uparrow\uparrow b \bmod c$

\begin{lstlisting}
namespace Prime
{
	typedef unsigned int ui;
	typedef unsigned long long ll;
	const int N=1e6+2;
	const ll M=(ll)(N-1)*(N-1);
	ui pr[N],mn[N],phi[N],cnt;
	int mu[N];
	void init_prime()
	{
		ui i,j,k;
		phi[1]=mu[1]=1;
		for (i=2;i<N;i++)
		{
			if (!mn[i])
			{
				pr[cnt++]=i;
				phi[i]=i-1;mu[i]=-1;
				mn[i]=i;
			}
			for (j=0;(k=i*pr[j])<N;j++)
			{
				mn[k]=pr[j];
				if (i%pr[j]==0)
				{
					phi[k]=phi[i]*pr[j];
					break;
				}
				phi[k]=phi[i]*(pr[j]-1);
				mu[k]=-mu[i];
			}
		}
		//for (i=2;i<N;i++) if (mu[i]<0) mu[i]+=p;
	}
	template<typename T> T getphi(T x)
	{
		assert(M>=x);
		T r=x;
		for (ui i=0;i<cnt&&(T)pr[i]*pr[i]<=x&&x>=N;i++) if (x%pr[i]==0)
		{
			ui y=pr[i],tmp;
			x/=y;
			while (x==(tmp=x/y)*y) x=tmp;
			r=r/y*(y-1);
		}
		if (x>=N) return r/x*(x-1);
		while (x>1)
		{
			ui y=mn[x],tmp;
			x/=y;
			while (x==(tmp=x/y)*y) x=tmp;
			r=r/y*(y-1);
		}
		return r;
	}
	template<typename T> vector<pair<T,ui>> getw(T x)
	{
		assert(M>=x);
		vector<pair<T,ui>> r;
		for (ui i=0;i<cnt&&(T)pr[i]*pr[i]<=x&&x>=N;i++) if (x%pr[i]==0)
		{
			ui y=pr[i],z=1,tmp;
			x/=y;
			while (x==(tmp=x/y)*y) x=tmp,++z;
			r.push_back({y,z});
		}
		if (x>=N)
		{
			r.push_back({x,1});
			return r;
		}
		while (x>1)
		{
			ui y=mn[x],z=1,tmp;
			x/=y;
			while (x==(tmp=x/y)*y) x=tmp,++z;
			r.push_back({y,z});
		}
		return r;
	}
}
using Prime::pr,Prime::phi,Prime::getw,Prime::getphi;
using Prime::mu,Prime::init_prime;
ui ksm(ll x,ui y,ui p)
{
	x=x%p+(x>=p)*p;
	ll r=1;
	while (y)
	{
		if (y&1)
		{
			if ((r*=x)>=p) r=r%p+p; else r%=p;
		}
		if ((x*=x)>=p) x=x%p+p; else x%=p;
		y>>=1;
	}
	return r;
}
struct Q
{
	vector<ui> p;
	Q(const ui &P)
	{
		p.push_back(P);
		while (p.back()>1) p.push_back(getphi(p.back()));
	}
	ui operator()(ll a,ll b)
	{
		if (!a) return (1^b&1)%p[0];
		ui r=1;
		int i=min(b,(ll)p.size());
		while ((--i)>=0) r=ksm(a,r,p[i]);
		return r%p[0];
	}
};
int main()
{
	ios::sync_with_stdio(0);cin.tie(0);
	cout<<setiosflags(ios::fixed)<<setprecision(15);
	int n,i;
	init_prime();
	int T;
	cin>>T;
	while (T--)
	{
		ui a,b,c;
		cin>>a>>b>>c;
		cout<<Q(c)(a,b)<<'\n';
	}
}
\end{lstlisting}

\subsection{exgcd}

$O(\log p)$,$O(\log p)$。

\begin{lstlisting}
int exgcd(int a,int b,int c)//ax+by=c,return x
{
	if (a==0) return c/b;
	return (c-(ll)b*exgcd(b%a,a,c))/a%b;
}
\end{lstlisting}

\begin{lstlisting}
pair<ll,ll> exgcd(ll a,ll b,ll c)//ax+by=c,{-1,-1} 无解,b=0 返回 {c/a,0},否则返回最小非负 x
{
	assert(a||b);
	if (!b) return {c/a,0};
	if (a<0) a=-a,b=-b,c=-c;
	ll d=gcd(a,b);
	if (c%d) return {-1,-1};
	ll x=1,x1=0,p=a,q=b,k;
	b=abs(b);
	while (b)
	{
		k=a/b;
		x-=k*x1;a-=k*b;
		swap(x,x1);
		swap(a,b);
	}
	b=abs(q/d);
	x=x*(c/d)%b;
	if (x<0) x+=b;
	return {x,(c-p*x)/q};
}
ll fun(ll a,ll b,ll p)//ax=b(mod p)
{
	return exgcd(-p,a,b).second%p;
}
\end{lstlisting}

\subsection{exCRT}

\begin{lstlisting}
namespace CRT
{
	typedef long long ll;
	pair<ll,ll> exgcd(ll a,ll b,ll c)
	{
		assert(a||b);
		if (!b) return {c/a,0};
		ll d=gcd(a,b);
		if (c%d) return {-1,-1};
		ll x=1,x1=0,p=a,q=b,k;
		b=abs(b);
		while (b)
		{
			k=a/b;
			x-=k*x1;a-=k*b;
			swap(x,x1);
			swap(a,b);
		}
		b=abs(q/d);
		x=x*(c/d)%b;
		if (x<0) x+=b;
		return {x,(c-p*x)/q};
	}
	struct Q
	{
		ll p,r;//0<=r<p
		Q operator+(const Q &o) const
		{
			if (p==0||o.p==0) return {0,0};
			auto [x,y]=exgcd(p,-o.p,r-o.r);
			if (x==-1&&y==-1) return {0,0};
			ll q=lcm(p,o.p);
			return {q,((r-x*p)%q+q)%q};
		}
	};
}
using CRT::Q;
\end{lstlisting}


\subsection{exBSGS}

$O(\sqrt n)$。

\begin{lstlisting}
namespace BSGS
{
	typedef unsigned int ui;
	typedef unsigned long long ll;
	template<int N,typename T,typename TT> struct ht//个数,定义域,值域
	{
		const static int p=1e6+7,M=p+2;
		TT a[N];
		T v[N];
		int fir[p+2],nxt[N],st[p+2];//和模数相适应
		int tp,ds;//自定义模数
		ht(){memset(fir,0,sizeof fir);tp=ds=0;}
		void mdf(T x,TT z)//位置,值
		{
			ui y=x%p;
			for (int i=fir[y];i;i=nxt[i]) if (v[i]==x) return a[i]=z,void();//若不可能重复不需要 for
			v[++ds]=x;a[ds]=z;
			if (!fir[y]) st[++tp]=y;
			nxt[ds]=fir[y];fir[y]=ds;
		}
		TT find(T x)
		{
			ui y=x%p;
			int i;
			for (i=fir[y];i;i=nxt[i]) if (v[i]==x) return a[i];
			return 0;//返回值和是否判断依据要求决定
		}
		void clear()
		{
			++tp;
			while (--tp) fir[st[tp]]=0;
			ds=0;
		}
	};
	const int N=5e4;
	ht<N,ui,ui> s;
	int exgcd(int a,int b)
	{
		if (a==1) return 1;
		return (1-(long long)b*exgcd(b%a,a))/a;//not ll
	}
	int bsgs(ui a,ui b,ui p)
	{
		s.clear();
		a%=p;b%=p;
		if (!a) return 1-min((int)b,2);//含 -1
		ui i,j,k,x,y;
		x=sqrt(p)+2;
		for (i=0,j=1;i<x;i++,j=(ll)j*a%p)
		{
			if (j==b) return i;
			s.mdf((ll)j*b%p,i+1);
		}
		k=j;
		for (i=1;i<=x;i++,j=(ll)j*k%p) if (y=s.find(j)) return (ll)i*x-y+1;
		return -1;
	}
	bool isprime(ui p)
	{
		if (p<=1) return 0;
		for (ui i=2;i*i<=p;i++) if (p%i==0) return 0;
		return 1;
	}
	int exbsgs(ui a,ui b,ui p)//a^x=b(mod p)
	{
		//if (isprime(p)) return bsgs(a,b,p);
		a%=p;b%=p;
		ui i,j,k,x,y=__lg(p),cnt=0;
		for (i=0,j=1%p;i<=y;i++,j=(ll)j*a%p) if (j==b) return i;
		y=1;
		while (1)
		{
			if ((x=gcd(a,p))==1) break;
			if (b%x) return -1;//no sol
			++cnt;
			p/=x;b/=x;
			y=(ll)y*(a/x)%p;
		}
		a%=p;
		b=(ll)b*(p+exgcd(y,p))%p;
		int r=bsgs(a,b,p);
		return r==-1?-1:r+cnt;
	}
}
using BSGS::bsgs,BSGS::exbsgs;
\end{lstlisting}

\subsection{exLucas}

\begin{lstlisting}
namespace exlucas
{
	typedef long long ll;
	typedef pair<int,int> pa;
	int P,p,q,i;
	vector<pa> a;
	vector<vector<int> > b;
	vector<int> ph;
	vector<int> xs;
	int ksm(unsigned int x,ll y,const unsigned int p)
	{
		unsigned int r=1;
		while (y)
		{
			if (y&1) r=(unsigned long long)r*x%p;
			x=(unsigned long long)x*x%p;
			y>>=1;
		}
		return r;
	}
	void init(int x)//分解质因数,如有必要可以使用更快的方法
	{
		a.clear();b.clear();
		int i,y,z;
		vector<int> v;
		for (i=2;i*i<=x;i++) if (x%i==0)
		{
			z=i;x/=i;
			while (1)
			{
				y=x/i;
				if (i*y==x) x=y; else break;
				z*=i;
			}
			a.push_back(pa(i,z));
			b.push_back(v);
		}
		if (x>1) a.push_back(pa(x,x)),b.push_back(v);
		ph.resize(a.size());
		xs.resize(a.size());
		for (int k=0;k<a.size();k++)
		{
			tie(q,p)=a[k];
			ph[k]=p/q*(q-1);
			xs[k]=(ll)ksm(P/p,ph[k]-1,p)*(P/p)%P;
		}
	}
	void spinit(int x)//O(p) space
	{
		for (int k=0;k<a.size();k++)
		{
			int q,p;
			tie(q,p)=a[k];
			b[k].resize(p);
			b[k][0]=1;
			for (int i=1,j=q;i<p;i++) if (i==j) j+=q,b[k][i]=b[k][i-1]; else b[k][i]=(ll)b[k][i-1]*i%p;
		}
	}
	ll g(ll n)
	{
		ll r=0,s;
		while (n>=q)
		{
			n/=q;
			r+=n;
		}
		return r;
	}
	// int f(ll n)
	// {
	// 	if (n==0) return 1;
	// 	int r=1;//若 p>1e9 j 要 unsigned
	// 	for (int i=1,j=q;i<p;i++) if (i==j) j+=q; else r=(ll)r*i%p;
	// 	r=(ll)ksm(r,n/p,p)*f(n/q)%p;
	// 	n%=p;
	// 	for (int i=1,j=q;i<=n;i++) if (i==j) j+=q; else r=(ll)r*i%p;
	// 	return r;
	// }//O(T\sum p) time,O(1) space ver.
	int f(ll n)
	{
		int r=1;
		ll cs=0;
		while (n)
		{
			r=(ll)r*b[i][n%p]%p;
			cs+=n/p;
			n/=q;
		}
		return (ll)ksm(b[i][p-1],cs%ph[i],p)*r%p;
	}//O(\sum p) time,O(p) space ver.
	int C(ll n,ll m,int M)
	{
		if (n<m) return 0;
		int r=0,w;
		if (P!=M) init(P=M),spinit(P);//sp for O(p) space
		for (i=0;i<a.size();i++)
		{
			tie(q,p)=a[i];
			w=(ll)ksm(q,g(n)-g(m)-g(n-m),p)*f(n)%p*ksm((ll)f(m)*f(n-m)%p,ph[i]-1,p)%p;
			r=(r+(ll)xs[i]*w)%M;
		}
		return r;
	}
}
#define C(x,y,z) exlucas::C(x,y,z)
\end{lstlisting}

\subsection{杜教筛}

\begin{lstlisting}
namespace du_seive
{
	typedef unsigned int ui;
	typedef unsigned long long ll;
	unordered_map<ll,ui> mp;
	const int N=1e7+2;
	const ui p=998244353;
	ui pr[N],phi[N];
	ui cnt;
	void init()
	{
		cnt=0;phi[1]=1;
		int i,j;
		for (i=2;i<N;i++)
		{
			if (!phi[i])
			{
				pr[cnt++]=i;
				phi[i]=i-1;
			}
			for (j=0;i*pr[j]<N;j++)
			{
				if (i%pr[j]==0)
				{
					phi[i*pr[j]]=phi[i]*pr[j];
					break;
				}
				phi[i*pr[j]]=phi[i]*(pr[j]-1);
			}
			if ((phi[i]+=phi[i-1])>=p) phi[i]-=p;
		}
	}
	ui get_phi_sum(ll n)
	{
		if (n<N) return phi[n];
		if (mp.count(n)) return mp[n];
		ui sum=0;
		for (ll i=2,j,k;i<=n;i=j+1)
		{
			j=n/(k=n/i);
			sum=(sum+(ll)get_phi_sum(k)*(j-i+1))%p;
		}
		ui nn=n%p;
		sum=(nn*(nn+1ll)/2+p-sum)%p;
		mp[n]=sum;
		return sum;
	}
}
using du_seive::init,du_seive::get_phi_sum;
\end{lstlisting}

\subsection{线性规划}

\begin{lstlisting}
typedef long double db;//__float128
struct linear
{
	static const int N=45;//n+m
	db r[N][N];
	int col[N],row[N];
	const db eps=1e-10,inf=1e9;//1e-17
	int n,m;
	template<typename T> linear(const vector<T> &a)//target: \sum a(i-1)xi
	{
		memset(r,0,sizeof r);
		memset(col,0,sizeof col);
		memset(row,0,sizeof row);
		n=a.size();m=0;
		for (int i=1;i<=n;i++) r[0][i]=-a[i-1];
	}
	template<typename T> void add(const vector<T> &a,db b)//limit: \sum a(i-1)xi<=b
	{
		assert(a.size()==n);
		++m;
		for (int i=1;i<=n;i++) r[m][i]=-a[i-1];
		r[m][0]=b;
	}
	void pivot(int k, int t)
	{
		swap(row[k+n],row[t]);
		db rkt=-r[k][t];
		int i,j;
		for (i=0;i<=n;i++) r[k][i]/=rkt;
		r[k][t]=-1/rkt;
		for (i=0;i<=m;i++) if (i!=k)
		{
			db rit=r[i][t];
			if (rit>=-eps&&rit<=eps) continue;
			for (j=0;j<=n;j++) if (j!=t) r[i][j]+=rit*r[k][j];
			r[i][t]=r[k][t]*rit;
		}
	}
	bool init()
	{
		int i;
		for (i=1;i<=n+m;i++) row[i]=i;
		while(1)
		{
			int q=1;
			auto b_min=r[1][0];
			for (i=2;i<=m;i++) if (r[i][0]<b_min) b_min=r[i][0],q=i;
			if (b_min+eps>=0) return 1;
			int p=0;
			for (i=1;i<=n;i++) if (r[q][i]>eps&&(!p||row[i]>row[p])) p=i;
			if (!p) break;
			pivot(q,p);
		}
		return 0;
	}
	bool simplex()
	{
		while (1)
		{
			int t=1,k=0,i;
			for (i=2;i<=n;i++) if (r[0][i]<r[0][t]) t=i;
			if (r[0][t]>=-eps) return 1;
			db ratio_min=inf;
			for (i=1;i<=m;i++) if (r[i][t]<-eps)
			{
				db ratio=-r[i][0]/r[i][t];
				if (!k||ratio<ratio_min||ratio<=ratio_min+eps&&row[i]>row[k])
				{
					ratio_min=ratio;
					k=i;
				}
			}
			if (!k) break;
			pivot(k,t);
		}
		return 0;
	}
	void solve(int type)
	{
		if (!init())
		{
			cout<<"Infeasible\n";
			return;
		}
		if (!simplex())
		{
			cout<<"Unbounded\n";
			return;
		}
		cout<<(long double)(-r[0][0])<<'\n';
		if (type)
		{
			int i;
			memset(col+1,0,n*sizeof col[0]);
			for (i=n+1;i<=n+m;i++) col[row[i]]=i;
			for (i=1;i<=n;i++) cout<<(long double)(col[i]?r[col[i]-n][0]:0)<<" \n"[i==n];
		}
	}
};
\end{lstlisting}


\subsection{斐波那契数列}

\begin{lstlisting}
const int NN=3e7+2,M=4e5,N=1e6+10;
char c[NN];
ll n;
ll y,mo,x,z;
int p,i,j,k;
struct Q
{
	int a[2][2];
	Q(int b=0,int c=0,int d=0,int e=0){a[0][0]=b,a[0][1]=c,a[1][0]=d,a[1][1]=e;}
	Q operator*(const Q &o)
	{
		return Q(((ll)a[0][0]*o.a[0][0]+(ll)a[0][1]*o.a[1][0])%p,
				((ll)a[0][0]*o.a[0][1]+(ll)a[0][1]*o.a[1][1])%p,
				((ll)a[1][0]*o.a[0][0]+(ll)a[1][1]*o.a[1][0])%p,
				((ll)a[1][0]*o.a[0][1]+(ll)a[1][1]*o.a[1][1])%p);
	}
};
struct ht
{
	ll v[N],a[N];
	int fir[N],nxt[N],st[N];//和模数相适应
	int tp,p,ds;//自定义模数
	ht(){tp=0,p=1e6+7,ds=0;}
	void mdf(const ll x,const ll z)//位置,值
	{
		const int y=x%p;
		for (int i=fir[y];i;i=nxt[i]) if (v[i]==x) return a[i]=z,void();//若不可能重复不需要这一步if,但需要for?
		v[++ds]=x;a[ds]=z;if (!fir[y]) st[++tp]=y;
		nxt[ds]=fir[y];fir[y]=ds;
	}
	ll find(const ll x)
	{
		const int y=x%p;int i;
		for (i=fir[y];i;i=nxt[i]) if (v[i]==x) break;
		if (!i) return 0;//返回值和是否判断依据要求决定
		return a[i];
	}
	void clear()
	{
		++tp;
		while (--tp) fir[st[tp]]=0;ds=0;
	}
};
ht mp;
Q f[M],g[M],ji;
int fib(ll n)
{
	Q x=f[n%k]*g[n/k];
	return x.a[0][1];
}
ll spefib(ll n)
{
	Q x=f[n%k]*g[n/k];
	return (ll)x.a[0][1]*p+x.a[1][1];
}
ll sj()
{
	ll x=rand();
	x=x<<15^rand();
	x=x<<15^rand();
	x=x<<15^rand();
	return x>0?x:-x;
}
ll ab(ll x)
{
	return x>0?x:-x;
}
int main()
{
	srand(383778817);
	scanf("%s\n%d",c+1,&p);
	k=sqrt((ll)20*p)+1;ji=Q(0,1,1,1);
	f[0]=Q(1,0,0,1);for (i=1;i<=k;i++) f[i]=f[i-1]*ji;
	g[0]=Q(1,0,0,1);for (i=1;i<=k;i++) g[i]=g[i-1]*f[k];
	while (1)
	{
		x=sj()%(20ll*p)+1;y=spefib(x);
		if (z=mp.find(y))
		{
			if (z!=x)
			{
				mo=ab(x-z);
				break;
			}
		} else mp.mdf(y,x);
	}
	n=0;
	for (i=1;c[i]>=48&&c[i]<=57;i++) n=(n*10+(c[i]^48))%mo;
	printf("%d",fib(n));
}
\end{lstlisting}

\subsection{线性插值($k$ 次幂和)}

$O(m)$,$O(m)$。

\begin{lstlisting}
int f(int *a,int n,int m)//这种写法不包含0处取值,n是值,m-1是次数,至少需要 m 项
{
	if (n<=m) return a[n];
	static int inv[N],l[N],r[N],ifac[N]; 
	int i;
	ifac[0]=inv[1]=1;
	for (i=2;i<=m;i++) inv[i]=p-(ll)p/i*inv[p%i]%p;
	for (i=1;i<=m;i++) ifac[i]=(ll)ifac[i-1]*inv[i]%p;//以上可以预跑
	int ans=0,rr=0;
	l[0]=1;r[m+1]=1;
	for (i=1;i<m;i++) l[i]=(ll)l[i-1]*(n-i)%p;
	for (i=m;i;i--) r[i]=(ll)r[i+1]*(n-i)%p;
	for (i=1;i<=m;i++)
	{
		if ((m^i)&1) rr=p-a[i]; else rr=a[i];
		ans=(ans+(ll)rr*ifac[i-1]%p*ifac[m-i]%p*l[i-1]%p*r[i+1])%p;
	}
	return ans;
}
\end{lstlisting}

\subsection{单原根(仅手动验证质数)}

\begin{lstlisting}
namespace get_root
{
	typedef unsigned int ui;
	typedef unsigned long long ll;
	ui ksm(ui x,ui y,ui p)
	{
		ui r=1;
		while (y)
		{
			if (y&1) r=(ll)r*x%p;
			x=(ll)x*x%p;y>>=1;
		}
		return r;
	}
	vector<ui> getw(ui n)
	{
		vector<ui> w;
		for (ui i=2;i*i<=n;i++) if (n%i==0)
		{
			w.push_back(i);
			n/=i;
			for (ui j=n/i;n==i*j;j=n/i) n/=i;
		}
		if (n>1) w.push_back(n);
		return w;
	}
	int getrt(ui n)
	{
		if (n<=2) return n-1;
		auto w=getw(n);
		ui ph=n;
		for (ui x:w) ph=ph/x*(x-1);
		w=getw(ph);
		for (ui &x:w) x=ph/x;
		for (ui i=2;i<n;i++) if (gcd(i,n)==1)
		{
			for (ui x:w) if (ksm(i,x,n)==1) goto no;
			return i;
			no:;
		}
		return -1;
	}
}
using get_root::getrt;
\end{lstlisting}

\subsection{稍快单原根(仅验证质数)}

\begin{lstlisting}
namespace get_root
{
	typedef unsigned int ui;
	typedef unsigned long long ll;
	bool ied=0;
	const int N=1e5+5;
	vector<ui> pr;
	bool ed[N];
	void init()
	{
		pr.reserve(N);
		for (ui i=2;i<N;i++)
		{
			if (!ed[i]) pr.push_back(i);
			for (ui x:pr)
			{
				if (i*x>=N) break;
				ed[i*x]=1;
				if (i%x==0) break;
			}
		}
	}
	ui ksm(ui x,ui y,ui p)
	{
		ui r=1;
		while (y)
		{
			if (y&1) r=(ll)r*x%p;
			x=(ll)x*x%p;y>>=1;
		}
		return r;
	}
	vector<ui> getw(ui n)
	{
		vector<ui> w;
		for (ui x:pr)
		{
			if (x*x>n) break;
			if (n%x==0)
			{
				w.push_back(x);
				n/=x;
				for (ui i=n/x;n==x*i;i=n/x) n/=x;
			}
		}
		if (n>1) w.push_back(n);
		return w;
	}
	int getrt(ui n)
	{
		if (n<=2) return n-1;
		if (!ed[4]) init();
		auto w=getw(n);
		ui ph=n;
		for (ui x:w) ph=ph/x*(x-1);
		w=getw(ph);
		for (ui &x:w) x=ph/x;
		for (ui i=2;i<n;i++) if (gcd(i,n)==1)
		{
			for (ui x:w) if (ksm(i,x,n)==1) goto no;
			return i;
			no:;
		}
		return -1;
	}
}
using get_root::getrt;
\end{lstlisting}

\subsection{筛全部原根}

\begin{lstlisting}
#include <bits/stdc++.h>
using namespace std;
typedef long long ll;
const int N=1e6+2;
int ss[N],mn[N],fmn[N],phi[N];
int t,n,gs,i,d;
bool ed[N],av[N],yg[N],hv[N];
double inv[N];
void getfac(int x,int *a,int &n)
{
	int y=x,z;
	if (1^x&1)
	{
		a[n=1]=2;x>>=1;while (1^x&1) x>>=1;
	}
	while (x>1)
	{
		x=1e-9+(x*inv[a[++n]=z=mn[x]]);
		while (x%z==0) x=1e-9+x*inv[z];
	}
	for (i=1;i<=n;i++) av[a[i]]=0,a[i]=1e-9+(y*inv[a[i]]);
}
int ksm(int x,int y,int p)
{
	int r=1;
	while (y)
	{
		if (y&1) r=(ll)r*x%p;
		x=(ll)x*x%p;y>>=1;
	}
	return r;
}
bool ck(int x,int *a,int n,int p)
{
	for (int i=1;i<=n;i++) if (ksm(x,a[i],p)==1) return 0;
	return 1;
} 
void getrt(int x,int d)
{
	if (!hv[x]) return puts("0\n"),void();
	static int a[30];
	int n=0,y,i,g=0,c=d;y=phi[x];
	fill(av+1,av+y+1,1);
	getfac(y,a,n);
	for (i=1;i<x;i++) if (__gcd(i,x)==1&&ck(i,a,n,x)) break;
	yg[g=i]=1;//g就是最小原根
	int j=(ll)g*g%x;
	for (i=2;i<y;i++,j=(ll)j*g%x) yg[j]=av[i]=av[mn[i]]&av[fmn[i]]; 
	printf("%d\n",phi[y]);
	for (i=1;i<x;i++) if (yg[i]) 
	{
		yg[i]=0;
		if (--c==0) printf("%d ",i),c=d;
	}puts("");
}
void init()
{
	int i,j,k,n=N-1;
	mn[1]=phi[1]=1;
	for (i=1;i<=n;i++) inv[i]=1.0/i;
	for (i=2;i<=n;i++)
	{
		if (!ed[i]) phi[mn[i]=ss[++gs]=i]=i-1,hv[i]=1;
		for (j=1;j<=gs&&(k=ss[j]*i)<=n;j++)
		{
			ed[k]=1;mn[k]=ss[j];
			if (i%ss[j]==0) {phi[k]=phi[i]*ss[j];hv[k]=hv[i];break;}
			phi[k]=phi[i]*(ss[j]-1);
		}
	}
	for (i=n;i;i--) fmn[i]=1e-9+(i*inv[mn[i]]),hv[i]|=(1^i&1)&&hv[i>>1];
	for (i=8;i<=n;i<<=1) hv[i]=0;
}
int main()
{
	init();
	scanf("%d",&t);
	while (t--)
	{
		scanf("%d%d",&n,&d);
		getrt(n,d);
	}
}
\end{lstlisting}



\subsection{圆上整点}

\begin{lstlisting}
	while (n&1^1) n>>=1;//n是半径
	for (i=3;i<=sqrt(n);i++) if (n%i==0)
	{
		if (i%4==3)
		{
			while (n%i==0) n/=i;
		}
		else
		{
			k=0;
			while (n%i==0)
			{
				n/=i;++k;
			}
			ans*=k<<1|1;
		}
	}
	if ((n>1)&&(n%4==1)) ans*=3;
	printf("%d",ans<<2);
\end{lstlisting}

\subsection{高斯消元(通解)}
\begin{lstlisting}
tuple<int,vector<ui>,vector<vector<ui>>> gauss(vector<vector<ui>> a)//sum = a[i][m], rank of base, one sol, base
{
	int n=a.size(),m=a[0].size()-1,i,j,k,R=m;
	vector<int> fix(m,-1);
	for (i=k=0;i<m;i++)
	{
		for (j=k;j<n;j++) if (a[j][i]) break;
		if (j==n) continue;
		fix[i]=k;--R;
		swap(a[k],a[j]);
		ui *u=a[k].data();
		ui x=ksm(u[i],p-2);
		for (j=i;j<=m;j++) u[j]=(ll)u[j]*x%p;
		for (auto &v:a) if (v.data()!=a[k].data())
		{
			x=p-v[i];
			for (j=i;j<=m;j++) v[j]=(v[j]+(ll)x*u[j])%p;
		}
		++k;
	}
	for (i=k;i<n;i++) if (a[i][m]) return {-1,{},{}};
	vector<ui> r(m);
	vector<vector<ui>> c;
	for (i=0;i<m;i++) if (fix[i]!=-1) r[i]=a[fix[i]][m];
	for (i=0;i<m;i++) if (fix[i]==-1)
	{
		vector<ui> r(m);
		r[i]=1;
		for (j=0;j<m;j++) if (fix[j]!=-1) r[j]=(p-a[fix[j]][i])%p;
		c.push_back(r);
	}
	return {R,r,c};
}
\end{lstlisting}

\subsection{高斯消元(列主元)}

$O(n^3)$,$O(n^2)$。

\begin{lstlisting}
namespace Gauss
{
	typedef double db;
	const db eps=1e-8;
	template<typename T> pair<vector<db>,int> solve(const vector<vector<T>> &A)//和为 0。返回秩,负数无解
	{
		assert(A.size());
		int n=A.size(),m=A[0].size()-1,i,j,k,l,r,fg=1;
		db a[n][m+1],b;
		for (i=0;i<n;i++) for (j=0;j<=m;j++) a[i][j]=A[i][j];
		for (i=l=r=0;i<n&&l<m;i++,l++)
		{
			k=i;
			for (j=i+1;j<n;j++) if (fabs(a[j][l])>fabs(a[k][l])) k=j;
			if (fabs(a[k][l])<eps) {--i;continue;}
			if (i!=k) for (j=l;j<=m;j++) swap(a[i][j],a[k][j]);
			b=1/a[i][l];++r;a[i][l]=1;
			for (j=l+1;j<=m;j++) a[i][j]*=b;
			for (j=0;j<n;j++) if (i!=j)
			{
				b=a[j][l];a[j][l]=0;
				for (k=l+1;k<=m;k++) a[j][k]-=b*a[i][k];
			}
		}
		vector<db> X(m);
		for (j=0;j<l;j++) for (k=0;k<i;k++) if (a[k][j]==1)
		{
			X[j]=-a[k][m];
			break;
		}
		for (j=i;j<n&&~fg;j++)
		{
			b=a[j][m];
			for (k=0;k<m;k++) b+=X[k]*a[j][k];
			if (fabs(b)>eps) fg=-1;
		}
		return {X,r*fg};
	}
}
\end{lstlisting}

\subsection{行列式求值(任意模数)}

$O(n^3)$,$O(n^2)$。

\begin{lstlisting}
#include <bits/stdc++.h>
using namespace std;
typedef long long ll;
const int N=502,p=998244353;
int cal(int a[][N],int n)
{
	int i,j,k,r=1,fh=0,l;
	for (i=1;i<=n;i++)
	{
		k=i;
		for (j=i+1;j<=n;j++) if (a[j][i]) {k=j;break;}
		if (a[k][i]==0) return 0;
		if (i!=k) {swap(a[k],a[i]);fh^=1;}
		for (j=i+1;j<=n;j++)
		{
			if (a[j][i]>a[i][i]) swap(a[j],a[i]),fh^=1;
			while (a[j][i])
			{
				l=a[i][i]/a[j][i];
				for (k=i;k<=n;k++) a[i][k]=(a[i][k]+(ll)(p-l)*a[j][k])%p;
				swap(a[j],a[i]);fh^=1;
			}
		}
		r=(ll)r*a[i][i]%p;
	}
	if (fh) return (p-r)%p;
	return r;
}
int main()
{
	ios::sync_with_stdio(0);cin.tie(0);
	int n,i,j;
	static int a[N][N];
	cin>>n;
	for (i=1;i<=n;i++) for (j=1;j<=n;j++) cin>>a[i][j];
	cout<<cal(a,n)<<endl;
}
\end{lstlisting}

\subsection{行列式求值(质数模数)}

$O(n^3)$,$O(n^2)$。

\begin{lstlisting}
#include <bits/stdc++.h>
using namespace std;
typedef long long ll;
const int N=502,p=998244353;
int ksm(int x,int y)
{
	int r=1;
	while (y)
	{
		if (y&1) r=(ll)r*x%p;
		y>>=1;x=(ll)x*x%p;
	}
	return r;
}
int cal(int a[][N],int n)
{
	int i,j,k,r=1,fh=0,l;
	for (i=1;i<=n;i++)
	{
		for (j=i;j<=n;j++) if (a[j][i]) break;
		if (j>n) return 0;
		if (i!=j) swap(a[j],a[i]),fh^=1;
		r=(ll)r*a[i][i]%p;
		k=ksm(a[i][i],p-2);
		for (j=i;j<=n;j++) a[i][j]=(ll)a[i][j]*k%p;
		for (j=i+1;j<=n;j++)
		{
			a[j][i]=p-a[j][i];
			for (k=i+1;k<=n;k++) a[j][k]=(a[j][k]+(ll)a[j][i]*a[i][k])%p;
			a[j][i]=0;
		}
	}
	if (fh) return (p-r)%p;
	return r;
}
int main()
{
	ios::sync_with_stdio(0);cin.tie(0);
	int n,i,j;
	static int a[N][N];
	cin>>n;
	for (i=1;i<=n;i++) for (j=1;j<=n;j++) cin>>a[i][j];
	cout<<cal(a,n)<<endl;
}
/*
3
3 1 4
1 5 9
2 6 5
998244263
*/
\end{lstlisting}

\subsection{稀疏矩阵系列}

\begin{lstlisting}
vector<ui> bm(const vector<ui> &a)
{
	vector<ui> r,lst;
	int n=a.size(),m=0,q=0,i,j,k=-1;
	ui D=0;
	for (i=0;i<n;i++)
	{
		ui cur=0;
		for (j=0;j<m;j++) cur=(cur+(ll)a[i-j-1]*r[j])%p;
		cur=(a[i]+p-cur)%p;
		if (!cur) continue;
		if (k==-1)
		{
			k=i;
			D=cur;
			r.resize(m=i+1);
			continue;
		}
		auto v=r;
		ui x=(ll)cur*ksm(D,p-2)%p;
		if (m<q+i-k) r.resize(m=q+i-k);
		(r[i-k-1]+=x)%=p;
		ui *b=r.data()+i-k;
		x=(p-x)%p;
		for (j=0;j<q;j++) b[j]=(b[j]+(ll)x*lst[j])%p;
		if (v.size()+k<lst.size()+i)
		{
			lst=v;
			q=v.size();
			k=i;
			D=cur;
		}
	}
	return r;
}
#define safe
struct Q
{
	int x,y;
	ui w;
};
mt19937_64 rnd(9980);
vector<ui> minpoly(int n,const vector<Q> &a)//[0,n),max:1
{
	for (auto [x,y,w]:a) assert(min(x,y)>=0&&max(x,y)<n);
	vector<ui> u(n),v(n),b(n*2+1),tmp(n);
	int i;
	for (ui &x:u) x=rnd()%p;
	for (ui &x:v) x=rnd()%p;
	assert(*min_element(all(u))&&*min_element(all(v)));
	for (ui &r:b)
	{
		for (i=0;i<n;i++) r=(r+(ll)u[i]*v[i])%p;
		fill(all(tmp),0);
		for (auto [x,y,w]:a) tmp[x]=(tmp[x]+(ll)w*v[y])%p;
		swap(v,tmp);
	}
	auto r=bm(b);
	#ifdef safe
		for (ui &x:u) x=rnd()%p;
		for (ui &x:v) x=rnd()%p;
		for (ui &r:b)
		{
			for (i=0;i<n;i++) r=(r+(ll)u[i]*v[i])%p;
			fill(all(tmp),0);
			for (auto [x,y,w]:a) tmp[x]=(tmp[x]+(ll)w*v[y])%p;
			swap(v,tmp);
		}
		auto rr=bm(b);
		assert(r==rr);
	#endif
	reverse(all(r));
	for (ui &x:r) if (x) x=p-x;
	r.push_back(1);
	return r;
}
ui det(int n,vector<Q> a)//[0,m)
{
	vector<ui> b(n);
	for (ui &x:b) x=rnd()%p;
	assert(*min_element(all(b)));
	for (auto &[x,y,w]:a) w=(ll)w*b[x]%p;
	ui r=minpoly(n,a)[0],tmp=1;
	for (ui x:b) tmp=(ll)tmp*x%p;
	r=(ll)r*ksm(tmp,p-2)%p;
	#ifdef safe
		for (ui &x:b) x=rnd()%p;
		assert(*min_element(all(b)));
		for (auto &[x,y,w]:a) w=(ll)w*b[x]%p;
		ui rr=minpoly(n,a)[0],tmpp=1;
		for (ui x:b) tmpp=(ll)tmpp*x%p;
		rr=(ll)rr*ksm(tmpp,p-2)%p*ksm(tmp,p-2)%p;
		assert(r==rr);
	#endif
	return n&1?(p-r)%p:r;
}
vector<ui> gauss(const vector<Q> &a,vector<ui> v)
{
	int n=v.size(),i,j;
	for (auto [x,y,w]:a) assert(0<=x&&x<n&&0<=y&&y<n);
	vector<ui> u(n),b(2*n+1),tmp(n),tv=v;
	for (ui &x:u) x=rnd()%p;
	assert(*min_element(all(u)));
	for (ui &r:b)
	{
		for (i=0;i<n;i++) r=(r+(ll)u[i]*v[i])%p;
		fill(all(tmp),0);
		for (auto [x,y,w]:a) tmp[x]=(tmp[x]+(ll)w*v[y])%p;
		swap(v,tmp);
	}
	auto f=bm(b);
	f.insert(f.begin(),p-1);
	int m=(int)f.size()-2;
	v=tv;fill(all(u),0);
	ui x;
	for (i=0;i<=m;i++)
	{
		x=f[m-i];
		for (j=0;j<n;j++) u[j]=(u[j]+(ll)v[j]*x)%p;
		fill(all(tmp),0);
		for (auto [x,y,w]:a) tmp[x]=(tmp[x]+(ll)w*v[y])%p;
		swap(v,tmp);
	}
	x=ksm((p-f.back())%p,p-2);
	for (ui &y:u) y=(ll)y*x%p;
	#ifdef safe
		for (auto [x,y,w]:a) tv[x]=(tv[x]+(ll)(p-w)*u[y])%p;
		assert(!*min_element(all(tv)));
	#endif
	return u;
}
\end{lstlisting}



\subsection{Min\_25筛}

$f(p^k)=p^k(p^k-1)$,求 $\sum\limits_{i=1}^n f(i)$。

\begin{lstlisting}
const int N=1e5+2,p=1e9+7,i6=166666668;
ll fs[N<<1],m;
int ss[N],ys[N<<1],s[N],f[N<<1],g[N<<1],ls[N<<1],cs[N<<1];
int gs,n,i,j,k,cnt,ct,ans,sq;
bool ed[N];
int S(ll n,int x)
{
	int r,i,j,l;
	ll k;
	if (ss[x]>=n) return 0;
	if (n>sq) r=g[ys[m/n]]; else r=g[n];
	if ((r=r-s[x])<0) r+=p;
	for (i=x+1;(ll)ss[i]*ss[i]<=n;i++) for (j=1,k=ss[i];k<=n;j++,k*=ss[i])
	{
		l=(k-1)%p;
		r=(r+(ll)l*(l+1)%p*((j!=1)+S(n/k,i)))%p;
	}
	return r;
}
int main()
{
	n=1e5;
	for (i=2;i<=n;i++)
	{
		if (!ed[i]) ss[++gs]=i;
		for (j=1;(j<=gs)&&(i*ss[j]<=n);j++)
		{
			ed[i*ss[j]]=1;
			if (i%ss[j]==0) break;
		}
	}ss[gs+1]=1e6;
	s[1]=ss[1]*ss[1];
	for (i=2;i<=gs;i++) s[i]=(s[i-1]+(ll)ss[i]*ss[i])%p;//s 是多项式在素数位置的前缀和
	memcpy(cs,s,sizeof(s));
	ll i,j,k,x,z; scanf("%lld",&m);
	sq=n=sqrt(m);while ((ll)(n+1)*(n+1)<=m) ++n;
	cnt=n-1;
	for (i=n;i<=m;i=j+1) {j=m/(m/i);++cnt;}ct=cnt++;
	for (i=1;i<=m;i=j+1)
	{
		j=m/(k=m/i);
		if (k<=n) g[fs[k]=k]=(k*(k+1)*(k<<1|1)/6-1)%p;//这里是多项式前缀和(不含1)
		else
		{
			z=k%p;//一样
			g[ys[j]=--cnt]=(z*(z+1)%p*(z<<1|1)%p+p-6)*i6%p;fs[cnt]=k;
		}
	}
	cnt=ct;
	for (j=1;(j<=gs)&&(z=(ll)ss[j]*ss[j]);j++) for (i=cnt;z<=fs[i];i--)
	{
		x=fs[i]/ss[j];if (x>n) x=ys[m/x];
		g[i]=(g[i]+(ll)(p-ss[j])*ss[j]%p*(g[x]-s[j-1]+p))%p;//另一处需要修改的
	}
	memcpy(ls,g,sizeof(g));
	s[1]=ss[1];
	for (i=2;i<=gs;i++) s[i]=s[i-1]+ss[i];
	cnt=n-1;
	for (i=n;i<=m;i=j+1) {j=m/(m/i);++cnt;}ct=cnt++;
	for (i=1;i<=m;i=j+1)
	{
		j=m/(k=m/i);
		if (k<=n) g[fs[k]=k]=((k*(k+1)>>1)-1)%p;
		else
		{
			z=k%p;
			g[ys[j]=--cnt]=(z*(z+1)-2>>1)%p;fs[cnt]=k;
		}
	}
	cnt=ct;
	for (j=1;(j<=gs)&&(z=(ll)ss[j]*ss[j]);j++) for (i=cnt;z<=fs[i];i--)
	{
		x=fs[i]/ss[j];if (x>n) x=ys[m/x];
		g[i]=(g[i]+(ll)(p-ss[j])*(g[x]-s[j-1]+p))%p;
	}
	for (i=1;i<=cnt;i++) if ((g[i]=ls[i]-g[i])<0) g[i]+=p;
	for (i=1;i<=gs;i++) if ((s[i]=cs[i]-s[i])<0) s[i]+=p;
	ans=S(m,0)+1;if (ans==p) ans=0;printf("%d",ans);
}
\end{lstlisting}

\subsection{Min\_25筛(卡常,素数个数,注意评测机 double 性能)}

\begin{lstlisting}
#include <bits/stdc++.h>
using namespace std;
typedef long long ll;
const int N=3.2e5+2;
ll s[N];
int ss[N],ys[N],gs=0;
bool ed[N];
ll cal(ll m)
{
	static ll g[N<<1],fs[N<<1];
	ll i,j,k,x;
	int n;
	int p,q,cnt;
	n=round(sqrt(m));
	q=lower_bound(ss+1,ss+gs+1,n)-ss;
	memset(g,0,sizeof(g));memset(ys,0,sizeof(ys));cnt=n-1;
	for (i=n;i<=m;i=j+1) {j=m/(m/i);++cnt;}int ct=cnt++;
	for (i=1;i<=m;i=j+1)
	{
		j=m/(k=m/i);
		if (k<=n) g[fs[k]=k]=k-1; else {g[ys[j]=--cnt]=k-1;fs[cnt]=k;}
	}cnt=ct;
	for (j=1;j<=q;j++) for (i=cnt;(ll)ss[j]*ss[j]<=fs[i];i--)
	{
		x=fs[i]/ss[j];if (x>n) x=ys[m/x];
		g[i]-=g[x]-j+1;
	}
	return g[cnt];//这里 g[cnt-i+1] 表示的是 [1,m/i] 的答案
}
int main()
{
	int n,i,j,t;
	n=3.2e5;
	for (i=2;i<=n;i++)
	{
		if (!ed[i]) ss[++gs]=i;
		for (j=1;(j<=gs)&&(i*ss[j]<=n);j++)
		{
			ed[i*ss[j]]=1;
			if (i%ss[j]==0) break;
		}
	}
	s[1]=ss[1];
	for (i=2;i<=gs;i++) s[i]=s[i-1]+ss[i];
	t=1;
	ll m;
	while (t--) cin>>m,cout<<cal(m)<<'\n';
}
\end{lstlisting}

\subsection{扩展 min-max 容斥(重返现世)}

$k\text{-th}\max\{S\}=\sum\limits_{T\subseteq S}(-1)^{|T|-k}\tbinom{|T|-1}{k-1}\min\{T\}$

\begin{lstlisting}
	scanf("%d%d%d",&n,&q,&m);inv[1]=1;q=n+1-q;
	for (i=2;i<=m;i++) inv[i]=p-(ll)p/i*inv[p%i]%p;
	for (i=1;i<=n;i++) scanf("%d",a+i);f[0][0]=1;
	for (j=1;j<=n;j++) for (i=q;i;i--) for (k=m;k>=a[j];k--) if ((f[i][k]=f[i][k]+f[i-1][k-a[j]]-f[i][k-a[j]])>=p) f[i][k]-=p; else if (f[i][k]<0) f[i][k]+=p;
	for (i=1;i<=m;i++) ans=(ans+(ll)f[q][i]*inv[i])%p;
	ans=(ll)ans*m%p;printf("%d",ans);
\end{lstlisting}

\subsection{模数为偶数 FWT \& 光速乘}

$O(n2^n)$,$O(2^n)$。

\begin{lstlisting}
const int N=1<<20,M=21;
int x[M];
ll p,f[N],g[N];
int n,m,c;
ll mul(ll x,ll y)
{
	x=x*y-(ll)((ldb)x/p*y+1e-8)*p;
	if (x<0) return x+p;return x;
}
void read(int &x)
{
	c=getchar();
	while ((c<48)||(c>57)) c=getchar();
	x=c^48;c=getchar();
	while ((c>=48)&&(c<=57))
	{
		x=x*10+(c^48);
		c=getchar();
	}
}
void dft(ll *a)
{
	int i,j,k,l;
	ll b;
	for (i=1;i<n;i=l)
	{
		l=i<<1;
		for (j=0;j<n;j+=l) for (k=0;k<i;k++)
		{
			b=a[j|k|i];
			a[j|k|i]=(a[j|k]-b+p)%p;
			a[j|k]=(a[j|k]+b)%p;
		}
	}
}
int main()
{
    ios::sync_with_stdio(0);cin.tie(0);
	ll t;int i;
	cin>>m>>t>>p;p*=(n=1<<m);
	for (i=0;i<n;i++) cin>>f[i];
	dft(f);
	for (i=0;i<=m;i++) cin>>x[i];
	for (i=1;i<n;i++) g[i]=g[i>>1]+(i&1);
	for (i=0;i<n;i++) g[i]=x[g[i]];dft(g);
	while (t)
	{
		if (t&1) for (i=0;i<n;i++) f[i]=mul(f[i],g[i]);
		for (i=0;i<n;i++) g[i]=mul(g[i],g[i]);t>>=1;
	}
	dft(f);
	for (i=0;i<n;i++) cout<<(f[i]>>m)<<'\n';
}
\end{lstlisting}

\subsection{二次剩余}

\begin{lstlisting}
namespace cipolla
{
	typedef unsigned int ui;
	typedef unsigned long long ll;
	ui p,w;
	struct Q
	{
		ll x,y;
		Q operator*(const Q &o) const {return {(x*o.x+y*o.y%p*w)%p,(x*o.y+y*o.x)%p};}
	};
	ui ksm(ll x,int y)
	{
		ll r=1;
		while (y)
		{
			if (y&1) r=r*x%p;
			x=x*x%p;y>>=1;
		}
		return r;
	}
	Q ksm(Q x,int y)
	{
		Q r={1,0};
		while (y)
		{
			if (y&1) r=r*x;
			x=x*x;y>>=1;
		}
		return r;
	}
	int mosqrt(ui x,ui P)//0<=x<P
	{
		if (x==0||P==2) return x;
		p=P;
		if (ksm(x,p-1>>1)!=1) return -1;
		ui y;
		mt19937 rnd(chrono::steady_clock::now().time_since_epoch().count());
		do y=rnd()%p,w=((ll)y*y+p-x)%p; while (ksm(w,p-1>>1)<=1);//not for p=2
		y=ksm({y,1},p+1>>1).x;
		if (y*2>p) y=p-y;//两解取小
		return y;
	}
}
using cipolla::mosqrt;
\end{lstlisting}

\subsection{$k$ 次剩余}
\begin{lstlisting}
namespace get_root
{
	typedef unsigned int ui;
	typedef unsigned long long ll;
	bool ied=0;
	const int N=1e5+5;
	vector<ui> pr;
	bool ed[N];
	void init()
	{
		pr.reserve(N);
		for (ui i=2;i<N;i++)
		{
			if (!ed[i]) pr.push_back(i);
			for (ui x:pr)
			{
				if (i*x>=N) break;
				ed[i*x]=1;
				if (i%x==0) break;
			}
		}
	}
	ui ksm(ui x,ui y,ui p)
	{
		ui r=1;
		while (y)
		{
			if (y&1) r=(ll)r*x%p;
			x=(ll)x*x%p;y>>=1;
		}
		return r;
	}
	vector<ui> getw(ui n)
	{
		vector<ui> w;
		for (ui x:pr)
		{
			if (x*x>n) break;
			if (n%x==0)
			{
				w.push_back(x);
				n/=x;
				for (ui i=n/x;n==x*i;i=n/x) n/=x;
			}
		}
		if (n>1) w.push_back(n);
		return w;
	}
	int getrt(ui n)
	{
		if (n<=2) return n-1;
		if (!ed[4]) init();
		auto w=getw(n);
		ui ph=n;
		for (ui x:w) ph=ph/x*(x-1);
		w=getw(ph);
		for (ui &x:w) x=ph/x;
		for (ui i=2;i<n;i++) if (gcd(i,n)==1)
		{
			for (ui x:w) if (ksm(i,x,n)==1) goto no;
			return i;
			no:;
		}
		return -1;
	}
}
namespace BSGS
{
	typedef unsigned int ui;
	typedef unsigned long long ll;
	template<int N,typename T,typename TT> struct ht//个数,定义域,值域
	{
		const static int p=1e6+7,M=p+2;
		TT a[N];
		T v[N];
		int fir[p+2],nxt[N],st[p+2];//和模数相适应
		int tp,ds;//自定义模数
		ht(){memset(fir,0,sizeof fir);tp=ds=0;}
		void mdf(T x,TT z)//位置,值
		{
			ui y=x%p;
			for (int i=fir[y];i;i=nxt[i]) if (v[i]==x) return a[i]=z,void();//若不可能重复不需要 for
			v[++ds]=x;a[ds]=z;
			if (!fir[y]) st[++tp]=y;
			nxt[ds]=fir[y];fir[y]=ds;
		}
		TT find(T x)
		{
			ui y=x%p;
			int i;
			for (i=fir[y];i;i=nxt[i]) if (v[i]==x) return a[i];
			return 0;//返回值和是否判断依据要求决定
		}
		void clear()
		{
			++tp;
			while (--tp) fir[st[tp]]=0;
			ds=0;
		}
	};
	const int N=5e4;
	ht<N,ui,ui> s;
	int exgcd(int a,int b)
	{
		if (a==1) return 1;
		return (1-(long long)b*exgcd(b%a,a))/a;//not ll
	}
	int bsgs(ui a,ui b,ui p)
	{
		s.clear();
		a%=p;b%=p;
		if (!a) return 1-min((int)b,2);//含 -1
		ui i,j,k,x,y;
		x=sqrt(p)+2;
		for (i=0,j=1;i<x;i++,j=(ll)j*a%p)
		{
			if (j==b) return i;
			s.mdf((ll)j*b%p,i+1);
		}
		k=j;
		for (i=1;i<=x;i++,j=(ll)j*k%p) if (y=s.find(j)) return (ll)i*x-y+1;
		return -1;
	}
	bool isprime(ui p)
	{
		if (p<=1) return 0;
		for (ui i=2;i*i<=p;i++) if (p%i==0) return 0;
		return 1;
	}
	int exbsgs(ui a,ui b,ui p)//a^x=b(mod p)
	{
		//if (isprime(p)) return bsgs(a,b,p);
		a%=p;b%=p;
		ui i,j,k,x,y=__lg(p),cnt=0;
		for (i=0,j=1%p;i<=y;i++,j=(ll)j*a%p) if (j==b) return i;
		y=1;
		while (1)
		{
			if ((x=gcd(a,p))==1) break;
			if (b%x) return -1;//no sol
			++cnt;
			p/=x;b/=x;
			y=(ll)y*(a/x)%p;
		}
		a%=p;
		b=(ll)b*(p+exgcd(y,p))%p;
		int r=bsgs(a,b,p);
		return r==-1?-1:r+cnt;
	}
}
pair<ll,ll> exgcd(ll a,ll b,ll c)//ax+by=c,{-1,-1} 无解,b=0 返回 {c/a,0},否则返回最小非负 x
{
	assert(a||b);
	if (!b) return {c/a,0};
	if (a<0) a=-a,b=-b,c=-c;
	ll d=gcd(a,b);
	if (c%d) return {-1,-1};
	ll x=1,x1=0,p=a,q=b,k;
	b=abs(b);
	while (b)
	{
		k=a/b;
		x-=k*x1;a-=k*b;
		swap(x,x1);
		swap(a,b);
	}
	b=abs(q/d);
	x=x*(c/d)%b;
	if (x<0) x+=b;
	return {x,(c-p*x)/q};
}
ll fun(ll a,ll b,ll p)//ax=b(mod p)
{
	return exgcd(-p,a,b).second%p;
}
using get_root::getrt;
using BSGS::bsgs,BSGS::exbsgs;
int nth_root(ui k,ui y,ui p)//x^k=y(mod p)
{
	if (k==0) return y==1?0:-1;
	if (y==0) return 0;
	ui g=getrt(p);
	ui z=bsgs(g,y,p);
	ll x=fun(k,z,p-1);
	if (x==-1) return -1;
	return get_root::ksm(g,x,p);
}
\end{lstlisting}

网上的超快版本

\begin{lstlisting}
#define popcount __builtin_popcount
using namespace std;
typedef long long int ll;
//using ll=__int128_t;
typedef pair<ll, int> P;
ll gcd(ll a, ll b){
	if (b==0) return a;
	return gcd(b, a%b);
}
ll powmod(ll a, ll k, ll mod){
	ll ap=a, ans=1;
	while(k){
		if (k&1){
			ans*=ap;
			ans%=mod;
		}
		ap=ap*ap;
		ap%=mod;
		k>>=1;
	}
	return ans;
}
ll inv(ll a, ll m){
	ll b=m, x=1, y=0;
	while(b>0){
		ll t=a/b;
		swap(a-=t*b, b);
		swap(x-=t*y, y);
	}
	return (x%m+m)%m;
}
vector<P> fac(ll x){
	vector<P> ret;
	for(ll i=2; i*i<=x; i++){
		if (x%i==0){
			int e=0;
			while(x%i==0){
				x/=i;
				e++;
			}
			ret.push_back({i, e});
		}
	}
	if (x>1) ret.push_back({x, 1});
	return ret;
}
//mt19937_64 mt(334);
mt19937 mt(334);
ll solve1(ll p, ll q, int e, ll a){
	int s=0;
	ll r=p-1, qs=1, qp=1;
	while(r%q==0){
		r/=q;
		qs*=q;
		s++;
	}
	for(int i=0; i<e; i++) qp*=q;
	ll d=qp-inv(r%qp, qp);
	ll t=(d*r+1)/qp;
	ll at=powmod(a, t, p), inva=inv(a, p);
	if (e>=s){
		if (powmod(at, qp, p)!=a) return -1;
		else return at;
	}
	//uniform_int_distribution<long long> rnd(1, p-1);
	uniform_int_distribution<> rnd(1, p-1);
	ll rv;
	while(1){
		rv=powmod(rnd(mt), r, p);
		if (powmod(rv, qs/q, p)!=1) break;
	}
	int i=0;
	ll qi=1, sq=1;
	while(sq*sq<q) sq++;
	while(i<s-e){
		ll qq=qs/qp/qi/q;
		vector<P> v(sq);
		ll rvi=powmod(rv, qp*qq*(p-2)%(p-1), p), rvp=powmod(rv, sq*qp*qq, p);
		ll x=powmod(powmod(at, qp, p)*inva%p, qq*(p-2)%(p-1), p), y=1;
		for(int j=0; j<sq; j++){
			v[j]=P(x, j);
			(x*=rvi)%=p;
		}
		sort(v.begin(), v.end());
		ll z=-1;
		for(int j=0; j<sq; j++){
			int l=lower_bound(v.begin(), v.end(), P(y, 0))-v.begin();
			if (v[l].first==y){
				z=v[l].second+j*sq;
				break;
			}
			(y*=rvp)%=p;
		}
		if (z==-1) return -1;
		(at*=powmod(rv, z, p))%=p;
		i++;
		qi*=q;
		rv=powmod(rv, q, p);
	}
	return at;
}
ll solve0(ll p, ll q, ll r, ll a){
	ll d=q-inv(r%q, q);
	ll t=(d*r+1)/q;
	ll at=powmod(a, t, p), inva=inv(a, p);
	if (powmod(at, q, p)!=a) return -1;
	else return at;
}
ll solve(ll p, ll k, ll a)//p k y
{
	if (k==0)
	{
		if (a==1) return 1;
		return -1;
	}
	if (a==0) return 0;
	if (p==2 || a==1) return 1;
	ll a1=a;
	ll g=gcd(p-1, k);
	ll c=inv(k/g%((p-1)/g), (p-1)/g);
	a=powmod(a, c, p);
	if (g==1){
		if (powmod(a, k, p)==a1) return a;
		else return -1;
	}
	ll g1=gcd(g, (p-1)/g), g2=g;
	vector<P> f1=fac(g1), f;
	for(auto r:f1){
		ll q=r.first;
		int e=0;
		while(g2%q==0){
			g2/=q;
			e++;
		}
		f.push_back({q, e});
	}
	ll ret=1, gp=1;
	if (g2>1){
		ll x=solve0(p, g2, (p-1)/g2, a);
		if (x==-1) return -1;
		ret=x, gp*=g2;
	}
	for(auto r:f){
		ll qp=1;
		for(int i=0; i<r.second; i++) qp*=r.first;
		ll x=solve1(p, r.first, r.second, a);
		if (x==-1) return -1;
		if (gp==1){
			ret=x, gp*=qp;
			continue;
		}
		ll s=inv(gp%qp, qp), t=(1-gp*s)/qp;
		if (t>=0) ret=powmod(ret, t, p);
		else ret=powmod(ret, p-1+t%(p-1), p);
		if (s>=0) x=powmod(x, s, p);
		else x=powmod(x, p-1+s%(p-1), p);
		(ret*=x)%=p;
		gp*=qp;
	}
	if (powmod(ret, k, p)!=a1) return -1;
	return ret;
}
\end{lstlisting}

\subsection{FWT/FST}

$O(n2^n)$,$O(2^n)$。

\begin{lstlisting}
void fwt_and(vector<ui> &A)//本质:母集和
{
	ui n=A.size(),*a=A.data(),i,j,k,l,*f,*g;
	for (i=1;i<n;i=l)
	{
		l=i*2;
		for (j=0;j<n;j+=l)
		{
			f=a+j;g=a+j+i;
			for (k=0;k<i;k++) if ((f[k]+=g[k])>=p) f[k]-=p;
		}
	}
}
void ifwt_and(vector<ui> &A)
{
	ui n=A.size(),*a=A.data(),i,j,k,l,*f,*g;
	for (i=1;i<n;i=l)
	{
		l=i*2;
		for (j=0;j<n;j+=l)
		{
			f=a+j;g=a+j+i;
			for (k=0;k<i;k++) if ((f[k]-=g[k])>=p) f[k]+=p;//unsigned
		}
	}
}
void fwt_or(vector<ui> &A)//本质:子集和
{
	ui n=A.size(),*a=A.data(),i,j,k,l,*f,*g;
	for (i=1;i<n;i=l)
	{
		l=i*2;
		for (j=0;j<n;j+=l)
		{
			f=a+j;g=a+j+i;
			for (k=0;k<i;k++) if ((g[k]+=f[k])>=p) g[k]-=p;
		}
	}
}
void ifwt_or(vector<ui> &A)
{
	ui n=A.size(),*a=A.data(),i,j,k,l,*f,*g;
	for (i=1;i<n;i=l)
	{
		l=i*2;
		for (j=0;j<n;j+=l)
		{
			f=a+j;g=a+j+i;
			for (k=0;k<i;k++) if ((g[k]-=f[k])>=p) g[k]+=p;//unsigned
		}
	}
}
void fwt_xor(vector<ui> &A)
{
	ui n=A.size(),*a=A.data(),i,j,k,l,*f,*g;
	for (i=1;i<n;i=l)
	{
		l=i*2;
		for (j=0;j<n;j+=l)
		{
			f=a+j;g=a+j+i;
			for (k=0;k<i;k++)
			{
				if ((f[k]+=g[k])>=p) f[k]-=p;
				g[k]=(f[k]+2*(p-g[k]))%p;
			}
		}
	}
}
void ifwt_xor(vector<ui> &A)
{
	ui n=A.size(),*a=A.data(),i,j,k,l,*f,*g,x=p+1>>1,y=1;
	for (i=1;i<n;i=l)
	{
		l=i*2;
		for (j=0;j<n;j+=l)
		{
			f=a+j;g=a+j+i;
			for (k=0;k<i;k++)
			{
				if ((f[k]+=g[k])>=p) f[k]-=p;
				g[k]=(f[k]+2*(p-g[k]))%p;
			}
		}
		y=(ll)y*x%p;
	}
	for (i=0;i<n;i++) a[i]=(ll)a[i]*y%p;
}
vector<ui> fst(const vector<ui> &s,const vector<ui> &t)
{
	int n=s.size(),m=__builtin_ctz(n),i,j,k;
	vector<ui> a[m+1],b[m+1],c[m+1],r(n);
	for (i=0;i<=m;i++) a[i].resize(n),b[i].resize(n),c[i].resize(n);
	for (i=0;i<n;i++)
	{
		k=__builtin_popcount(i);
		a[k][i]=s[i];
		b[k][i]=t[i];
	}
	for (i=0;i<m;i++) fwt_or(a[i]),fwt_or(b[i]);
	for (i=0;i<=m;i++) for (j=0;j<=i;j++) for (k=0;k<n;k++) c[i][k]=(c[i][k]+(ll)a[j][k]*b[i-j][k])%p;
	for (i=1;i<=m;i++) ifwt_or(c[i]);
	for (i=0;i<n;i++) r[i]=c[__builtin_popcount(i)][i];
	return r;
}
\end{lstlisting}

\subsection{NTT}

\begin{lstlisting}
typedef unsigned long long ll;
namespace NTT//禁止混用三模与普通 NTT
{
	mt19937 rnd(chrono::steady_clock::now().time_since_epoch().count());
/**/const int N=1<<22;//务必修改
// #define MTT
// #define CRT
/**/const ll g=3,I=86'583'718;//g^((p-1)/4)
#ifndef MTT
/**/const ll p=998244353;
/**/ll w[N];
#else
	const ll p=1e9+7;
	const ll p1=469'762'049,p2=998'244'353,p3=1004'535'809;//三模,原根都是 3,非常好
	const ll inv_p1=554'580'198,inv_p12=395'249'030;//三模,1 关于 2 逆,1*2 关于 3 逆,1*2 mod 3
#endif
/**/int r[N];
	ll inv[N],fac[N],ifac[N];//W for mosqrt
/**/ll ksm(ll x,ll y)
/**/{
/**/	ll r=1;
/**/	while (y)
/**/	{
/**/		if (y&1) r=r*x%p;
/**/		x=x*x%p;
/**/		y>>=1;
/**/	}
/**/	return r;
/**/}
	vector<ll> getinvs(vector<ll> a)
	{
		int n=a.size(),i;
		if (n<=2)
		{
			for (i=0; i<n; i++) a[i]=ksm(a[i],p-2);
			return a;
		}
		vector<ll> l(n),r(n);
		l[0]=a[0]; r[n-1]=a[n-1];
		for (i=1; i<n; i++) l[i]=l[i-1]*a[i]%p;
		for (i=n-2; i; i--) r[i]=r[i+1]*a[i]%p;
		ll x=ksm(l[n-1],p-2);
		a[0]=x*r[1]%p; a[n-1]=x*l[n-2]%p;
		for (i=1; i<n-1; i++) a[i]=x*l[i-1]%p*r[i+1]%p;
		return a;
	}
	int mosqrt(ll x)
	{
		static ll W;
		struct P
		{
			ll x,y;
			P operator*(const P &a) const
			{
				return {(x*a.x+y*a.y%p*W)%p,(x*a.y+y*a.x)%p};
			}
		};
		if (x==0) return 0;
		if (ksm(x,p-1>>1)!=1) { cout<<"-1\n"; exit(0); }
		ll y;
		do y=rnd()%p; while (ksm(W=(y*y%p+p-x)%p,p-1>>1)<=1);//not for p=2
		y=[&](P x,ll y)
		{
			P r{1,0};
			while (y)
			{
				if (y&1) r=r*x;
				x=x*x; y>>=1;
			}
			return r.x;
		}({y,1},p+1>>1);
		return y*2<p?y:p-y;
	}
#ifdef MTT
#ifdef CRT
	void init(int);
	template<const ll p> struct M
	{
		ll w[N];
		ll ksm(ll x,ll y)
		{
			ll r=1;
			while (y)
			{
				if (y&1) r=r*x%p;
				x=x*x%p;
				y>>=1;
			}
			return r;
		}
		void init(int n)
		{
			static int pre=0;
			int i,j,k;
			for (j=1; j<n; j=k)
			{
				k=j*2;
				ll wn=ksm(g,(p-1)/k);
				w[j]=1;
				for (i=j+1; i<k; i++) w[i]=w[i-1]*wn%p;
			}
		}
		void dft(vector<ll> &a,int o=0)
		{
			int n=a.size(),i,j,k;
			ll *f,*g,*wn,*A=a.data(),x,y;
			NTT::init(n);
			for (i=1; i<n; i++) if (i<r[i]) swap(A[i],A[r[i]]);
			for (k=1; k<n; k*=2)
			{
				wn=w+k;
				for (i=0; i<n; i+=k*2)
				{
					f=A+i; g=A+i+k;
					for (j=0; j<k; j++)
					{
						y=g[j]*wn[j]%p;
						g[j]=f[j]+p-y;
						f[j]+=y;
					}
				}
				if (k*2==n||k==1<<14) for (ll &x:a) x%=p;
			}
			if (o)
			{
				x=ksm(n,p-2);
				for (ll &y:a) y=y*x%p;
				reverse(1+all(a));
			}
		}
	};
	M<p1> s1;
	M<p2> s2;
	M<p3> s3;
	void init(int n)
	{
		static int pre=0;
		if (pre==n) return;
		int b=__lg(n)-1,i;
		for (i=1; i<n; i++) r[i]=r[i>>1]>>1|(i&1)<<b;
		if (pre<n)
		{
			s1.init(n);
			s2.init(n);
			s3.init(n);
		}
		pre=n;
	}
#endif
#else
/**/void init(int n)
/**/{
/**/	static int pr=0,pw=0;
/**/	if (pr==n) return;
/**/	int b=__lg(n)-1,i,j,k;
/**/	for (i=1; i<n; i++) r[i]=r[i>>1]>>1|(i&1)<<b;
/**/	if (pw<n)
/**/	{
/**/		for (j=1; j<n; j=k)
/**/		{
/**/			k=j*2;
/**/			ll wn=ksm(g,(p-1)/k);
/**/			w[j]=1;
/**/			for (i=j+1; i<k; i++) w[i]=w[i-1]*wn%p;
/**/		}
/**/		pw=n;
/**/	}
/**/	pr=n;
/**/}
#endif
	void getinv(int n)
	{
		static int pre=0;
		if (!pre) pre=inv[1]=1;
		if (n<=pre) return;
		for (int i=pre+1,j; i<=n; i++)
		{
			j=p/i;
			inv[i]=(p-j)*inv[p-i*j]%p;
		}
		pre=n;
	}
	void getfac(int n)
	{
		static int pre=-1;
		if (pre==-1) pre=0,ifac[0]=fac[0]=1;
		if (n<=pre) return;
		getinv(n);
		for (int i=pre+1,j; i<=n; i++) fac[i]=fac[i-1]*i%p,ifac[i]=ifac[i-1]*inv[i]%p;
		pre=n;
	}
/**/int cal(int x) { return 1<<__lg(max(x,1)*2-1); }
/**/struct Q
/**/{
/**/	vector<ll> a;
/**/	ll *pt() { return a.data(); }
/**/	Q(int x=1):a(cal(x)) {}//小心:{}会调用这条而非下一条
		Q(const vector<ll> &o):a(cal(o.size())) { copy(all(o),a.begin()); }
		ll fx(ll x)
		{
			ll r=0;
			int i;
			for (i=a.size()-1; i>=0&&!a[i]; i--);
			for (; i>=0; i--) r=(r*x+a[i])%p;
			return r;
		}
#ifndef MTT
/**/	void dft(int o=0)
/**/	{
/**/		int n=a.size(),i,j,k;
/**/		ll y,*f,*g,*wn,*A=pt();
/**/		init(n);
/**/		for (i=1; i<n; i++) if (i<r[i]) swap(A[i],A[r[i]]);
/**/		for (k=1; k<n; k*=2)
/**/		{
/**/			wn=w+k;
/**/			for (i=0; i<n; i+=k*2)
/**/			{
/**/				f=A+i; g=A+i+k;
/**/				for (j=0; j<k; j++)
/**/				{
/**/					y=g[j]*wn[j]%p;
/**/					g[j]=f[j]+p-y;
/**/					f[j]+=y;
/**/				}
/**/			}
/**/			if (k*2==n||k==1<<14) for (ll &x:a) x%=p;
/**/		}
/**/		if (o)
/**/		{
/**/			y=ksm(n,p-2);
/**/			// getinv(n); x=inv[n];
/**/			for (ll &x:a) x=x*y%p;
/**/			reverse(1+all(a));
/**/		}
/**/	}
		void hf_dft(int o=0)
		{
			int n=a.size()>>1,i,j,k;
			ll x,y,*f,*g,*wn,*A=pt();
			init(n);
			for (i=1; i<n; i++) if (i<r[i]) swap(A[i],A[r[i]]);
			for (k=1; k<n; k*=2)
			{
				wn=w+k;
				for (i=0; i<n; i+=k*2)
				{
					f=A+i; g=A+i+k;
					for (j=0; j<k; j++)
					{
						x=f[j]; y=g[j]*wn[j]%p;
						if (x+y>=p) f[j]=x+y-p; else f[j]=x+y;
						if (x<y) g[j]=x-y+p; else g[j]=x-y;
					}
				}
			}
			if (o)
			{
				x=ksm(n,p-2);
				// t_tinv(n); x=inv[n];
				for (i=0; i<n; i++) A[i]=A[i]*x%p;
				reverse(A+1,A+n);
			}
		}
#endif
/**/	void resize(int n)
/**/	{
/**/		assert((n&-n)==n);
/**/		a.resize(n);
/**/	}
/**/	ll &operator[](const int &x) { return a[x]; }
/**/	const ll &operator[](const int &x) const { return a[x]; }
		Q dao() const
		{
			ll n=a.size();
			Q r(n);
			for (ll i=1; i<n; i++) r[i-1]=a[i]*i%p;
			return r;
		}
		Q ji() const
		{
			int n=a.size();
			getinv(n);
			Q r(n);
			for (int i=1; i<n; i++) r[i]=a[i-1]*inv[i]%p;
			return r;
		}
		Q operator-() const { Q r=*this; for (ll &x:r.a) if (x) x=p-x; return r; }
		Q operator+(ll x) const { Q r=*this; r+=x; return r; }
		Q &operator+=(ll x) { (a[0]+=x)%=p; return *this; }
		Q operator-(ll x) const { Q r=*this; r-=x; return r; }
		Q &operator-=(ll x) { (a[0]+=p-x)%=p; return *this; }
		Q operator*(ll k) const { Q r=*this; r*=k; return r; }
		Q &operator*=(ll k) { for (ll &x:a) x=x*k%p; return *this; }
		Q operator+(Q f) const { f+=*this; return f; }
		Q &operator+=(const Q &f) { int n=f.a.size(),i; if (a.size()<n) resize(n); for (i=0; i<n; i++) (a[i]+=f[i])%=p; return *this; }
		Q operator-(Q f) const { Q r=*this; r-=f; return r; }
		Q &operator-=(const Q &f) { int n=f.a.size(),i; if (a.size()<n) resize(n); for (i=0; i<n; i++) (a[i]+=p-f[i])%=p; return *this; }
/**/	Q operator*(Q f) const { f*=*this; return f; }
#ifdef MTT
#ifdef CRT
		template<const ll p> void fun(M<p> &s,Q &g)
		{
			Q f=*this;
			int n=g.a.size(),i;
			s.dft(f.a); s.dft(g.a);
			for (i=0; i<n; i++) g[i]=g[i]*f[i]%p;
			s.dft(g.a,1);
		}
		void operator*=(Q g3)
		{
			assert(a.size()==g3.a.size());
			int n=a.size()*2,i;
			resize(n); g3.resize(n);
			Q g1=g3,g2=g3;
			fun(s1,g1);
			fun(s2,g2);
			fun(s3,g3);
			resize(n>>=1);
			ll _p12=p1*p2%p,x;
			for (i=0; i<n; i++)
			{
				x=(g2[i]+p2-g1[i])*inv_p1%p2*p1+g1[i];
				a[i]=((x+p3-g3[i])%p3*(p3-inv_p12)%p3*_p12+x)%p;
			}
		}//三模,板子 OJ 5e5 0.9s
#else
		void operator*=(const Q &g)
		{
			ll n=a.size(),m=(1<<15)-1,i;
			assert(n==g.a.size());
			n*=2;
			foly a0(n),a1(n),b0(n),b1(n),u(n),v(n);
			n/=2;
			for (i=0; i<n; i++) a0.a[i].x=a[i]>>15,a1.a[i].x=a[i]&m;
			for (i=0; i<n; i++) b0.a[i].x=g.a[i]>>15,b1.a[i].x=g.a[i]&m;
			ddt(a0,a1); ddt(b0,b1);
			n*=2;
			for (i=0; i<n; i++)
			{
				u.a[i]=a0.a[i]*b0.a[i]+FFT::I*a1.a[i]*b0.a[i];
				v.a[i]=a0.a[i]*b1.a[i]+FFT::I*a1.a[i]*b1.a[i];
			}
			u.dft(1); v.dft(1);
			n>>=1; resize(n);
			for (i=0; i<n; i++) a[i]=((((ll)dtol(u.a[i].x)<<15)%p+dtol(u.a[i].y)+dtol(v.a[i].x)<<15)+dtol(v.a[i].y))%p;
		}//4 次拆系数
		void operator|=(const Q &g)//直接卷积
		{
			ui n=cal(a.size()+g.a.size()-1),m=(1<<15)-1,i;
			foly a0(n),a1(n),b0(n),b1(n),u(n),v(n);
			for (i=0; i<a.size(); i++) a0.a[i].x=a[i]>>15,a1.a[i].x=a[i]&m;
			for (i=0; i<g.a.size(); i++) b0.a[i].x=g.a[i]>>15,b1.a[i].x=g.a[i]&m;
			ddt(a0,a1); ddt(b0,b1);
			for (i=0; i<n; i++)
			{
				u.a[i]=a0.a[i]*b0.a[i]+FFT::I*a1.a[i]*b0.a[i];
				v.a[i]=a0.a[i]*b1.a[i]+FFT::I*a1.a[i]*b1.a[i];
			}
			u.dft(1); v.dft(1);
			resize(n);
			for (i=0; i<n; i++) a[i]=((((ll)dtol(u.a[i].x)<<15)%p+dtol(u.a[i].y)+dtol(v.a[i].x)<<15)+dtol(v.a[i].y))%p;
		}//4 次拆系数,板子 OJ 5e5 1.4s 精度也爆,待修复
#endif
#else
		Q &operator|=(Q f)//直接卷积,不 shift
		{
			int n=cal(a.size()+f.a.size()-1),i;
			resize(n); f.resize(n);
			dft(); f.dft();
			for (i=0; i<n; i++) a[i]=a[i]*f[i]%p;
			dft(1);
			return *this;
		}
		Q operator|(Q f) const { f|=*this; return f; }
/**/	Q &operator*=(Q f)//群内卷积
/**/	{
/**/		assert(a.size()==f.a.size());
/**/		int n=a.size()*2,i;
/**/		resize(n); f.resize(n);
/**/		dft(); f.dft();
/**/		for (i=0; i<n; i++) a[i]=a[i]*f[i]%p;
/**/		dft(1); resize(n>>1);
/**/		return *this;
/**/	}
		Q &operator&=(const Q &f)//卷积并 shift
		{
			*this|=f;
			int n=a.size(),i;
			for (i=n-1; i; i--) if (a[i]) break;
			resize(cal(i+1));
			return *this;
		}
		Q operator&(Q f) const { f&=*this; return f; }
		Q &operator^=(Q f)//差卷积
		{
			assert(a.size()==f.a.size());
			int n=a.size();
			reverse(all(f.a));
			*this|=f;
			copy(pt()+n-1,pt()+n*2-1,pt());
			resize(n);
			return *this;
		}
		Q operator^(const Q &f) const { Q g=*this; g^=f; return g; }
#endif
#ifdef MTT
		Q operator~()
		{
			Q q=*this,r(1);
			int n=a.size()*2,i,j,k;
			resize(n);
			r[0]=ksm(a[0],p-2);
			for (j=2; j<=n; j*=2)
			{
				k=j>>1;
				r.resize(j);
				q.resize(j);
				copy_n(pt(),k,q.pt());
				r=-(q*r-2)*r;
				r.resize(k);
			}
			n>>=1;
			resize(n);
			return r;
		}//trivial
#else
		Q operator~()
		{
			Q q=*this,r(1),g(1);
			int n=a.size(),i,j,k;
			r[0]=ksm(a[0],p-2);
			for (j=2; j<=n; j*=2)
			{
				k=j>>1;
				r.resize(j);
				g=r;
				q.resize(j);
				copy_n(pt(),j,q.pt());
				r.dft(); q.dft();
				for (i=0; i<j; i++) q[i]=q[i]*r[i]%p;
				q.dft(1);
				fill_n(q.pt(),k,0);
				q.dft();
				for (i=0; i<j; i++) r[i]=q[i]*r[i]%p;
				r.dft(1);
				copy_n(g.pt(),k,r.pt());
				for (i=k; i<j; i++) r[i]=(p-r[i])%p;
			}
			return r;
		}//inv(1 6 3 4 9)=(1 998244347 33 998244169 1020)
#endif
		Q operator/(Q f) const { return (*this)*~f; }
		Q &operator/=(Q f) { f=~f; (*this)*=f; return *this; }
	};
#ifndef MTT
	Q sqr(Q f)
	{
		int n=f.a.size()*2,i;
		f.resize(n);
		f.dft();
		for (i=0; i<n; i++) f[i]=f[i]*f[i]%p;
		f.dft(1);
		f.resize(n>>1);
		return f;
	}
	void cdq(Q &f,Q &g,int l,int r)//g_0=1,i*g_i=g_{i-j}*f_j,use for exp_cdq
	{
		static vector<Q> cd;
		int i,m=l+r>>1,n=r-l,nn=n>>1;
		if (l==0&&r==f.a.size())
		{
			getinv(n-1);
			g=Q(n);
			cd.clear();
			for (i=2; i<=n; i*=2)
			{
				cd.emplace_back(i);
				Q &h=cd.back();
				h.resize(i);
				copy_n(f.pt(),i,h.pt());
				h.dft();
			}
		}
		if (l+1==r)
		{
			g[l]=l?g[l]*inv[l]%p:1;
			return;
		}
		cdq(f,g,l,m);
		Q u=cd[__lg(n)-1],v(n);
		copy_n(g.pt()+l,nn,v.pt());
		v.dft();
		for (i=0; i<n; i++) u[i]=u[i]*v[i]%p;
		u.dft(1);
		for (i=m; i<r; i++) (g[i]+=u[i-l])%p;
		cdq(f,g,m,r);
	}
	Q exp_cdq(Q f)
	{
		Q g(1);
		int n=f.a.size(),i;
		for (i=1; i<n; i++) f[i]=f[i]*i%p;
		cdq(f,g,0,n);
		return g;
	}
	Q sqrt(Q b)
	{
		Q q(1),f(1),r(1);
		int n=b.a.size(),i,j=n,l;
		for (i=0; i<n; i++) if (b[i]) { j=i; break; }
		if (j==n) return b;
		if (j&1) { cout<<"-1\n"; exit(0); }
		l=j>>1;
		for (i=0; i<n-j; i++) b[i]=b[i+j];
		for (i=n-j; i<n; i++) b[i]=0;
		r[0]=i=mosqrt(b[0]);
		assert(i!=-1);
		for (j=2; j<=n; j*=2)
		{
			r.resize(j);
			q=~r; f.resize(j*2);
			copy_n(b.pt(),j,f.pt());
			q.resize(j*2); r.resize(j*2);
			q.dft(); r.dft(); f.dft();
			for (i=0; i<j*2; i++) r[i]=(r[i]*r[i]+f[i])%p*q[i]%p*(p+1>>1)%p;
			r.dft(1);
			fill(j+all(r.a),0);
		}
		r.resize(n);
		for (i=n-1; i>=l; i--) r[i]=r[i-l];
		for (i=0; i<l; i++) r[i]=0;
		return r;
	}//sqrt(0 0 4 2 3)=(0 2 499122177 311951361 171573248)
#endif
	Q ln(const Q &f) { return (f.dao()/f).ji(); }//ln(1 927384623 878326372 3882 273455637 998233543)=(0 927384623 817976920 427326948 149643566 610586717)
#ifdef MTT
	Q exp(Q f)
	{
		Q q(1),r(1);
		int n=f.a.size()*2,i,j,k;
		r[0]=1;
		for (j=2; j<=n; j*=2)
		{
			k=j>>1;
			r.resize(j); q.resize(j);
			copy_n(f.pt(),k,q.pt());
			// for (i=0; i<k; i++) q[i]=f[i];
			r=r*(q-ln(r)+1);
			r.resize(k);
		}
		return r;
	}
#else
	Q exp(Q b)
	{
		Q q(1),r(1);
		int n=b.a.size(),i,j;
		r[0]=1;
		for (j=2; j<=n; j*=2)
		{
			r.resize(j);
			q=ln(r);
			for (i=0; i<j; i++) q[i]=(b[i]+p-q[i])%p;
			// for (i=0; i<j; i++) if ((q[i]=b[i]+p-q[i])>=p) q[i]-=p;
			(++q[0])%=p;
			r.resize(j*2); q.resize(j*2);
			r.dft(); q.dft();
			for (i=0; i<j*2; i++) r[i]=r[i]*q[i]%p;
			r.dft(1);
			r.resize(j);
		}
		return r;
	}//exp(0 4 2 3 5)=(1 4 10 665496257 665496281)
	void mul(Q &f,Q &g)
	{
		int n=f.a.size(),i;
		assert(n==g.a.size());
		f.dft(); g.dft();
		for (i=0; i<n; i++) f[i]=f[i]*g[i]%p;
		f.dft(1);
	}
	Q exp_new(Q b)
	{
		Q h(1),f(1),r(1),u(1),v(1);
		int n=b.a.size(),i,j,k;
		r[0]=1; h[0]=1;
		for (j=2; j<=n; j*=2)
		{
			f.resize(j);
			for (i=0; i<j; i++) f[i]=b[i];
			f=f.dao();
			k=j>>1;
			for (i=0; i<k-1; i++) { (f[i+k]+=f[i])%=p; f[i]=0; }
			for (i=k-1; i<j; i++) f[i]=(p-f[i])%p;
			u.resize(k); v.resize(k);
			copy_n(r.pt(),k,u.pt()); copy_n(h.pt(),k,v.pt());
			u=u.dao(); mul(u,v);
			for (i=0; i<k-1; i++) (f[i+k]+=u[i])%=p;
			(f[k-1]+=u[k-1])%=p;
			copy_n(r.pt(),k,u.pt());
			u.dft();
			for (i=0; i<k; i++) u[i]=u[i]*v[i]%p;
			u.dft(1);
			(u[0]+=p-1)%=p;
			u.resize(j); v.resize(j);
			copy_n(b.pt(),k,v.pt());
			v=v.dao(); mul(u,v);
			for (i=0; i<k; i++) (f[i+k]+=p-u[i])%=p;
			f=f.ji();
			copy_n(r.pt(),k,u.pt()); fill_n(u.pt()+k,k,0);
			mul(u,f); r.resize(j);
			for (i=k; i<j; i++) r[i]=(p-u[i])%p;
			if (j!=n) h=~r;
		}
		return r;
	}
	Q sqrt_new(Q b)
	{
		Q q(1),r(1),h(1);
		int n=b.a.size(),i,j=n,k,l;
		for (i=0; i<n; i++) if (b[i]) { j=i; break; }
		if (j==n) return b;
		if (j&1) { cout<<"-1\n"; exit(0); }
		l=j>>1;
		for (i=0; i<n-j; i++) b[i]=b[i+j];
		for (i=n-j; i<n; i++) b[i]=0;
		r[0]=mosqrt(b[0]); h[0]=ksm(r[0],p-2);
		r.resize(1);
		ll i2=p+1>>1;
		for (j=2; j<=n; j*=2)
		{
			k=j>>1;
			q=r; q.dft();
			for (i=0; i<k; i++) q[i]=q[i]*q[i]%p;
			q.dft(1);
			q.resize(j);
			for (i=k; i<j; i++) q[i]=(q[i-k]+p*2-b[i]-b[i-k])*i2%p,q[i-k]=0;
			h.resize(j);
			mul(q,h);
			r.resize(j);
			for (i=k; i<j; i++) r[i]=(p-q[i])%p;
			if (j!=n) h=~r;
		}
		r.resize(n);
		for (i=n-1; i>=l; i--) r[i]=r[i-l];
		for (i=0; i<l; i++) r[i]=0;
		return r;
	}
	Q pow(Q b,ll m)//不应传入超过 int 内容
	{
		int n=b.a.size(),i,j=n,k;
		for (i=0; i<n; i++) if (b[i]) { j=i; break; }
		if (j==n) return b[0]=!m,b;
		if (j*m>=n)
		{
			fill_n(b.pt(),n,0);
			return b;
		}
		for (i=0; i<n-j; i++) b[i]=b[i+j];
		for (i=n-j; i<n; i++) b[i]=0;
		k=b[0]; assert(k);
		b=exp_new(ln(b*ksm(k,p-2))*m)*ksm(k,m);
		j*=m;
		for (i=n-1; i>=j; i--) b[i]=b[i-j];
		for (i=0; i<j; i++) b[i]=0;
		return b;
	}
	Q pow(Q b,string s)
	{
		int n=b.a.size(),i,j=n,k;
		for (i=0; i<n; i++) if (b[i]) { j=i; break; }
		if (j==n) return b[0]=s=="0",b;
		if (j)
		{
			if (s.size()>8||j*stoll(s)>=n)
			{
				fill_n(b.pt(),n,0);
				return b;
			}
		}
		ll m0=0,m1=0;
		for (auto c:s) m0=(m0*10+c-'0')%p,m1=(m1*10+c-'0')%(p-1);
		for (i=0; i<n-j; i++) b[i]=b[i+j];
		for (i=n-j; i<n; i++) b[i]=0;
		k=b[0]; assert(k);
		b=exp_new(ln(b*ksm(k,p-2))*m0)*ksm(k,m1);
		j*=m0;
		for (i=n-1; i>=j; i--) b[i]=b[i-j];
		for (i=0; i<j; i++) b[i]=0;
		return b;
	}
	Q pow2(Q b,ll m)
	{
		Q r(b.a.size()); r[0]=1;
		while (m)
		{
			if (m&1) r=r*b;
			if (m>>=1) b=sqr(b);
		}
		return r;
	}
	pair<Q,Q> div(Q a,Q b)
	{
		int n=0,m=0,l,i,nn=a.a.size();
		for (i=a.a.size()-1; i>=0; i--) if (a[i]) { n=i+1; break; }
		for (i=b.a.size()-1; i>=0; i--) if (b[i]) { m=i+1; break; }
		assert(m);
		if (n<m) return {Q(1),a};
		l=cal(n+m-1);
		Q c(n),d(m);
		reverse_copy(a.pt(),a.pt()+n,c.pt());
		reverse_copy(b.pt(),b.pt()+m,d.pt());
		c.resize(cal(n-m+1)); d.resize(c.a.size());
		d=~d;
		fill(n-m+1+all(d.a),0);
		c*=d;
		fill(n-m+1+all(c.a),0);
		reverse(c.pt(),c.pt()+n-m+1);
		n=a.a.size(); b.resize(n); c.resize(n);
		d=a-c*b;
		assert(count(m+all(d.a),0)==d.a.size()-m);
		c.resize(cal(n-m+1)); d.resize(cal(m));
		return {c,d};
	}
	Q sin(const Q &f) { return (exp_new(f*I)-exp_new(f*(p-I)))*ksm(2*I%p,p-2); }
	Q cos(const Q &f) { return (exp_new(f*I)+exp_new(f*(p-I)))*ksm(2,p-2); }
	Q tan(const Q &f) { return sin(f)/cos(f); }
	Q asin(const Q &f) { return (f.dao()/sqrt_new((f*f-1)*(p-1))).ji(); }
	Q acos(const Q &f) { return ((f.dao()/sqrt_new((f*f-1)*(p-1)))*(p-1)).ji(); }
	Q atan(const Q &f) { return (f.dao()/(f*f+1)).ji(); }
	Q cdq_inv(const Q &f) { return (~(f-1))*(p-1); }//g_0=1,g_i=g_{i-j}*f_j
	Q operator%(const Q &f,const Q &g) { return div(f,g).second; }
	Q &operator%=(Q &f,const Q &g) { f=f%g; return f; }
	ll dt(const vector<ll> &f,const vector<ll> &a,ll m)//常系数齐次线性递推,find a_m,a_n=a_{n-i}*f_i,f_1...k,a_0...k-1
	{
		if (m<a.size()) return a[m];
		assert(f.size()==a.size()+1);
		int k=a.size(),n=cal(k+1<<1),i,l;
		ll ans=0;
		Q h(n);
		for (i=1; i<=k; i++) h[k-i]=(p-f[i])%p;
		h[k]=1;
		Q g(2),r(2); g[1]=1; r[0]=1;
		while (m)
		{
			if (m&1) r=(r&g)%h;
			l=g.a.size()*2; g.resize(l);
			g.dft();
			for (i=0; i<l; i++) g[i]=g[i]*g[i]%p;
			g.dft(1); g%=h; m>>=1;
		}
		k=min(k,(int)r.a.size());
		for (i=0; i<k; i++) ans=(ans+a[i]*r[i])%p;
		return ans;
	}//板子 OJ 1e5/1e18 8246ms,Luogu 32000/1e9 710ms
	ll dt_new(const vector<ll> &f,const vector<ll> &a,ll m)//常系数齐次线性递推,find a_m,a_n=a_{n-i}*f_i,f_1...k,a_0...k-1
	{
		if (m<a.size()) return a[m];
		assert(f.size()==a.size()+1);
		int k=a.size(),n=cal(k+1),lim=n*2,i;
		Q g(n),h(n);
		for (i=1; i<=k; i++) h[k-i]=(p-f[i])%p;
		h[0]=1;
		for (i=0; i<k; i++) g[i]=a[i];
		g*=h;
		fill(k+all(g.a),0);
		++k; g.resize(lim); h.resize(lim);
		vector<ll> res(k);
		while (m)
		{
			if (m&1)
			{
				ll x=p-g[0];
				for (i=1; i<k; i+=2) res[i>>1]=x*h[i]%p;
				copy(g.pt()+1,g.pt()+k,g.pt());
				g[k-1]=0;
			}
			g.dft(); h.dft();
			ll *a=g.pt(),*b=h.pt(),*c=a+n,*d=b+n;
			for (i=0; i<n; i++) g[i]=(a[i]*d[i]+b[i]*c[i])%p*(p+1>>1)%p;
			for (i=0; i<n; i++) h[i]=h[i]*h[i^n]%p;
			g.hf_dft(1); h.hf_dft(1);
			fill(g.pt()+k,g.pt()+lim,0);
			if (m&1) for (i=0; i<k; i++) (g[i]+=res[i])%=p;
			fill(h.pt()+k,h.pt()+lim,0);
			m>>=1;
		}
		return g[0];
	}//板子 OJ 1e5/1e18 1310ms,Luogu 32000/1e9 160ms
	Q prod(const vector<Q> &a)
	{
		if (!a.size()) return Q(vector<ll>{1});
		auto dfs=[&](auto dfs,const Q *a,int n) -> Q
		{
			if (n==1) return a[0];
			int m=n>>1;
			return dfs(dfs,a,m)&dfs(dfs,a+m,n-m);
		};
		return dfs(dfs,a.data(),a.size());
	}
	Q prod_new(const vector<Q> &a)
	{
		if (!a.size()) return Q(vector<ll>{1});
		struct cmp
		{
			bool operator()(const Q &f,const Q &g) const { return f.a.size()>g.a.size(); }
		};
		priority_queue<Q,vector<Q>,cmp> q(all(a));
		while (q.size()>1)
		{
			auto f=q.top(); q.pop();
			f&=q.top(); q.pop();
			q.push(f);
		}
		return q.top();
	}
	vector<ll> get_fx(const Q &f,const vector<ll> &X)
	{
		int m=X.size(),n=f.a.size()-1,i,j;
		vector<Q> pro(m*4+4);
		while (n>1&&!f[n]) --n;
		vector<ll> y(m);
		function<void(int,int,int)> build=[&](int x,int l,int r)
		{
			if (l+1==r)
			{
				pro[x]={vector{(p-X[l])%p,1llu}};
				return;
			}
			int mid=l+r>>1,c=x*2;
			build(c,l,mid); build(c+1,mid,r);
			pro[x]=pro[c]&pro[c+1];
		};
		function<void(int,int,int,Q,int)> dfs=[&](int x,int l,int r,Q f,int d)
		{
			const static int limit=256;
			if (d>=r-l)
			{
				f%=pro[x];
				d=r-l-1;
				while (d>0&&!f[d]) --d;
				f.resize(cal(d+1));
			}
			if (r-l<limit)
			{
				for (int i=l; i<r; i++) y[i]=f.fx(X[i]);
				return;
			}
			int mid=l+r>>1,c=x*2;
			dfs(c,l,mid,f,d);
			dfs(c+1,mid,r,f,d);
		};
		build(1,0,m);
		dfs(1,0,m,f,n);
		return y;
	}//板子 OJ 2^17 920ms
	vector<ll> get_fx_new(Q f,const vector<ll> &X)//多项式多点求值
	{
		int m=X.size(),i,j;
		vector<ll> y(m);
		if (X.size()<=10)
		{
			for (i=0; i<m; i++) y[i]=f.fx(X[i]);
			return y;
		}
		int n=f.a.size();
		while (n>1&&!f[n-1]) --n;
		f.resize(cal(n));
		vector<Q> pro(m*4+4);
		function<void(int,int,int)> build=[&](int x,int l,int r)
		{
			if (l==r)
			{
				pro[x]={vector{1llu,(p-X[l])%p}};
				return;
			}
			int m=l+r>>1,c=x*2;
			build(c,l,m); build(c+1,m+1,r);
			pro[x]=pro[c]&pro[c+1];
		};
		function<void(int,int,int,Q)>dfs=[&](int x,int l,int r,Q f)
		{
			const static int limit=30;
			if (r-l+1<=limit)
			{
				int m=r-l+1,m1,m2,mid=l+r>>1,i,j,k;
				static ll g[limit+2],g1[limit+2],g2[limit+2];
				m1=m2=r-l;
				copy_n(f.pt(),m,g1);
				copy_n(f.pt(),m,g2);
				for (i=mid+1; i<=r; i++,--m1) for (k=0; k<m1; k++) g1[k]=(g1[k]+g1[k+1]*(p-X[i]))%p;
				for (i=l; i<=mid; i++,--m2) for (k=0; k<m2; k++) g2[k]=(g2[k]+g2[k+1]*(p-X[i]))%p;
				for (i=l; i<=mid; i++)
				{
					copy_n(g1,(m=m1)+1,g);
					for (j=l; j<=mid; j++) if (i!=j)
					{
						for (k=0; k<m; k++) g[k]=(g[k]+g[k+1]*(p-X[j]))%p;
						--m;
					}
					y[i]=g[0];
				}
				for (i=mid+1; i<=r; i++)
				{
					copy_n(g2,(m=m2)+1,g);
					for (j=mid+1; j<=r; j++) if (i!=j)
					{
						for (k=0; k<m; k++) g[k]=(g[k]+g[k+1]*(p-X[j]))%p;
						--m;
					}
					y[i]=g[0];
				}
				return;
			}
			int mid=l+r>>1,c=x*2,n=f.a.size(),i;
			pro[c].resize(n); pro[c+1].resize(n);
			f.dft(); reverse(all(pro[c].a));
			pro[c].dft();
			for (i=0; i<n; i++) pro[c][i]=pro[c][i]*f[i]%p;
			pro[c].dft(1); rotate(all(pro[c].a)-1,pro[c].a.end());
			pro[c].resize(cal(r-mid));
			fill(r-mid+all(pro[c].a),0);
			c^=1;
			reverse(all(pro[c].a));
			pro[c].dft();
			for (i=0; i<n; i++) pro[c][i]=pro[c][i]*f[i]%p;
			pro[c].dft(1); rotate(all(pro[c].a)-1,pro[c].a.end());
			pro[c].resize(cal(mid-l+1));
			fill(mid-l+1+all(pro[c].a),0);
			c^=1;
			dfs(c,l,mid,pro[c+1]);
			dfs(c+1,mid+1,r,pro[c]);
		};
		build(1,0,m-1);
		pro[1].resize(f.a.size());
		f^=~pro[1];
		f.resize(cal(m));
		fill(min(m,n)+all(f.a),0);
		dfs(1,0,m-1,f);
		return y;
	}//板子 OJ 2^17 510ms
	Q get_poly(const vector<ll> &X,const vector<ll> &y)//多项式快速插值
	{
		assert(X.size()==y.size());
		int n=X.size(),i,j;
		if (n<=1) return Q(y);
		if (1)
		{
			auto vv=X; sort(all(vv));
			assert(unique(all(vv))-vv.begin()==n);
		}
		vector<Q> sum(4*n+4),pro(4*n+4);
		function<void(int,int,int)> build=[&](int x,int l,int r)
		{
			if (l==r)
			{
				sum[x]={vector{(p-X[l])%p,1llu}};
				return;
			}
			int mid=l+r>>1,c=x*2;
			build(c,l,mid); build(c+1,mid+1,r);
			sum[x]=sum[c]&sum[c+1];
		};
		build(1,0,n-1);
		sum[1]=sum[1].dao();
		auto v=get_fx_new(sum[1],X);
		assert(v.size()==n);
		auto Y=getinvs(v);
		for (i=0; i<n; i++) Y[i]=Y[i]*y[i]%p;
		function<void(int,int,int)> dfs=[&](int x,int l,int r)
		{
			if (l==r)
			{
				pro[x]={vector{Y[l],0llu}};
				return;
			}
			int c=x*2,mid=l+r>>1;
			dfs(c,l,mid); dfs(c|1,mid+1,r);
			pro[x]=(pro[c]&sum[c|1])+(pro[c|1]&sum[c]);
		};
		dfs(1,0,n-1);
		pro[1].resize(cal(n));
		return pro[1];
	}//板子 OJ 2^17 1.2s
	Q comp(const Q &f,Q g)//多项式复合 f(g(x))=[x^i]f(x)g(x)^i
	{
		int n=f.a.size(),l=ceil(::sqrt(n)),i,j;
		assert(n>=g.a.size());//返回 n-1 次多项式
		vector<Q> a(l+1),b(l);
		a[0].resize(n); a[0][0]=1; a[1]=g;
		g.resize(n*2);
		Q w=g,u,v(n);
		w.dft(); u=w;
		for (i=2; i<=l; i++)
		{
			if (i>2) u.dft();
			for (j=0; j<n*2; j++) u[j]=u[j]*w[j]%p;
			u.dft(1);
			fill(n+all(u.a),0);
			a[i]=u;
		}
		w=a[l];
		w.dft(); u=w;
		for (i=2; i<=l; i++) a[i].resize(n);
		for (i=2; i<l; i++)
		{
			if (i>2) u.dft();
			b[i-1]=u;
			for (j=0; j<n*2; j++) u[j]=u[j]*w[j]%p;
			u.dft(1);
			fill(n+all(u.a),0);
		}
		if (l>2) u.dft(); b[l-1]=u;
		for (i=0; i<l; i++)
		{
			fill(all(v.a),0);
			for (j=0; j<l; j++) if (i*l+j<n) v+=a[j]*f[i*l+j];
			if (i==0) u=v; else
			{
				v.resize(n*2); v.dft();
				for (j=0; j<n*2; j++) v[j]=v[j]*b[i][j]%p;
				v.dft(1); v.resize(n); u+=v;
			}
		}
		return u;
	}//n^2+n\sqrt n\log n,8000 板子 OJ 300ms,20000 luogu 3.5s
	Q comp_inv(Q f)//多项式复合逆 g(f(x))=x,求 g,[x^n]g=([x^{n-1}](x/f)^n)/n,要求常数 0 一次非 0
	{
		assert(!f[0]&&f[1]);
		int n=f.a.size(),l=ceil(::sqrt(n)),i,j,k,m;//l>=2
		rotate(f.a.begin(),1+all(f.a));
		f=~f;
		getinv(n*2);
		vector<Q> a(l+1),b(l);
		Q u,v;
		u=a[1]=f;
		u.resize(n*2); u.dft(); v=u;
		for (i=2; i<=l; i++)
		{
			if (i>2) u.dft();
			for (j=0; j<n*2; j++) u[j]=u[j]*v[j]%p;
			u.dft(1);
			fill(n+all(u.a),0);
			a[i]=u;
		}
		b[0].resize(n); b[0][0]=1; b[1]=a[l]; u.dft(); v=u;
		for (i=2; i<l; i++)
		{
			if (i>2) u.dft();
			for (j=0; j<n*2; j++) u[j]=u[j]*v[j]%p;
			u.dft(1);
			fill(n+all(u.a),0);
			b[i]=u;
		}
		u.resize(n); u[0]=0;
		for (i=0; i<l; i++) for (j=1; j<=l; j++) if (i*l+j<n)
		{
			m=i*l+j-1;
			ll r=0,*f=b[i].pt(),*g=a[j].pt();
			for (k=0; k<=m; k++) r=(r+f[k]*g[m-k])%p;
			u[m+1]=r*inv[m+1]%p;
		}
		return u;
	}
	Q shift(Q f,ll c)//get f(x+c),c\in [0,p)
	{
		int n=f.a.size(),i,j;
		Q g(n);
		getfac(n);
		for (i=0; i<n; i++) f[i]=f[i]*fac[i]%p;
		g[0]=1;
		for (i=1; i<n; i++) g[i]=g[i-1]*c%p;
		for (i=0; i<n; i++) g[i]=g[i]*ifac[i]%p;
		f^=g;
		for (i=0; i<n; i++) f[i]=f[i]*ifac[i]%p;
		return f;
	}//板子 OJ 5e5 200ms
	vector<ll> shift(vector<ll> y,ll c,ll m)//[0,n) 点值 -> [c,c+m) 点值
	{
		assert(y.size());
		if (y.size()==1) return vector(m,y[0]);
		vector<ll> r,res;
		r.reserve(m);
		int n=y.size(),i,j,mm=m;
		while (c<n&&m) r.push_back(y[c++]),--m;
		if (c+m>p)
		{
			res=shift(y,0,c+m-p);
			m=p-c;
		}
		if (!m) { r.insert(r.end(),all(res)); return r; }
		int len=cal(m+n-1),l=m+n-1;
		for (i=n&1; i<n; i+=2) y[i]=(p-y[i])%p;
		getfac(n);
		for (i=0; i<n; i++) y[i]=y[i]*ifac[i]%p*ifac[n-1-i]%p;
		y.resize(len);
		Q f,g;
		vector<ll> v(m+n-1);
		c-=n-1;
		for (i=0; i<l; i++) v[i]=(c+i)%p;
		f.a=y; g.a=getinvs(v); g.resize(len);
		f|=g;
		vector<ll> u(m);
		for (i=n-1; i<l; i++) u[i-(n-1)]=f[i];
		v.resize(m);
		for (i=0; i<m; i++) v[i]=c+i;
		v=getinvs(v); c+=n;
		ll tmp=1;
		for (i=c-n; i<c; i++) tmp=tmp*i%p;
		for (i=0; i<m; i++) u[i]=u[i]*tmp%p,tmp=tmp*(c+i)%p*v[i]%p;
		r.insert(r.end(),all(u));
		r.insert(r.end(),all(res));
		assert(r.size()==mm);
		return r;
	}//板子 OJ 5e5 430ms
	const ll B=1e5;
	ll a[B+2],b[B+2];
	ll mic(ll x) { return a[x%B]*b[x/B]%p; }
	vector<ll> Z_trans(Q f,ll c,ll m)//求 f(c^[0,m))。核心 ij=C(i+j,2)-C(i,2)-C(j,2)
	{
		int i,n=f.a.size();
		if (n*m<B*5)
		{
			vector<ll> r(m);
			ll j;
			for (i=0,j=1; i<m; i++) r[i]=f.fx(j),j=j*c%p;
			return r;
		}
		ll l=cal(m+=n-1);
		Q g(l);
		assert(B*B>p);
		a[0]=b[0]=g[0]=g[1]=1;
		for (i=1; i<=B; i++) a[i]=a[i-1]*c%p;
		c=a[B];
		for (i=1; i<=B; i++) b[i]=b[i-1]*c%p;
		for (i=2; i<n; i++) f[i]=f[i]*mic((p*2-2-i)*(i-1)/2%(p-1))%p;
		reverse(all(f.a)); f.resize(l);
		for (i=2; i<m; i++) g[i]=mic(i*(i-1llu)/2%(p-1));
		f.dft(); g.dft();
		for (i=0; i<l; i++) f[i]=f[i]*g[i]%p;
		f.dft(1);
		vector<ll> r(f.pt()+n-1,f.pt()+m); m-=n-1;
		for (i=2; i<m; i++) r[i]=r[i]*mic((p*2-2-i)*(i-1)/2%(p-1))%p;
		return r;
	}//luogu 1e6 400ms
	vector<ll> get_Bell(int n)//B(0...n)
	{
		++n;
		getfac(n-1);
		Q f(n);
		int i;
		for (i=1; i<n; i++) f[i]=ifac[i];
		f=exp_new(f);
		for (i=2; i<n; i++) f[i]=f[i]*fac[i]%p;
		return vector<ll>(f.pt(),f.pt()+n);
	}
	vector<ll> get_S1_row(int n,int m)//S1(n,0...m),O(nlogn),unsigned
	{
		int cm=cal(++m);
		if (n==0)
		{
			vector<ll> r(m);
			r[0]=1;
			return r;
		}
		auto dfs=[&](auto self,int n)
		{
			if (n==1)
			{
				Q f(1);
				f[1]=1;
				return f;
			}
			Q f=self(self,n>>1);
			f|=shift(f,n>>1);
			if (n&1)
			{
				f.a.resize(cal(n+1));
				copy_n(f.pt(),n,f.pt()+1);
				--n;
				for (int i=0; i<=n; i++) f[i]=(f[i]+f[i+1]*n)%p;
			}
			if (f.a.size()>cm) f.resize(cm);
			return f;
		};
		Q f=dfs(dfs,n);
		return vector<ll>(f.pt(),f.pt()+m);
	}
	vector<ll> get_S1_column(int n,int m)//S1(0...n,m),O(nlogn)
	{
		if (m==0)
		{
			vector<ll> r(n+1);
			r[0]=1;
			return r;
		}
		Q f(n+1);
		getfac(max(n,m));
		int i;
		for (i=1; i<=n; i++) f[i]=inv[i];
		f=pow(f,m);
		for (i=m; i<=n; i++) f[i]=f[i]*fac[i]%p*ifac[m]%p;
		return vector<ll>(f.pt(),f.pt()+n+1);
	}
	vector<ll> get_S2_row(int n,int m)//S2(n,0...m),O(mlogm)
	{
		int tm=++m;
		if (n==0)
		{
			vector<ll> r(m);
			r[0]=1;
			return r;
		}
		m=min(m,n+1);
		ll pr[m],pw[m],cnt=0;//i^n
		int i,j;
		fill_n(pw,m,0);
		pw[1]=1;
		for (i=2; i<m; i++)
		{
			if (!pw[i]) pr[cnt++]=i,pw[i]=ksm(i,n);
			for (j=0; i*pr[j]<m; j++)
			{
				pw[i*pr[j]]=pw[i]*pw[pr[j]]%p;
				if (i%pr[j]==0) break;
			}
		}
		getfac(m-1);
		Q f(m),g(m);
		for (i=0; i<m; i+=2) f[i]=ifac[i];
		for (i=1; i<m; i+=2) f[i]=p-ifac[i];
		for (i=1; i<m; i++) g[i]=pw[i]*ifac[i]%p;
		f*=g;
		vector<ll> r(f.pt(),f.pt()+m);
		r.resize(tm);
		return r;
	}
	vector<ll> get_S2_column(int n,int m)//S2(0...n,m),O(nlogn)
	{
		if (m==0)
		{
			vector<ll> r(n+1);
			r[0]=1;
			return r;
		}
		Q f(n+1);
		getfac(max(n,m));
		int i;
		for (i=1; i<=n; i++) f[i]=ifac[i];
		f=pow(f,m);
		for (i=m; i<=n; i++) f[i]=f[i]*fac[i]%p*ifac[m]%p;
		return vector<ll>(f.pt(),f.pt()+n+1);
	}
	vector<ll> get_signed_S1_row(int n,int m)
	{
		auto v=get_S1_row(n,m);
		for (int i=1^n&1; i<=m; i+=2) v[i]=(p-v[i])%p;
		return v;
	}
	vector<ll> get_Bernoulli(int n)//B(0...n)
	{
		getfac(++n);
		int i;
		Q f(n);
		for (i=0; i<n; i++) f[i]=ifac[i+1];
		f=~f;
		for (i=0; i<n; i++) f[i]=f[i]*fac[i]%p;
		return vector<ll>(f.pt(),f.pt()+n);
	}
	vector<ll> get_partition(int n)//P(0...n),分拆数
	{
		Q f(++n);
		int i,l=0,r=0;
		while (--l) if (3*l*l-l>=n*2) break;
		while (++r) if (3*r*r-r>=n*2) break;
		++l;
		for (i=l+abs(l)%2; i<r; i+=2) f[3*i*i-i>>1]=1;
		for (i=l+abs(l+1)%2; i<r; i+=2) f[3*i*i-i>>1]=p-1;
		f=~f;
		return vector<ll>(f.pt(),f.pt()+n);
	}
#endif
}
using NTT::p;
#define poly NTT::Q
/*
函数名称:sqr, cdq, exp_cdq, sqrt, ln, exp, exp_new, sqrt_new, pow, pow2, div,
dt, dt_new, prod, get_fx, get_fx_new, get_poly, comp, comp_inv, shift, Z_trans,
get_Bell, get_S1_row, get_S1_column, get_S2_row, get_S2_column, get_signed_S1_row,
get_Bernoulli.
*/
\end{lstlisting}

\subsection{FFT}

\begin{lstlisting}
namespace FFT
{
	#define all(x) (x).begin(),(x).end()
	typedef double db;
	const int N=1<<21;
	const db pi=3.14159265358979323846;
	struct comp
	{
		db x,y;
		comp operator+(const comp &o) const {return {x+o.x,y+o.y};}
		comp operator-(const comp &o) const {return {x-o.x,y-o.y};}
		comp operator*(const comp &o) const {return {x*o.x-y*o.y,o.x*y+x*o.y};}
		comp operator*(const db &o) const {return {x*o,y*o};}
		void operator*=(const comp &o) {*this={x*o.x-y*o.y,o.x*y+x*o.y};}
		void operator*=(const db &o) {x*=o;y*=o;}
		void operator/=(const db &o) {x/=o;y/=o;}
		comp operator/(const comp &o) const
		{
			db z=1/(o.x*o.x+o.y*o.y);
			return {z*(x*o.x+y*o.y),z*(o.x*y-x*o.y)};
		}//not necessary, no check
	};
	long long dtol(const double &x) {return fabs(round(x));}
	const comp I{0,-1};
	ostream & operator<<(ostream &cout,const comp &o) {cout<<o.x;if (o.y>=0) cout<<'+';return cout<<o.y<<'i';}
	int r[N];
	char c;
	comp Wn[N];
	void init(int n)
	{
		static int preone=-1;
		if (n==preone) return;
		preone=n;
		int b,i;
		b=__builtin_ctz(n)-1;
		for (i=1;i<n;i++) r[i]=r[i>>1]>>1|(i&1)<<b;
		for (i=0;i<n;i++) Wn[i]={cos(pi*i/n),sin(pi*i/n)};
	}
	int cal(int x) {return 1u<<32-__builtin_clz(max(x,2)-1);}
	struct Q
	{
		vector<comp> a;
		int deg;
		comp* pt() {return a.data();}
		Q(int n=0)
		{
			deg=n;
			a.resize(cal(n));
		}
		void dft(int xs=0)//1,0
		{
			int i,j,k,l,n=a.size(),d;
			comp w,wn,b,c,*f=pt(),*g,*a=f;
			init(n);
			if (xs) reverse(a+1,a+n);//spe
			for (i=0;i<n;i++) if (i<r[i]) swap(a[i],a[r[i]]);
			for (i=1,d=0;i<n;i=l,d++)
			{
				//wn={cos(pi/i),(xs?-1:1)*sin(pi/i)};
				l=i<<1;
				for (j=0;j<n;j+=l)
				{
					//w={1,0};
					f=a+j;g=f+i;
					for (k=0;k<i;k++)
					{
						w=Wn[k*(n>>d)];
						b=f[k];c=g[k]*w;
						f[k]=b+c;
						g[k]=b-c;
						//w*=wn;
					}
				}
			}
			if (xs) for (i=0;i<n;i++) a[i]/=n;
		}
		void operator|=(Q o)
		{
			int n=deg+o.deg-1,m=cal(n),i;
			a.resize(m);o.a.resize(m);
			dft();o.dft();
			for (i=0;i<m;i++) a[i]*=o.a[i];
			dft(1);
			for (i=n;i<m;i++) a[i]={};
			deg=n;
		}
		Q operator|(Q o) const {o|=*this;return o;}
	};
	Q mul(Q a,const Q &b)//三次变两次,仅实数,注意精度
	{
		int n=a.deg+b.deg-1,m=cal(n),i;
		a.a.resize(m);
		for (i=0;i<b.deg;i++) a.a[i]={a.a[i].x,b.a[i].x};
		a.dft();
		for (i=0;i<m;i++) a.a[i]*=a.a[i];
		a.dft(1);
		for (i=0;i<n;i++) a.a[i]={a.a[i].y*.5};
		for (i=n;i<m;i++) a.a[i]={};
		a.deg=n;
		return a;
	}
	void ddt(Q &a,Q &b)//double dft,仅实数,注意精度
	{
		comp x,y;
		int n=a.a.size(),i;
		assert(n==b.a.size());
		for (i=0;i<n;i++) a.a[i]={a.a[i].x,b.a[i].x};
		a.dft();
		for (i=0;i<n;i++) b.a[i]={a.a[i].x,-a.a[i].y};
		reverse(b.pt()+1,b.pt()+n);
		for (i=0;i<n;i++)
		{
			x=a.a[i];y=b.a[i];
			a.a[i]=(x+y)*.5;
			b.a[i]=(y-x)*.5*I;
		}
	}
}
using FFT::dtol;
\end{lstlisting}

\subsection{约数个数和}

$O(\sqrt[3]n\log n)$。

\begin{lstlisting}
#include<bits/stdc++.h>
#define ll long long
#define lll __int128
using namespace std;

void myw(lll x){
	if(!x) return;
	myw(x/10);printf("%d",(int)(x%10));
}

struct vec{
	ll x,y;
	vec (ll x0=0,ll y0=0){x=x0,y=y0;}
	vec operator +(const vec b){return vec(x+b.x,y+b.y);}
};

ll N;
vec stk[1000005];int len;
vec P;
vec L,R; 

bool ninR(vec a){return N<(lll)a.x*a.y;}
bool steep(ll x,vec a){return (lll)N*a.x<=(lll)x*x*a.y;}

lll Solve(){
	len=0;
	ll cbr=cbrt(N),sqr=sqrt(N);
	P.x=N/sqr,P.y=sqr+1;
	lll ans=0;
	stk[++len]=vec(1,0);stk[++len]=vec(1,1);
	while(1){
		L=stk[len--];
		while(ninR(vec(P.x+L.x,P.y-L.y)))
			ans+=(lll)P.x*L.y+(lll)(L.y+1)*(L.x-1)/2,
			P.x+=L.x,P.y-=L.y;
		if(P.y<=cbr) break;
		R=stk[len];
		while(!ninR(vec(P.x+R.x,P.y-R.y))) L=R,R=stk[--len];
		while(1){
			vec mid=L+R;
			if(ninR(vec(P.x+mid.x,P.y-mid.y))) R=stk[++len]=mid;
			else if(steep(P.x+mid.x,R)) break;
			else L=mid;
		}
	}
	for(int i=1;i<P.y;i++) ans+=N/i;
	return ans*2-sqr*sqr;
}

int T;

int main(){
	scanf("%d",&T);
	while(T--){
		scanf("%lld",&N);
		myw(Solve());printf("\n");
	}
}
\end{lstlisting}



\subsection{万能欧几里得}

题意:$\sum\limits_{i=0}^{n-1}\lfloor \dfrac {ai+b}m\rfloor$ ($0\le a,b$)

注意若 $b\ge m$ 需要增加先往上走一步

\begin{lstlisting}
struct nd
{
	ll x,y,sy;
	nd operator+(const nd &o) const
	{
		return {x+o.x,y+o.y,sy+o.sy+y*o.x};
	}
};
nd ksm (nd a,int k)
{
	nd res{};
	while (k)
	{
		if (k&1) res=res+a;
		a=a+a;k>>=1;
	}
	return res;
}
nd sol (int p,int q,int r,int l,nd a,nd b)//(0,l],(pi+r)/q
{
	if (!l) return {};
	if (p>=q) return sol(p%q,q,r,l,a,ksm(a,p/q)+b);
	int m=((ll)l*p+r)/q;
	if (!m) return ksm(b,l);
	int cnt=l-((ll)q*m-r-1)/p;
	return ksm(b,(q-r-1)/p)+a+sol(q,p,(q-r-1)%p,m-1,b,a)+ksm(b,cnt);
}
int main()
{
	ios::sync_with_stdio(0);cin.tie(0);
	cout<<setiosflags(ios::fixed)<<setprecision(15);
	int T;cin>>T;
	while (T--)
	{
		int n,m,a,b;
		cin>>n>>m>>a>>b;
		nd nx={1,0,0},ny={0,1,0};
		nd ans=sol(a,m,b,n-1,ny,nx);
		cout<<ans.sy<<'\n';
	}
}
\end{lstlisting}


\newpage

\section{字符串}

\subsection{AC 自动机}

\begin{lstlisting}
scanf("%d",&n);
	for (i=1;i<=n;i++)
	{
		x=0;cc=getchar();
		while ((cc<'a')||(cc>'z')) cc=getchar();
		while ((cc>='a')&&(cc<='z'))
		{
			cc-='a';
			if (c[x][cc]==0) c[x][cc]=++ds;
			x=c[x][cc];
			cc=getchar();
		}
		ys[i]=x;
	}tou=1;wei=0;
	for (int v:c[0]) if (v) dl[++wei]=v;
	while (tou<=wei)
	{
		x=dl[tou++];
		for (i=0;i<=25;i++) if (c[x][i]) f[dl[++wei]=c[x][i]]=c[f[x]][i]; else c[x][i]=c[f[x]][i];
	}
	x=0;cc=getchar();
	while ((cc<'a')||(cc>'z')) cc=getchar();
	while ((cc>='a')&&(cc<='z'))
	{
		++cs[x=c[x][cc-'a']];cc=getchar();
	}++wei;
	while (--wei) cs[f[dl[wei]]]+=cs[dl[wei]];
	for (i=1;i<=n;i++) printf("%d\n",cs[ys[i]]);
\end{lstlisting}

\subsection{hash}

$O(n)$,$O(n)$。

\begin{lstlisting}
typedef unsigned int ui;
typedef unsigned long long ull;
typedef pair<ui,ui> pa;
namespace sh
{
	const int N=1e6+5;
	const ull b1=137,b2=149,i1=1'603'801'661,i2=1'024'053'074;
	const ui p1=2'034'452'107,p2=2'013'074'419;
	ull m1[N+1],m2[N+1],r1,r2;
	int i;
	void init()
	{
		m1[0]=m2[0]=1;
		for (i=1;i<=N;i++)
		{
			m1[i]=m1[i-1]*b1%p1;
			m2[i]=m2[i-1]*b2%p2;
		}
	}
	struct str
	{
		vector<pa> a;
		  str(int *s,int n)
		{
			a.resize(n+1);
			r1=r2=0;
			a[0]={0,0};
			for (i=0;i<n;i++)
			{
				r1=(r1+s[i]*m1[i])%p1;
				r2=(r2+s[i]*m2[i])%p2;
				a[i+1]={r1,r2};
			   }
		}
		str(const string &s)
		{
			int n=s.size();
			a.resize(n+1);
			r1=r2=0;
			a[0]={0,0};
			for (i=0;i<n;i++)
			{
				r1=(r1+s[i]*m1[i])%p1;
				r2=(r2+s[i]*m2[i])%p2;
				a[i+1]={r1,r2};
			   }
		}
		pa getv(int l,int r)//[l,r)
		{
			return {(p1+a[r].first-a[l].first)*m1[N-l]%p1,(p2+a[r].second-a[l].second)*m2[N-l]%p2};
		}
	};
	ull ptou(const pa &a)
	{
		return (ull)a.first<<32|a.second;
	}
}
using sh::init,sh::ptou,sh::p1,sh::p2,sh::str,sh::m1,sh::m2;
\end{lstlisting}

\subsection{KMP}

$O(n)$,$O(n)$。

\begin{lstlisting}
struct str
{
	vector<int> nxt,s;
	int n;
	str(int *S,int _n)//[1,n]
	{
		n=_n;
		nxt.resize(n+1);
		s=vector<int>(S,S+n+1);
		int i,j=0;
		nxt[1]=0;
		for (i=2;i<=n;i++)
		{
			while (j&&s[i]!=s[j+1]) j=nxt[j];
			nxt[i]=j+=s[i]==s[j+1];
		}
	}
	vector<int> match(int *t,int m)//find s(str) in t (start pos)
	{
		vector<int> r;
		int i,j=0;
		for (i=1;i<=m;i++)
		{
			while (j&&t[i]!=s[j+1]) j=nxt[j];
			if ((j+=t[i]==s[j+1])==n) j=nxt[j],r.push_back(i-n+1);
		}
		return r;
	}
};
int main()
{
	ios::sync_with_stdio(0);cin.tie(0);
	string s,t;
	cin>>s>>t;
	int n=s.size(),m=t.size(),i;
	vector<int> a(n+1),b(m+1);
	for (i=1;i<=n;i++) a[i]=s[i-1];
	for (i=1;i<=m;i++) b[i]=t[i-1];
	str q(b.data(),m);
	auto r=q.match(a.data(),n);
	for (int x:r) cout<<x<<'\n';
	for (i=1;i<=m;i++) cout<<q.nxt[i]<<" \n"[i==m];
}

\end{lstlisting}

\subsection{KMP(重构,未验证)}

$O(n)$,$O(n)$。

\begin{lstlisting}
struct str//[0,n)
{
	vector<int> nxt,s;
	int n;
	str(const vector<int> &_s):nxt(_s.size(),-1),s(all(_s)),n(_s.size())
	{
		int i,j=-1;
		for (i=1;i<n;i++)
		{
			while (j!=-1&&s[i]!=s[j+1]) j=nxt[j];
			nxt[i]=j+=s[i]==s[j+1];
		}
	}
	vector<int> match(const vector<int> &t)//find s(str) in t (start pos)
	{
		int m=t.size();
		vector<int> r;
		int i,j=-1;
		for (i=0;i<m;i++)
		{
			while (j!=-1&&t[i]!=s[j+1]) j=nxt[j];
			if ((j+=t[i]==s[j+1])==n-1) j=nxt[j],r.push_back(i-n+1);
		}
		return r;
	}
};

	
\end{lstlisting}

\subsection{manacher}

$O(n)$,$O(n)$。

\begin{lstlisting}
vector<int> manacher(const string &t)//ex[i](total length) centered at i/2
{
	string S="$#";
	int n=t.size(),i,r=1,m=0;
	for (i=0;i<n;i++) S+=t[i],S+='#';
	S+='#';
	char *s=S.data()+2;
	n=n*2-1;
	vector<int> ex(n);
	ex[0]=2;
	for (i=1;i<n;i++)
	{
		ex[i]=i<r?min(ex[m*2-i],r-i+1):1;
		while (s[i+ex[i]]==s[i-ex[i]]) ++ex[i];
		if (i+ex[i]-1>r) r=i+ex[m=i]-1;
	}
	for (i=0;i<n;i++) --ex[i];
	return ex;
}
\end{lstlisting}

\subsection{SA}

$O((n+\sum)\log n)$,$O(n+\sum)$。

\begin{lstlisting}
namespace SA
{
	const int N=1e6+2;
	int x[N],y[N],s[N],lg[N];
	int m,i,j,k,cnt;
	bool ied=0;
	void SA_init()
	{
		for (int i=2;i<N;i++) lg[i]=lg[i>>1]+1;
	}
	struct Q
	{
		vector<vector<int>> st;
		vector<int> _sa,rk,h;
		int *sa;
		int lcp(int x,int y)
		{
			assert(x^y);
			x=rk[x];y=rk[y];
			if (x>y) swap(x,y);
			++x;
			int z=lg[y-x+1];
			return min(st[z][x],st[z][y-(1<<z)+1]);
		}
		Q(int *a,int n)//[1,n]
		{
			if (!ied) ied=1,SA_init();
			_sa.resize(n+1);rk.resize(n+1);h.resize(n+1);sa=_sa.data();
			m=*min_element(a+1,a+n+1);--m;
			for (i=1;i<=n;i++) a[i]-=m;
			m=*max_element(a+1,a+n+1);
			assert(n<N);assert(m<N);
			memset(s+1,0,m*sizeof s[0]);
			for (i=1;i<=n;i++) ++s[x[i]=a[i]];
			for (i=2;i<=m;i++) s[i]+=s[i-1];
			for (i=n;i;i--) sa[s[x[i]]--]=i;
			memset(s+1,0,m*sizeof s[0]);
			for (j=1;j<=n;j<<=1)
			{
				cnt=0;
				for (i=n-j+1;i<=n;i++) y[++cnt]=i;
				for (i=1;i<=n;i++) if (sa[i]>j) y[++cnt]=sa[i]-j;
				for (i=1;i<=n;i++) ++s[x[i]];
				for (i=2;i<=m;i++) s[i]+=s[i-1];
				for (i=n;i;i--) sa[s[x[y[i]]]--]=y[i];
				y[sa[1]]=cnt=1;
				memset(s+1,0,m*sizeof s[0]);
				for (i=2;i<=n;i++) if (x[sa[i]]==x[sa[i-1]]&&sa[i]<=n-j&&sa[i-1]<=n-j&&x[sa[i]+j]==x[sa[i-1]+j]) y[sa[i]]=cnt; else y[sa[i]]=++cnt;
				memcpy(x,y,sizeof(y));
				if ((m=cnt)==n) break;
			}
			for (i=1;i<=n;i++) rk[sa[i]]=i;
			j=0;
			for (i=1;i<=n;i++) if (x[i]>1)
			{
				cnt=sa[x[i]-1];
				while (i+j<=n&&cnt+j<=n&&a[i+j]==a[cnt+j]) ++j;
				h[x[i]]=j;
				if (j) --j;
			}
			st=vector<vector<int>>(lg[n]+1,vector<int>(n+1));
			for (i=1;i<=n;i++) st[0][i]=h[i];
			for (j=1;j<=lg[n];j++) for (i=1,k=n-(1<<j)+1;i<=k;i++) st[j][i]=min(st[j-1][i],st[j-1][i+(1<<j-1)]);
		}
	};
}
#define str SA::Q
\end{lstlisting}

\subsection{SAM}

$O(n\sum)$,$O(2n\sum )$。

\begin{lstlisting}
template<int N> struct sam
{
	int p,q,np,nq,c[N*2][26],ds,cd,len[N*2],fa[N*2];
	sam(){np=ds=1;}
	void ins(int zf)
	{
		p=np;len[np=++ds]=++cd;
		while (!c[p][zf]&&p)
		{
			c[p][zf]=np;
			p=fa[p];
		}
		if (!p)
		{
			fa[np]=1;
			return;
		}
		q=c[p][zf];
		if (len[q]==len[p]+1)
		{
			fa[np]=q;
			return;
		}
		len[nq=++ds]=len[p]+1;
		memcpy(c[nq],c[q],sizeof(c[q]));
		fa[nq]=fa[q];
		fa[np]=fa[q]=nq;
		c[p][zf]=nq;
		while (c[p=fa[p]][zf]==q) c[p][zf]=nq;
	}
	void out()
	{
		int i,j;
		for (i=1;i<=ds;i++) for (j=0;j<=25;j++) if (c[i][j]) printf("%d->%d %c\n",i,c[i][j],j+'a');
	}
	vector<int> match(string s)//返回每个前缀最长匹配长度
	{
		vector<int> r;
		r.reserve(s.size());
		p=1;
		int nl=0;
		for (auto ch:s)
		{
			ch-='a';
			if (c[p][ch]) ++nl,p=c[p][ch];
			else
			{
				while (p&&c[p][ch]==0) p=fa[p];
				if (p==0) p=1,nl=0; else nl=len[p]+1,p=c[p][ch];
			}
			r.push_back(nl);
		}
		return r;
	}
};
\end{lstlisting}

\subsection{SqAM}

$O(n\sum)$,$O(n\sum )$。

\begin{lstlisting}
struct sqam
{
	int c[N][26],ds,i,j,lst[26],pre[N];
	void csh()
	{
		ds=1;
	}
	void ins(int zf)
	{
		++ds;
		for (i=0;i<=25;i++) if (lst[i]) for (j=lst[i];(j)&&(c[j][zf]==0);j=pre[j]) c[j][zf]=ds;
		if (!lst[zf]) c[1][zf]=ds; else pre[ds]=lst[zf];
		lst[zf]=ds;
	}
};
\end{lstlisting}

\subsection{ukkonen 后缀树}

$O(n)$,$O(2n\sum )$。

\begin{lstlisting}
void dfs(int x,int lf)
{
	if (!fir[x])
	{
		siz[x][1]=1;
		return;
	}
	int i,j;
	for (i=fir[x];i;i=nxt[i])
	{
		j=c[x][lj[i]];
		if ((f[j]<=m)&&(t[j]>=m)) ++siz[x][0];
		dfs(zd[j],t[j]-f[j]+1);
		siz[x][0]+=siz[zd[j]][0];
		siz[x][1]+=siz[zd[j]][1];
		if ((t[j]==n)&&(f[j]<=m)) --siz[x][1];
	}
	ans+=(ll)siz[x][0]*siz[x][1]*lf;
}
void add(int a,int b,int cc,int d)
{
	zd[++bbs]=b;
	t[bbs]=d;
	c[a][s[f[bbs]=cc]]=bbs;
}
void add(int x,int y)
{
	lj[++bs]=y;
	nxt[bs]=fir[x];
	fir[x]=bs;
}
	s[++m]=26;
	fa[1]=point=ds=1;
	for (i=1;i<=m;i++)
	{
		ad=0;++remain;
		while (remain)
		{
			if (r==0) edge=i;
			if ((j=c[point][s[edge]])==0)
			{
				fa[++ds]=1;
				fa[ad]=point;
				add(ad=point,ds,edge,m);
				add(point,s[edge]);
			}
			else
			{
				if ((t[j]!=m)&&(t[j]-f[j]+1<=r))
				{
					r-=t[j]-f[j]+1;
					edge+=t[j]-f[j]+1;
					point=zd[j];
					continue;
				}
				if (s[f[j]+r]==s[i]) {++r;fa[ad]=point;break;}
				fa[fa[ad]=++ds]=1;
				add(ad=ds,zd[j],f[j]+r,t[j]);
				add(ds,s[i]);add(ds,s[f[j]+r]);fa[++ds]=1;
				add(ds-1,ds,i,m);
				zd[j]=ds-1;t[j]=f[j]+r-1;
			}
			--remain;
			if ((r)&&(point==1))
			{
				--r;edge=i-remain+1;
			} else point=fa[point];
		}
	}
	for (i=1;i<=ds;i++) for (j=fir[i];j;j=nxt[j]) {len[j]=t[c[i][lj[j]]]-f[c[i][lj[j]]]+1;lj[j]=zd[c[i][lj[j]]];}
\end{lstlisting}

\subsection{ukkonen 后缀树(重构)}

\begin{lstlisting}
struct suffixtree
{
	const static int M=27;
	struct P
	{
		int v,w;
	};
	struct Q
	{
		int f,t,v;//t=0: n
	};
	vector<Q> edges;
	vector<vector<P>> e;
	vector<array<int,M>> c;
	vector<int> s,fa,dep,siz;
	int n,point,ds,remain,r,edge;
	bool bd;
	suffixtree():c(2),fa({0,1}),edges(1),e(2)
	{
		n=remain=r=edge=bd=0;
		point=ds=1;
	}
	suffixtree(const string &s):c(2),fa({0,1}),edges(1),e(2)
	{
		n=remain=r=edge=bd=0;
		point=ds=1;
		reserve(s.size());
		for (auto c:s) insert(c-'a');
		insert(26);
	}
	void reserve(int len)
	{
		++len;
		s.reserve(len);
		len=len*2+2;
		c.reserve(len);
		fa.reserve(len);
		e.reserve(len);
	}
	inline void add(int a,int b,int cc,int d)
	{
		assert(edges.size());
		c[a][s[cc]]=edges.size();
		edges.push_back({cc,d,b});
	}
	void insert(int ch)//[0,|S|)
	{
		assert(ds==fa.size()-1&&ds==c.size()-1&&n==s.size()&&ds==e.size()-1);
		assert(ch>=0&&ch<M);
		s.push_back(ch);
		int ad=0;
		++remain;
		while (remain)
		{
			if (!r) edge=n;
			if (int m=c[point][s[edge]];!m)
			{
				assert(!m);
				fa.push_back(1);c.push_back({});e.push_back({});
				fa[ad]=point;
				add(ad=point,++ds,edge,-1);
				e[point].push_back({s[edge]});
				//add(point,s[edge]);
			}
			else
			{
				assert(m);
				auto [f,t,v]=edges[m];
				if (t>=0&&t-f+1<=r)
				{
					assert(t!=n);
					r-=t-f+1;
					edge+=t-f+1;
					point=v;
					continue;
				}
				assert(f+r<=n);
				if (s[f+r]==s[n])
				{
					++r;
					fa[ad]=point;
					break;
				}
				fa.push_back(1);c.push_back({});e.push_back({});
				fa.push_back(1);c.push_back({});e.push_back({});
				fa[ad]=++ds;
				add(ad=ds,v,f+r,t);
				e[ds].push_back({s[n]});
				e[ds].push_back({s[f+r]});
				//add(ds,s[n]);add(ds,s[f+r]);
				++ds;add(ds-1,ds,n,-1); 
				edges[m]={f,f+r-1,ds-1};
			}
			--remain;
			if (r&&point==1)
			{
				--r;
				edge=n-remain+1;
			} else point=fa[point];
		}
		++n;
	}
	void build_edge()
	{
		bd=1;

		//其余信息
		dep.resize(ds+1);
		siz.resize(ds+1);

		int i,j;
		for (i=1;i<=ds;i++) for (auto &[v,w]:e[i])
		{
			j=c[i][v];
			v=edges[j].v;
			w=(edges[j].t>=0?edges[j].t:n-1)-edges[j].f+1;
		}
	}
	void out()
	{
		int i;
		for (i=1;i<=ds;i++) for (int j:c[i]) if (j)
		{
			auto [f,t,v]=edges[j];
			if (t==-1) t=n-1;
			cerr<<i<<' '<<v<<' ';
			//cerr<<i<<" -> "<<v<<": ";
			for (int k=f;k<=t;k++) cerr<<char('a'+s[k]);
			cerr<<endl;
		}
	}
	ll ans;
	void dfs(int u)
	{
		assert(bd);
		++ans;
		for (auto [v,w]:e[u])
		{
			//dep[v]=dep[u]+w;
			dfs(v);
			ans+=w-1;
		}
	}
	ll fun()
	{
		ans=0;
		build_edge();
		dfs(1);
		return ans-n;
	}
};
\end{lstlisting}



\subsection{Z 函数}

表示每个后缀和母串的 lcp。

\begin{lstlisting}
struct str
{
	vector<int> z;
	int n;
	str(int *s,int _n)//[1,n]
	{
		n=_n;
		z=vector<int>(n+1,0);
		int i,l,r;
		z[1]=n;
		for (i=2,l=r=0;i<=n;i++)
		{
			if (i<=r) z[i]=min(z[i-l+1],r-i+1);
			while (i+z[i]<=n&&s[i+z[i]]==s[1+z[i]]) ++z[i];
			if (i+z[i]-1>r) l=i,r=i+z[i]-1;
		}
	}
};
\end{lstlisting}



\subsection{最小表示法}

$O(n)$,$O(1)$。

\begin{lstlisting}
template<typename T> void min_order(T *a,int n)//[0,n)
{
	int i,j,k;
	T x,y;
	i=k=0;j=1;
	while (i<n&&j<n&&k<n)
	{
		x=a[(i+k)%n];y=a[(j+k)%n];
		if (x==y) ++k; else
		{
			if (x>y) i+=k+1; else j+=k+1;
			if (i==j) ++j;
			k=0;
		}
	}
	if (j>i) j=i;
	//[j,n)+[0,j)
	rotate(a,a+j,a+n);
}
\end{lstlisting}


\newpage

\section{图论}

\subsection{最小密度环}

$O(nm)$。

\begin{lstlisting}
#include <bits/stdc++.h>
using namespace std;
const int N=3e3+5,M=1e4+5;
const double inf=1e18;
int u[M],v[M];
double f[N][N],w[M];
int main()
{
	ios::sync_with_stdio(0);cin.tie(0);
	cout<<setiosflags(ios::fixed)<<setprecision(8);
	int n,m,i,j;
	cin>>n>>m;
	for (i=1;i<=m;i++) cin>>u[i]>>v[i]>>w[i];
	++n;
	for (i=1;i<=n;i++)
	{
		fill_n(f[i]+1,n,inf);
		for (j=1;j<=m;j++) f[i][v[j]]=min(f[i][v[j]],f[i-1][u[j]]+w[j]);
	}
	double ans=inf;
	for (i=1;i<n;i++) if (f[n][i]!=inf)
	{
		double r=-inf;
		for (j=1;j<n;j++) r=max(r,(f[n][i]-f[j][i])/(n-j));
		ans=min(ans,r);
	}
	cout<<ans<<endl;
}
\end{lstlisting}

\subsection{全源最短路与判负环}

\begin{lstlisting}
#include <bits/stdc++.h>
using namespace std;
typedef long long ll;
typedef pair<int,int> pa;
typedef tuple<int,int,int> tp;
const int N=152;
const ll inf=5e8;
ll dis[N][N],d[N][N];
int main()
{
	ios::sync_with_stdio(0);cin.tie(0);
	while (1)
	{
		int n,m,q,i,j,k;
		cin>>n>>m>>q;
		if (tp(n,m,q)==tp(0,0,0)) return 0;
		for (i=0;i<n;i++) fill_n(dis[i],n,inf*inf);
		for (i=0;i<n;i++) dis[i][i]=0;
		while (m--)
		{
			int u,v,w;
			cin>>u>>v>>w;
			dis[u][v]=min(dis[u][v],(ll)w);
		}
		for (k=0;k<n;k++) for (i=0;i<n;i++) for (j=0;j<n;j++) dis[i][j]=max(min(dis[i][j],dis[i][k]+dis[k][j]),-inf*2);
		for (i=0;i<n;i++) copy_n(dis[i],n,d[i]);
		for (k=0;k<n;k++) for (i=0;i<n;i++) for (j=0;j<n;j++) dis[i][j]=max(min(dis[i][j],dis[i][k]+dis[k][j]),-inf*2);
		while (q--)
		{
			int u,v;
			cin>>u>>v;
			if (d[u][v]>inf) cout<<"Impossible\n"; else if (dis[u][v]!=d[u][v]||d[u][v]<-inf) cout<<"-Infinity\n"; else cout<<d[u][v]<<'\n';
		}
		cout<<'\n';
	}
}
\end{lstlisting}

\subsection{三/四元环计数}

$O(m\sqrt m)$,$O(n+m)$。

注意四元环数的是边四元环。点四元环需要去掉四点完全图个数*2,似乎不太能做?

\begin{lstlisting}
ll triple(const vector<pair<int,int>> &edges)//start from 0
{
	int n=0,i;
	for (auto [u,v]:edges) n=max({n,u,v});
	++n;
	vector d(n,0),id(d),rk(d),cnt(d);
	vector e(n,vector(0,0));
	iota(all(id),0); sort(all(id),[&](int x,int y) { return d[x]<d[y]; });
	for (i=0; i<n; i++) rk[id[i]]=i;
	for (auto [u,v]:edges)
	{
		if (rk[u]>rk[v]) swap(u,v);
		e[u].push_back(v);
	}
	ll ans=0;
	for (i=0; i<n; i++)
	{
		for (int u:e[i]) cnt[u]=1;
		for (int u:e[i]) for (int v:e[u]) ans+=cnt[v];
		for (int u:e[i]) cnt[u]=0;
	}
	return ans;
}
ll quadruple(const vector<pair<int,int>> &edges)
{
	int n=0,i;
	for (auto [u,v]:edges) n=max({n,u,v});
	++n;
	vector d(n,0),id(d),rk(d),cnt(d);
	vector e(n,vector(0,0)),lk(n,vector(0,0));
	for (auto [u,v]:edges) ++d[u],++d[v];
	iota(all(id),0); sort(all(id),[&](int x,int y) { return d[x]<d[y]; });
	for (i=0; i<n; i++) rk[id[i]]=i;
	for (auto [u,v]:edges)
	{
		if (rk[u]>rk[v]) swap(u,v);
		e[u].push_back(v);
		lk[u].push_back(v);
		lk[v].push_back(u);
	}
	ll ans=0;
	for (i=0; i<n; i++)
	{
		for (int u:lk[i]) for (int v:e[u]) if (rk[v]>rk[i]) ans+=cnt[v]++;
		for (int u:lk[i]) for (int v:e[u]) cnt[v]=0;
	}
	return ans;
}
\end{lstlisting}

\subsection{Johnson 全源带负权最短路}

$O(nm\log m)$,$O(n+m)$。

\begin{lstlisting}
for (int u=1;u<=n;u++) for (auto &[v,w]:e[u]) w+=dis[u]-dis[v];
\end{lstlisting}

\subsection{弦图}

单纯点:$v$ 和 $v$ 邻点构成团。

完美消除序列:$v_i$ 在 $\{v_i,v_{i+1},\cdots,v_n\}$ 为单纯点。

$N(v_i)=\{v_j|j>i\land (v_i,v_j)\in E\}$,$next(v_i)$ 为 $N(v_i)$ 最靠前的点。

极大团一定是 $\{v\}\cup N(v)$ 。

最大团大小等于色数。

弦图判定:等价于是否存在完美消除序列。首先求出一个完美消除序列,然后判定是否合法。

判定方法:设 $v_{i+1},\cdots,v_n$ 中与 $v_i$ 相邻的依次为 $v'_1,\cdots,v'_m$。只需判断是否 $v_1'$ 与 $v'_2,\cdots,v'_m$ 相邻。

LexBFS 算法(我不会写)

每个点有一个字符串 label,初始为 $0$。从 $i=n$ 到 $i=1$ 确定,选 label 字典序最大的 $u$,再把 $u$ 邻点的 label 后面接一个 $i$。

最大势算法:从 $v_n$ 求到 $v_1$,设 $label_i$ 表示 $i$ 与多少个已选点相邻,每次选 $label_i$ 最大的点。

弦图极大团:$\{v|\forall next(w)=v,|N(v)|\ge |N(w)|\}$。选出的集合为基本点,按上述极大团构造。

弦图染色:从 $v_n$ 到 $v_1$ 依次选最小可染的色。

最大独立集:从 $v_1$ 到 $v_n$ 能选就选。

最小团覆盖:设最大独立集为 $\{p_m\}$,最小团覆盖为 $\{\{p_i\}\cup N(p_i)\}$。

区间图:两个区间有边当且仅当交集非空。

区间图是弦图。

\subsubsection{代码}

\begin{lstlisting}
namespace chordal_graph//下标从 1 开始
{
	const int N=1e5+2;//点数
	bool ed[N];
	vector<int> e[N];
	int n;
	void init(const vector<pair<int,int>> &edges)
	{
		n=0;
		for (auto [u,v]:edges) n=max({n,u,v});
		for (int i=1;i<=n;i++) e[i].clear();
		for (auto [u,v]:edges) e[u].push_back(v),e[v].push_back(u);
	}
	vector<int> perfect_seq(const vector<pair<int,int>> &edges)//MCS
	{
		init(edges);
		static int d[N];
		static vector<int> buc[N];
		int i,mx=0;
		memset(d+1,0,n*sizeof d[0]);
		memset(ed+1,0,n*sizeof ed[0]);
		for (i=1;i<=n;i++) buc[i].clear();
		buc[0].resize(n);
		iota(all(buc[0]),1);
		vector<int> r(n);
		for (i=n-1;i>=0;i--)
		{
			int u=0;
			while (!u)
			{
				while (buc[mx].size()) if (ed[buc[mx].back()]) buc[mx].pop_back();
				else
				{
					ed[u=buc[mx].back()]=1;
					buc[mx].pop_back();
					goto yes;
				}
				--mx;
			}
			yes:;
			r[i]=u;
			for (int v:e[u]) if (!ed[v]) buc[++d[v]].push_back(v),mx=max(mx,d[v]);
		}
		return r;
	}
	bool check_perfect_seq(vector<int> a)
	{
		static bool ee[N];
		memset(ed+1,0,n*sizeof ed[0]);
		memset(ee+1,0,n*sizeof ee[0]);
		reverse(all(a));
		for (int u:a)
		{
			ed[u]=1;
			int w=0;
			for (int v:e[u]) if (ed[v]) {w=v;break;}
			if (!w) continue;
			ee[w]=1;
			for (int v:e[w]) ee[v]=1;
			for (int v:e[u]) if (ed[v]&&!ee[v]) return 0;
			ee[w]=0;
			for (int v:e[w]) ee[v]=0;
		}
		return 1;
	}
	bool check_chordal(const vector<pair<int,int>> &edges) {return check_perfect_seq(perfect_seq(edges));}
	vector<int> color(int _n,const vector<pair<int,int>> &edges)//返回长度为 _n+1。其中 0 无意义
	{
		auto a=perfect_seq(edges);
		reverse(all(a));
		memset(ed+1,0,n*sizeof ed[0]);
		vector<int> r(_n+1);
		for (int u:a)
		{
			for (int v:e[u]) ed[r[v]]=1;
			int x=1;
			while (ed[x]) ++x;
			r[u]=x;
			for (int v:e[u]) ed[r[v]]=0;
		}
        for (int i=n+1;i<=_n;i++) r[i]=1;
		return r;
	}
	vector<int> max_independent(int _n,const vector<pair<int,int>> &edges)//注意有孤立点这种奇怪东西
	{
		auto a=perfect_seq(edges);
		memset(ed+1,0,n*sizeof ed[0]);
		vector<int> r;
		for (int u:a) if (!ed[u])
		{
			r.push_back(u);
			for (int v:e[u]) ed[v]=1;
		}
		for (int i=n+1;i<=_n;i++) r.push_back(i);
		return r;
	}
}
using chordal_graph::check_chordal,chordal_graph::color,chordal_graph::max_independent;
\end{lstlisting}

\subsection{二分图与网络流建图}

以下约定,若为二分图则 $n,m$ 表示两侧点数,否则仅 $n$ 表示全图点数。

\subsubsection{二分图边染色}

留坑待填。

结论:$\Delta(G)\le \chi'(G) \le \Delta(G)+1$,二分图时 $\chi'(G)=\Delta(G)$。$\Delta(G)$ 为图的最大度。

\subsubsection{二分图最小点集覆盖}

$ans=\text{maxmatch}$,方案如下。

\begin{lstlisting}
#include <bits/stdc++.h>
using namespace std;
const int N=5e3+2;
vector<int> e[N];
int ed[N],lk[N],kl[N],flg[N],now;
bool dfs(int u)
{
	for (int v:e[u]) if (ed[v]!=now)
	{
		ed[v]=now;
		if (!lk[v]||dfs(lk[v])) return lk[v]=u;
	}
	return 0;
}
void dfs2(int u)
{
	for (int v:e[u]) if (!flg[v]) flg[v]=1,dfs2(lk[v]);
}
int main()
{
	int n,m,i,r=0;
	cin>>n>>m;
	while (m--)
	{
		int u,v;
		cin>>u>>v;
		e[u].push_back(v);
	}
	for (i=1;i<=n;i++) dfs(now=i);
	for (i=1;i<=n;i++) kl[lk[i]]=i;
	for (i=1;i<=n;i++) if (!kl[i]) dfs2(i);
	vector<int> A[2];
	for (i=1;i<=n;i++) if (lk[i])
	{
		if (flg[i]) A[1].push_back(i); else A[0].push_back(lk[i]);
	}
	for (int j=0;j<2;j++)
	{
		cout<<A[j].size();
		for (int x:A[j]) cout<<' '<<x;cout<<'\n';
	}
}
\end{lstlisting}

\subsubsection{二分图最大独立集}

$ans=n+m-\text{maxmatch}$,方案是最小点集覆盖的补集。

\subsubsection{二分图最小边覆盖}

$ans=n+m-\text{maxmatch}$,方案是最大匹配加随便一些边。无解当且仅当有孤立点,算法会视为单选孤立点(无边)。

\subsubsection{有向无环图最小不相交链覆盖}

$ans=n-\text{maxmatch}$,其中二分图建图方法是拆入点和出点(实现时直接跑一次二分图就行,不用额外处理),注意**不**需要传递闭包。方案如下。

\begin{lstlisting}
#include <bits/stdc++.h>
using namespace std;
const int N=152;
vector<int> e[N];
int lk[N],kl[N],ed[N],now;
bool dfs(int u)
{
	for (int v:e[u]) if (ed[v]!=now)
	{
		ed[v]=now;
		if (!lk[v]||dfs(lk[v])) return lk[v]=u;
	}
	return 0;
}
int main()
{
	int n,m,i;
	ios::sync_with_stdio(0);cin.tie(0);
	cin>>n>>m;
	while (m--)
	{
		int u,v;
		cin>>u>>v;
		e[u].push_back(v);
	}
	int r=0;
	for (i=1;i<=n;i++) r+=dfs(now=i);
	for (i=1;i<=n;i++) kl[lk[i]]=i;
	for (i=1;i<=n;i++) if (ed[i]!=-1&&!lk[i])
	{
		vector<int> ans;
		int u=i;
		while (u)
		{
			ed[u]=-1;
			ans.push_back(u);
			u=kl[u];
		}
		for (int j=0;j<ans.size();j++) cout<<ans[j]<<" \n"[j+1==ans.size()];
	}
	cout<<n-r<<endl;
}
\end{lstlisting}

\subsubsection{有向无环图最大互不可达集}

$ans=n-\text{maxmatch}$,其中二分图建图方法是拆入点和出点(实现时直接跑一次二分图就行,不用额外处理),注意**需要**传递闭包。方案?

\subsubsection{最大权闭合子图}

若 $v_i>0$,$s\to i$ 流量 $v_i$;若 $v_i<0$,$i\to t$ 流量 $-v_i$。若原图 $u\to v$ 可花费 $w$ 代价违抗,流量 $w$,否则 $+\infty$ 。答案为 $\sum\limits_{v_i>0} v_i-\text{maxflow}$。方案?

\subsection{二分图匹配(时间戳写法)}

\begin{lstlisting}
bool dfs(int u)
{
	for (int v:e[u]) if (ed[v]!=now)
	{
		ed[v]=now;
		if (!lk[v]||dfs(lk[v])) return lk[v]=u;
	}
	return 0;
}
\end{lstlisting}

\subsection{二分图最大权匹配}

\begin{lstlisting}
namespace KM
{
	const int N=405;//点数
	typedef long long ll;//答案范围
	const ll inf=1e16;
	int lk[N],kl[N],pre[N],q[N],n,h,t;
	ll sl[N],e[N][N],lx[N],ly[N];
	bool edx[N],edy[N];
	bool ck(int v)
	{
		if (edy[v]=1,kl[v]) return edx[q[++t]=kl[v]]=1;
		while (v) swap(v,lk[kl[v]=pre[v]]);
		return 0;
	}
	void bfs(int u)
	{
		fill_n(sl+1,n,inf);
		memset(edx+1,0,n*sizeof edx[0]);
		memset(edy+1,0,n*sizeof edy[0]);
		q[h=t=1]=u;edx[u]=1;
		while (1)
		{
			while (h<=t)
			{
				int u=q[h++],v;
				ll d;
				for (v=1;v<=n;v++) if (!edy[v]&&sl[v]>=(d=lx[u]+ly[v]-e[u][v])) if (pre[v]=u,d) sl[v]=d; else if (!ck(v)) return;
			}
			int i;
			ll m=inf;
			for (i=1;i<=n;i++) if (!edy[i]) m=min(m,sl[i]);
			for (i=1;i<=n;i++)
			{
				if (edx[i]) lx[i]-=m;
				if (edy[i]) ly[i]+=m; else sl[i]-=m;
			}
			for (i=1;i<=n;i++) if (!edy[i]&&!sl[i]&&!ck(i)) return;
		}
	}
	template<typename TT> ll max_weighted_match(int N,const vector<tuple<int,int,TT>> &edges)//lk[[1,n]]->[1,n]
	{
		int i;n=N;
		memset(lk+1,0,n*sizeof lk[0]);
		memset(kl+1,0,n*sizeof kl[0]);
		memset(ly+1,0,n*sizeof ly[0]);
		for (i=1;i<=n;i++) fill_n(e[i]+1,n,0);//若不需保证匹配边最多,置 0 即可,否则 -inf/N
		for (auto [u,v,w]:edges) e[u][v]=max(e[u][v],(ll)w);
		for (i=1;i<=n;i++) lx[i]=*max_element(e[i]+1,e[i]+n+1);
		for (i=1;i<=n;i++) bfs(i);
		ll r=0;
		for (i=1;i<=n;i++) r+=e[i][lk[i]];
		return r;
	}
}
using KM::max_weighted_match,KM::lk,KM::kl,KM::e;
\end{lstlisting}

\subsection{一般图最大匹配}

\begin{lstlisting}
namespace blossom_tree
{
	const int N=1005;
	vector<int> e[N];
	int lk[N],rt[N],f[N],dfn[N],typ[N],q[N];
	int id,h,t,n;
	int lca(int u,int v)
	{
		++id;
		while (1)
		{
			if (u)
			{
				if (dfn[u]==id) return u;
				dfn[u]=id;u=rt[f[lk[u]]];
			}
			swap(u,v);
		}
	}
	void blm(int u,int v,int a)
	{
		while (rt[u]!=a)
		{
			f[u]=v;
			v=lk[u];
			if (typ[v]==1) typ[q[++t]=v]=0;
			rt[u]=rt[v]=a;
			u=f[v];
		}
	}
	void aug(int u)
	{
		while (u)
		{
			int v=lk[f[u]];
			lk[lk[u]=f[u]]=u;
			u=v;
		}
	}
	void bfs(int root)
	{
		memset(typ+1,-1,n*sizeof typ[0]);
		iota(rt+1,rt+n+1,1);
		typ[q[h=t=1]=root]=0;
		while (h<=t)
		{
			int u=q[h++];
			for (int v:e[u])
			{
				if (typ[v]==-1)
				{
					typ[v]=1;f[v]=u;
					if (!lk[v]) return aug(v);
					typ[q[++t]=lk[v]]=0;
				} else if (!typ[v]&&rt[u]!=rt[v])
				{
					int a=lca(rt[u],rt[v]);
					blm(v,u,a);blm(u,v,a);
				}
			} 
		}
	}
	int max_general_match(int N,vector<pair<int,int>> edges)//[1,n]
	{
		n=N;id=0;
		memset(f+1,0,n*sizeof f[0]);
		memset(dfn+1,0,n*sizeof dfn[0]);
		memset(lk+1,0,n*sizeof lk[0]);
		int i;
		for (i=1;i<=n;i++) e[i].clear();
		mt19937 rnd(114);
		shuffle(all(edges),rnd);
		for (auto [u,v]:edges)
		{
			e[u].push_back(v),e[v].push_back(u);
			if (!(lk[u]||lk[v])) lk[u]=v,lk[v]=u;
		}
		int r=0;
		for (i=1;i<=n;i++) if (!lk[i]) bfs(i);
		for (i=1;i<=n;i++) r+=!!lk[i];
		return r/2;
	}
}
using blossom_tree::max_general_match,blossom_tree::lk;
\end{lstlisting}

\subsection{一般图最大权匹配}

$n=400$:UOJ 600ms, Luogu 135ms

\begin{lstlisting}
#include<bits/stdc++.h>
using namespace std;
#define all(x) (x).begin(),(x).end()
namespace weighted_blossom_tree
{
	#define d(x) (lab[x.u]+lab[x.v]-e[x.u][x.v].w*2)
	const int N=403*2;//两倍点数
	typedef long long ll;//总和大小
	typedef int T;//权值大小
	//均不允许无符号
	const T inf=numeric_limits<int>::max()>>1;
	struct Q
	{
		int u,v;
		T w;
	} e[N][N];
	T lab[N];
	int n,m=0,id,h,t,lk[N],sl[N],st[N],f[N],b[N][N],s[N],ed[N],q[N];
	vector<int> p[N];
	void upd(int u,int v) {if (!sl[v]||d(e[u][v])<d(e[sl[v]][v])) sl[v]=u;}
	void ss(int v)
	{
		sl[v]=0;
		for (int u=1;u<=n;u++) if (e[u][v].w>0&&st[u]!=v&&!s[st[u]]) upd(u,v);
	}
	void ins(int u) {if (u<=n) q[++t]=u; else for (int v:p[u]) ins(v);}
	void mdf(int u,int w)
	{
		st[u]=w;
		if (u>n) for (int v:p[u]) mdf(v,w);
	}
	int gr(int u,int v)
	{
		if ((v=find(all(p[u]),v)-p[u].begin())&1)
		{
			reverse(1+all(p[u]));
			return (int)p[u].size()-v;
		}
		return v;
	}
	void stm(int u,int v)
	{
		lk[u]=e[u][v].v;
		if (u<=n) return;
		Q w=e[u][v];
		int x=b[u][w.u],y=gr(u,x),i;
		for (i=0;i<y;i++) stm(p[u][i],p[u][i^1]);
		stm(x,v);
		rotate(p[u].begin(),y+all(p[u]));
	}
	void aug(int u,int v)
	{
		int w=st[lk[u]];
		stm(u,v);
		if (!w) return;
		stm(w,st[f[w]]);
		aug(st[f[w]],w);
	}
	int lca(int u,int v)
	{
		for (++id;u|v;swap(u,v))
		{
			if (!u) continue;
			if (ed[u]==id) return u;
			ed[u]=id;//??????????v?? 这是原作者的注释,我也不知道是啥
			if (u=st[lk[u]]) u=st[f[u]];
		}
		return 0;
	}
	void add(int u,int a,int v)
	{
		int x=n+1,i,j;
		while (x<=m&&st[x]) ++x;
		if (x>m) ++m;
		lab[x]=s[x]=st[x]=0;lk[x]=lk[a];
		p[x].clear();p[x].push_back(a);
		for (i=u;i!=a;i=st[f[j]]) p[x].push_back(i),p[x].push_back(j=st[lk[i]]),ins(j);//复制,改一处
		reverse(1+all(p[x]));
		for (i=v;i!=a;i=st[f[j]]) p[x].push_back(i),p[x].push_back(j=st[lk[i]]),ins(j);
		mdf(x,x);
		for (i=1;i<=m;i++) e[x][i].w=e[i][x].w=0;
		memset(b[x]+1,0,n*sizeof b[0][0]);
		for (int u:p[x])
		{
			for (v=1;v<=m;v++) if (!e[x][v].w||d(e[u][v])<d(e[x][v])) e[x][v]=e[u][v],e[v][x]=e[v][u];
			for (v=1;v<=n;v++) if (b[u][v]) b[x][v]=u;
		}
		ss(x);
	}
	void ex(int u)  // s[u] == 1
	{
		for (int x:p[u]) mdf(x,x);
		int a=b[u][e[u][f[u]].u],r=gr(u,a),i;
		for (i=0;i<r;i+=2)
		{
			int x=p[u][i],y=p[u][i+1];
			f[x]=e[y][x].u;
			s[x]=1;s[y]=0;
			sl[x]=0;ss(y);
			ins(y);
		}
		s[a]=1;f[a]=f[u];
		for (i=r+1;i<p[u].size();i++) s[p[u][i]]=-1,ss(p[u][i]);
		st[u]=0;
	}
	bool on(const Q &e)
	{
		int u=st[e.u],v=st[e.v],a;
		if(s[v]==-1)
		{
			f[v]=e.u;s[v]=1;
			a=st[lk[v]];
			sl[v]=sl[a]=s[a]=0;
			ins(a);
		}
		else if(!s[v])
		{
			a=lca(u,v);
			if (!a) return aug(u,v),aug(v,u),1;
			else add(u,a,v);
		}
		return 0;
	}
	bool bfs()
	{
		memset(s+1,-1,m*sizeof s[0]);
		memset(sl+1,0,m*sizeof sl[0]);
		h=1;t=0;
		int i,j;
		for (i=1;i<=m;i++) if (st[i]==i&&!lk[i]) f[i]=s[i]=0,ins(i);
		if (h>t) return 0;
		while (1)
		{
			while (h<=t)
			{
				int u=q[h++],v;
				if (s[st[u]]!=1) for (v=1; v<=n;v++) if (e[u][v].w>0&&st[u]!=st[v])
				{
					if (d(e[u][v])) upd(u,st[v]); else if (on(e[u][v])) return 1;
				}
			}
			T x=inf;
			for (i=n+1;i<=m;i++) if (st[i]==i&&s[i]==1) x=min(x,lab[i]>>1);
			for (i=1;i<=m;i++) if (st[i]==i&&sl[i]&&s[i]!=1) x=min(x,d(e[sl[i]][i])>>s[i]+1);
			for (i=1;i<=n;i++) if (~s[st[i]]) if ((lab[i]+=(s[st[i]]*2-1)*x)<=0) return 0;
			for (i=n+1;i<=m;i++) if (st[i]==i&&~s[st[i]]) lab[i]+=(2-s[st[i]]*4)*x;
			h=1;t=0;
			for (i=1;i<=m;i++) if (st[i]==i&&sl[i]&&st[sl[i]]!=i&&!d(e[sl[i]][i])&&on(e[sl[i]][i])) return 1;
			for (i=n+1;i<=m;i++) if (st[i]==i&&s[i]==1&&!lab[i]) ex(i);
		}
		return 0;
	}
	template<typename TT> ll max_weighted_general_match(int N,const vector<tuple<int,int,TT>> &edges)//[1,n],返回权值
	{
		memset(ed+1,0,m*sizeof ed[0]);
		memset(lk+1,0,m*sizeof lk[0]);
		n=m=N;id=0;
		iota(st+1,st+n+1,1);
		int i,j;
		T wm=0;
		ll r=0;
		for (i=1;i<=n;i++) for (j=1;j<=n;j++) e[i][j]={i,j,0};
		for (auto [u,v,w]:edges) wm=max(wm,e[v][u].w=e[u][v].w=max(e[u][v].w,(T)w));
		for (i=1;i<=n;i++) p[i].clear();
		for (i=1;i<=n;i++) for (j=1;j<=n;j++) b[i][j]=i*(i==j);
		fill_n(lab+1,n,wm);
		while (bfs());
		for (i=1;i<=n;i++) if (lk[i]) r+=e[i][lk[i]].w;
		return r/2;
	}
	#undef d
}
using weighted_blossom_tree::max_weighted_general_match,weighted_blossom_tree::lk;
int main()
{
	ios::sync_with_stdio(0);cin.tie(0);
	int n,m;
	cin>>n>>m;
	vector<tuple<int,int,long long>> edges(m);
	for (auto &[u,v,w]:edges) cin>>u>>v>>w;
	cout<<max_weighted_general_match(n,edges)<<'\n';
	for (int i=1;i<=n;i++) cout<<lk[i]<<" \n"[i==n];
}
\end{lstlisting}

\subsection{网络流代码}

\begin{lstlisting}
namespace net
{
	const int N=4e5+50;//number of points
	namespace flow
	{
		typedef ll wT;//single flow
		typedef ll cT;//total flow
		const cT inf=numeric_limits<cT>::max()/2;//maximum
		struct Q
		{
			int v;
			wT w;
			int id;
		};
		vector<Q> e[N];
		int fc[N],q[N];
		int n,s,t;
		int bfs()
		{
			fill_n(fc,n,0);
			int p1=0,p2=0,u;
			fc[s]=1; q[0]=s;
			while (p1<=p2)
			{
				int u=q[p1++];
				for (auto [v,w,id]:e[u]) if (w&&!fc[v]) fc[q[++p2]=v]=fc[u]+1;
			}
			return fc[t];
		}
		cT dfs(int u,cT maxf)
		{
			if (u==t) return maxf;
			cT j=0,k;
			for (auto &[v,w,id]:e[u]) if (w&&fc[v]==fc[u]+1&&(k=dfs(v,min(maxf-j,(cT)w))))
			{
				j+=k;
				w-=k;
				e[v][id].w+=k;
				if (j==maxf) return j;
			}
			fc[u]=0;
			return j;
		}
		cT max_flow(const vector<tuple<int,int,wT>> &edges,int S,int T)//[0,n]
		{
			s=S; t=T; n=max(s,t);
			for (auto [u,v,w]:edges) n=max({n,u,v});
			++n;
			assert(n<N);
			for (int i=0; i<n; i++) e[i].clear();
			for (auto [u,v,w]:edges) if (u!=v)
			{
				e[u].push_back({v,w,(int)e[v].size()});
				e[v].push_back({u,0,(int)e[u].size()-1});
			}
			cT r=0;
			while (bfs()) r+=dfs(s,inf);
			return r;
		}
	}
	using flow::max_flow,flow::fc;
	namespace match
	{
		int lk[N];
		int max_match(int n,int m,const vector<pair<int,int>> &edges)//lk[[0,n]]->[0,m]
		{
			++n; ++m;
			assert(max(n,m)<N);
			int s=n+m,t=n+m+1,i;
			vector<tuple<int,int,ll>> eg;
			eg.reserve(n+m+edges.size());
			for (i=0; i<n; i++) eg.push_back({s,i,1});
			for (i=0; i<m; i++) eg.push_back({i+n,t,1});
			for (auto [u,v]:edges) eg.push_back({u,v+n,1});
			int r=max_flow(eg,s,t);
			fill_n(lk,n,-1);
			for (i=0; i<n; i++) for (auto [v,w,id]:flow::e[i]) if (v<s&&!w) { lk[i]=v-n; break; }
			return r;
		}
		bool ed[N];
		int kl[N];
		vector<int> e[N];
		void dfs(int u)
		{
			for (int v:e[u]) if (!ed[v]) ed[v]=1,dfs(kl[v]);
		}
		pair<vector<int>,vector<int>> min_cover(int n,int m,const vector<pair<int,int>> &edges)//[0,n]-[0,m]
		{
			max_match(n,m,edges);
			++n; ++m;
			fill_n(kl,m,-1); fill_n(ed,m,0);
			int i;
			for (i=0; i<n; i++)
			{
				e[i].clear();
				if (lk[i]!=-1) kl[lk[i]]=i;
			}
			for (auto [u,v]:edges) e[u].push_back(v);
			for (i=0; i<n; i++) if (lk[i]==-1) dfs(i);
			vector<int> r[2];
			for (i=0; i<m; i++) if (kl[i]!=-1)
			{
				if (ed[i]) r[1].push_back(i); else r[0].push_back(kl[i]);
			}
			sort(all(r[0]));
			return {r[0],r[1]};
		}
	}
	using match::max_match,match::min_cover,match::lk,match::kl;
	namespace cost_flow
	{
		typedef ll wT;
		typedef ll cT;
		const cT inf=numeric_limits<cT>::max()/2;
		struct Q
		{
			int v;
			wT w;
			cT c;
			int id;
		};
		vector<Q> e[N];
		cT dis[N];
		int pre[N],pid[N],ipd[N];
		bool ed[N];
		int n,s,t;
		pair<wT,cT> spfa()
		{
			queue<int> q;
			fill_n(dis,n,inf);
			memset(ed,0,n*sizeof ed[0]);
			q.push(s); dis[s]=0;
			while (q.size())
			{
				int u=q.front(); q.pop(); ed[u]=0;
				for (auto [v,w,c,id]:e[u]) if (w&&dis[v]>dis[u]+c)
				{
					dis[v]=dis[u]+c;
					pre[v]=u;
					pid[v]=e[v][id].id;
					ipd[v]=id;
					if (!ed[v]) q.push(v),ed[v]=1;
				}
			}
			if (dis[t]==inf) return {0,0};
			wT mw=numeric_limits<wT>::max();
			for (int i=t; i!=s; i=pre[i]) mw=min(mw,e[pre[i]][pid[i]].w);
			for (int i=t; i!=s; i=pre[i]) e[pre[i]][pid[i]].w-=mw,e[i][ipd[i]].w+=mw;
			return {mw,(cT)mw*dis[t]};
		}
		pair<wT,cT> mcmf_spfa(const vector<tuple<int,int,wT,cT>> &edges,int S,int T)//[0,n]
		{
			s=S; t=T; n=max(s,t);
			for (auto [u,v,w,c]:edges) n=max({n,u,v});
			++n;
			assert(n<N);
			for (int i=0; i<n; i++) e[i].clear();
			for (auto [u,v,w,c]:edges) if (u!=v)
			{
				e[u].push_back({v,w,c,(int)e[v].size()});
				e[v].push_back({u,0,-c,(int)e[u].size()-1});
			}
			pair<wT,cT> r{0,0},rr;
			while ((rr=spfa()).first) r={r.first+rr.first,r.second+rr.second};
			return r;
		}
		pair<wT,cT> mcmf_dijk(const vector<tuple<int,int,wT,cT>> &edges,int S,int T)//[0,n]
		{
			s=S; t=T; n=max(s,t);
			for (auto [u,v,w,c]:edges) n=max({n,u,v});
			++n;
			assert(n<N);
			for (int i=0; i<n; i++) e[i].clear();
			for (auto [u,v,w,c]:edges) if (u!=v)
			{
				e[u].push_back({v,w,c,(int)e[v].size()});
				e[v].push_back({u,0,-c,(int)e[u].size()-1});
			}
			static cT h[N];
			auto get_h=[&]()
			{
				fill_n(h,n,inf);
				memset(ed,0,n*sizeof ed[0]);
				queue<int> q;
				q.push(s); h[s]=0;
				while (q.size())
				{
					int u=q.front(); q.pop(); ed[u]=0;
					for (auto [v,w,c,id]:e[u]) if (w&&h[v]>h[u]+c)
					{
						h[v]=h[u]+c;
						if (!ed[v]) q.push(v),ed[v]=1;
					}
				}
				return;
			};
			auto dijkstra=[&]() -> pair<wT,cT>
			{
				static int fl[N],zl[N];
				int i;
				memset(ed,0,n*sizeof ed[0]);
				fill_n(dis,n,inf);
				typedef pair<cT,int> pa;
				priority_queue<pa,vector<pa>,greater<pa>> q;
				dis[s]=0; q.push({0,s});
				while (q.size())
				{
					int u=q.top().second;
					q.pop(); ed[u]=1;
					i=0;
					for (auto [v,w,c,id]:e[u])
					{
						if (w&&dis[v]>dis[u]+c) fl[v]=id,zl[v]=i,q.push({dis[v]=dis[pre[v]=u]+c,v});
						++i;
					}
					while (q.size()&&ed[q.top().second]) q.pop();
				}
				if (dis[t]==inf) return {0,0};
				wT tf=numeric_limits<wT>::max();
				for (i=t; i!=s; i=pre[i]) tf=min(tf,e[pre[i]][zl[i]].w);
				for (i=t; i!=s; i=pre[i]) e[pre[i]][zl[i]].w-=tf,e[i][fl[i]].w+=tf;
				for (int u=0; u<n; u++) for (auto &[v,w,c,id]:e[u]) c+=dis[u]-dis[v];
				return {tf,tf*(h[t]+=dis[t])};
			};
			get_h();
			for (int u=0; u<n; u++) for (auto &[v,w,c,id]:e[u]) c+=h[u]-h[v];
			pair<wT,cT> r{0,0},rr;
			while ((rr=dijkstra()).first) r={r.first+rr.first,r.second+rr.second};
			return r;
		}
	}
	using cost_flow::mcmf_spfa,cost_flow::mcmf_dijk;
	namespace bounded_flow
	{
		typedef ll wT;//single flow
		typedef ll cT;//total flow
		bool valid_flow(const vector<tuple<int,int,wT,wT>> &edges)//方案需加上 l
		{
			if (!edges.size()) return 1;
			int n=0,i;
			cT tot=0;
			for (auto [u,v,l,r]:edges)
			{
				n=max({n,u,v});
				if (l>r) return 0;
			}
			++n;
			static cT cd[N];
			memset(cd,0,n*sizeof cd[0]);
			for (auto [u,v,l,r]:edges) cd[u]+=l,cd[v]-=l;
			vector<tuple<int,int,wT>> eg;
			eg.reserve(n+edges.size());
			for (i=0; i<n; i++) if (cd[i]>0) eg.push_back({i,n+1,cd[i]}),tot+=cd[i];
			else if (cd[i]<0) eg.push_back({n,i,-cd[i]});
			for (auto [u,v,l,r]:edges) eg.push_back({u,v,r-l});
			return tot==flow::max_flow(eg,n,n+1);
		}
		cT valid_flow_st(vector<tuple<int,int,wT,wT>> edges,int s,int t)//-1 invalid, wT=cT
		{
			int n=max(s,t);
			cT tot=0;
			for (auto [u,v,l,r]:edges) n=max({n,u,v}),tot+=(u==s)*r;
			++n;
			edges.push_back({t,s,0,tot});
			if (!valid_flow(edges)) return -1;
			assert(flow::e[s].back().v==t);
			assert(flow::e[t].back().v==s);
			return tot-flow::e[t].back().w;
		}
		cT valid_max_flow(const vector<tuple<int,int,wT,wT>> &edges,int s,int t)//-1 invalid, wT=cT
		{
			cT r=valid_flow_st(edges,s,t);
			if (r<0) return r;
			flow::s=s; flow::t=t;
			flow::e[s].pop_back(); flow::e[t].pop_back();
			while (flow::bfs()) r+=flow::dfs(s,flow::inf);
			return r;
		}
		cT valid_min_flow(const vector<tuple<int,int,wT,wT>> &edges,int s,int t)//-1 invalid, wT=cT
		{
			cT r=valid_flow_st(edges,s,t);
			if (r<0) return r;
			flow::s=t; flow::t=s;
			flow::e[s].pop_back(); flow::e[t].pop_back();
			while (flow::bfs()) r-=flow::dfs(t,flow::inf);
			return r;
		}//not check
	}
	using bounded_flow::valid_flow,bounded_flow::valid_flow_st,bounded_flow::valid_max_flow,bounded_flow::valid_min_flow;
	namespace bounded_cost_flow
	{
		pair<ll,ll> valid_mcf(const vector<tuple<int,int,ll,ll,ll>> &edges,int s,int t)//[u,v,l,r,c],mincost flow
		{
			int n=max(s,t);
			for (auto [u,v,l,r,c]:edges) n=max({n,u,v});
			++n;
			int ss=n,tt=n+1;
			static ll cd[N];
			memset(cd,0,n*sizeof cd[0]);
			for (auto [u,v,l,r,c]:edges) cd[u]+=l,cd[v]-=l;
			vector<tuple<int,int,ll,ll>> e;
			ll t1=0,t2=0;
			for (int i=0; i<n; i++) if (cd[i]>0) e.push_back({i,tt,cd[i],0}),t2+=cd[i];
			else if (cd[i]<0) e.push_back({ss,i,-cd[i],0});
			for (auto [u,v,l,r,c]:edges) e.push_back({u,v,r-l,c});
			for (auto [u,v,w,c]:e) t1+=(u==s)*w;
			e.push_back({t,s,t1,0});
			auto res=mcmf_spfa(e,ss,tt);//checked dijk
			if (res.first!=t2) return {-1,-1};
			res.first=cost_flow::e[s].back().w;
			for (auto [u,v,l,r,c]:edges) res.second+=l*c;
			return res;
		}
		pair<ll,ll> valid_mcmf(const vector<tuple<int,int,ll,ll,ll>> &edges,int s,int t)//[u,v,l,r,c],mincost max_flow
		{
			auto r=valid_mcf(edges,s,t);
			if (r.first<0) return {-1,-1};
			cost_flow::e[s].pop_back();
			cost_flow::e[t].pop_back();
			cost_flow::s=s; cost_flow::t=t;
			pair<ll,ll> rr;
			while ((rr=cost_flow::spfa()).first) r={r.first+rr.first,r.second+rr.second};//spfa ver. not checked dijk
			return r;
		}
	}
	using bounded_cost_flow::valid_mcf,bounded_cost_flow::valid_mcmf;
	namespace ne_cost_flow
	{
		pair<ll,ll> ne_mcmf(const vector<tuple<int,int,ll,ll>> &edges,int s,int t)
		{
			vector<tuple<int,int,ll,ll,ll>> e;
			for (auto [u,v,w,c]:edges) if (c>=0) e.push_back({u,v,0,w,c}); else
			{
				e.push_back({u,v,w,w,c});
				e.push_back({v,u,0,w,-c});
			}
			return valid_mcmf(e,s,t);
		}
	}
	using ne_cost_flow::ne_mcmf;
}
\end{lstlisting}

\subsection{费用流(SPFA)}

\begin{lstlisting}
bool dfs()
{
	memset(jl,-0x3f,sizeof(jl));
	jl[dl[tou=wei=1]=0]=0;
	while (tou<=wei)
	{
		ed[x=dl[tou++]]=0;
		for (i=fir[x];i;i=nxt[i]) if ((lj[i][1])&&(jl[lj[i][0]]<jl[x]+lj[i][2]))
		{
			jl[lj[i][0]]=jl[x]+lj[i][2];
			qq[lj[i][0]]=x;
			dy[lj[i][0]]=i;
			if (!ed[lj[i][0]]) ed[dl[++wei]=lj[i][0]]=1;
		}
	}
	zg=m;
	if (jl[t]==jl[t+1]) return 0;
	for (i=t;i;i=qq[i]) zg=min(zg,lj[dy[i]][1]);
	for (i=t;i;i=qq[i])
	{
		lj[dy[i]][1]-=zg;
		ans+=zg*lj[dy[i]][2];
		if (dy[i]&1) lj[dy[i]+1][1]+=zg; else lj[dy[i]-1][1]+=zg;
	}
	return 1;
}
while (dfs());
\end{lstlisting}

\subsection{费用流(Dijkstra)}

\begin{lstlisting}
priority_queue<pa,vector<pa>,greater<pa> > heap;
const int N=5e3+2,M=1e5+2;
pa ans;
int lj[M][3],nxt[M],fir[N],dis[N],h[N],pre[N],fl[N];
int n,m,s,t,bs,x,y,z,w,ans1,ans2;
bool ed[N];
void add(const int u,const int v,const int x,const int y)
{
	lj[++bs][0]=v;
	lj[bs][1]=x;
	lj[bs][2]=y;
	nxt[bs]=fir[u];
	fir[u]=bs;
	lj[++bs][0]=u;
	lj[bs][1]=0;
	lj[bs][2]=-y;
	nxt[bs]=fir[v];
	fir[v]=bs;
}
void spfa()//本题中用dijkstra代替
{
	int x,i,j;
	memset(h,0x3f,sizeof(h));h[s]=0;
	heap.push(make_pair(0,s));
	while (!heap.empty())
	{
		ed[x=heap.top().second]=1;heap.pop();
		for (i=fir[x];i;i=nxt[i]) if ((lj[i][1])&&(h[lj[i][0]]>h[x]+lj[i][2]))
			heap.push(make_pair(h[lj[i][0]]=h[x]+lj[i][2],lj[i][0]));
		while ((!heap.empty())&&(ed[heap.top().second])) heap.pop();
	}
	for (i=1;i<=n;i++) for (j=fir[i];j;j=nxt[j]) lj[j][2]+=h[i]-h[lj[j][0]];
	memset(ed,0,sizeof(ed));
}
pa dijkstra()
{
	int i,j,x,tf=1e9;
	memset(dis,0x3f,sizeof(dis));memset(pre,0,sizeof(pre));dis[s]=0;heap.push(make_pair(0,s));
	while (!heap.empty())
	{
		ed[x=heap.top().second]=1;heap.pop();
		for (i=fir[x];i;i=nxt[i]) if ((lj[i][1])&&(dis[lj[i][0]]>dis[x]+lj[i][2]))
			heap.push(make_pair(dis[lj[i][0]]=dis[pre[lj[i][0]]=x]+lj[i][2],lj[i][0])),fl[lj[i][0]]=i;
		while ((!heap.empty())&&(ed[heap.top().second])) heap.pop();
	}
	if (dis[t]==dis[t+1]) return make_pair(0,0);
	for (i=t;i!=s;i=pre[i]) tf=min(tf,lj[fl[i]][1]);
	for (i=t;i!=s;i=pre[i]) lj[fl[i]][1]-=tf,lj[fl[i]^1][1]+=tf;
	for (i=1;i<=n;i++) for (j=fir[i];j;j=nxt[j]) lj[j][2]+=dis[i]-dis[lj[j][0]];
	h[t]+=dis[t];memset(ed,0,sizeof(ed));
	return make_pair(tf,tf*h[t]);
}
signed main()
{
	while (!heap.empty()) heap.pop();
	read(n);read(m);read(s);read(t);bs=1;
	while (m--)
	{
		read(x);read(y);read(z);read(w);
		add(x,y,z,w);
	}
	spfa();
	while ((ans=dijkstra()).first) ans1+=ans.first,ans2+=ans.second;
	printf("%d %d",ans1,ans2);
}
\end{lstlisting}



\subsection{假花树}

\begin{lstlisting}
vector<int> lj[N];
int lk[N],ed[N];
int n,m,cnt,i,t,x,y,ans,la;
bool dfs(int x)
{
	ed[x]=cnt;int v;
	random_shuffle(lj[x].begin(),lj[x].end());
	for (auto u:lj[x]) if (ed[v=lk[u]]!=cnt)
	{
		lk[v]=0,lk[u]=x,lk[x]=u;
		if (!v||dfs(v)) return 1;
		lk[v]=u,lk[u]=v,lk[x]=0;
	}
	return 0;
}
int main()
{
	srand(time(0));la=-1;
	read(n);read(m);
	while (m--) read(x),read(y),lj[x].push_back(y),lj[y].push_back(x);
	while (la!=ans)
	{
		memset(ed+1,0,n<<2);la=ans;
		for (i=1;i<=n;i++) if (!lk[i]) ans+=dfs(cnt=i);
	}
	printf("%d\n",ans);
	for (i=1;i<=n;i++) printf("%d ",lk[i]);
}
\end{lstlisting}

\subsection{Stoer-Wagner 全局最小割}

$O(n^3)$。可优化到 $O(nm\log n)$。

\begin{lstlisting}
namespace StoerWagner
{
	const int N=602;//点数
	typedef int T;//边权和
	T e[N][N],w[N];
	int ed[N],p[N],f[N];//f 仅输出方案用
	int getf(int u){return f[u]==u?u:f[u]=getf(f[u]);}
	template<typename TT> pair<T,vector<int>> mincut(int n,const vector<tuple<int,int,TT>> &edges)//[1,n],返回某一半点集
	{
		vector<int> ans;ans.reserve(n);
		int i,j,m;
		T r;
		r=numeric_limits<T>::max();
		for (i=1;i<=n;i++) memset(e[i]+1,0,n*sizeof e[0][0]);
		for (auto [u,v,w]:edges) e[u][v]+=w,e[v][u]+=w;
		fill_n(ed+1,n,0);
		iota(f+1,f+n+1,1);
		for (m=n;m>1;m--)
		{
			fill_n(w+1,n,0);
			for (i=1;i<=n;i++) ed[i]&=2;
			for (i=1;i<=m;i++)
			{
				int x=0;
				for (j=1;j<=n;j++) if (!ed[j]) break;x=j;
				for (j++;j<=n;j++) if (!ed[j]*w[j]>w[x]) x=j;
				ed[p[i]=x]=1;
				for (j=1;j<=n;j++) w[j]+=!ed[j]*e[x][j];
			}
			int s=p[m-1],t=p[m];
			if (r>w[t])
			{
				r=w[t];ans.clear();
				for (i=1;i<=n;i++) if (getf(i)==getf(t)) ans.push_back(i);
			}
			for (i=1;i<=n;i++) e[i][s]=e[s][i]+=e[t][i];
			ed[t]=2;
			f[getf(s)]=getf(t);
		}
		return {r,ans};
	}
}
\end{lstlisting}

\subsection{点双}

$O(n+m)$,$O(n+m)$。

ans 存放每个点双包含的边。ct 为 $1$ 表示是割点。没有自环。

\begin{lstlisting}
struct Q
{
	int v,w;
};
vector<vector<int>> ans;
vector<int> cur;
vector<Q> e[N];
int dfn[N],low[N],ct[N],st[N];
bool ed[N],eed[N];
int id,tp;
void dfs(int u,bool rt)
{
	dfn[u]=low[u]=++id;
	int cnt=0;
	for (auto [v,w]:e[u]) if (!ed[w])
	{
		st[tp++]=w;ed[w]=1;
		if (dfn[v]) low[u]=min(low[u],dfn[v]);
		else
		{
			dfs(v,0);
			++cnt;
			low[u]=min(low[u],low[v]);
			if (dfn[u]<=low[v])
			{
				ct[u]=cnt>rt;
				cur.clear();
				do cur.push_back(st[--tp]); while (st[tp]!=w);
				ans.push_back(cur);
			}
		}
	}
}
int main()
{
	ios::sync_with_stdio(0);cin.tie(0);
	int n,m,i;
	cin>>n>>m;
	for (i=0;i<m;i++)
	{
		int u,v;
		cin>>u>>v;
		e[u].push_back({v,i});
		e[v].push_back({u,i});
	}
	for (i=0;i<n;i++) if (!dfn[i]) dfs(i,1);
	cout<<ans.size()<<'\n';
	for (auto &v:ans) cout<<v.size()<<' '<<v<<'\n';
}

\end{lstlisting}

ans 存放每个点双包含的点。可以自环。

\begin{lstlisting}
const int N=5e5+5;
struct Q
{
	int v,w;
};
vector<vector<int>> ans;
vector<int> cur;
vector<int> e[N];
int dfn[N],low[N],st[N];
int id,tp;
void dfs(int u)
{
	dfn[u]=low[u]=++id;
	st[++tp]=u;
	for (int v:e[u]) if (dfn[v]) low[u]=min(low[u],dfn[v]); else
	{
		dfs(v);
		low[u]=min(low[u],low[v]);
		if (dfn[u]<=low[v])
		{
			vector cur={u};
			do
			{
				cur.push_back(st[tp]);
			} while (st[tp--]!=v);
			ans.push_back(cur);
		}
	}
}
int main()
{
	ios::sync_with_stdio(0);cin.tie(0);
	cout<<setiosflags(ios::fixed)<<setprecision(15);
	int n,m,i;
	cin>>n>>m;
	for (i=0;i<m;i++)
	{
		int u,v;
		cin>>u>>v;
		e[u].push_back(v);
		e[v].push_back(u);
	}
	for (i=0;i<n;i++) if (!dfn[i]) dfs(i);
	for (i=0;i<n;i++) if (count(all(e[i]),i)==e[i].size()) ans.push_back({i});
	cout<<ans.size()<<'\n';
	for (auto &v:ans) cout<<v.size()<<' '<<v<<'\n';
}
\end{lstlisting}

\subsection{边双}

$O(n+m)$,$O(n+m)$。

ans 存放每个边双包含的点。ct 为 $1$ 表示是割边。

\begin{lstlisting}
struct Q
{
	int v,w;
};
vector<vector<int>> ans;
vector<int> cur;
vector<Q> e[N];
int dfn[N],low[N],ed[N];
bool ct[N];
int id;
void dfs(int u,int fw)
{
	dfn[u]=low[u]=++id;
	for (auto [v,w]:e[u]) if (w!=fw)
	{
		if (!dfn[v])
		{
			dfs(v,w);
			low[u]=min(low[u],low[v]);
			ct[w]=dfn[u]<low[v];
		} else low[u]=min(low[u],dfn[v]);
	}
}
void dfs(int u)
{
	cur.push_back(u);ed[u]=1;
	for (auto [v,w]:e[u]) if (!ct[w]&&!ed[v]) dfs(v);
}
int main()
{
	ios::sync_with_stdio(0);cin.tie(0);
	int n,m,i;
	cin>>n>>m;
	for (i=0;i<m;i++)
	{
		int u,v;
		cin>>u>>v;
		e[u].push_back({v,i});
		e[v].push_back({u,i});
	}
	for (i=0;i<n;i++) if (!dfn[i]) dfs(i,-1);
	for (i=0;i<n;i++) if (!ed[i])
	{
		cur.clear();
		dfs(i);
		ans.push_back(cur);
	}
	cout<<ans.size()<<'\n';
	for (auto &v:ans) cout<<v.size()<<' '<<v<<'\n';
}

\end{lstlisting}

\subsection{输出负环}

\begin{lstlisting}
#include <bits/stdc++.h>
using namespace std;
const int N=34;
struct Q
{
	int v,w,c;
	Q(){}
	Q(int x,int y,int z):v(x),w(y),c(z){}
};
vector<Q> lj[N];
int dis[N],cnt[N],pt[N],S;
Q pre[N],st[N];
int n,m,ans,tp;
bool ed[N];
int main()
{
	freopen("arbitrage.in","r",stdin);
	freopen("arbitrage.out","w",stdout);
	ios::sync_with_stdio(0);cin.tie(0);
	cin>>n>>m;
	while (m--)
	{
		int x,y,z,w;
		cin>>x>>y>>z>>w;
		lj[x].emplace_back(y,w,z);
		lj[y].emplace_back(x,0,-z);
	}
	for (int i=1;i<=n;i++) lj[0].emplace_back(i,1,0);
	while (1)
	{
		memset(dis,-0x3f,sizeof dis);dis[0]=0;
		for (int i=0;i<=n;i++) ed[i]=cnt[i]=0;S=-1;
		queue<int> q;q.push(0);
		while (!q.empty())
		{
			int u=q.front();q.pop();ed[u]=0;
			for (auto &[v,w,c]:lj[u]) if (w&&dis[v]<dis[u]+c)
			{
				dis[v]=dis[u]+c;pre[v]=Q(u,w,c);
				if (!ed[v])
				{
					if (++cnt[v]>n+1) {S=v;goto aa;}
					ed[v]=1;q.push(v);
				}
			}
		}
		aa:;
		if (S==-1) break;
		{
			static bool ed[N];
			memset(ed,0,sizeof ed);
			while (!ed[S]) ed[S]=1,S=pre[S].v;
		}
		st[tp=1]=pre[S];pt[1]=S;
		int x=pre[S].v;
		while (x!=S)
		{
			st[++tp]=pre[x];pt[tp]=x;
			x=pre[x].v;
			assert(tp<=n+5);
		}
		int fl=1e9;
		for (int j=1;j<=tp;j++) fl=min(fl,st[j].w);
		assert(fl);
		for (int j=1;j<=tp;j++)
		{
			ans+=fl*st[j].c;
			int nn=0;
			for (auto &[v,w,c]:lj[st[j].v]) if (v==pt[j]&&st[j].c==c&&st[j].w==w) {++nn;w-=fl;break;}
			for (auto &[v,w,c]:lj[pt[j]]) if (v==st[j].v&&st[j].c+c==0) {++nn;w+=fl;break;}assert(nn==2);
		}
	}
	cout<<ans<<endl;
}
\end{lstlisting}



\subsection{DAG 删点最长路}

$O((n+m)\log n)$,$O(n+m)$。

\begin{lstlisting}
priority_queue<int> hp1,hp2,del1,del2;
int lj[M],nxt[M],fir[N],flj[M],fnxt[M],ffir[N],dl[N],rd[N],cd[N],dis1[N],dis2[N];
int dtp;
char c[M*15+1];
int main()
{
	int n,m,i,j,x,y,tou,wei,zd=0,ans=M,cur,pos=0;
	scanf("%d%d",&n,&m);
	fread(c+1,1,m*15,stdin);
	for (i=1;i<=m;i++)
	{
		read(x);read(y);++cd[x];
		lj[i]=y;nxt[i]=fir[x];fir[x]=i;++rd[y];
		flj[i]=x;fnxt[i]=ffir[y];ffir[y]=i;
	}
	tou=1;wei=0;
	for (i=1;i<=n;i++) if (!cd[i]) dl[++wei]=i;
	while (tou<=wei) for (i=ffir[x=dl[tou++]];i;i=fnxt[i]) 
	{
		dis2[flj[i]]=max(dis2[flj[i]],dis2[x]+1);
		if (--cd[flj[i]]==0) dl[++wei]=flj[i];
	}
	tou=1;wei=0;
	for (i=1;i<=n;i++) if (!rd[i]) dl[++wei]=i;
	while (tou<=wei) for (i=fir[x=dl[tou++]];i;i=nxt[i]) 
	{
		dis1[lj[i]]=max(dis1[lj[i]],dis1[x]+1);
		if (--rd[lj[i]]==0) dl[++wei]=lj[i];
	}
	for (i=1;i<=n;i++) hp1.push(dis2[i]);hp1.push(0);hp2.push(0);
	for (j=1;j<=wei;j++)
	{
		x=dl[j];
		if (dis2[x]==hp1.top())
		{
			hp1.pop();
			while ((!del1.empty())&&(del1.top()==hp1.top())) {hp1.pop();del1.pop();}
		} else del1.push(dis2[x]);
		for (i=ffir[x];i;i=fnxt[i]) del2.push(dis1[flj[i]]+dis2[x]+1);
		while ((!del2.empty())&&(del2.top()==hp2.top())) {hp2.pop();del2.pop();}
		cur=max(zd,max(hp1.top(),hp2.top()));
		if (cur<ans)
		{
			pos=dl[j];ans=cur;
		}
		zd=max(zd,dis1[x]);
		for (i=fir[x];i;i=nxt[i]) hp2.push(dis1[x]+dis2[lj[i]]+1);
		if (ans<=zd) break;
	}
	printf("%d %d",pos,ans);
}
\end{lstlisting}

\subsection{(基环)树哈希}

\begin{lstlisting}
#include <bits/stdc++.h>
using namespace std;
namespace tree_hash
{
	typedef unsigned int ui;
	typedef unsigned long long ll;
	const int N=1e6+2;
    const ui p1=2034452107,p2=2013074419,B=(1ll<<32)-1;
	mt19937 rnd(chrono::steady_clock::now().time_since_epoch().count());
	ui bas1[N],bas2[N],lst;
	ui uni1,uni2;
	vector<int> e[N];
	vector<ll> rt;
	ll g[N];
	int siz[N],h[N],f[N],num[N*2];
	int n,m;
	void init()
	{
		uni1=rnd()%(p1/2)+p1/2;uni2=rnd()%(p2/2)+p2/2;
		lst=0;
	}
	void dfs1(int u)
	{
		siz[u]=1;
		int mx=0;
		for (auto &v:e[u]) if (v!=f[u])
		{
			f[v]=u;dfs1(v);siz[u]+=siz[v];
			mx=max(mx,siz[v]);
		}
		mx=max(mx,n-siz[u]);
		if (mx*2<=n) rt.push_back(u);
	}
	void dfs2(int u)
	{
		for (auto &v:e[u]) if (v!=f[u]) f[v]=u,dfs2(v),h[u]=max(h[u],h[v]);
		++h[u];
		int n=0;
		static ll a[N];
		for (auto &v:e[u]) if (v!=f[u]) a[n++]=g[v];
		sort(a,a+n);
		ll r1=0,r2=0;
		a[n++]=1ll<<32|1;
		for (int i=0;i<n;i++) r1=(r1*bas1[h[u]]+(a[i]>>32))%p1,r2=(r2*bas2[h[u]]+(a[i]&B))%p2;
		g[u]=r1<<32|r2;
	}
	void get_e(vector<pair<int,int>> &E)
	{
		int i;
		n=E.size()+1;m=0;
		for (auto &[u,v]:E) num[m++]=u,num[m++]=v;
		sort(num,num+m);m=unique(num,num+m)-num;
		for (i=0;i<m;i++) e[num[i]].clear();
		for (auto &[u,v]:E) e[u].push_back(v),e[v].push_back(u);
		while (lst<n) bas1[++lst]=rnd()%(p1/2)+p1/2,bas2[lst]=rnd()%(p2/2)+p2/2;
	}
	ll rooted_tree_hash(int u)
	{
		if (n==1) return 1ll<<32|1;
		for (int i=0;i<m;i++) f[num[i]]=0,h[num[i]]=0;
		dfs2(u);
		return g[u];
	}
	ll t_h(vector<pair<int,int>> &E)
	{
		int i;
		get_e(E);
		for (i=0;i<m;i++) f[num[i]]=0;
		rt.clear();dfs1(1);
		ll r1=0,r2=0;
		for (auto &u:rt) u=rooted_tree_hash(u);
		sort(rt.begin(),rt.end());
		for (auto &u:rt) r1=(r1*uni1+(u>>32))%p1,r2=(r2*uni2+(u&B))%p2;
		return r1<<32|r2;
	}
}
using tree_hash::get_e;
using tree_hash::rooted_tree_hash;
using tree_hash::t_h;
typedef pair<int,int> pa;
typedef unsigned int ui;
typedef unsigned long long ull;
const ui mod1=2034452107,mod2=2013074419,B=(1ll<<32)-1;
ui b1,b2;
const int N=1e6+2;
vector<int> e[N];
int f[N];
vector<int> lp;
int getf(int u) {return f[u]==u?u:f[u]=getf(f[u]);}
void dfs1(int u)
{
	for (auto &v:e[u]) if (v!=f[u]) f[v]=u,dfs1(v);
}
bool ed[N];
void dfs2(int u,vector<pa> &E)
{
	for (auto &v:e[u]) if (!ed[v]) ed[v]=1,E.emplace_back(u,v),dfs2(v,E);
}
void min_order(ull *a,int n)
{
	int i,j,k;
	ull x,y;
	i=k=0;j=1;
	while ((i<n)&&(j<n)&&(k<n))
	{
		x=a[(i+k)%n];y=a[(j+k)%n];
		if (x==y) ++k; else
		{
			if (x>y) i+=k+1;  else j+=k+1;
			if (i==j) ++j;
			k=0;
		}
	}
	if (j>i) j=i;
	//[j,n)+[0,j)
	rotate(a,a+j,a+n);
}
int cal()
{
	int n,m,p1,p2;
	cin>>m;
	vector<pair<ull,ull>> a(m);
	for (auto &V:a)
	{
		int i;
		cin>>n;
		for (i=1;i<=n;i++) e[i].clear();
		iota(f+1,f+n+1,1);
		for (i=1;i<=n;i++)
		{
			int u,v;
			cin>>u>>v;
			if (getf(u)==getf(v)) {p1=u;p2=v;continue;}
			e[u].push_back(v);
			e[v].push_back(u);
			f[f[u]]=f[v];
		}
		memset(f+1,0,n*sizeof f[0]);
		dfs1(p1);
		static int st[N];
		memset(ed+1,0,n*sizeof ed[0]);
		int tp=1;st[1]=p2;
		while (p2!=p1) st[++tp]=p2=f[p2];
		for (i=1;i<=tp;i++) ed[st[i]]=1;
		vector<pa> E;
		static ull ans[N];
		E.reserve(n);
		for (i=1;i<=tp;i++)
		{
			dfs2(st[i],E);
			get_e(E);
			ans[i]=rooted_tree_hash(st[i]);
			E.clear();
		}
		min_order(ans+1,tp);
		ull r1=0,r2=0,r,rr;
		for (int i=1;i<=tp;i++) r1=(r1*b1+(ans[i]>>32))%mod1,r2=(r2*b2+(ans[i]&B))%mod2;
		r=r1<<32|r2;
		reverse(ans+1,ans+tp+1);
		min_order(ans+1,tp);r1=r2=0;
		for (int i=1;i<=tp;i++) r1=(r1*b1+(ans[i]>>32))%mod1,r2=(r2*b2+(ans[i]&B))%mod2;
		rr=r1<<32|r2;
		if (r>rr) swap(r,rr);
		V=make_pair(r,rr);
	}
	sort(a.begin(),a.end());
	return unique(a.begin(),a.end())-a.begin();
}
int main()
{
	b1=tree_hash::rnd()%(mod1/2)+mod1/2;
	b2=tree_hash::rnd()%(mod2/2)+mod2/2;
	tree_hash::init();
	ios::sync_with_stdio(0);cin.tie(0);
	int n,T;
	cin>>T;
	while (T--) cout<<cal()<<'\n';
}
\end{lstlisting}



\subsection{无向图最小环}

$O(n^3)$,$O(n^2)$。

\begin{lstlisting}
int f[N][N],jl[N][N];
int n,m,c,ans=inf,i,j,k,x,y,z;
int main()
{
	read(n);read(m);
	memset(f,0x3f,sizeof(f));
	memset(jl,0x3f,sizeof(jl));
	while (m--)
	{
		read(x);read(y);read(z);
		jl[x][y]=jl[y][x]=f[x][y]=f[y][x]=min(f[y][x],z);
	}
	for (k=1;k<=n;k++)
	{
		for (i=1;i<k;i++) if (jl[k][i]!=jl[0][0]) for (j=1;j<i;j++)
			if (jl[k][j]!=jl[0][0]) ans=min(ans,jl[k][i]+jl[k][j]+f[i][j]);
		for (i=1;i<=n;i++) if (i!=k) for (j=1;j<=n;j++)
			if ((j!=i)&&(j!=k)) f[i][j]=min(f[i][j],f[i][k]+f[k][j]);
	}
	if (ans==inf) puts("No solution."); else printf("%d",ans);
}
\end{lstlisting}

\subsection{切比雪夫距离最小生成树}

$O(n\log n)$,$O(n)$。

\begin{lstlisting}
const int N=3e5+2,M=N<<2;
struct P
{
	int u,v,w;
	P(int a=0,int b=0,int c=0):u(a),v(b),w(c){}
	bool operator<(const P &o) const {return w<o.w;}
};
struct Q
{
	int x,y,id;
	Q(int a=0,int b=0,int c=0):x(a),y(b),id(c){}
	bool operator<(const Q &o) const {return x!=o.x?x>o.x:y>o.y;}
};
ll ans;
P lb[M];
Q a[N],b[N];
int f[N],c[N];
int n,m,i,x,y;
struct bit
{
	int a[N],pos[N],n;
	void init(int &nn)
	{
		memset(a+1,0x7f,(n=nn)*sizeof a[0]);
		memset(pos+1,0,n*sizeof pos[0]);
	}
	void mdf(int x,const int y,const int z)
	{
		if (a[x]>y) a[x]=y,pos[x]=z;
		while (x-=x&-x) if (a[x]>y) a[x]=y,pos[x]=z;
	}
	int sum(int x)
	{
		int r=a[x],rr=pos[x];
		while ((x+=x&-x)<=n) if (a[x]<r) r=a[x],rr=pos[x];
		return rr;
	}
};
bit s;
void cal()
{
	int i,x,y;
	s.init(n);
	memcpy(b+1,a+1,sizeof(Q)*n);
	sort(a+1,a+n+1);
	for (i=1;i<=n;i++) c[i]=a[i].y-a[i].x;
	sort(c+1,c+n+1);
	for (i=1;i<=n;i++)
	{
		if (x=s.sum(y=lower_bound(c+1,c+n+1,a[i].y-a[i].x)-c))
			lb[++m]=P(a[x].id,a[i].id,a[x].x+a[x].y-a[i].x-a[i].y);//谨防 int 爆
		s.mdf(y,a[i].y+a[i].x,i);
	}
	memcpy(a+1,b+1,sizeof(Q)*n);
}
int getf(int x) {return f[x]==x?x:f[x]=getf(f[x]);}
int main()
{
	read(n);
	for (i=1;i<=n;i++) {read(a[f[i]=a[i].id=i].x);read(a[i].y);
		swap(a[i].x,a[i].y);a[i]=Q(a[i].x+a[i].y,a[i].x-a[i].y,i);}
	cal();for (i=1;i<=n;i++) swap(a[i].x,a[i].y);
	cal();for (i=1;i<=n;i++) a[i].y=-a[i].y;
	cal();for (i=1;i<=n;i++) swap(a[i].x,a[i].y);
	cal();sort(lb+1,lb+m+1);
	for (i=1;i<=m;i++) if ((x=getf(lb[i].u))!=(y=getf(lb[i].v))) f[x]=y,ans+=lb[i].w;
	printf("%lld\n",ans>>1);
}
\end{lstlisting}

\subsection{点分治}

$O(n\log n)$,$O(n)$。

\begin{lstlisting}
int siz[N],dep[N];
	int n,ksiz,md,rt,mn;
	bool ed[N];
	void find(int u)
	{
		ed[u]=1;siz[u]=1;
		int mx=0;
		for (int v:e[u]) if (!ed[v])
		{
			find(v);
			siz[u]+=siz[v];
			mx=max(mx,siz[v]);
		}
		mx=max(mx,ksiz-siz[u]);
		if (mn>mx) mn=mx,rt=u;
		ed[u]=0;
	}
	void cal(int u)
	{
		md=max(md,dep[u]);
		ed[u]=1;++cnt[dep[u]];
		for (int v:e[u]) if (!ed[v])
		{
			dep[v]=dep[u]+1;
			cal(v);
		}
		ed[u]=0;
	}
	void solve(int u)
	{
		mn=1e9;
		find(u);
		ed[rt]=1;
		vector<int> c;
		for (int v:e[rt]) if (!ed[v])
		{
			c.push_back(v);
			if (siz[v]>=siz[rt]) siz[v]=siz[u]-siz[rt];
		}
		sort(all(c),[&](const int &a,const int &b){return siz[a]<siz[b];});
		NTT::Q a(vector<ui>{1});
		NT::Q b(vector<ui>{1});
		for (int v:c)
		{
			md=0;dep[v]=1;
			cal(v);++md;
			vector<ui> d(cnt,cnt+md);
			NTT::Q e(d);
			NT::Q f(d);
			auto g=e&a;
			auto h=f&b;
			for (int i=0;i<g.a.size();i++) r1[i]=(r1[i]+g.a[i])%NTT::p;
			for (int i=0;i<h.a.size();i++) r2[i]=(r2[i]+h.a[i])%NT::p;
			a+=e;b+=f;
			fill_n(cnt,md,0);
		}
		for (int v:c)
		{
			ksiz=siz[v];
			solve(v);
		}
	}
\end{lstlisting}

\subsection{点分树}

$O(n\log^2 n)$,$O(n\log n)$。

\begin{lstlisting}
namespace DFS
{
	template<typename typC> struct bit0
	{
		vector<typC> a;
		int n;
		bit0() {}
		bit0(int nn):n(nn),a(nn+1) {}
		template<typename T> bit0(int nn,T *b):n(nn),a(nn+1)
		{
			for (int i=1; i<=n; i++) a[i]=b[i-1];
			for (int i=1; i<=n; i++) if (i+(i&-i)<=n) a[i+(i&-i)]+=a[i];
		}
		void add(int x,typC y)
		{
			++x;
			//cerr<<"add "<<x<<" by "<<y<<endl;
			assert(1<=x&&x<=n);
			a[x]+=y;
			while ((x+=x&-x)<=n) a[x]+=y;
		}
		typC sum(int x)
		{
			++x;
			//cerr<<"sum "<<x;
			if (x>n) x=n;
			assert(0<=x&&x<=n);
			typC r=a[x];
			while (x^=x&-x) r+=a[x];
			//cerr<<"= "<<r<<endl;
			return r;
		}
		typC sum(int x,int y)
		{
			return sum(y)-sum(x-1);
		}
	};
	typedef long long ll;
	const int N=1e5+5,M=18;
	ll a[N];
	bit0<ll> s[N],rc[N];
	int st[M][N*2],lg[N*2];
	int dep[N],dfn[N],siz[N],f[N];
	vector<int> e[N],c[N];
	bool ed[N];
	int n,ksiz,rt,mn,id;
	int lca(int u,int v)
	{
		u=dfn[u]; v=dfn[v];
		if (u>v) swap(u,v);
		int z=lg[v-u+1];
		return dep[st[z][u]]<dep[st[z][v-(1<<z)+1]]?st[z][u]:st[z][v-(1<<z)+1];
	}
	int dis(int u,int v)
	{
		return dep[u]+dep[v]-dep[lca(u,v)]*2;
	}
	void findroot(int u)
	{
		ed[u]=siz[u]=1;
		int mx=0;
		for (int v:e[u]) if (!ed[v])
		{
			findroot(v);
			siz[u]+=siz[v];
			mx=max(mx,siz[v]);
		}
		mx=max(mx,ksiz-siz[u]);
		ed[u]=0;
		if (mn>mx) mn=mx,rt=u;
	}
	int fun(int u)
	{
		mn=1e9;
		findroot(u);
		u=rt;
		ed[u]=1;
		for (int v:e[u]) if (!ed[v]&&siz[v]>siz[u]) siz[v]=ksiz-siz[u];
		for (int v:e[u]) if (!ed[v])
		{
			ksiz=siz[v];
			c[u].push_back(fun(v));
			f[c[u].back()]=u;
		}
		return u;
	}
	void pre_dfs(int u)
	{
		st[0][dfn[u]=++id]=u;
		ed[u]=1;
		for (int v:e[u]) if (!ed[v])
		{
			dep[v]=dep[u]+1;
			pre_dfs(v);
			st[0][++id]=u;
		}
		ed[u]=0;
	}
	void clear(int _n)
	{
		n=_n; id=0;
		int i;
		for (int i=1; i<=n; i++)
		{
			e[i].clear();
			a[i]=f[i]=ed[i]=0;
		}
	}
	void new_dfs(int u)
	{
		siz[u]=1;
		for (int v:c[u]) new_dfs(v),siz[u]+=siz[v];
		static int nd[N];
		vector<int> q={u};
		int mx=0,ql=0;
		nd[u]=0;
		while (ql<q.size())
		{
			int x=q[ql++];
			for (int v:c[x]) q.push_back(v),mx=max(mx,nd[v]=nd[u]+1);
		}
		static ll tmp[N];
		static int ds[N];
		mx=0;
		for (int v:q) mx=max(mx,ds[v]=dis(u,v));
		++mx;
		fill_n(tmp,mx,0);
		for (int v:q) tmp[ds[v]]+=a[v];
		s[u]=bit0<ll>(mx,tmp);
		if (u!=rt)
		{
			mx=0;
			for (int v:q) mx=max(mx,ds[v]=dis(f[u],v));
			++mx;
			fill_n(tmp,mx,0);
			for (int v:q) tmp[ds[v]]+=a[v];
			rc[u]=bit0<ll>(mx,tmp);
		}
	}
	void init()
	{
		pre_dfs(1);
		int i,j;
		for (i=2; i<=id; i++) lg[i]=lg[i>>1]+1;
		for (j=0; j<lg[id]; j++)
		{
			int R=id-(2<<j)+1;
			for (i=1; i<=R; i++) st[j+1][i]=dep[st[j][i]]<dep[st[j][i+(1<<j)]]?st[j][i]:st[j][i+(1<<j)];
		}
		ksiz=n;
		int tmp=fun(1);
		rt=tmp;
		new_dfs(rt);
	}
	void add(int u,ll val)
	{
		static int st[N],ds[N];
		int tp=1,i; st[1]=u;
		while (u=f[u]) st[++tp]=u;
		u=st[1];
		for (i=1; i<=tp; i++) ds[i]=dis(u,st[i]);
		for (i=1; i<=tp; i++)
		{
			s[st[i]].add(ds[i],val);
			if (i<tp) rc[st[i]].add(ds[i+1],val);
		}
	}
	ll ask(int u,int k)
	{
		ll res=0;
		static int st[N],ds[N];
		int tp=1,i; st[1]=u;
		while (u=f[u]) st[++tp]=u;
		u=st[1];
		for (i=1; i<=tp; i++) ds[i]=dis(u,st[i]);
		for (i=1; i<=tp; i++)
		{
			if (ds[i]<=k) res+=s[st[i]].sum(k-ds[i]);
			if (i<tp&&ds[i+1]<=k) res-=rc[st[i]].sum(k-ds[i+1]);
		}
		return res;
	}
}
//核心:点分树上 lca 出现在原树路径上
int main()
{
	ios::sync_with_stdio(0); cin.tie(0);
	cout<<setiosflags(ios::fixed)<<setprecision(15);
	int n,m,i;
	cin>>n>>m;
	DFS::clear(n);
	for (i=1; i<=n; i++) cin>>DFS::a[i];
	for (i=1; i<n; i++)
	{
		int u,v;
		cin>>u>>v;
		DFS::e[u].push_back(v);
		DFS::e[v].push_back(u);
	}
	DFS::init();
	ll ans=0;
	while (m--)
	{
		int op,x,y;
		cin>>op>>x>>y; x^=ans; y^=ans;
		if (op==1) DFS::add(x,y-DFS::a[x]),DFS::a[x]=y;
		else cout<<(ans=DFS::ask(x,y))<<'\n';
	}
}

\end{lstlisting}

\subsection{prufer 与树的互相转化}

$O(n)$,$O(n)$。

\begin{lstlisting}
vector<int> edges_to_prufer(const vector<pair<int,int>> &eg)//[1,n],定根为 n
{
	int n=eg.size()+1,i,j,k;
	int fir[n+1],nxt[n*2+1],e[n*2+1];
	int rd[n+1],cnt=0;
	memset(rd,0,sizeof rd);memset(nxt,0,sizeof nxt);memset(fir,0,sizeof fir);
	for (auto [u,v]:eg)
	{
		e[++cnt]=v;nxt[cnt]=fir[u];fir[u]=cnt;++rd[v];
		e[++cnt]=u;nxt[cnt]=fir[v];fir[v]=cnt;++rd[u];
	}
	for (i=1;i<=n;i++) if (rd[i]==1) break;
	int u=i;
	vector<int> r;r.reserve(n-2);
	for (j=1;j<n-1;j++)
	{	
		for (k=fir[u],u=rd[u]=0;k;k=nxt[k]) if (rd[e[k]]) 
		{
			r.push_back(e[k]);
			if ((--rd[e[k]]==1)&&(e[k]<i)) u=e[k];
		}
		if (!u) { while (rd[i]!=1) ++i;u=i;}
	}
	return r;
}
vector<pair<int,int>> prufer_to_edges(const vector<int> &p)//[1,n],定根为 n
{
	int n=p.size(),i,j,k;
	int m=n+3;
	int cs[m];memset(cs,0,sizeof cs);
	for (i=0;i<n;i++) ++cs[p[i]];
	i=0;
	while (cs[++i]);
	int u=i,v;
	vector<pair<int,int>> r;
	r.reserve(n-2);
	for (j=0;j<n;j++)
	{
		cs[u]=1e9;
		r.push_back({u,v=p[j]});
		if ((--cs[v]==0)&&(v<i)) u=v;
		if (v!=u) {while (cs[i]) ++i;u=i;}
	}
	r.push_back({u,n+2});
	return r;
}

\end{lstlisting}

\subsection{树链剖分}

\begin{lstlisting}
namespace HLD
{
	const int N=5e5+2;
	vector<int> e[N];
	int dfn[N],dep[N],f[N],siz[N],hc[N],top[N];
	int id;
	void dfs1(int u)
	{
		siz[u]=1;
		for (int v:e[u]) if (v!=f[u])
		{
			dep[v]=dep[f[v]=u]+1;
			dfs1(v);
			siz[u]+=siz[v];
			if (siz[v]>siz[hc[u]]) hc[u]=v;
		}
	}
	void dfs2(int u)
	{
		dfn[u]=++id;
		if (hc[u])
		{
			top[hc[u]]=top[u];
			dfs2(hc[u]);
			for (int v:e[u]) if (v!=hc[u]&&v!=f[u]) dfs2(top[v]=v);
		}
	}
	int lca(int u,int v)
	{
		while (top[u]!=top[v])
		{
			if (dep[top[u]]<dep[top[v]]) swap(u,v);
			u=f[top[u]];
		}
		if (dep[u]>dep[v]) swap(u,v);
		return u;
	}
	int dis(int u,int v)
	{
		return dep[u]+dep[v]-(dep[lca(u,v)]<<1);
	}
	void init(int n)
	{
		for (int i=1;i<=n;i++)
		{
			e[i].clear();
			f[i]=hc[i]=0;
		}
		id=0;
	}
	void fun(int root)
	{
		dep[root]=1;dfs1(root);dfs2(top[root]=root);
	}
	vector<pair<int,int>> get_path(int u,int v)//u->v,注意可能出现 [r>l](表示反过来走)
	{
		//cerr<<"path from "<<u<<" to "<<v<<": ";
		vector<pair<int,int>> v1,v2;
		while (top[u]!=top[v])
		{
			if (dep[top[u]]>dep[top[v]]) v1.push_back({dfn[u],dfn[top[u]]}),u=f[top[u]];
			else v2.push_back({dfn[top[v]],dfn[v]}),v=f[top[v]];
		}
		v1.reserve(v1.size()+v2.size()+1);
		v1.push_back({dfn[u],dfn[v]});
		reverse(v2.begin(),v2.end());
		for (auto v:v2) v1.push_back(v);
		//for (auto [x,y]:v1) cerr<<"["<<x<<','<<y<<"] ";cerr<<endl;
		return v1;
	}
}
using HLD::e,HLD::lca,HLD::dis,HLD::dfn,HLD::dep,HLD::f,HLD::siz,HLD::get_path;
using HLD::fun;//5e5
\end{lstlisting}



\subsection{LCT}

$O(n\log n)$,$O(n)$。

makeroot 会变根,split 会把 $y$ 变根,findroot 会把根变根,link 会把 $x,y$ 变根($y$ 是新的),cut 会把 $x,y$ 变根($x$ 是新的),注意 swap 子节点可能要 pushup。

\begin{lstlisting}
template<int N,typename Q> struct LCT
{
	int f[N],c[N][2],siz[N],st[N];
	Q s[N],v[N];
	#ifdef Rev
	Q rs[N];
	#endif
	//heap g[N]; //虚子树
	bool lz[N];
	void init(int n)
	{
		++n;
		for (int i=0;i<n;i++)
		{
			f[i]=c[i][0]=c[i][1]=lz[i]=0;
			s[i]=v[i]=Q();
			#ifdef Rev
			rs[i]=Q();
			#endif
			siz[i]=!!i;
		}
	}
	void modify(int x,const Q &o)
	{
		makeroot(x);
		v[x]=o;
		pushup(x);
	}
	bool nroot(int x) const
	{
		return c[f[x]][0]==x||c[f[x]][1]==x;
	}
	void pushup(int x)
	{
		int lc=c[x][0],rc=c[x][1];
		s[x]=v[x];siz[x]=1;
		#ifdef Rev
		rs[x]=v[x];
		#endif
		if (lc)
		{
			s[x]=s[lc]+s[x];
			siz[x]+=siz[lc];
			#ifdef Rev
			rs[x]=rs[x]+rs[lc];
			#endif
		}
		if (rc)
		{
			s[x]=s[x]+s[rc];
			siz[x]+=siz[rc];
			#ifdef Rev
			rs[x]=rs[rc]+rs[x];
			#endif
		}
	}
	void swp(int x)
	{
		swap(c[x][0],c[x][1]);
		#ifdef Rev
		swap(s[x],rs[x]);
		#endif
		lz[x]^=1;
	}
	void pushdown(int x)
	{
		int lc=c[x][0],rc=c[x][1];
		if (lz[x])
		{
			if (lc) swp(lc);
			if (rc) swp(rc);
			lz[x]=0;
		}
	}
	void zigzag(int x)
	{
		int y=f[x],z=f[y],typ=(c[y][0]==x);
		if (nroot(y)) c[z][c[z][1]==y]=x;
		f[x]=z;f[y]=x;
		if (c[x][typ]) f[c[x][typ]]=y;
		c[y][typ^1]=c[x][typ];c[x][typ]=y;
		pushup(y);
	}
	void splay(int x)
	{
		int y,tp=0;
		st[tp=1]=y=x;
		while (nroot(y)) st[++tp]=y=f[y];
		while (tp) pushdown(st[tp--]);
		for (;nroot(x);zigzag(x)) if (!nroot(f[x])) continue; else zigzag((c[f[x]][0]==x)^(c[f[f[x]]][0]==f[x]) ? x:f[x]);
		pushup(x);
	}
	void access(int x)
	{
		for (int y=0;x;x=f[y=x])
		{
			splay(x);
			//g[x].ins(s[c[x][1]]);g[x].del(s[y]);虚子树变化
			c[x][1]=y;pushup(x);
		}
	}
	int findroot(int x)
	{
		access(x);splay(x);pushdown(x);
		while (c[x][0]) pushdown(x=c[x][0]);
		splay(x);
		return x;
	}
	void split(int x,int y)//x 为树新根,y 为 splay 新根
	{
		makeroot(x);
		access(y);
		splay(y);
	}
	void makeroot(int x)
	{
		access(x);splay(x);
		swp(x);
	}
	void link(int x,int y)//y 为新根
	{
		makeroot(x);
		if (x!=findroot(y))//可能已经连通
		{
			makeroot(y);f[x]=y;//虚子树变化
		}
	}
	void cut(int x,int y)
	{
		makeroot(x);
		if (x==findroot(y))//可能本不连通
		{
			pushdown(x);
			if (c[x][1]==y&&!c[y][0]&&!c[y][1])//可能连通但无边
			{
				c[x][1]=f[y]=0;//可能需要修改
				pushup(x);
			}
		}
	}
};
\end{lstlisting}

\subsection{LCT(重构,代码为动态割边割点)}

\begin{lstlisting}
#include "bits/stdc++.h"
using namespace std;
template<int N,typename info,typename tag> struct LCT
{
	int f[N],c[N][2];
	info s[N],v[N];
#ifdef Rev
	info rs[N];
#endif
	tag tg[N];
	bool rev[N],lz[N];
	void init(int n,info *a)
	{
		for (int i=0; i<=n; i++)
		{
			rev[i]=lz[i]=0;
			f[i]=c[i][0]=c[i][1]=0;
			s[i]=v[i]=a[i];
#ifdef Rev
			rs[i]=a[i];
#endif
		}
	}
	bool nroot(int x) const
	{
		return c[f[x]][0]==x||c[f[x]][1]==x;
	}
	void pushup(int x)
	{
		int lc=c[x][0],rc=c[x][1];
		s[x]=v[x];
#ifdef Rev
		rs[x]=v[x];
#endif
		if (lc)
		{
			s[x]=s[lc]+s[x];
#ifdef Rev
			rs[x]=rs[x]+rs[lc];
#endif
		}
		if (rc)
		{
			s[x]=s[x]+s[rc];
#ifdef Rev
			rs[x]=rs[rc]+rs[x];
#endif
		}
	}
	void swp(int x)
	{
		swap(c[x][0],c[x][1]);
#ifdef Rev
		swap(s[x],rs[x]);
#endif
		rev[x]^=1;
	}
	void pushdown(int x)
	{
		if (rev[x])
		{
			for (int y:c[x]) if (y) swp(y);
			rev[x]=0;
		}
		if (lz[x])
		{
			for (int y:c[x]) if (y)
			{
				if (lz[y]) tg[y]+=tg[x]; else tg[y]=tg[x],lz[y]=1;
				s[y]+=tg[x];
			}
			lz[x]=0;
		}
	}
	void zigzag(int x)
	{
		int y=f[x],z=f[y],typ=(c[y][0]==x);
		if (nroot(y)) c[z][c[z][1]==y]=x;
		f[x]=z; f[y]=x;
		if (c[x][typ]) f[c[x][typ]]=y;
		c[y][typ^1]=c[x][typ]; c[x][typ]=y;
		pushup(y);
	}
	void splay(int x)
	{
		static int st[N];
		int y,tp;
		st[tp=1]=y=x;
		while (nroot(y)) st[++tp]=y=f[y];
		while (tp) pushdown(st[tp--]);
		for (; nroot(x); zigzag(x)) if (nroot(y=f[x])) zigzag((c[y][0]==x)^(c[f[y]][0]==y)?x:f[x]);
		pushup(x);
	}
	int access(int x)
	{
		int y=0;
		for (; x; x=f[y=x]) splay(x),c[x][1]=y,pushup(x);
		return y;
	}
	int findroot(int x)//splay 根为树根,splay 维护树根到 x 的链
	{
		access(x); splay(x); pushdown(x);
		while (c[x][0]) pushdown(x=c[x][0]);
		splay(x); return x;
	}
	void split(int x,int y)//x 为树新根,y 为 splay 新根
	{ makeroot(x); access(y); splay(y); }
	void makeroot(int x)//x 为树、splay 新根
	{ access(x); splay(x); swp(x); }
	void modify(int x,const info &o)
	{ makeroot(x); v[x]=o; pushup(x); }
	void modify(int x,int y,const tag &o)
	{
		split(x,y); s[y]+=o;
		if (lz[y]) tg[y]+=o; else tg[y]=o,lz[y]=1;
	}
	info ask(int x,int y) { split(x,y); return s[y]; }
	bool connected(int x,int y)//注意会改变形态
	{ makeroot(x); return findroot(y)==x; }
	void link(int x,int y)//y 为新根
	{ if (!connected(x,y)) makeroot(f[x]=y); }
	void cut(int x,int y)
	{
		if (connected(x,y))//可能本不连通
		{
			pushdown(x);
			if (c[x][1]==y&&!c[y][0]&&!c[y][1])//可能连通但无边
			{
				c[x][1]=f[y]=0;
				pushup(x);
			}
		}
	}
	int lca(int x,int y) { access(x); return access(y); }
	vector<int> res;
	void dfs(int x)
	{
		if (!x) return;
		pushdown(x);
		dfs(c[x][0]); res.push_back(x); dfs(c[x][1]);
	}
	vector<int> get_path(int x,int y)
	{
		res.clear(); split(x,y); dfs(y);
		if (res[0]!=x) return {};
		return res;
	}
};
const int N=2e5+5,M=4e5+5;
struct Q
{
	void operator+=(const Q &o) const {}
};
void operator+=(int &x,const Q &o) { x=0; }
LCT<N,int,Q> s;
LCT<M,int,Q> t;
int a[N],b[M];
int main()
{
	ios::sync_with_stdio(0); cin.tie(0);
	int n,m,i,r=0;
	cin>>n>>m;
	fill_n(a+n+1,n,1);
	fill_n(b+1,n,1);
	s.init(n*2,a);
	t.init(n+m,b);
	int bs=n,ds=n;
	while (m--)
	{
		int op,u,v;
		cin>>op>>u>>v;
		u^=r; v^=r;
		// dbg(op,u,v);
		if (u<1||u>n||v<1||v>n) return 0;
		if (op==1)
		{
			if (s.connected(u,v))
			{
				s.modify(u,v,{});
				auto c=t.get_path(u,v);
				for (i=1; i<c.size(); i++) t.cut(c[i-1],c[i]);
				++ds;
				for (int x:c) t.link(ds,x);
			}
			else
			{
				s.link(++bs,u);
				s.link(bs,v);
				t.link(++ds,u);
				t.link(ds,v);
			}
		}
		else
		{
			if (!s.connected(u,v))
			{
				cout<<"-1\n";
				continue;
			}
			r=op==2?s.ask(u,v):t.ask(u,v);
			cout<<r<<'\n';
		}
	}
}
\end{lstlisting}


\subsection{带子树的 LCT}

$O(n\log n)$,$O(n)$。

\begin{lstlisting}
#include <bits/stdc++.h>
using namespace std;
typedef long long ll;
template<int N> struct LCT
{
	ll s[N],v[N],sg[N];
	int f[N],c[N][2],siz[N],st[N];
	//heap g[N]; //虚子树
	bool lz[N];
	void init(int n)
	{
		memset(f,0,n+1<<2);
		memset(c,0,n+1<<3);
		memset(s,0,n+1<<3);
		memset(v,0,n+1<<3);
		memset(lz,0,n+1);
	}
	bool nroot(int x)
	{
		return c[f[x]][0]==x||c[f[x]][1]==x;
	}
	void pushup(int x)
	{
		s[x]=s[c[x][0]]+s[c[x][1]]+v[x]+sg[x];
		siz[x]=siz[c[x][0]]+siz[c[x][1]]+1;
	}
	void pushdown(int x)
	{
		if (lz[x])
		{
			swap(c[c[x][0]][0],c[c[x][0]][1]);
			swap(c[c[x][1]][0],c[c[x][1]][1]);
			lz[c[x][0]]^=1;
			lz[c[x][1]]^=1;
			lz[x]=0;
		}
	}
	void zigzag(int x)
	{
		int y=f[x],z=f[y],typ=(c[y][0]==x);
		if (nroot(y)) c[z][c[z][1]==y]=x;
		f[x]=z;f[y]=x;
		if (c[x][typ]) f[c[x][typ]]=y;
		c[y][typ^1]=c[x][typ];c[x][typ]=y;
		pushup(y);
	}
	void splay(int x)
	{
		int y,tp=0;
		st[tp=1]=y=x;
		while (nroot(y)) st[++tp]=y=f[y];
		while (tp) pushdown(st[tp--]);
		for (;nroot(x);zigzag(x)) if (!nroot(f[x])) continue; else zigzag((c[f[x]][0]==x)^(c[f[f[x]]][0]==f[x]) ? x:f[x]);
		pushup(x);
	}
	void access(int x)
	{
		for (int y=0;x;x=f[y=x])
		{
			splay(x);sg[x]-=s[y];s[x]-=s[y];
			sg[x]+=s[c[x][1]];s[x]+=s[c[x][1]];
			//g[x].ins(s[c[x][1]]);g[x].del(s[y]);虚子树变化
			c[x][1]=y;pushup(x);
		}
	}
	int findroot(int x)
	{
		access(x);splay(x);pushdown(x);
		while (c[x][0]) pushdown(x=c[x][0]);
		splay(x);
		return x;
	}
	void split(int x,int y)
	{
		makeroot(x);
		access(y);
		splay(y);
	}
	void makeroot(int x)
	{
		access(x);splay(x);lz[x]^=1;swap(c[x][0],c[x][1]); pushup(x);
	}
	void link(int x,int y)
	{
		makeroot(x);
		if (x!=findroot(y))//可能已经连通
		{
			makeroot(y);f[x]=y;//虚子树变化
			sg[y]+=s[x];s[y]+=s[x];
		}
	}
	void cut(int x,int y)
	{
		makeroot(x);
		if (x==findroot(y))//可能本不连通
		{
			pushdown(x);
			if (c[x][1]==y&&!c[y][0]&&!c[y][1])//可能连通但无边
			{
				c[x][1]=f[y]=0;//可能需要修改
				pushup(x);
			}
		}
	}
};
const int N=2e5+2;
LCT<N> s;
int n,q,i,x,y,z,w;
void read(int &x)
{
	int c=getchar();
	while (c<48||c>57) c=getchar();
	x=c^48;c=getchar();
	while (c>=48&&c<=57) x=x*10+(c^48),c=getchar();
}
int main()
{
	read(n);read(q);s.init(n);
	for (i=1;i<=n;i++) read(x),s.s[i]=s.v[i]=x;
	for (i=1;i<n;i++)
	{
		read(x);read(y);++x;++y;
		s.link(x,y);
	}
	while (q--)
	{
		read(x);read(y);read(z);++y;
		if (x==0)
		{
			read(x);read(w);
			++z;++x;++w;
			s.cut(y,z);s.link(x,w);
			continue;
		}
		if (x==1)
		{
			s.split(y,y);
			s.s[y]=(s.v[y]+=z);
		}
		else
		{
			++z;
			s.split(y,z);
			printf("%lld\n",s.s[y]);
		}
	}
}
\end{lstlisting}

\subsection{轻重链剖分}

\begin{lstlisting}

namespace HLD
{
	const int N=5e5+2;
	vector<int> e[N];
	int dfn[N],dep[N],f[N],siz[N],hc[N],top[N];
	int id;
	void dfs1(int u)
	{
		siz[u]=1;
		for (int v:e[u]) if (v!=f[u])
		{
			dep[v]=dep[f[v]=u]+1;
			dfs1(v);
			siz[u]+=siz[v];
			if (siz[v]>siz[hc[u]]) hc[u]=v;
		}
	}
	void dfs2(int u)
	{
		dfn[u]=++id;
		if (hc[u])
		{
			top[hc[u]]=top[u];
			dfs2(hc[u]);
			for (int v:e[u]) if (v!=hc[u]&&v!=f[u]) dfs2(top[v]=v);
		}
	}
	int lca(int u,int v)
	{
		while (top[u]!=top[v])
		{
			if (dep[top[u]]<dep[top[v]]) swap(u,v);
			u=f[top[u]];
		}
		if (dep[u]>dep[v]) swap(u,v);
		return u;
	}
	int dis(int u,int v)
	{
		return dep[u]+dep[v]-(dep[lca(u,v)]<<1);
	}
	void init(int n)
	{
		for (int i=1;i<=n;i++)
		{
			e[i].clear();
			f[i]=hc[i]=0;
		}
		id=0;
	}
	void fun(int root)
	{
		dep[root]=1;dfs1(root);dfs2(top[root]=root);
	}
	vector<pair<int,int>> get_path(int u,int v)//u->v,注意可能出现 [r>l](表示反过来走)
	{
		//cerr<<"path from "<<u<<" to "<<v<<": ";
		vector<pair<int,int>> v1,v2;
		while (top[u]!=top[v])
		{
			if (dep[top[u]]>dep[top[v]]) v1.push_back({dfn[u],dfn[top[u]]}),u=f[top[u]];
			else v2.push_back({dfn[top[v]],dfn[v]}),v=f[top[v]];
		}
		v1.reserve(v1.size()+v2.size()+1);
		v1.push_back({dfn[u],dfn[v]});
		reverse(v2.begin(),v2.end());
		for (auto v:v2) v1.push_back(v);
		//for (auto [x,y]:v1) cerr<<"["<<x<<','<<y<<"] ";cerr<<endl;
		return v1;
	}
}
using HLD::e,HLD::lca,HLD::dis,HLD::dfn,HLD::dep,HLD::f,HLD::siz,HLD::get_path;
using HLD::fun;//5e5


\end{lstlisting}

\subsection{换根树剖}

$O(n+q\log n)$,$O(n)$。

\begin{lstlisting}
void dfs1(int x)
{
	int i;
	siz[x]=1;
	for (i=fir[x];i;i=nxt[i]) if (lj[i]!=f[x])
	{
		dep[lj[i]]=dep[f[lj[i]]=x]+1;
		dfs1(lj[i]);
		siz[x]+=siz[lj[i]];
		if (siz[hc[x]]<siz[lj[i]]) hc[x]=lj[i];
	}
}
void dfs2(int x)
{
	nfd[dfn[x]=++bs]=x;
	if (hc[x])
	{
		int i;
		top[hc[x]]=top[x];
		dfs2(hc[x]);
		for (i=fir[x];i;i=nxt[i]) if ((lj[i]!=f[x])&&(lj[i]!=hc[x])) dfs2(top[lj[i]]=lj[i]);
	}
}
void mdf(int xx,int yy)
{
	while (top[xx]!=top[yy])
	{
		if (dep[top[xx]]<dep[top[yy]]) swap(xx,yy);
		z=dfn[top[xx]];y=dfn[xx];xdsmdf(1);
		xx=f[top[xx]];
	}
	if (dep[xx]>dep[yy]) swap(xx,yy);
	z=dfn[xx];y=dfn[yy];
	xdsmdf(1);
}
int find(int x,int y)
{
	while ((top[x]!=top[y])&&(f[top[x]]!=y)) x=f[top[x]];
	if (top[x]==top[y]) return hc[y];
	return top[x];
}
int main()
{
	read(n);read(m);
	for (i=2;i<=n;i++)
	{
		read(x);read(y);
		add();
	}bs=0;
	for (i=1;i<=n;i++) read(v[i]);
	dfs1(dep[1]=1);dfs2(top[1]=1);
	read(rt);r[l[1]=1]=n;build(1);
	while (m--)
	{
		read(x);read(y);
		if (x==1) {rt=y;continue;}
		if (x==2)
		{
			read(x);read(dt);
			mdf(x,y);continue;
		}
		x=y;dt=inf;
		if (x==rt)
		{
			z=1;y=n;sum(1);
		}
		else if ((dfn[x]<dfn[rt])&&(dfn[x]+siz[x]>dfn[rt]))
		{
			c=find(rt,x);
			z=1;y=dfn[c]-1;if (z<=y) sum(1);
			z=dfn[c]+siz[c];y=n;if (z<=y) sum(1);
		}
		else
		{
			z=dfn[x];y=z+siz[x]-1;sum(1);
		}
		printf("%d\n",dt);
	}
}
\end{lstlisting}

\subsection{树上启发式合并,DSU on tree}

\begin{lstlisting}
void dfs1(int x)
{
	siz[x]=zdep[x]=1;
	int i;
	for (i=fir[x];i;i=nxt[i]) if (lj[i]!=f[x])
	{
		dep[lj[i]]=dep[f[lj[i]]=x]+1;
		dfs1(lj[i]);
		siz[x]+=siz[lj[i]];
		if (siz[hc[x]]<siz[lj[i]]) hc[x]=lj[i];
		zdep[x]=max(zdep[x],zdep[lj[i]]+1);
	}
}
void cal(int x)
{
	int i;
	dl[tou=wei=1]=x;
	while (tou<=wei)
	{
		++dp[dep[x=dl[tou++]]];
		if ((dp[dep[x]]>dp[zd])||(dp[dep[x]]==dp[zd])&&(dep[x]<zd)) zd=dep[x];
		for (i=fir[x];i;i=nxt[i]) if (lj[i]!=f[x]) dl[++wei]=lj[i];
	}
}
void dfs2(int x)
{
	if (!hc[x])
	{
		if (++dp[dep[x]]>dp[zd]) zd=dep[x];
		return;
	}
	int i;
	for (i=fir[x];i;i=nxt[i]) if ((lj[i]!=f[x])&&(lj[i]!=hc[x]))
	{
		dfs2(lj[i]);
		memset(dp+dep[lj[i]],0,zdep[lj[i]]<<2);
	}
	dfs2(hc[x]);
	dp[dep[x]]=1;
	if (dp[zd]<=1) zd=dep[x];
	for (i=fir[x];i;i=nxt[i]) if ((lj[i]!=f[x])&&(lj[i]!=hc[x])) cal(lj[i]);
	ans[x]=zd-dep[x];
}
\end{lstlisting}

\subsection{长链剖分($k$ 级祖先)}

$O(n+q)$,$O(n)$。

\begin{lstlisting}
void dfs1(int x)
{
	int i;
	for (i=1;i<=er[dep[x]-1];i++) f[x][i]=f[f[x][i-1]][i-1];md[x]=dep[x];
	for (i=fir[x];i;i=nxt[i]) {dep[lj[i]]=dep[x]+1;dfs1(lj[i]);if (md[lj[i]]>md[dc[x]]) dc[x]=lj[i];}
	if (dc[x]) md[x]=md[dc[x]];
}
void dfs2(int x)
{
	int i;
	if (dc[x])
	{
		top[dc[x]]=top[x];
		dfs2(dc[x]);
		for (i=fir[x];i;i=nxt[i]) if (lj[i]!=dc[x]) dfs2(top[lj[i]]=lj[i]);
	}
	if (x==top[x])
	{
		c=md[x]-dep[x];y=x;up[x].push_back(x);down[x].push_back(x);
		for (i=1;(i<=c)&&(y=f[y][0]);i++) up[x].push_back(y);y=x;
		for (i=1;i<=c;i++) down[x].push_back(y=dc[y]);
	}
}
int main()
{
	int n,q,ans=0,x,y,c,i;
	ll ta=0;
	read(n);read(q);read(s);
	for (i=1;i<=n;i++) {read(f[i][0]);if (f[i][0]==0) rt=i; else add(f[i][0],i);}
	for (i=2;i<=n;i++) er[i]=er[i>>1]+1;dep[rt]=1;
	dfs1(rt);dfs2(top[rt]=rt);
	for (i=1;i<=q;i++)
	{
		x=(get(s)^ans)%n+1;y=(get(s)^ans)%dep[x];
		if (y==0) {ans=x;ta^=(ll)i*ans;continue;}
		c=dep[x]-y;x=top[f[x][er[y]]];
		if (dep[x]>c) ans=up[x][dep[x]-c]; else ans=down[x][c-dep[x]];
		ta^=(ll)i*ans;
	}
	printf("%lld",ta);
}
\end{lstlisting}

\subsection{长链剖分(dp 合并)}

$O(n)$,$O(n)$。

\begin{lstlisting}
void dfs1(int x)
{
	top[x]=1;
	for (int i=fir[x];i;i=nxt[i]) if (!top[lj[i]])
	{
		dfs1(lj[i]);
		if (len[lj[i]]>len[hc[x]]) hc[x]=lj[i];
	}
	len[x]=len[hc[x]]+1;top[hc[x]]=0;
}
void dfs2(int x)
{
	*f[x]=1;gs[x]=1;
	if (!hc[x]) return;
	ed[x]=1;f[hc[x]]=f[x]+1;
	for (int i=fir[x];i;i=nxt[i]) if (!ed[lj[i]]) dfs2(lj[i]);
	ans[x]=ans[hc[x]]+1;gs[x]=gs[hc[x]];
	if (gs[x]==1) ans[x]=0;
	for (int i=fir[x];i;i=nxt[i]) if ((!ed[lj[i]])&&(lj[i]!=hc[x]))
	{
		int v=lj[i],*p;
		for (int j=0;j<len[v];j++)
		{
			*(p=f[x]+j+1)+=*(f[v]+j);
			if (j+1==ans[x]) {gs[x]=*p;continue;}
			if ((*p>gs[x])||(*p==gs[x])&&(j+1<ans[x])) {gs[x]=*p;ans[x]=j+1;}
		}
	}
	gs[x]=*(f[x]+ans[x]);
	ed[x]=0;
}
\end{lstlisting}

\subsection{动态 dp(全局平衡二叉树)}

$O((n+q)\log n)$,$O(n)$。

\begin{lstlisting}
#include <stdio.h>
#include <string.h>
#include <algorithm>
#include <fstream>
using namespace std;
const int N=1e6+2,M=6e7+2,INF=-1e9;
struct matrix
{
	int a[2][2];
};
matrix s[N],js;
matrix operator *(matrix x,matrix y)
{
	js.a[0][0]=max(x.a[0][0]+y.a[0][0],x.a[0][1]+y.a[1][0]);
	js.a[0][1]=max(x.a[0][0]+y.a[0][1],x.a[0][1]+y.a[1][1]);
	js.a[1][0]=max(x.a[1][0]+y.a[0][0],x.a[1][1]+y.a[1][0]);
	js.a[1][1]=max(x.a[1][0]+y.a[0][1],x.a[1][1]+y.a[1][1]);
	return js;
}
int st[N],c[N][2],hc[N],lj[N<<1],nxt[N<<1],fir[N],siz[N],v[N],g[N][2],fa[N],f[N],val[N];
int n,m,i,j,x,y,z,dtp,stp,tp,fh,bs,rt,aaa,la;
char dr[M+5],sc[M];
void pushup(int x)
{
	s[x].a[0][0]=s[x].a[0][1]=g[x][0];
	s[x].a[1][0]=g[x][1];s[x].a[1][1]=INF;
	if (c[x][0]) s[x]=s[c[x][0]]*s[x];
	if (c[x][1]) s[x]=s[x]*s[c[x][1]];
}
void read(int &x)
{
	++dtp;fh=0;
	while ((dr[dtp]<48)||(dr[dtp]>57))
	{
		if (dr[dtp++]=='-')
		{
			fh=1;
			break;
		}
	}
	x=dr[dtp++]^48;
	while ((dr[dtp]>=48)&&(dr[dtp]<=57)) x=x*10+(dr[dtp++]^48);
	if (fh) x=-x;
}
void add(int x,int y)
{
	lj[++bs]=y;
	nxt[bs]=fir[x];
	fir[x]=bs;
	lj[++bs]=x;
	nxt[bs]=fir[y];
	fir[y]=bs;
}
bool nroot(int x)
{
	return ((c[f[x]][0]==x)||(c[f[x]][1]==x));
}
void dfs1(int x)
{
	siz[x]=1;
	int i;
	for (i=fir[x];i;i=nxt[i]) if (lj[i]!=fa[x])
	{
		fa[lj[i]]=x;
		dfs1(lj[i]);
		siz[x]+=siz[lj[i]];
		if (siz[hc[x]]<siz[lj[i]]) hc[x]=lj[i];
	}
}
int build(int l,int r)
{
	if (l>r) return 0;
	int i,tot=0,upn=0;
	for (i=l;i<=r;i++) tot+=val[i];tot>>=1;
	for (i=l;i<=r;i++)
	{
		upn+=val[i];
		if (upn>=tot)
		{
			f[c[st[i]][0]=build(l,i-1)]=st[i];
			f[c[st[i]][1]=build(i+1,r)]=st[i];
			pushup(st[i]);
			++aaa;
			return st[i];
		}
	}
}
int dfs2(int x)
{
	int i,j;
	for (i=x;i;i=hc[i]) for (j=fir[i];j;j=nxt[j]) if ((lj[j]!=fa[i])&&(lj[j]!=hc[i]))
	{
		f[y=dfs2(lj[j])]=i;
		g[i][0]+=max(s[y].a[0][0],s[y].a[1][0]);
		g[i][1]+=s[y].a[0][0];
	}
	tp=0;
	for (i=x;i;i=hc[i]) st[++tp]=i;
	for (i=1;i<tp;i++) val[i]=siz[st[i]]-siz[st[i+1]];
	val[tp]=siz[st[tp]];
	return build(1,tp);
}
void change(int x,int y)
{
	g[x][1]+=y-v[x];v[x]=y;
	while (f[x])
	{
		if (nroot(x)) pushup(x);
		else
		{
			g[f[x]][0]-=max(s[x].a[0][0],s[x].a[1][0]);
			g[f[x]][1]-=s[x].a[0][0];
			pushup(x);
			g[f[x]][0]+=max(s[x].a[0][0],s[x].a[1][0]);
			g[f[x]][1]+=s[x].a[0][0];
		}
		x=f[x];
	}
	pushup(x);
}
int main()
{
	scanf("%d%d",&n,&m);
	fread(dr+1,1,min(M,n*20+m*20),stdin);
	for (i=1;i<=n;i++)
	{
		read(g[i][1]);
		v[i]=g[i][1];
	}
	for (i=1;i<n;i++)
	{
		read(x);read(y);
		add(x,y);
	}
	dfs1(1);
	rt=dfs2(1);tp=0;
	while (m--)
	{
		read(x);read(y);
		change(x^la,y);
		x=la=max(s[rt].a[0][0],s[rt].a[1][0]);
		while (x)
		{
			st[++tp]=x%10;
			x/=10;
		}
		while (tp) sc[++stp]=st[tp--]|48;
		sc[++stp]=10;
	}
	fwrite(sc+1,1,stp,stdout);
}
\end{lstlisting}

\subsection{全局平衡二叉树(修改版)}

$O((n+q)\log n)$,$O(n)$。

\begin{lstlisting}
#include <bits/stdc++.h>
using namespace std;
typedef long long ll;
typedef pair<int,int> pa;
const int N=1e6+2,M=1e6+2;
ll ans;
pa w[N];
int c[N][2],f[N],fa[N],v[N],s[N],lz[N],lj[M],nxt[M],siz[N],hc[N],fir[N],st[N];
int a[N],top[N];
int n,i,x,y,z,bs,tp,rt;
void read(int &x)
{
	int c=getchar();
	while (c<48||c>57) c=getchar();
	x=c^48;c=getchar();
	while (c>=48&&c<=57) x=x*10+(c^48),c=getchar();
}
void add()
{
	lj[++bs]=y;nxt[bs]=fir[x];fir[x]=bs;
	lj[++bs]=x;nxt[bs]=fir[y];fir[y]=bs;
}
void pushup(int &x)
{
	s[x]=min(v[x],min(s[c[x][0]],s[c[x][1]]));
}
void pushdown(int &x)
{
	if (lz[x]<0)
	{
		int cc=c[x][0];
		if (cc)
		{
			lz[cc]+=lz[x];s[cc]+=lz[x];v[cc]+=lz[x];
		}
		cc=c[x][1];
		if (cc)
		{
			v[cc]+=lz[x];lz[cc]+=lz[x];s[cc]+=lz[x];
		}lz[x]=0;
		return;
	}
}
bool nroot(int &x) {return c[f[x]][0]==x||c[f[x]][1]==x;}
bool cmp(pa &o,pa &p) {return o>p;}
void dfs1(int x)
{
	siz[x]=1;
	for (int i=fir[x];i;i=nxt[i]) if (lj[i]!=fa[x])
	{
		fa[lj[i]]=x;dfs1(lj[i]);siz[x]+=siz[lj[i]];
		if (siz[hc[x]]<siz[lj[i]]) hc[x]=lj[i];
	}
}
int build(int l,int r)
{
	if (l>r) return 0;
	if (l==r)
	{
		l=st[l];s[l]=v[l]=siz[l]>>1;
		return l;
	}
	int x=lower_bound(a+l,a+r+1,a[l]+a[r]>>1)-a,y=st[x];
	c[y][0]=build(l,x-1);
	c[y][1]=build(x+1,r);
	v[y]=siz[y]>>1;
	if (c[y][0]) f[c[y][0]]=y;
	if (c[y][1]) f[c[y][1]]=y;
	pushup(y);
	return y;
}
void dfs2(int x)
{
	if (!hc[x]) return;
	int i;
	top[hc[x]]=top[x];
	if (top[x]==x)
	{
		st[tp=1]=x;
		for (i=hc[x];i;i=hc[i]) st[++tp]=i;
		for (i=1;i<=tp;i++) a[i]=siz[st[i]]-siz[hc[st[i]]]+a[i-1];
		f[build(1,tp)]=fa[x];
	}
	dfs2(hc[x]);
	for (i=fir[x];i;i=nxt[i]) if (lj[i]!=fa[x]&&lj[i]!=hc[x]) dfs2(top[lj[i]]=lj[i]);
}
void mdf(int x)
{
	int y=x;
	st[tp=1]=x;
	while (y=f[y]) st[++tp]=y;y=x;
	while (tp) pushdown(st[tp--]);
	while (x)
	{
		--v[x];--lz[c[x][0]];--v[c[x][0]];--s[c[x][0]];
		while (c[f[x]][0]==x) x=f[x];x=f[x];
	}
	pushup(y);
	while (y=f[y]) pushup(y);
}
int ask(int x)
{
	int y=x;
	st[tp=1]=x;
	while (y=f[y]) st[++tp]=y;
	while (tp) pushdown(st[tp--]);
	int r=v[x];
	while (x)
	{
		r=min(r,min(v[x],s[c[x][0]]));
		while (c[f[x]][0]==x) x=f[x];x=f[x];
	}
	return r;
}
signed main()
{
	read(n);s[0]=1e9;
	for (i=1;i<=n;i++) read(w[w[i].second=i].first);
	for (i=1;i<n;i++) read(x),read(y),add();
	sort(w+1,w+n+1,cmp);dfs1(1);dfs2(top[1]=1);rt=1;while (f[rt]) rt=f[rt];
	for (i=1;i<=n&&v[rt];i++) if (ask(w[i].second)) mdf(w[i].second),ans+=w[i].first;
	printf("%lld",ans);
}
\end{lstlisting}

\subsection{单调队列优化树上背包}

\begin{lstlisting}
#include <stdio.h>
#include <string.h>
#include <algorithm>
using namespace std;
const int N=502,M=4002,inf=-1e9;
int lj[N<<1],nxt[N<<1],fir[N],siz[N],v[N],p[N],l[N],f[N][M],num[M],dl[M];
int n,m,i,j,x,y,c,bs,t,ksiz,rt,zx,ans,tou,wei;
bool ed[N];
void add()
{
	lj[++bs]=y;
	nxt[bs]=fir[x];
	fir[x]=bs;
	lj[++bs]=x;
	nxt[bs]=fir[y];
	fir[y]=bs;
}
void read(int &x)
{
	c=getchar();
	while ((c<48)||(c>57)) c=getchar();
	x=c^48;c=getchar();
	while ((c>=48)&&(c<=57))
	{
		x=x*10+(c^48);
		c=getchar();
	}
}
void dfs2(int x)
{
	ed[x]=siz[x]=1;
	int i,zd=0;
	for (i=fir[x];i;i=nxt[i]) if (!ed[lj[i]])
	{
		dfs2(lj[i]);
		siz[x]+=siz[lj[i]];
		zd=max(zd,siz[lj[i]]);
	}
	zd=max(zd,ksiz-siz[x]);
	if (zd<zx)
	{
		zx=zd;
		rt=x;
	}
	ed[x]=0;
	for (i=1;i<=m;i++) f[x][i]=inf;
}
void dfs3(int x)
{
	if (p[x]>m) return;
	ed[x]=1;
	f[x][0]=max(f[x][0],0);
	int i;
	if (!l[x])
	{
		for (i=fir[x];i;i=nxt[i]) if (!ed[lj[i]])
		{
			for (j=0;j<=m;j++) f[lj[i]][j]=f[x][j];
			dfs3(lj[i]);
			for (j=m-p[x];~j;j--) f[x][j]=max(f[x][j],f[lj[i]][j]);
		}
		for (i=m;i>=p[x];i--) f[x][i]=f[x][i-p[x]]+v[x];
		for (i=0;i<p[x];i++) f[x][i]=inf;
		ed[x]=0;
		return;
	}
	for (i=0;i<p[x];i++)
	{
		y=(m-i)/p[x];
		num[dl[tou=wei=1]=0]=f[x][i];
		for (j=1;j<=y;j++)
		{
			while ((tou<wei)&&(j-dl[tou]>l[x])) ++tou;
			f[x][i+j*p[x]]=max(num[j]=f[x][i+j*p[x]],num[dl[tou]]+(j-dl[tou])*v[x]);
			while ((tou<=wei)&&(num[dl[wei]]+(j-dl[wei])*v[x]<=num[j])) --wei;
			dl[++wei]=j;
		}
	}
	for (i=fir[x];i;i=nxt[i]) if (!ed[lj[i]])
	{
		for (j=0;j<=m;j++) f[lj[i]][j]=f[x][j];
		dfs3(lj[i]);
		for (j=m-p[x];~j;j--) f[x][j]=max(f[x][j],f[lj[i]][j]);
	}
	for (i=m;i>=p[x];i--) f[x][i]=f[x][i-p[x]]+v[x];
	for (i=0;i<p[x];i++) f[x][i]=inf;
	ed[x]=0;
}
void dfs1(int x)
{
	int i,j=ksiz;
	rt=x;zx=n;
	dfs2(x);
	dfs3(x=rt);
	for (i=p[x];i<=m;i++) ans=max(ans,f[x][i]);
	ed[x]=1;
	for (i=fir[x];i;i=nxt[i]) if (!ed[lj[i]])
	{
		if (j>siz[lj[i]]) ksiz=siz[lj[i]]; else ksiz=j-siz[x];
		dfs1(lj[i]);
	}
}
int main()
{
	read(t);
	while (t--)
	{
		ans=0;
		read(n);read(m);
		for (i=1;i<=n;i++) read(v[i]);
		for (i=1;i<=n;i++) read(p[i]);
		for (i=1;i<=n;i++) read(l[i]);
		for (i=1;i<=n;i++) --l[i];
		memset(f,0,sizeof(f));
		ksiz=n;
		for (i=1;i<n;i++)
		{
			read(x);read(y);add();
		}
		dfs1(1);
		printf("%d\n",ans);
		memset(fir,0,sizeof(fir));bs=0;
		memset(ed,0,sizeof(ed));
	}
}
\end{lstlisting}

\subsection{树上背包}

\begin{lstlisting}
void dfs(int x)
{
	int i;
	for (i=fir[x];i;i=nxt[i])
	{
		for (j=1;j<=m;j++) f[lj[i]][j]=f[x][j-1]+v[lj[i]];
		dfs(lj[i]);
		for (j=0;j<=m;j++) f[x][j]=max(f[x][j],f[lj[i]][j]);
	}
}
\end{lstlisting}

\subsection{虚树}

$O(n+\sum k\log n)$,$O(n)$。

\begin{lstlisting}
void ins(int x)
{
	if (tp==0)
	{
		st[tp=1]=x;
		return;
	}
	ance=lca(st[tp],x);
	while (tp>1&&dep[ance]<dep[st[tp-1]])
	{
		add(st[tp-1],st[tp]);
		--tp;
	}
	if (dep[ance]<dep[st[tp]]) add(ance,st[tp--]);
	if (!tp||st[tp]!=ance) st[++tp]=ance;
	st[++tp]=x;
}
	sort(a+1,a+m+1,cmp);
	if (a[1]!=1) st[tp=1]=1;//先行添加根节点
	for (i=1;i<=m;i++) ins(a[i]);
	if (tp) while (--tp) add(st[tp],st[tp+1]);//回溯
\end{lstlisting}

\subsection{圆方树}

$O(n+m)$,$O(n+m)$。

\begin{lstlisting}
#include <bits/stdc++.h>
using namespace std;
#if !defined(ONLINE_JUDGE)&&defined(LOCAL)
#include "my_header\debug.h"
#else
#define dbg(...); 1;
#endif
typedef unsigned int ui;
typedef long long ll;
#define all(x) (x).begin(),(x).end()
const int N=3e4+2,M=3e4+2;//M 包括方点
struct P
{
	int v,w,id;
	P(int a,int b,int c):v(a),w(b),id(c){}
};
struct Q
{
	int v,w;
	Q(int a,int b):v(a),w(b){}
};
vector<P> e[N];
vector<Q> fe[M];
int dfn[M],low[N],st[N],len[M],top[M],siz[M],hc[M],dep[M],f[M],rb[N];
bool ed[M];//ed,dfn,loop,sum,fe,hc,tp,id,cnt,dep[1] 需初始化(注意倍率),ed 大小为边数
int tp,id,cnt,n;
void dfs1(int u)
{
	dfn[u]=low[u]=++id;
	st[++tp]=u;
	for (auto [v,w,id]:e[u]) if (!ed[id])
	{
		if (dfn[v]) low[u]=min(low[u],dfn[v]),rb[v]=w; else
		{
			ed[id]=1;
			dfs1(v);
			if (dfn[u]>low[v]) low[u]=min(low[u],low[v]),rb[v]=w; else
			{
				int ntp=tp;
				while (st[ntp]!=v) --ntp;
				if (ntp==tp)//圆圆边
				{
					--tp;
					fe[u].emplace_back(v,w);
					f[v]=u;
					continue;
				}
				++cnt;f[cnt]=u;
				for (int i=ntp;i<=tp;i++) f[st[i]]=cnt;
				len[st[ntp]]=w;
				for (int i=ntp+1;i<=tp;i++) len[st[i]]=len[st[i-1]]+rb[st[i]];
				len[cnt]=len[st[tp]]+rb[u];
				fe[u].emplace_back(cnt,0);
				for (int i=ntp;i<=tp;i++) fe[cnt].emplace_back(st[i],min(len[st[i]],len[cnt]-len[st[i]]));
				tp=ntp-1;
			}
		}
	}
}
void dfs2(int u)
{
	siz[u]=1;
	for (auto [v,w]:fe[u])
	{
		dep[v]=dep[u]+w;
		dfs2(v);
		siz[u]+=siz[v];
		if (siz[v]>siz[hc[u]]) hc[u]=v;
	}
}
void dfs3(int u)
{
	dfn[u]=++id;
	if (hc[u])
	{
		top[hc[u]]=top[u];
		dfs3(hc[u]);
		for (auto [v,w]:fe[u]) if (v!=hc[u]) dfs3(top[v]=v);
	}
}
int lca(int u,int v)
{
	while (top[u]!=top[v]) if (dfn[top[u]]>dfn[top[v]]) u=f[top[u]]; else v=f[top[v]];//注意不能用 dep
	return dfn[u]<dfn[v]?u:v;
}
int find(int u,int v)//u 是根
{
	if (dfn[hc[u]]+siz[hc[u]]>dfn[v]) return hc[u];
	while (f[top[v]]!=u) v=f[top[v]];
	return top[v];
}
int dis(int u,int v)
{
	int o=lca(u,v),r=dep[u]+dep[v];
	if (o<=n) return r-(dep[o]<<1);
	u=find(o,u);v=find(o,v);
	if (len[u]>len[v]) swap(u,v);
	return r+min(len[v]-len[u],len[o]-(len[v]-len[u]))-dep[u]-dep[v];
}
int main()
{
	ios::sync_with_stdio(0);cin.tie(0);
	int m,q,i;
	cin>>n>>m>>q;cnt=n;
	for (i=1;i<=m;i++)
	{
		int u,v,w;
		cin>>u>>v>>w;
		e[u].emplace_back(v,w,i);
		e[v].emplace_back(u,w,i);
	}
	mt19937 rnd(time(0));
	for (i=1;i<=n;i++) shuffle(all(e[i]),rnd);
	dfs1(1);id=0;
	dfs2(1);
	dfs3(top[1]=1);
	while (q--)
	{
		int u,v;
		cin>>u>>v;
		cout<<dis(u,v)<<'\n';
	}
}
\end{lstlisting}

\subsection{广义圆方树}

$O(n+m)$,$O(n+m)$。

\begin{lstlisting}
void dfs(int u)
{
	dfn[u]=low[u]=++id;
	st[++tp]=u;
	for (int v:e[u]) if (dfn[v]) low[u]=min(low[u],dfn[v]); else
	{
		dfs(v);
		low[u]=min(low[u],low[v]);
		if (dfn[u]<=low[v])
		{
			vector cur={u};
			do
			{
				cur.push_back(st[tp]);
			} while (st[tp--]!=v);
			ans.push_back(cur);
		}
	}
}
\end{lstlisting}

\subsection{支配树(DAG 版)}

$O(m\log n)$,$O(n\log n)$。

\begin{lstlisting}
int lca(int x,int y)
{
	int i;
	if (dep[x]<dep[y]) swap(x,y);
	for (i=lm[x];dep[x]!=dep[y];i--) if (dep[f[x][i]]>=dep[y]) x=f[x][i];
	if (x==y) return x;
	for (i=lm[x];f[x][0]!=f[y][0];i--) if (f[x][i]!=f[y][i])
	{
		x=f[x][i];y=f[y][i];
	}
	return f[x][0];
}
void dfs(int x)
{
	s[x]=1;
	int i;
	for (i=sfir[x];i;i=snxt[i])
	{
		dfs(slj[i]);
		s[x]+=s[slj[i]];
	}
}
int main()
{
	dep[0]=-1;
	read(n);
	for (i=1;i<=n;i++)
	{
		read(x);
		while (x)
		{
			add(x,i);
			read(x);
		}
	}
	dl[tou=wei=1]=++n;
	for (i=1;i<n;i++) if (!rd[i]) add(n,i);
	while (tou<=wei)
	{
		for (i=fir[x=dl[tou++]];i;i=nxt[i]) if (--rd[lj[i]]==0) dl[++wei]=lj[i];
		if (i=ffir[x])
		{
			y=flj[i];
			while (i=fnxt[i]) y=lca(y,flj[i]);
			f[x][0]=y;
		} else y=0;
		sadd(y,x);
		f[x][0]=y;
		for (i=1;i<=16;i++) if (0==(f[x][i]=f[f[x][i-1]][i-1]))
		{
			lm[x]=i;
			break;
		}
		dep[x]=dep[y]+1;
	}
	dfs(n);
	for (i=1;i<n;i++) printf("%d\n",s[i]-1);
}
\end{lstlisting}

\subsection{支配树(一般图)}

\begin{lstlisting}
#include <bits/stdc++.h>
using namespace std;
const int N=2e5+2;
vector<int> lj[N],llj[N],fl[N],tl[N],buc[N],c[N];
int f[N],mn[N],siz[N],sdom[N],idom[N],dfn[N],nfd[N],pv[N];
int n,m,cnt,i,j,x,y,na;
bool reach[N];
void dfs1(int x)
{
	nfd[dfn[x]=++cnt]=x;
	for (auto v:lj[x]) if (!dfn[v]) dfs1(v),c[x].push_back(v);
}
int getf(int x)
{
	if (f[x]==x) return x;
	int u=getf(f[x]);
	mn[x]=dfn[sdom[mn[x]]]<dfn[sdom[mn[f[x]]]]?mn[x]:mn[f[x]];
	return f[x]=u;
}
void dfs0(int u)
{
	reach[u]=1;
	for (auto &v:lj[u]) if (!reach[v]) dfs0(v);
}
int main()
{
	ios::sync_with_stdio(0);cin.tie(0);
	int S;
	cin>>n>>m>>S;++S;
	while (m--) cin>>x>>y,++x,++y,lj[x].push_back(y);
	for (i=1;i<=n;i++) mn[i]=f[i]=i;
	dfs0(S);
	for (i=1;i<=n;i++) if (reach[i]) for (auto &v:lj[i]) if (reach[v]) llj[i].push_back(v),fl[v].push_back(i);
	for (i=1;i<=n;i++) lj[i]=llj[i];
	dfs1(S);dfn[0]=1e9;
	for (i=cnt;i;i--)
	{
		x=nfd[i];na=0;
		for (auto v:fl[x])
		{
			sdom[x]=dfn[sdom[x]]<dfn[v]?sdom[x]:v;
			if (dfn[v]>dfn[x])
			{
				getf(v);
				na=dfn[sdom[na]]<dfn[sdom[mn[v]]]?na:mn[v];
			}
		}
		sdom[x]=dfn[sdom[x]]<dfn[sdom[na]]?sdom[x]:sdom[na];
		buc[sdom[x]].push_back(x);
		for (auto v:buc[x]) getf(v),pv[v]=mn[v];
		for (auto v:c[x]) f[v]=x,mn[v]=dfn[sdom[mn[v]]]<dfn[sdom[mn[x]]]?mn[v]:mn[x];
	}
	for (i=1;i<=n;i++) idom[nfd[i]]=(sdom[pv[nfd[i]]]==sdom[nfd[i]])?sdom[nfd[i]]:idom[pv[nfd[i]]];
	for (i=1;i<=n;i++) cout<<(i==S?S:idom[i])-1<<" \n"[i==n];
}
\end{lstlisting}

\subsection{最小树形图(朱刘算法,无方案)}

$O(nm)$,$O(n+m)$。

\begin{lstlisting}
int main()
{
	read(n);read(m);read(rt);
	for (i=1;i<=m;i++)
	{
		read(lj[i][1]);read(lj[i][2]);read(lj[i][0]);
	}
	while (1)
	{
		memset(infl,0x3f,sizeof(infl));
		memset(ed,0,sizeof(ed));
		memset(fa,0,sizeof(fa));
		for (i=1;i<=m;i++) if ((lj[i][1]!=lj[i][2])&&(lj[i][2]!=rt)&&(infl[lj[i][2]]>lj[i][0]))
		{
			infl[lj[i][2]]=lj[i][0];
			pre[lj[i][2]]=lj[i][1];
		}
		for (i=1;i<=n;i++) if (i!=rt)
		{
			if (infl[i]==infl[0])
			{
				puts("-1");return 0;
			}
			ans+=infl[i];
			for (j=i;(ed[j]!=i)&&(fa[j]==0)&&(j!=rt);j=pre[j]) ed[j]=i;
			if (ed[j]==i)
			{
				++cnt;
				while (fa[j]==0)
				{
					fa[j]=cnt;
					j=pre[j];
				}
			}
		}
		if (!cnt)
		{
			printf("%d",ans);return 0;
		}
		for (i=1;i<=n;i++) if (!fa[i]) fa[i]=++cnt;
		for (i=1;i<=m;i++)
		{
			lj[i][0]-=infl[lj[i][2]];
			lj[i][1]=fa[lj[i][1]];
			lj[i][2]=fa[lj[i][2]];
		}
		rt=fa[rt];
		n=cnt;cnt=0;
	}
}
\end{lstlisting}

\subsection{最小乘积生成树}

\begin{lstlisting}
#include <bits/stdc++.h>
using namespace std;
typedef long long ll;
const int N=202,M=10002;
template<typename typC> void read(typC &x)
{
	int c=getchar(),fh=1;
	while ((c<48)||(c>57))
	{
		if (c=='-') {c=getchar();fh=-1;break;}
		c=getchar();
	}
	x=c^48;c=getchar();
	while ((c>=48)&&(c<=57))
	{
		x=x*10+(c^48);
		c=getchar();
	}
	x*=fh;
}
struct P
{
	int x,y;
	P(int a=0,int b=0):x(a),y(b){}
	bool operator<(const P &o) const {return (ll)x*y<(ll)o.x*o.y||(ll)x*y==(ll)o.x*o.y&&x<o.x;}
};
struct Q
{
	int u,v,x,y,val;
	bool operator<(const Q &o) const {return val<o.val;}
};
P ans=P(1e9,1e9),l,r;
Q a[M];
int f[N];
int n,m,i;
int getf(int x)
{
	if (f[x]==x) return x;
	return f[x]=getf(f[x]);
}
P sol1()
{
	P r=P(0,0);
	for (i=1;i<=n;i++) f[i]=i;
	sort(a+1,a+m+1);
	for (i=1;i<=m;i++) if (getf(a[i].u)!=getf(a[i].v))
	{
		f[f[a[i].u]]=f[a[i].v];
		r.x+=a[i].x,r.y+=a[i].y;
	}
	return r;
}
void sol2(P l,P r)
{
	for (i=1;i<=m;i++) a[i].val=(r.x-l.x)*a[i].y+(l.y-r.y)*a[i].x;
	P np=sol1();
	ans=min(ans,np);
	if ((ll)(r.x-l.x)*(np.y-l.y)-(ll)(r.y-l.y)*(np.x-l.x)>=0) return;
	sol2(l,np);sol2(np,r);
}
int main()
{
	read(n);read(m);
	for (i=1;i<=m;i++) read(a[i].u),read(a[i].v),read(a[i].x),read(a[i].y),++a[i].u,++a[i].v;
	for (i=1;i<=m;i++) a[i].val=a[i].x;l=sol1();
	for (i=1;i<=m;i++) a[i].val=a[i].y;r=sol1();
	ans=min(ans,min(l,r));sol2(l,r);
	printf("%d %d",ans.x,ans.y);
}
\end{lstlisting}

\subsection{最小斯坦纳树}

$O(3^kn+2^km\log m)$。

\begin{lstlisting}
const int N=102,M=1002,K=1024;
typedef long long ll;
typedef pair<ll,int> pa;
priority_queue<pa,vector<pa>,greater<pa> > heap;
pa cr;
ll f[K][N],inf;
int lj[M],len[M],nxt[M],fir[N];
int n,m,q,i,j,k,x,y,z,bs,c;
void add()
{
	lj[++bs]=y;
	len[bs]=z;
	nxt[bs]=fir[x];
	fir[x]=bs;
	lj[++bs]=x;
	len[bs]=z;
	nxt[bs]=fir[y];
	fir[y]=bs;
}
void read(int &x)
{
	c=getchar();
	while ((c<48)||(c>57)) c=getchar();
	x=c^48;c=getchar();
	while ((c>=48)&&(c<=57))
	{
		x=x*10+(c^48);
		c=getchar();
	}
}
void dijk(int s)
{
	int i;
	while (!heap.empty())
	{
		x=heap.top().second;heap.pop();
		for (i=fir[x];i;i=nxt[i]) if (f[s][lj[i]]>f[s][x]+len[i])
		{
			cr.first=f[s][cr.second=lj[i]]=f[s][x]+len[i];
			heap.push(cr);
		}
		while ((!heap.empty())&&(heap.top().first!=f[s][heap.top().second])) heap.pop();
	}
}
int main()
{
	memset(f,0x3f,sizeof(f));inf=f[0][0];
	read(n);read(m);read(q);
	while (m--)
	{
		read(x);read(y);read(z);
		add();
	}
	for (i=1;i<=q;i++)
	{
		read(x);
		f[1<<i-1][x]=0;
	}
	q=(1<<q)-1;
	for (i=1;i<=q;i++)
	{
		for (k=1;k<=n;k++)
		{
			for (j=i&(i-1);j;j=i&(j-1)) f[i][k]=min(f[i][k],f[j][k]+f[i^j][k]);
			if (f[i][k]<inf) heap.push(pa(f[i][k],k));
		}
		dijk(i);
	}
	for (i=1;i<=n;i++) inf=min(inf,f[q][i]);
	printf("%lld",inf);
}
\end{lstlisting}

\subsection{$2$-sat}

$O(n+m)$,$O(n+m)$。

\begin{lstlisting}
struct sat
{
	vector<vector<int>> e;
	vector<int> dfn,low,st,f,ed;
	int fs,tp,id,n;
	sat(int n):n(n),e(n*2),dfn(n*2,-1),low(n*2),st(n*2),f(n*2,-1),ed(n*2),fs(0),tp(-1),id(0){}
	void dfs(int u)
	{
		dfn[u]=low[u]=id++;
		ed[u]=1;st[++tp]=u;
		for (int v:e[u]) if (dfn[v]!=-1)
		{
			if (ed[v]) low[u]=min(low[u],dfn[v]);
		} else dfs(v),low[u]=min(low[u],low[v]);
		if (dfn[u]==low[u])
		{
			do
			{
				f[st[tp]]=fs;
				ed[st[tp]]=0;
			} while (st[tp--]!=u);
			++fs;
		}
	}
	void add(int u,bool x,int v,bool y)//d:dif
	{
		assert(u>=0&&u<n&&v>=0&&v<n);
		e[u+x*n].push_back(v+y*n);
		e[v+(y^1)*n].push_back(u+(x^1)*n);
	}
	void set(int u,bool x)
	{
		assert(u>=0&&u<n);
		e[u+(x^1)*n].push_back(u+x*n);
	}
	vector<int> getans()
	{
		int i;
		for (i=0;i<n*2;i++) if (dfn[i]==-1) dfs(i);
		vector<int> r(n);
		for (i=0;i<n;i++)
		{
			if (f[i]==f[i+n]) return {};
			r[i]=f[i]>f[i+n];
		}
		return r;
	}
};
\end{lstlisting}

\subsection{Kosaraju 强连通分量(bitset 优化)}

$O(\frac{n^2}w)$,$O(\frac {n^2}w)$。

\begin{lstlisting}
void dfs1(int x)
{
	int i;ed[x]=0;
	for (i=(lj[x]&ed)._Find_first();i<=n;i=(lj[x]&ed)._Find_next(i)) dfs1(i);
	sx[--tp]=x;
}
void dfs2(int x)
{
	int i;ed[x]=0;tv[f[x]=f[0]]+=v[x];
	for (i=(fj[x]&ed)._Find_first();i<=n;i=(fj[x]&ed)._Find_next(i)) dfs2(i);
}
int main()
{
	read(n);read(m);tp=n+1;
	for (i=1;i<=n;i++) {ed[i]=1;read(v[i]);}
	for (i=1;i<=m;i++)
	{
		read(x);read(y);lj[x][y]=1;fj[y][x]=1;lb[i][0]=x;lb[i][1]=y;
	}
	for (i=1;i<=n;i++) if (ed[i]) dfs1(i);
	ed.set();
	for (i=1;i<=n;i++) if (ed[sx[i]]) {++f[0];dfs2(sx[i]);}
	for (i=1;i<=m;i++) if (f[lb[i][0]]!=f[lb[i][1]])
	{
		flj[f[lb[i][0]]].push_back(f[lb[i][1]]);++rd[f[lb[i][1]]];
	}
	for (i=1;i<=f[0];i++) if (!rd[i]) dl[++wei]=i;
	while (tou<=wei)
	{
		x=dl[tou++];g[x]+=tv[x];
		for (i=0;i<flj[x].size();i++)
		{
			g[flj[x][i]]=max(g[flj[x][i]],g[x]);
			if (--rd[flj[x][i]]==0) dl[++wei]=flj[x][i];
		}
	}
	for (i=1;i<=f[0];i++) ans=max(ans,g[i]);printf("%d",ans); 
}

\end{lstlisting}

\subsection{Tarjan 强连通分量}

$O(n+m)$,$O(n+m)$。

\begin{lstlisting}
int dfn[N],low[N],st[N],f[N],fs,tp,id;
bool ed[N];
void tarjan(int u)
{
	dfn[u]=low[u]=++id;
	ed[u]=1;st[++tp]=u;
	for (int v:e[u]) if (dfn[v])
	{
		if (ed[v]) low[u]=min(low[u],dfn[v]);
	} else tarjan(v),low[u]=min(low[u],low[v]);
	if (dfn[u]==low[u])
	{
		++fs;
		do
		{
			f[st[tp]]=fs;
			ed[st[tp]]=0;
		} while (st[tp--]!=u);
	}
}

\end{lstlisting}

\subsection{欧拉路径(字典序最小)}

\begin{lstlisting}
#include <bits/stdc++.h>
using namespace std;
#if !defined(ONLINE_JUDGE)&&defined(LOCAL)
#include "my_header\debug.h"
#else
#define dbg(...); 1;
#endif
typedef unsigned int ui;
typedef long long ll;
#define all(x) (x).begin(),(x).end()
const int N=1e5+2;
vector<int> e[N];
int rd[N],cd[N];
vector<int> ans;
void dfs(int u)
{
	while (e[u].size())
	{
		int v=e[u].back();
		e[u].pop_back();
		dfs(v);
		ans.push_back(v);
	}
}
int main()
{
	ios::sync_with_stdio(0);cin.tie(0);
	int n,m,i,x=0;
	cin>>n>>m;ans.reserve(m);
	while (m--)
	{
		int u,v;
		cin>>u>>v;
		e[u].push_back(v);
		++cd[u];++rd[v];
	}
	for (i=1;i<=n;i++) if (cd[i]!=rd[i])
	{
		if (abs(cd[i]-rd[i])>1) goto no;
		++x;
	}
	if (x>2) goto no;x=1;
	for (i=1;i<=n;i++) if (cd[i]>rd[i]) {x=i;break;}
	for (i=1;i<=n;i++) sort(all(e[i])),reverse(all(e[i]));
	dfs(x);ans.push_back(x);reverse(all(ans));
	for (i=0;i<ans.size();i++) cout<<ans[i]<<" \n"[i+1==ans.size()];
	return 0;
	no:cout<<"No"<<endl;
}
\end{lstlisting}

\subsection{欧拉回路构造}

$O(n+m)$,$O(n+m)$。

\begin{lstlisting}
struct Q
{
	int v,w;
};
pair<vector<int>,int> undirected_euler_cycle(int n,const vector<pair<int,int>> &edges)//[1,n]/[1,m], 正数表示正向,负数表示反向
{
	int i=0;
	vector<int> rd(n+1),ed(edges.size()+1),r;
	vector<vector<Q>> e(n+1);
	for (auto [u,v]:edges)
	{
		++rd[u],++rd[v];
		e[u].push_back({v,++i});
		e[v].push_back({u,-i});
	}
	for (i=1;i<=n;i++) if (rd[i]&1) return {{},0};
	auto dfs=[&](auto dfs,int u) -> void
	{
		while (e[u].size())
		{
			auto [v,w]=e[u].back();
			e[u].pop_back();
			if (ed[abs(w)]) continue;
			ed[abs(w)]=1;
			dfs(dfs,v);
			r.push_back(w);
		}
	};
	for (i=1;i<=n;i++) if (rd[i]) {dfs(dfs,i);break;}
	reverse(all(r));
	if (r.size()!=edges.size()) return {{},0};
	return {r,1};
}
pair<vector<int>,int> directed_euler_cycle(int n,const vector<pair<int,int>> &edges)//[1,n]/[1,m]
{
	int i=0;
	vector<int> rd(n+1),cd(n+1),r;
	vector<vector<Q>> e(n+1);
	for (auto [u,v]:edges)
	{
		++cd[u],++rd[v];
		e[u].push_back({v,++i});
	}
	for (i=1;i<=n;i++) if (rd[i]!=cd[i]) return {{},0};
	auto dfs=[&](auto dfs,int u) -> void
	{
		while (e[u].size())
		{
			auto [v,w]=e[u].back();
			e[u].pop_back();
			dfs(dfs,v);
			r.push_back(w);
		}
	};
	for (i=1;i<=n;i++) if (cd[i]) {dfs(dfs,i);break;}
	reverse(all(r));
	if (r.size()!=edges.size()) return {{},0};
	return {r,1};
}
\end{lstlisting}

\subsection{有向图欧拉回路计数(BEST 定理)}

$O(n^3)$,$O(n^2)$。

以 $u$ 为起点的欧拉回路个数 $sum=T(u)\times \prod\limits_{v=1}^n(out(v)-1)!$,其中 $T(u)$ 是以 $u$ 为根的外向树个数,$out(v)$ 是 $v$ 的出度。若允许循环同构(如 $1\to 2\to 1\to 3\to 1$ 与 $1\to 3\to 1\to 2\to 1$),还需多乘 $out(u)$。

\begin{lstlisting}
//https://blog.csdn.net/Jaihk662/article/details/79338437
#include <bits/stdc++.h>
using namespace std;
typedef long long ll;
const int N=102,M=4e5+2,p=1e6+3;
int a[N][N],fac[M],cd[N],st[N],rd[N],f[N],b[N][N];
int n,i,j,x,c,ans,t,tp;
void read(int &x)
{
	c=getchar();
	while ((c<48)||(c>57)) c=getchar();
	x=c^48;c=getchar();
	while ((c>=48)&&(c<=57))
	{
		x=x*10+(c^48);
		c=getchar();
	}
}
int ksm(int x,int y)
{
	int r=1;
	while (y)
	{
		if (y&1) r=(ll)r*x%p;
		x=(ll)x*x%p;
		y>>=1;
	}
	return r;
}
int GS()
{
	int i,j,k,r=1,xs;
	int fh=0;
	for (i=1;i<=n;i++)
	{
		k=0;
		for (j=i;j<=n;j++) if (a[j][i])
		{
			k=j;break;
		}
		assert(k);
		if (k==0) return 0;
		if (j!=k) fh^=1;
		for (j=i;j<=n;j++) swap(a[i][j],a[k][j]);
		xs=ksm(a[i][i],p-2);
		for (j=i+1;j<=n;j++) a[i][j]=(ll)a[i][j]*xs%p;
		r=(ll)r*a[i][i]%p;
		for (j=i+1;j<=n;j++)
		{
			xs=p-a[j][i];
			for (k=i+1;k<=n;k++) a[j][k]=(a[j][k]+(ll)(p-a[j][i])*a[i][k])%p;
		}
	}
	if (fh) return p-r; return r;
}
int getf(int x){if (f[x]==x) return x;return f[x]=getf(f[x]);}
int main()
{
	fac[0]=1;
	for (i=1;i<M;i++) fac[i]=(ll)fac[i-1]*i%p;
	read(t);
	while (t--)
	{
		read(n);
		memset(cd,0,sizeof(cd));
		memset(rd,0,sizeof(rd));
		memset(b,0,sizeof(b));
		for (i=1;i<=n;i++) f[i]=i;
		for (i=1;i<=n;i++)
		{
			read(cd[i]);
			for (j=1;j<=cd[i];j++)
			{
				read(x);--b[x][i];++b[i][i];++rd[x];
				f[getf(x)]=getf(i);
			}
		}
		for (i=1;i<=n;i++) if (rd[i]!=cd[i]) {puts("0");break;}
		if (i<=n) continue;
		tp=0;
		for (i=2;i<=n;i++) if (rd[i])
		{
			if (getf(i)!=getf(1)) {puts("0");break;}
			st[++tp]=i;
		}
		if (i<=n) continue;
		ans=1;if (cd[1]>1) ans=(ll)ans*cd[1]%p;
		for (i=1;i<=n;i++) if (cd[i]>2) ans=(ll)ans*fac[cd[i]-1]%p;
		//if (!cd[1]) {puts("1");continue;}
		for (i=1;i<=tp;i++) for (j=1;j<=tp;j++) a[i][j]=b[st[i]][st[j]];
		n=tp;
		for (i=1;i<=n;i++) for (j=1;j<=n;j++) if (a[i][j]<0) a[i][j]+=p;
		ans=(ll)ans*GS()%p;
		//if (1^n&1) ans=p-ans;
		printf("%d\n",ans%p);
	}
}
\end{lstlisting}

\subsection{点染色}

结论:$\chi(G)\le \Delta(G)+1$,其中 $\Delta(G)$ 是图的最大度。只有奇圈和完全图取等。

\begin{lstlisting}
vector<int> chromatic_number(int n,const vector<pair<int,int>> &edges)//[0,n)
{
	vector r(n,-1),cur(n,-1);
	vector<vector<int>> e(n);
	int ans=0,i;
	for (auto [u,v]:edges) e[u].push_back(v),e[v].push_back(u);
	for (i=0;i<n;i++) ans=max(ans,(int)e[i].size());
	ans+=2;
	vector p(n,vector(ans,0));
	function<void(int)> dfs=[&](int u)
	{
		int col=u?*max_element(cur.begin(),cur.begin()+u)+1:0;
		if (col>=ans) return;
		if (u==n)
		{
			r=cur;
			ans=col;
			return;
		}
		int i;
		for (int i=0;i<=col;i++) if (!p[u][i])
		{
			cur[u]=i;
			for (int v:e[u]) ++p[v][i];
			dfs(u+1);
			for (int v:e[u]) --p[v][i];
		}
	};
	dfs(0);
	return r;
}
\end{lstlisting}

\subsection{最大独立集}

\begin{lstlisting}
vector<int> indep_set(int n,const vector<pair<int,int>> &edges)//[0,n)
{
	vector<vector<int>> e(n);
	mt19937 rnd(998);
	vector<int> p(n),q(n),ed(n);
	iota(all(p),0);
	shuffle(all(p),rnd);
	for (int i=0;i<n;i++) q[p[i]]=i;
	for (auto [u,v]:edges)
	{
		e[p[u]].push_back(p[v]);
		e[p[v]].push_back(p[u]);
	}
	vector<int> r,cur;
	function<void(int)> dfs=[&](int u)
	{
		if (cur.size()+n-u<=r.size()) return;
		if (u==n)
		{
			r=cur;
			return;
		}
		if (!ed[u])
		{
			cur.push_back(u);
			for (int v:e[u]) ++ed[v];
			dfs(u+1);
			for (int v:e[u]) --ed[v];
			cur.pop_back();
		}
		if (ed[u]||e[u].size()) dfs(u+1);
	};dfs(0);
	for (int &x:r) x=q[x];
	sort(all(r));
	return r;
}
\end{lstlisting}




\newpage

\section{计算几何}

\subsection{自适应 simpson 法}

\begin{lstlisting}
const db eps=1e-7;
db sl,sr,sm,a;
db f(db x)
{
	return pow(x,a/x-x);
}
db g(db l,db r)
{
	db mid=(l+r)*0.5;
	return (f(l)+f(r)+f(mid)*4)/6*(r-l);
}
db ab(db x)
{
	if (x>0) return x;
	return -x;
}
db sim(db l,db r)
{
	db mid=(l+r)*0.5;
	sl=g(l,mid);sr=g(mid,r);sm=g(l,r);
	if (ab(sl+sr-sm)<eps) return sl+sr;
	return sim(l,mid)+sim(mid,r);
}
\end{lstlisting}

\subsection{板子}

\begin{lstlisting}
namespace geometry//不要用 int!
{
	#define tmpl template<typename T>
	typedef long long ll;
	typedef long double db;
	const db eps=1e-6;
	#define all(x) (x).begin(),(x).end()
	inline int sgn(const ll &x)
	{
		if (x<0) return -1;
		return x>0;
	}
	inline int sgn(const db &x)
	{
		if (fabs(x)<eps) return 0;
		return x>0?1:-1;
	}
	tmpl struct point//* 为叉乘,& 为点乘,只允许使用 double 和 ll
	{
		T x,y;
		point(){}
		point(T a,T b):x(a),y(b){}
		operator point<ll>() const {return point<ll>(x,y);}
		operator point<db>() const {return point<db>(x,y);}
		point<T> operator+(const point<T> &o) const {return point(x+o.x,y+o.y);}
		point<T> operator-(const point<T> &o) const {return point(x-o.x,y-o.y);}
		point<T> operator*(const T &k) const {return point(x*k,y*k);}
		point<T> operator/(const T &k) const {return point(x/k,y/k);}
		T operator*(const point<T> &o) const {return x*o.y-y*o.x;}
		T operator&(const point<T> &o) const {return x*o.x+y*o.y;}
		void operator+=(const point<T> &o) {x+=o.x;y+=o.y;}
		void operator-=(const point<T> &o) {x+=o.x;y+=o.y;}
		void operator*=(const T &k) {x*=k;y*=k;}
		void operator/=(const T &k) {x/=k;y/=k;}
		bool operator==(const point<T> &o) const {return x==o.x&&y==o.y;}
		bool operator!=(const point<T> &o) const {return x!=o.x||y!=o.y;}
		db len() const {return sqrt(len2());}//模长
		T len2() const {return x*x+y*y;}
	};
	const point<db> npos=point<db>(514e194,9810e191),apos=point<db>(145e174,999e180);
	const int DS[4]={1,2,4,3};
	tmpl int quad(const point<T> &o)//坐标轴归右上象限,返回值 [1,4]
	{
		return DS[(sgn(o.y)<0)*2+(sgn(o.x)<0)];
	}
	tmpl bool angle_cmp(const point<T> &a,const point<T> &b)
	{
		int c=quad(a),d=quad(b);
		if (c!=d) return c<d;
		return a*b>0;
	}
	tmpl db dis(const point<T> &a,const point<T> &b) {return (a-b).len();}
	tmpl T dis2(const point<T> &a,const point<T> &b) {return (a-b).len2();}
	tmpl point<T> operator*(const T &k,const point<T> &o) {return point<T>(k*o.x,k*o.y);}
	tmpl bool operator<(const point<T> &a,const point<T> &b)
	{
		int s=sgn(a*b);
		return s>0||s==0&&sgn(a.len2()-b.len2())<0;
	}
	tmpl istream & operator>>(istream &cin,point<T> &o) {return cin>>o.x>>o.y;}
	tmpl ostream & operator<<(ostream &cout,const point<T> &o) 
	{
		if ((point<db>)o==apos) return cout<<"all position";
		if ((point<db>)o==npos) return cout<<"no position";
		return cout<<'('<<o.x<<','<<o.y<<')';
	}
	tmpl struct line
	{
		point<T> o,d;
		line(){}
		line(const point<T> &a,const point<T> &b,int twopoint);
		bool operator!=(const line<T> &m) {return !(*this==m);}
	};
	template<> line<ll>::line(const point<ll> &a,const point<ll> &b,int twopoint)
	{
		o=a;
		d=twopoint?b-a:b;
		ll tmp=gcd(d.x,d.y);
		assert(tmp);
		if (d.x<0||d.x==0&&d.y<0) tmp=-tmp;
		d.x/=tmp;d.y/=tmp;
	}
	template<> line<db>::line(const point<db> &a,const point<db> &b,int twopoint)
	{
		o=a;
		d=twopoint?b-a:b;
		int s=sgn(d.x);
		if (s<0||!s&&d.y<0) d.x=-d.x,d.y=-d.y;
	}
	tmpl line<T> rotate_90(const line<T> &m) {return line(m.o,point(m.d.y,-m.d.x),0);}
	tmpl line<db> rotate(const line<T> &m,db angle)
	{
		return {(point<db>)m.o,{m.d.x*cos(angle)-m.d.y*sin(angle),m.d.x*sin(angle)+m.d.y*cos(angle)},0};
	}
	tmpl db get_angle(const line<T> &m,const line<T> &n) {return asin((m.d*n.d)/(m.d.len()*n.d.len()));}
	tmpl bool operator<(const line<T> &m,const line<T> &n)
	{
		int s=sgn(m.d*n.d);
		return s?s>0:m.d*m.o<n.d*n.o;
	}
	bool operator==(const line<ll> &m,const line<ll> &n) {return m.d==n.d&&(m.o-n.o)*m.d==0;}
	bool operator==(const line<db> &m,const line<db> &n) {return fabs(m.d*n.d)<eps&&fabs((n.o-m.o)*m.d)<eps;}
	tmpl ostream & operator<<(ostream &cout,const line<T> &o) {return cout<<'('<<o.d.x<<" k + "<<o.o.x<<" , "<<o.d.y<<" k + "<<o.o.y<<")";}
	tmpl point<db> intersect(const line<T> &m,const line<T> &n)
	{
		if (!sgn(m.d*n.d))
		{
			if (!sgn(m.d*(n.o-m.o))) return apos;
			return npos;
		}
		return (point<db>)m.o+(n.o-m.o)*n.d/(db)(m.d*n.d)*(point<db>)m.d;
	}
	tmpl db dis(const line<T> &m,const point<T> &o) {return abs(m.d*(o-m.o)/m.d.len());}
	tmpl db dis(const point<T> &o,const line<T> &m) {return abs(m.d*(o-m.o)/m.d.len());}
	struct circle
	{
		point<db> o;
		db r;
		circle(){}
		circle(const point<db> &O,const db &R=0):o(point<db>((db)O.x,(db)O.y)),r(R){}//圆心半径构造
		circle(const point<db> &a,const point<db> &b)//直径构造
		{
			o=(a+b)*0.5;
			r=dis(b,o);
		}
		circle(const point<db> &a,const point<db> &b,const point<db> &c)//三点构造外接圆(非最小圆)
		{
			auto A=(b+c)*0.5,B=(a+c)*0.5;
			o=intersect(rotate_90(line(A,c,1)),rotate_90(line(B,c,1)));
			r=dis(o,c);
		}
		circle(vector<point<db>> a)
		{
			int n=a.size(),i,j,k;
			mt19937 rnd(75643);
			shuffle(all(a),rnd);
			*this=circle(a[0]);
			for (i=1;i<n;i++) if (!cover(a[i]))
			{
				*this=circle(a[i]);
				for (j=0;j<i;j++) if (!cover(a[j]))
				{
					*this=circle(a[i],a[j]);
					for (k=0;k<j;k++) if (!cover(a[k])) *this=circle(a[i],a[j],a[k]);
				}
			}
		}
		circle(const vector<point<ll>> &b)
		{
			vector<point<db>> a(b.size());
			int n=a.size(),i,j,k;
			for (i=0;i<a.size();i++) a[i]=(point<db>)b[i];
			*this=circle(a);
		}
		tmpl bool cover(const point<T> &a) {return sgn(dis((point<db>)a,o)-r)<=0;}
	};
	tmpl struct segment
	{
		point<T> a,b;
		segment(){}
		segment(point<T> o,point<T> p)
		{
			int s=sgn(o.x-p.x);
			if (s>0||!s&&o.y>p.y) swap(o,p);
			a=o;b=p;
		}
	};
	tmpl bool intersect(const segment<T> &m,const segment<T> &n)
	{
		auto a=n.b-n.a,b=m.b-m.a;
		auto d=n.a-m.a;
		if (sgn(n.b.x-m.a.x)<0||sgn(m.b.x-n.a.x)<0) return 0;
		if (sgn(max(n.a.y,n.b.y)-min(m.a.y,m.b.y))<0||sgn(max(m.a.y,m.b.y)-min(n.a.y,n.b.y))<0) return 0;
		return sgn(b*d)*sgn((n.b-m.a)*b)>=0&&sgn(a*d)*sgn((m.b-n.a)*a)<=0;
	}
	tmpl struct convex
	{
		vector<point<T>> p;
		convex(vector<point<T>> a);
		db peri()//周长
		{
			int i,n=p.size();
			db C=(p[n-1]-p[0]).len();
			for (i=1;i<n;i++) C+=(p[i-1]-p[i]).len();
			return C;
		}
		db area(){return area2()*0.5;}//面积
		T area2()//两倍面积
		{
			int i,n=p.size();
			T S=p[n-1]*p[0];
			for (i=1;i<n;i++) S+=p[i-1]*p[i];
			return abs(S);
		}
		db diam() {return sqrt(diam2());}
		T diam2()//直径平方
		{
			T r=0;
			int n=p.size(),i,j;
			if (n<=2)
			{
				for (i=0;i<n;i++) for (j=i+1;j<n;j++) r=max(r,dis2(p[i],p[j]));
				return r;
			}
			p.push_back(p[0]);
			for (i=0,j=1;i<n;i++)
			{
				while ((p[i+1]-p[i])*(p[j]-p[i])<=(p[i+1]-p[i])*(p[j+1]-p[i])) if (++j==n) j=0;
				r=max({r,dis2(p[i],p[j]),dis2(p[i+1],p[j])});
			}
			p.pop_back();
			return r;
		}
		bool cover(const point<T> &o) const//点是否在凸包内
		{
			if (o.x<p[0].x||o.x==p[0].x&&o.y<p[0].y) return 0;
			if (o==p[0]) return 1;
			if (p.size()==1) return 0;
			ll tmp=(o-p[0])*(p.back()-p[0]);
			if (tmp==0) return dis2(o,p[0])<=dis2(p.back(),p[0]);
			if (tmp<0||p.size()==2) return 0;
			int x=upper_bound(1+all(p),o,[&](const point<T> &a,const point<T> &b){return (a-p[0])*(b-p[0])>0;})-p.begin()-1;
			return (o-p[x])*(p[x+1]-p[x])<=0;
		}
		convex<T> operator+(const convex<T> &A) const
		{
			int n=p.size(),m=A.p.size(),i,j;
			vector<point<T>> c;
			if (min(n,m)<=2)
			{
				c.reserve(n*m);
				for (i=0;i<n;i++) for (j=0;j<m;j++) c.push_back(p[i]+A.p[j]);
				return convex<T>(c);
			}
			point<T> a[n],b[m];
			for (i=0;i+1<n;i++) a[i]=p[i+1]-p[i];
			a[n-1]=p[0]-p[n-1];
			for (i=0;i+1<m;i++) b[i]=A.p[i+1]-A.p[i];
			b[m-1]=A.p[0]-A.p[m-1];
			c.reserve(n+m);
			c.push_back(p[0]+A.p[0]);
			for (i=j=0;i<n&&j<m;) c.push_back(c.back()+(a[i]*b[j]>0?a[i++]:b[j++]));
			while (i<n-1) c.push_back(c.back()+a[i++]);
			while (j<m-1) c.push_back(c.back()+b[j++]);
			return convex<T>(c);
		}
		void operator+=(const convex &a) {*this=*this+a;}
	};
	tmpl convex<T>::convex(vector<point<T>> a)
	{
		int n=a.size(),i;
		if (!n) return;
		p=a;
		for (i=1;i<n;i++) if (p[i].x<p[0].x||p[i].x==p[0].x&&p[i].y<p[0].y) swap(p[0],p[i]);
		a.resize(0);a.reserve(n);
		for (i=1;i<n;i++) if (p[i]!=p[0]) a.push_back(p[i]-p[0]);
		sort(all(a));
		for (i=0;i<a.size();i++) a[i]+=p[0];
		point<T>* st=p.data()-1;
		int tp=1;
		for (auto &v:a)
		{
			while (tp>1&&sgn((st[tp]-st[tp-1])*(v-st[tp-1]))<=0) --tp;
			st[++tp]=v;
		}
		p.resize(tp);
	}
	template<> bool convex<db>::cover(const point<db> &o) const//点是否在凸包内
	{
		if (o.x<p[0].x||o.x==p[0].x&&o.y<p[0].y) return 0;
		if (o==p[0]) return 1;
		if (p.size()==1) return 0;
		ll tmp=(o-p[0])*(p.back()-p[0]);
		if (tmp==0) return dis2(o,p[0])<=dis2(p.back(),p[0]);
		if (tmp<0||p.size()==2) return 0;
		int x=upper_bound(1+all(p),o,[&](const point<db> &a,const point<db> &b){return (a-p[0])*(b-p[0])>eps;})-p.begin()-1;
		return (o-p[x])*(p[x+1]-p[x])<=0;
	}
	tmpl struct half_plane//默认左侧
	{
		point<T> o,d;
		operator half_plane<ll>() const {return {(point<ll>)o,(point<ll>)d,0};}
		operator half_plane<db>() const {return {(point<db>)o,(point<db>)d,0};}
		half_plane(){}
		half_plane(const point<T> &a,const point<T> &b,bool twopoint)
		{
			o=a;
			d=twopoint?b-a:b;
		}
		bool operator<(const half_plane<T> &a) const 
		{
			int p=quad(d),q=quad(a.d);
			if (p!=q) return p<q;
			p=sgn(d*a.d);
			if (p) return p>0;
			return sgn(d*(a.o-o))>0;
		}
	};
	tmpl ostream & operator<<(ostream &cout,half_plane<T> &m) {return cout<<m.o<<" | "<<m.d;}
	tmpl point<db> intersect(const half_plane<T> &m,const half_plane<T> &n)
	{
		if (!sgn(m.d*n.d))
		{
			if (!sgn(m.d*(n.o-m.o))) return apos;
			return npos;
		}
		return (point<db>)m.o+(n.o-m.o)*n.d/(db)(m.d*n.d)*(point<db>)m.d;
	}
	const db inf=1e9;
	tmpl convex<db> intersect(vector<half_plane<T>> a)
	{
		T I=inf;
		a.push_back({{-I,-I},{I,-I},1});
		a.push_back({{I,-I},{I,I},1});
		a.push_back({{I,I},{-I,I},1});
		a.push_back({{-I,I},{-I,-I},1});
		sort(all(a));
		int n=a.size(),i,h=0,t=-1;
		half_plane<db> q[n];
		point<db> p[n];
		vector<point<db>> r;
		for (i=0;i<n;i++) if (i==n-1||sgn(a[i].d*a[i+1].d))
		{
			auto x=(half_plane<db>)a[i];
			while (h<t&&sgn((p[t-1]-x.o)*x.d)>=0) --t;
			while (h<t&&sgn((p[h]-x.o)*x.d)>=0) ++h;
			q[++t]=x;
			if (h<t) p[t-1]=intersect(q[t-1],q[t]);
		}
		while (h<t&&sgn((p[t-1]-q[h].o)*q[h].d)>=0) --t;
		if (h==t) return convex<db>(vector<point<db>>(0));
		p[t]=intersect(q[h],q[t]);
		return convex<db>(vector<point<db>>(p+h,p+t+1));
	}
	#undef tmpl
}
using geometry::point,geometry::line,geometry::circle,geometry::convex,geometry::half_plane;
using geometry::db,geometry::sgn,geometry::eps,geometry::ll,geometry::segment;
using geometry::intersect,geometry::dis;
\end{lstlisting}


\newpage

\section{公式与杂项}

\subsection{枚举大小为 r 的集合}

思路:通过进位创造 1,再把一串 1 移到最后

\begin{lstlisting}
for (int s=(1<<r)-1;s<1<<n;)
{
	int t=s+(s&-s);
	s=(s&~t)>>__lg(s&-s)+1|t;
}
\end{lstlisting}

\subsection{整体二分(区间 $k$-th)}

$O((n+q)\log a)$,$O(n+q)$。

\begin{lstlisting}
struct cz
{
	int x,y,kth,pos,typ;
};
cz q[M],st1[M],st2[M];
int a[N],b[N],d[N],ans[N],s[N];
int n,m,t1,t2,i,j,c,gs;
int lb(int x)
{
	return x&(-x);
}
void add(int x,int y)
{
	for (;x<=n;x+=lb(x)) s[x]+=y;
}
int sum(int x)
{
	int ans=0;
	for (;x;x-=lb(x)) ans+=s[x];
	return ans;
}
void ztef(int ql,int qr,int l,int r)
{
	if (ql>qr) return;
	int mid=l+r>>1,i,midd;
	t1=t2=0;
	if (l==r)
	{
		for (i=ql;i<=qr;i++) if (q[i].typ) ans[q[i].pos]=d[l];
		return;
	}
	for (i=ql;i<=qr;i++) if (q[i].typ)
	{
		midd=sum(q[i].y)-sum(q[i].x-1);
		if (midd>=q[i].kth) st1[++t1]=q[i]; else
		{
			st2[++t2]=q[i];
			st2[t2].kth-=midd;
		}
	}
	else if (q[i].pos<=mid)
	{
		add(q[i].x,1);
		st1[++t1]=q[i];
	}
	else st2[++t2]=q[i];
	for (i=1;i<=t1;i++) if (!st1[i].typ) add(st1[i].x,-1);
	for (i=1;i<=t1;i++) q[i+ql-1]=st1[i];
	midd=ql+t1-1;
	for (i=1;i<=t2;i++) q[i+midd]=st2[i];
	ztef(ql,midd,l,mid);ztef(midd+1,qr,mid+1,r);
}
int main()
{
	read(n);read(m);
	for (i=1;i<=n;i++)
	{
		read(a[i]);b[i]=a[i];
	}
	sort(b+1,b+n+1);
	d[gs=1]=b[1];
	for (i=2;i<=n;i++) if (b[i]!=b[i-1]) d[++gs]=b[i];
	for (i=1;i<=n;i++) a[i]=lower_bound(d+1,d+gs+1,a[i])-d;
	for (i=1;i<=n;i++)
	{
		q[i].x=i;q[i].pos=a[i];q[i].typ=0;
	}
	for (i=1;i<=m;i++)
	{
		read(q[i+n].x);read(q[i+n].y);read(q[i+n].kth);q[i+n].pos=i;q[i+n].typ=1;
	}
	ztef(1,n+m,1,gs);
	for (i=1;i<=m;i++) printf("%d\n",ans[i]);
}
\end{lstlisting}

\subsection{cdq 分治(三维偏序)}

$O(n\log ^2n)$,$O(n)$。

\begin{lstlisting}
int lb(int x)
{
	return x&(-x);
}
void add(int x,int y)
{
	for (;x<=mx;x+=lb(x)) a[x]+=y;
}
int sum(int x)
{
	int ans=0;
	for (;x;x^=lb(x)) ans+=a[x];
	return ans;
}
void gb(int l,int r)
{
	int i=l,m=l+r>>1,j=m+1,p=l;
	if (i<m) gb(i,m);
	if (j<r) gb(j,r);
	while ((i<=m)||(j<=r)) if ((j>r)||(i<=m)&&(q[i].x<=q[j].x))
	{
		if (!q[i].typ) add(q[i].y,1);
		qq[p++]=q[i++];
	}
	else
	{
		if (q[j].typ) ans[q[j].pos]+=q[j].typ*sum(q[j].y);
		qq[p++]=q[j++];
	}
	for (i=l;i<=m;i++) if (!q[i].typ) add(q[i].y,-1);
	for (i=l;i<=r;i++) q[i]=qq[i];
}
int main()
{
	read(n);read(m);
	for (i=1;i<=n;i++)
	{
		read(q[i].x);read(q[i].y);++q[i].y;
		yc[i]=q[i].y;
		if (q[i].y>mx) mx=q[i].y;
	}
	qs=ys=n;
	for (i=1;i<=m;i++)
	{
		read(x);read(y);read(z);read(j);
		q[++qs].x=x-1;q[qs].y=y;q[qs].pos=i;q[qs].typ=1;
		q[++qs].x=z;q[qs].y=y;q[qs].pos=i;q[qs].typ=-1;
		q[++qs].x=x-1;q[qs].y=j+1;q[qs].pos=i;q[qs].typ=-1;
		q[++qs].x=z;q[qs].y=j+1;q[qs].pos=i;q[qs].typ=1;
		if (j+1>mx) mx=j+1;
	}
	gb(1,qs);
	for (i=1;i<=m;i++) printf("%d\n",ans[i]);
}
\end{lstlisting}

\subsection{$k$ 阶差分( $[L,R]$ 加 $\tbinom{j-L+k}{k}$)}

$O((n+q)k)$,$O(nk)$。

\begin{lstlisting}
int main()
{
	read(n);read(m);
	for (i=1;i<=n;i++) read(b[i]);
	C[0][0]=1;
	for (i=1;i<=n+100;i++)
	{
		C[i][0]=1;
		for (j=1;j<=min(i,100);j++)
		{
			C[i][j]=C[i-1][j-1]+C[i-1][j];
			if (C[i][j]>=p) C[i][j]-=p;
		}
	}
	while (m--)
	{
		read(x);read(y);read(z);
		++a[x][z];
		for (i=0;i<=z;i++)
		{
			a[y+1][z-i]-=C[y-x+i][i];
			if (a[y+1][z-i]<0) a[y+1][z-i]+=p;
		}
	}
	for (i=100;i>=0;i--) for (j=1;j<=n;j++)
	{
		a[j][i]+=a[j-1][i];
		if (a[j][i]>=p) a[j][i]-=p;
		a[j][i]+=a[j][i+1];
		if (a[j][i]>=p) a[j][i]-=p;
	}
	for (i=1;i<=n;i++) printf("%d ",(b[i]+a[i][0])%p);
}
\end{lstlisting}

\subsection{高精度}

\begin{lstlisting}
#include <bits/stdc++.h>
using namespace std;
namespace unsigned_bigint
{
	const int p=10000,ws=4;
	struct Q
	{
		vector<int> a;
		Q(){a.clear();}
		void operator=(const int &nn)
		{
			int n=nn;
			while (n) a.push_back(n%p),n/=p;
		}
		Q operator+(const Q &o) const
		{
			Q r;r.a.resize(max(a.size(),o.a.size()));if (!r.a.size()) return r;//resize&size?
			int len=r.a.size()-1,lenn=min(a.size(),o.a.size());
			for (int i=0;i<lenn;i++) r.a[i]=a[i]+o.a[i];
			if (a.size()>o.a.size()) for (int i=lenn;i<=len;i++) r.a[i]=a[i]; else for (int i=lenn;i<=len;i++) r.a[i]=o.a[i]; 
			for (int i=0;i<len;i++) if (r.a[i]>=p) r.a[i]-=p,++r.a[i+1];
			if (r.a[len]>=p) r.a.push_back(r.a[len]/p),r.a[len]%=p;
			return r;
		}
		Q operator-(const Q &o) const
		{
			Q r;r.a.resize(a.size());
			int len=o.a.size();
			for (int i=0;i<len;i++) r.a[i]=a[i]-o.a[i];
			memcpy(&r.a[o.a.size()],&a[o.a.size()],a.size()-o.a.size()<<2);
			len=a.size();
			for (int i=0;i<len;i++) if (r.a[i]<0) r.a[i]+=p,--r.a[i+1];
			while (r.a.size()&&!r.a[r.a.size()-1]) r.a.pop_back();
			return r;
		}
		Q operator*(const Q &o) const
		{
			Q r;r.a.resize(a.size()+o.a.size());
			if (!r.a.size()) return r;
			int n=a.size(),m=o.a.size();
			for (int i=0;i<n;i++) for (int j=0;j<m;j++) r.a[i+j]+=a[i]*o.a[j];
			n=r.a.size()-1;
			for (int i=0;i<n;i++) r.a[i+1]+=r.a[i]/p,r.a[i]%=p;
			if (!r.a[n]) r.a.pop_back();
			return r;
		}
		Q operator+(const int &o) const {Q r;r=o; return (*this)+r;}
		Q operator-(const int &o) const {Q r;r=o;return (*this)-r;}
		Q operator*(const int &o) const {Q r;r=o;return (*this)*r;}
		template<typename C> void operator+=(C &o)
		{
			Q r;
			r=(*this)+o;
			(*this)=r;
		}
		template<typename C> void operator-=(C &o)
		{
			Q r;
			r=(*this)-o;
			(*this)=r;
		}
		template<typename C> void operator*=(C &o)
		{
			Q r;
			r=(*this)*o;
			(*this)=r;
		}
		bool operator<(const Q &o)
		{
			if (a.size()^o.a.size()||!a.size()) return a.size()<o.a.size();
			for (int i=a.size()-1;~i;i--) if (a[i]^o.a[i]) return a[i]<o.a[i];
			return 0;
		}
		bool operator!=(const Q &o)
		{
			if (a.size()^o.a.size()) return 1;int n=a.size();
			for (int i=0;i<n;i++) if (a[i]^o.a[i]) return 1;
			return 0;
		}
		bool operator!=(const int &o)
		{
			Q r;r=o;
			return (*this)!=r;
		}
		bool operator==(const Q &o)
		{
			return !((*this)!=o);
		}
		bool operator==(const int &o)
		{
			Q r;r=o;
			return (*this)==r;
		}
		bool operator>(const Q &o)
		{
			if (a.size()^o.a.size()||!a.size()) return a.size()>o.a.size();
			for (int i=a.size()-1;~i;i--) if (a[i]^o.a[i]) return a[i]>o.a[i];
			return 0;
		}
		Q operator/(const int &o)
		{
			Q r=(*this);
			if (!a.size()) return r;
			for (int i=a.size()-1;i;i--) r.a[i-1]+=r.a[i]%o*p,r.a[i]/=o;
			r.a[0]/=o;
			while (r.a.size()&&!r.a[r.a.size()-1]) r.a.pop_back();
			return r;
		}
		void operator/=(const int &o)
		{
			if (!a.size()) return;
			for (int i=a.size()-1;i;i--) a[i-1]+=a[i]%o*p,a[i]/=o;
			a[0]/=o;
			while (a.size()&&!a[a.size()-1]) a.pop_back();
		}
		int operator%(const int &o)
		{
			if (!a.size()) return 0;
			if (p%o==0) return a[0]%o;
			int r=0;
			for (int i=a.size()-1;~i;i--) r=(r*p+a[i])%o;
			return r;
		}
	};
	istream & operator>>(istream &cin,Q &o)
	{
		o.a.clear();
		int cnt=0,n=0,r;
		string s;
		cin>>s;
		reverse(s.begin(),s.end());
		for (char c:s)
		{
			if (cnt==0) o.a.push_back(0),r=1;
			o.a[o.a.size()-1]+=(c^'0')*r;r*=10;
			if (++cnt==ws) cnt=0;//这里也要改,是压位的位数
		}//printf("%d\n",(int)o.a.size());
		return cin;
	}
	ostream & operator<<(ostream &cout,const Q &o)
	{
		if (!o.a.size()) return cout<<0;
		cout<<o.a.back();
		if (o.a.size()==1) return;
		for (int i=o.a.size()-2;~i;i--) cout<<setfill('0')<<setw(ws)<<o.a[i];//注意这里也要改
		return cout;
	}
	Q gcd(Q a,Q b)
	{
		Q r;r=1;
		while (a%2==0&&b%2==0) a/=2,b/=2,r=r*2;
		while (a%2==0) a/=2;
		while (b%2==0) b/=2;
		if (b<a) swap(a,b);
		while (a.a.size())
		{
			b=(b-a)/2;
			while (b.a.size()&&b%2==0) b/=2;
			if (b<a) swap(a,b);
		}
		return r*b;
	}
}
\end{lstlisting}

\subsection{分散层叠算法(Fractional Cascading)}

$O(n+q(k+\log n))$,$O(n)$。

给出 $k$ 个长度为 $n$ 的有序数组。

现在有 $q$ 个查询 : 给出数 $x$,分别求出每个数组中大于等于 $x$ 的最小的数(非严格后继)。

若后继不存在,则定义为 $0$。你需要在线地回答这些询问。

\begin{lstlisting}
int a[M][N],b[M][N<<1],c[M][N<<1][2],len[M],ans[M];
int n,m,qs,p,q,d,i,j,x,y,la;
int main()
{
	read(n);read(m);read(qs);read(d);
	for (j=1;j<=m;j++) for (i=0;i<n;i++) read(a[j][i]);
	for (j=1;j<=m;j++) a[j][n]=inf+j;++n;
	for (i=0;i<n;i++) b[m][i]=a[m][i],c[m][i][0]=i;
	len[m]=n;
	for (j=m-1;j;j--)
	{
		p=0,q=1;
		while (p<n&&q<len[j+1]) 
if (a[j][p]<b[j+1][q]) b[j][len[j]]=a[j][p],c[j][len[j]][0]=p++,c[j][len[j]++][1]=q;
		else b[j][len[j]]=b[j+1][q],c[j][len[j]][0]=p,c[j][len[j]++][1]=q,q+=2;
		while (p<n) b[j][len[j]]=a[j][p],c[j][len[j]][0]=p++,c[j][len[j]++][1]=q;
		while (q<len[j+1]) b[j][len[j]]=b[j+1][q],c[j][len[j]][0]=p,c[j][len[j]++][1]=q,q+=2;
	}
	for (int ii=1;ii<=qs;ii++)
	{
		read(x);x^=la;
		y=lower_bound(b[1],b[1]+len[1],x)-b[1];
		ans[1]=a[1][c[1][y][0]];y=c[1][y][1];//下标是c[1][y][0]
		for (j=2;j<=m;j++) 
		{
			if (y&&b[j][y-1]>=x) --y;
			ans[j]=a[j][c[j][y][0]];//下标是c[j][y][0]
			y=c[j][y][1];
		}
		la=0;
		for (i=1;i<=m;i++) la^=ans[i]>inf?0:ans[i];
		if (ii%d==0) printf("%d\n",la);
	}
}
\end{lstlisting}

\subsection{模意义真分数还原}

$q\equiv \dfrac{x}{a}\pmod p$,$|a|\le A$。

\begin{lstlisting}
pair<int, int> approx(int p,int q,int A)
{
	int x=q,y=p,a=1,b=0;
	while (x>A)
	{
		swap(x,y);swap(a,b);
		a-=x/y*b;x%=y;
	}
	return make_pair(x,a);
}
\end{lstlisting}

\subsection{快速取模}

\begin{lstlisting}
__uint128_t brt=((__uint128_t)1<<64)/mod;
for(int i=1;i<=n;i++)
{
	ans*=i;
	ans=ans-mod*(brt*ans>>64);
	while(ans>=mod) ans-=mod;//可以替换为 if,但据说会变慢。如果循环展开则需要替换
}

struct barret{
    ll p,m; //p 表示上面的模数, m 为取模参数
    int c=0;
    inline void init(ll t){
    	c=48+log2(t),p=t;
		m=(ll((ulll(1)<<c)/t));
	}
    friend inline ll operator % (ll n,const barret &d) { // get n % d
        return n-((ulll(n)*d.m)>>d.c)*d.p;
    }
}modp;
\end{lstlisting}

\subsection{IO 优化}

\subsubsection{WDOI}

\begin{lstlisting}
class fast_iostream{
private:
	const int MAXBF = 1 << 20; FILE *inf, *ouf;
	char *inbuf, *inst, *ined;
	char *oubuf, *oust, *oued;
	inline void _flush(){fwrite(oubuf, 1, oued - oust, ouf);}
	inline char _getchar(){
		if(inst == ined) inst = inbuf, ined = inbuf + fread(inbuf, 1, MAXBF, inf);
		return inst == ined ? EOF : *inst++;
	}
	inline void _putchar(char c){
		if(oued == oust + MAXBF) _flush(), oued = oubuf;
		*oued++ = c;
	}
public:
	 fast_iostream(FILE *_inf = stdin, FILE * _ouf = stdout)
	:inbuf(new char[MAXBF]), inf(_inf), inst(inbuf), ined(inbuf),
	 oubuf(new char[MAXBF]), ouf(_ouf), oust(oubuf), oued(oubuf){}
	~fast_iostream(){_flush(); delete inbuf; delete oubuf;}
	template <typename Int>
	fast_iostream& operator >> (Int  &n){
		static char c;
		while((c = _getchar()) < '0' || c > '9');n = c - '0';
		while((c = _getchar()) >='0' && c <='9') n = n * 10 + c - '0';
		return *this;
	}
	template <typename Int>
	fast_iostream& operator << (Int   n){
		if(n < 0) _putchar('-'), n = -n; static char S[20]; int t = 0;
		do{S[t++] = '0' + n % 10, n /= 10;} while(n);
		for(int i = 0;i < t;++i) _putchar(S[t - i - 1]);
		return *this;
	}
	fast_iostream& operator << (char  c){_putchar(c);    return *this;}
	fast_iostream& operator << (const char *s){
		for(int i = 0;s[i];++i) _putchar(s[i]); return *this;
	}
}fio;//unsigned
\end{lstlisting}

\subsubsection{自用}

\begin{lstlisting}
	c[fread(c+1,1,N,stdin)+1]=0;char *cc=c;
void read(int &x)
{
	char *c=cc;
	while ((*c<48)||(*c>57)) ++c;
	x=*(c++)^48;
	while ((*c>=48)&&(*c<=57)) x=x*10+(*(c++)^48);cc=c;
}
void read(int &x)
{
	char *c=cc;fh=1;
	while ((*c<48)||(*c>57)){if (*c=='-') {++c;fh=-1;break;}++c;}
	x=*(c++)^48;
	while ((*c>=48)&&(*c<=57)) x=x*10+(*(c++)^48);
	x*=fh;cc=c;
}
void write(const int x)
{
	while (x)
	{
		st[++tp]=x%10;
		x/=10;
	}
	char *c=nc;
	while (tp) *(++c)=st[tp--]|48;
	*(++c)=10;nc=c;
}
	char *nc=sc;
	fwrite(sc+1,1,stp,stdout);
\end{lstlisting}

\subsection{手动开栈}

\begin{lstlisting}
//#pragma comment(linker, "/STACK:102400000,102400000") 偶尔没用
	{
		static int OP=0;
		if (OP++==0)
		{
			int size=128<<20;//128MB
			char* p=new char[size]+size;
			__asm__ __volatile__("movq %0, %%rsp\n""pushq $exit\n""jmp main\n"::"r"(p));
		}
	}//main 开头,需要配合 exit(0) 食用
\end{lstlisting}

\subsection{德扑}

\begin{lstlisting}

struct Q
{
	int x,y;
	bool operator<(const Q &o) const { return x>o.x; }
};
const ll inf=1e18;
ll getrk(vector<Q> a)
{
	assert(a.size()==5);
	int i,j;
	bool isf=1,iss=1,spe=0;
	sort(all(a));//decrease
	for (i=1; i<5; i++) if (a[i].y!=a[0].y) { isf=0; break; }
	for (i=1; i<5; i++) if (a[0].x!=i+a[i].x) { iss=0; break; }
	if (a[0].x==14)
	{
		for (i=1; i<5; i++) if (a[i].x!=6-i) break;
		if (i==5) iss=1,spe=1;
	}
	if (iss&&isf&&a[4].x==10) return 6*inf;
	if (iss&&isf) return 5*inf+a[4].x*!spe;
	static int cnt[15];
	static ll hash[5];
	for (auto [x,y]:a) ++cnt[x];
	memset(hash,0,5*sizeof hash[0]);
	for (auto [x,y]:a) if (cnt[x]) hash[cnt[x]]=hash[cnt[x]]*15+x,cnt[x]=0;
	if (hash[4]) return 4*inf+hash[4]*15+hash[1];
	if (hash[3]&&hash[2]) return 3*inf+hash[3]*225+hash[2];
	return hash[3]*170'859'375+hash[2]*759'375+hash[1]+iss*(inf+a[4].x*!spe)+isf*2*inf;
}
Q stoq(const string &s)
{
	static string num=" ?23456789TJQKA",col=" SHCD";
	return {(int)num.find(s[0]),(int)col.find(s[1])};
}
\end{lstlisting}


\subsection{质数,$\omega(n)$ 与 $d(n)$}

\begin{tabular}{c|c|c|c|c}
	$n$ & $n$ 前第一个质数 & $n$ 后第一个质数 & $\max\{\omega (n)\}$ & $\max\{d(n)\}$\\
	\hline
	$10^{1}$   &   $10^{1}-3$  &   $10^{1}+1$  & $2$  &    $4$    \\
	$10^{2}$   &   $10^{2}-3$  &   $10^{2}+1$  & $3$  &    $12$   \\
	$10^{3}$   &   $10^{3}-3$  &  $10^{3}+13$  & $4$  &    $32$   \\
	$10^{4}$   &  $10^{4}-27$  &   $10^{4}+7$  & $5$  &    $64$   \\
	$10^{5}$   &   $10^{5}-9$  &   $10^{5}+3$  & $6$  &   $128$   \\
	$10^{6}$   &  $10^{6}-17$  &   $10^{6}+3$  & $7$  &   $240$   \\
	$10^{7}$   &   $10^{7}-9$  &  $10^{7}+19$  & $8$  &   $448$   \\
	$10^{8}$   &  $10^{8}-11$  &   $10^{8}+7$  & $8$  &   $768$   \\
	$10^{9}$   &  $10^{9}-63$  &   $10^{9}+7$  & $9$  &   $1344$  \\
	$10^{10}$  &  $10^{10}-33$ &  $10^{10}+19$ & $10$ &   $2304$  \\
	$10^{11}$  &  $10^{11}-23$ &  $10^{11}+3$  & $10$ &   $4032$  \\
	$10^{12}$  &  $10^{12}-11$ &  $10^{12}+39$ & $11$ &   $6720$  \\
	$10^{13}$  &  $10^{13}-29$ &  $10^{13}+37$ & $12$ &  $10752$  \\
	$10^{14}$  &  $10^{14}-27$ &  $10^{14}+31$ & $12$ &  $17280$  \\
	$10^{15}$  &  $10^{15}-11$ &  $10^{15}+37$ & $13$ &  $26880$  \\
	$10^{16}$  &  $10^{16}-63$ &  $10^{16}+61$ & $13$ &  $41472$  \\
	$10^{17}$  &  $10^{17}-3$  &  $10^{17}+3$  & $14$ &  $64512$  \\
	$10^{18}$  &  $10^{18}-11$ &  $10^{18}+3$  & $15$ &  $103680$ \\
	$10^{19}$  &  $10^{19}-39$ &  $10^{19}+51$ & $16$ &  $161280$ \\
\end{tabular}

\subsection{NTT 质数}

\begin{tabular}{c|c|c|c}
	$p=r\times 2^k+1$   &  $r$  & $k$  & $g$(最小原根)\\
 \hline
		  $17$          &  $1$  & $4$  &  $3$  \\
		  $97$          &  $3$  & $5$  &  $5$  \\
		  $193$         &  $3$  & $6$  &  $5$  \\
		  $257$         &  $1$  & $8$  &  $3$  \\
		 $7681$         & $15$  & $9$  & $17$  \\
		 $12289$        &  $3$  & $12$ & $11$  \\
		 $40961$        &  $5$  & $13$ &  $3$  \\
		 $65537$        &  $1$  & $16$ &  $3$  \\
		$786433$        &  $3$  & $18$ & $10$  \\
		$5767169$       & $11$  & $19$ &  $3$  \\
		$7340033$       &  $7$  & $20$ &  $3$  \\
	   $23068673$       & $11$  & $21$ &  $3$  \\
	   $104857601$      & $25$  & $22$ &  $3$  \\
	   $167772161$      &  $5$  & $25$ &  $3$  \\
	   $469762049$      &  $7$  & $26$ &  $3$  \\
	   $998244353$      & $119$ & $23$ &  $3$  \\
	  $1004535809$      & $479$ & $21$ &  $3$  \\
	  $2013265921$      & $15$  & $27$ & $31$  \\
	  $2281701377$      & $17$  & $27$ &  $3$  \\
	  $3221225473$      &  $3$  & $30$ &  $5$  \\
	  $75161927681$     & $35$  & $31$ &  $3$  \\
	  $77309411329$     &  $9$  & $33$ &  $7$  \\
	 $206158430209$     &  $3$  & $36$ & $22$  \\
	 $2061584302081$    & $15$  & $37$ &  $7$  \\
	 $2748779069441$    &  $5$  & $39$ &  $3$  \\
	 $6597069766657$    &  $3$  & $41$ &  $5$  \\
	$39582418599937$    &  $9$  & $42$ &  $5$  \\
	$79164837199873$    &  $9$  & $43$ &  $5$  \\
	$263882790666241$   & $15$  & $44$ &  $7$  \\
   $1231453023109121$   & $35$  & $45$ &  $3$  \\
   $1337006139375617$   & $19$  & $46$ &  $3$  \\
   $3799912185593857$   & $27$  & $47$ &  $5$  \\
   $4222124650659841$   & $15$  & $48$ & $19$  \\
   $7881299347898369$   &  $7$  & $50$ &  $6$  \\
   $31525197391593473$  &  $7$  & $52$ &  $3$  \\
  $180143985094819841$  &  $5$  & $55$ &  $6$  \\
  $1945555039024054273$ & $27$  & $56$ &  $5$  \\
  $4179340454199820289$ & $29$  & $57$ &  $3$  \\
 \end{tabular}

\subsection{公式}

向上取整整除分块 $[i,\lfloor\dfrac{n-1}{\lceil\dfrac ni \rceil-1}\rfloor]$

$n$ 个点 $k$ 个连通块的生成树方案 $n^{k-2}\prod\limits_{i=1}^k siz_i$

杜教筛 $g(1)S(n)=\sum\limits_{i=1}^n(f*g)(i)-\sum\limits_{j=2}^ng(j)S(\lfloor\frac nj\rfloor)$

$(x,y)$ 曼哈顿距离 $\to$ $(x+y,x-y)$ 切比雪夫距离  
$(x,y)$ 切比雪夫距离 $\to$ $(\dfrac{x+y}{2},\dfrac{x-y}{2})$ 曼哈顿距离

错排数=$\lceil0.5+\frac{n!}{e}\rceil$

Kummer's Theorem: $\tbinom{n+m}{n}$ 含 $p~(p\in \text {prime})$ 的次数是 $n+m$ 在 $p$ 进制下的进位数

$\ln (1-x^V)=-\sum\limits_{i\ge1}\dfrac{x^{Vi}}{i}$

$x^{\bar n}=\sum\limits_i S_1(n,i)x^i$

$\begin{cases}x\equiv a_1\pmod {m_1}\\x\equiv a_2\pmod {m_2}\\\cdots\\x\equiv a_n\pmod {m_n}\end{cases}$ 

$m_i$ 为不同的质数。设 $M=\prod\limits_{i=1}^nm_i$,$t_i\times \dfrac {M}{m_i}\equiv 1\pmod {m_i}$,则 $x\equiv \sum\limits_{i=1}^na_it_i\dfrac {M}{m_i}$。

$V-E+F=2$,$S=n+\frac s2-1$。($n$ 为内部,$s$ 为边上)

用途:对于相邻的不相等的值,在中间画一条线(最外也画),$\text{连通块个数}=1+E-V+内部框个数$

注意全都是不含矩形边界上的。

$\pi^{-1}$ 最小时 $\pi$ 最小,$\pi$ 最大等价于 $\pi^{-1}$ 最大?

五边形数 GF:$\frac{x(2x+1)}{(1-x)^3}$

五边形数:$\frac{3n^2-n}2$,广义含非正,逆为分拆数 GF(注意系数正负和 $n$ 取值奇偶性相同)

贝尔数(划分集合方案数)EGF:$\exp(e^x-1)$,$B_n=\sum\limits_{i=0}^n S_2(n,i)$,伯努利数 EGF:$\dfrac{x}{e^x-1}$

$S_1(i,m)$ EGF:$\dfrac{(\sum\limits_{i\ge 0}\dfrac{x^i}i)^m}{m!}$,$S_2(i,m)$ EGF:$\dfrac{(e^x-1)^m}{m!}$

多项式牛顿迭代:如果已知 $G(F(x))\equiv0\pmod{x^{2n}}$,$G(F_*(x))\equiv0\pmod {x^n}$,则有 $F(x)\equiv F_*(x)-\frac{G(F_*(x))}{G'(F_*(x))}\pmod{x^{2n}}$。求导时孤立的多项式视为常数。

$\int_0^1 t^a(1-t)^b\mathrm{d}t=\dfrac{a!b!}{(a+b+1)!}$,$\sum\limits_{i=0}^{n-1}i^{\underline{k}}=\dfrac{n^{\underline{k+1}}}{k+1}$

Burnside 引理:等价类数量为 $\sum\limits_{g\in G}\dfrac{X^g}{|G|}$,$X^g$ 表示 $g$ 变换下不动点的数量。

Polya 定理:染色方案数为 $\sum\limits_{g\in G}\dfrac{m^{c(g)}}{|G|}$,其中 $c(g)$ 表示 $g$ 变换下环的数量。

假设已经只保留了一个牛人酋长,其名字为 $A=a_1a_2\cdots a_l$。

假设王国旁边开了一座赌场,每单位时间(就称为“秒”吧)会有一个赌徒带着 $1$ 铜币进入赌场。

赌场规则很简单:支付 $x$ 铜币赌下一秒会唱出 $y$,如果猜对了就返还 $nx$ 铜币,否则钱就没了。

每个赌徒会如下行动:支付 $1$ 铜币赌下一秒会唱出 $a_1$,如果赌对了就支付得到的 $n$ 铜币赌下一秒会唱出 $a_2$,如果还对了就支付得到的 $n^2$ 铜币赌下一秒会唱出 $a_3$,等等,以此类推,最后支付 $n^{l-1}$ 铜币赌下一秒会唱出 $a_l$。

一旦连续唱出了 $a_1a_2\cdots a_l$,赌场老板就会认为自己亏大了而关门,并驱散所有赌徒。

那么关门前发生了什么呢?以 $A=\{1,4,1,5,1,1,4,1\},n=5$ 为例:

- 最后一位赌徒拿着 $5$ 铜币离开;
- 倒数第三位赌徒拿着 $5^3$ 铜币离开;
- 倒数第八位赌徒拿着 $5^8$ 铜币离开;
- 其他所有赌徒空手而归。

我们可以发现 $1,3$ 恰好是原序列的所有 border 的长度,而且对于其他的名字也有这样的规律。

这时候最神奇的一步来了:由于这个赌博游戏是公平的,因此赌场应该期望下不赚不赔,因此关门时期望来了 $5+5^3+5^8$ 个赌徒,因此期望需要 $5+5^3+5^8$ 单位时间唱出这个名字。

同理,即可知道对于一般的 $A$,答案为:

$$\sum\limits_{a_1a_2\cdots a_c=a_{l-c+1}a_{l-c+2}\cdots a_l} n^c$$


 \newpage

 \section{stl 使用指南}

\subsection{bitset}

\begin{lstlisting}
#include <bits/stdc++.h>
using namespace std;
bitset<10> f(12);
char s2[]="100101";
bitset<10> g(s2);
string s="100101";//reverse 了
bitset<10> h(s);
int main()
{
	for (int i=0;i<=9;i++) if (f[i]) printf("1"); else printf("0");puts("");
	for (int i=0;i<=9;i++) if (g[i]) printf("1"); else printf("0");puts("");
	for (int i=0;i<=9;i++) if (h[i]) printf("1"); else printf("0");puts("");
	cout<<h<<endl;
    foo.count();//1的个数
	foo.flip();//全部翻转
	foo.set();//变1
	foo.reset();//变0
	foo.to_string();
	foo.to_ulong();
	foo.to_ullong();
	foo._Find_first();
	foo._Find_next();
    //位运算:<< 变大,>> 变小
    __builtin_clz();//前导 0
    __builtin_ctz();//后面的 0
}
\end{lstlisting}

输出:

\begin{verbatim}
0011000000
1010010000
1010010000
0000100101
\end{verbatim}

\subsection{pb\_ds}

\begin{lstlisting}
#pragma GCC optimize("Ofast")
#pragma GCC target("popcnt","sse3","sse2","sse","avx","sse4","sse4.1","sse4.2","ssse3","f16c","fma","avx2","xop","fma4")
#pragma GCC optimize("inline","fast-math","unroll-loops","no-stack-protector")
#pragma GCC diagnostic error "-fwhole-program"
#pragma GCC diagnostic error "-fcse-skip-blocks"
#pragma GCC diagnostic error "-funsafe-loop-optimizations"
#include "bits/stdc++.h"
#include "ext/pb_ds/assoc_container.hpp"
#include "ext/pb_ds/tree_policy.hpp" //balanced tree
#include "ext/pb_ds/hash_policy.hpp" //hash table
#include "ext/pb_ds/priority_queue.hpp" //priority_queue
using namespace __gnu_pbds;
using namespace std;
inline char gc()
{
    static char buf[1048576], *p1, *p2;
    return p1 == p2 && (p2 = (p1 = buf) + fread(buf, 1, 1048576, stdin),
    p1 == p2) ? EOF : *p1++;
}
inline int read()
{
    char ch = gc(); int r = 0, w = 1;
    for (; ch < '0' || ch > '9'; ch = gc()) if (ch == '-') w = -1;
    for (; '0' <= ch && ch <= '9'; ch = gc()) r = r * 10 + (ch - '0');
    return r * w;
}
typedef tree<int,null_type,less<int>,rb_tree_tag,tree_order_statistics_node_update> rbtree;
cc_hash_table<string,int>mp1;//拉链法
gp_hash_table<string,int>mp2;//查探法
rbtree s1,s2;//注意是不可重的
//null_type无映射(低版本g++为null_mapped_type)
//less<int>从小到大排序
//插入t.insert();
//删除t.erase();
//求有多少个数比 k 小:t.order_of_key(k);
//求树中第 k+1 小:t.find_by_order(k);
//a.join(b) b并入a,前提是两棵树的 key 的取值范围不相交,b 会清空但迭代器没事,如不满足会抛出异常。我听说复杂度是线性???
//a.split(v,b) key 小于等于 v 的元素属于 a,其余的属于 b
//T.lower_bound(x)  >=x 的 min 的迭代器
//T.upper_bound(x)  >x 的 min 的迭代器
__gnu_pbds::priority_queue<int,greater<int>,pairing_heap_tag> pq;
//join(priority_queue &other)  //合并两个堆,other会被清空
//split(Pred prd,priority_queue &other)  //分离出两个堆
//modify(point_iterator it,const key)  //修改一个节点的值
int main()
{
	ios::sync_with_stdio(0);cin.tie(0);
	mt19937 rnd(chrono::steady_clock::now().time_since_epoch().count());
    cout<<setiosflags(ios::fixed)<<setprecision(15);
	rbtree::iterator it;
    uniform_real_distribution<> a(1,2);
	numeric_limits<int>::max();
	for (int i=1;i<=10;i++) s1.insert(i*2);
	//it=s2.lower_bound(35);
	for (auto u:s1) printf("%d\n",u);puts("");
	printf("%d\n",*s1.find_by_order(10));
	//printf("%d\n",*it);
}
\end{lstlisting}

\subsection{python 使用方法}

\begin{lstlisting}
fi = open("discuss.in", "r")
fo = open("discuss.out", "w")
n=int(fi.readline())
fo.write(str(ans))
\end{lstlisting}


\newpage

\section{其他板子(补充)}

\subsection{MTT+exp}

\begin{lstlisting}
#include<bits/stdc++.h>
using namespace std;
typedef long long ll;
typedef double db;
int read(){
	int res=0;
	char c=getchar(),f=1;
	while(c<48||c>57){if(c=='-')f=0;c=getchar();}
	while(c>=48&&c<=57)res=(res<<3)+(res<<1)+(c&15),c=getchar();
	return f?res:-res;
}

const int L=1<<19,mod=1e9+7;
const db pi2=3.141592653589793*2;
int inc(int x,int y){return x+y>=mod?x+y-mod:x+y;}
int dec(int x,int y){return x-y<0?x-y+mod:x-y;}
int mul(int x,int y){return (ll)x*y%mod;}
int qpow(int x,int y){
	int res=1;
	for(;y;y>>=1)res=y&1?mul(res,x):res,x=mul(x,x);
	return res;
}
int inv(int x){return qpow(x,mod-2);}

struct cp{
	db x,y;
	cp(){}
	cp(db a,db b){x=a,y=b;}
	cp operator+(const cp& p)const{return cp(x+p.x,y+p.y);}
	cp operator-(const cp& p)const{return cp(x-p.x,y-p.y);}
	cp operator*(const cp& p)const{return cp(x*p.x-y*p.y,x*p.y+y*p.x);}
	cp conj(){return cp(x,-y);}
}w[L];
int re[L];
int getre(int n){
	int len=1,bit=0;
	while(len<n)++bit,len<<=1;
	for(int i=1;i<len;++i)re[i]=(re[i>>1]>>1)|((i&1)<<(bit-1));
	return len;
}
void getw(){
	for(int i=0;i<L;++i)w[i]=cp(cos(pi2/L*i),sin(pi2/L*i));
}
void fft(cp* a,int len,int m){
	for(int i=1;i<len;++i)if(i<re[i])swap(a[i],a[re[i]]);
	for(int k=1,r=L>>1;k<len;k<<=1,r>>=1)
		for(int i=0;i<len;i+=k<<1)
			for(int j=0;j<k;++j){
				cp &L=a[i+j],&R=a[i+j+k],t=w[r*j]*R;
				R=L-t,L=L+t;
			}
	if(!~m){
		reverse(a+1,a+len);
		cp tmp=cp(1.0/len,0);
		for(int i=0;i<len;++i)a[i]=a[i]*tmp;
	}
}
void mul(int* a,int* b,int* c,int n1,int n2,int n){
	static cp f1[L],f2[L],f3[L],f4[L];
	int len=getre(n1+n2-1);
	for(int i=0;i<len;++i){
		f1[i]=i<n1?cp(a[i]>>15,a[i]&32767):cp(0,0);
		f2[i]=i<n2?cp(b[i]>>15,b[i]&32767):cp(0,0);
	}
	fft(f1,len,1),fft(f2,len,1);
	cp t1=cp(0.5,0),t2=cp(0,-0.5),r=cp(0,1);
	cp x1,x2,x3,x4;
	for(int i=0;i<len;++i){
		int j=(len-i)&(len-1);
		x1=(f1[i]+f1[j].conj())*t1;
		x2=(f1[i]-f1[j].conj())*t2;
		x3=(f2[i]+f2[j].conj())*t1;
		x4=(f2[i]-f2[j].conj())*t2;
		f3[i]=x1*(x3+x4*r);
		f4[i]=x2*(x3+x4*r);
	}
	fft(f3,len,-1),fft(f4,len,-1);
	ll c1,c2,c3,c4;
	for(int i=0;i<n;++i){
		c1=(ll)(f3[i].x+0.5)%mod,c2=(ll)(f3[i].y+0.5)%mod;
		c3=(ll)(f4[i].x+0.5)%mod,c4=(ll)(f4[i].y+0.5)%mod;
		c[i]=((((c1<<15)+c2+c3)<<15)+c4)%mod;
	}
}
void inv(int* a,int* b,int n){
	if(n==1){b[0]=1;return;}
	static int c[L];
	int l=(n+1)>>1;
	inv(a,b,l);
	mul(a,b,c,n,l,n);
	for(int i=0;i<n;++i)c[i]=mod-c[i];
	c[0]+=2;
	mul(b,c,b,n,n,n);
}
void der(int* a,int n){
	for(int i=1;i<n;++i)a[i-1]=mul(a[i],i);
	a[n-1]=0;
}
void its(int* a,int n){
	for(int i=n-1;i;--i)a[i]=mul(a[i-1],inv(i));
	a[0]=0;
}
void ln(int* a,int* b,int n){
	static int c[L];
	for(int i=0;i<n;++i)c[i]=a[i];
	der(c,n);
	inv(a,b,n);
	mul(b,c,b,n,n,n);
	its(b,n);
}
void exp(int* a,int* b,int n){
	if(n==1){b[0]=1;return;}
	static int c[L];
	int l=(n+1)>>1;
	exp(a,b,l);
	ln(b,c,n);
	for(int i=0;i<n;++i)c[i]=dec(a[i],c[i]);
	++c[0];
	mul(b,c,b,l,n,n);
	for(int i=0;i<n;++i)c[i]=0;
}

int n,k,a[L],f[L],g[L];
int main(){
	getw();
	n=read(),k=read();
	for(int i=1;i<=k;++i)a[i]=inv(i);
	for(int i=2;i<=n;++i)
		for(int j=1;i*j<=k;++j)
			f[i*j]=inc(f[i*j],a[j]);
	for(int i=1;i<=k;++i)f[i]=mod-f[i];
	for(int i=1;i<=k;++i)f[i]=inc(f[i],mul(n-1,a[i]));
	exp(f,g,k+1);
	printf("%d\n",g[k]);
}
\end{lstlisting}

\subsection{多项式}

\begin{lstlisting}
#include<bits/stdc++.h>
using namespace std;
typedef long long ll;
int read(){
	int res=0;
	char c=getchar(),f=1;
	while(c<48||c>57){if(c=='-')f=0;c=getchar();}
	while(c>=48&&c<=57)res=(res<<3)+(res<<1)+(c&15),c=getchar();
	return f?res:-res;
}
void write(int x){
	char c[21];
	int len=0;
	if(!x)return putchar('0'),void();
	if(x<0)x=-x,putchar('-');
	while(x)c[++len]=x%10,x/=10;
	while(len)putchar(c[len--]+48);
}
#define space(x) write(x),putchar(' ')
#define enter(x) write(x),putchar('\n')

const int mod=998244353;
struct M{
	int x;
	M(int a=0):x(a){}
	M operator+(const M& p)const{return x+p.x>=mod?x+p.x-mod:x+p.x;}
	M operator-()const{return x?mod-x:0;}
	M operator-(const M& p)const{return x-p.x<0?x-p.x+mod:x-p.x;}
	M operator*(const M& p)const{return (ll)x*p.x%mod;}
	bool operator==(const int& p)const{return x==p;}
	void operator+=(const M& p){*this=*this+p;}
	void operator-=(const M& p){*this=*this-p;}
	void operator*=(const M& p){*this=*this*p;}
};
void write(const M& x){write(x.x);}

M qpow(M x,int y){
	M res(1);
	for(;y;y>>=1)res=y&1?res*x:res,x=x*x;
	return res;
}
M inv(M x){return qpow(x,mod-2);}

const int N=1<<21|7;
namespace NTT{
int re[N];
M w[2][N];
int getre(int n){
	int len=1,bit=0;
	while(len<n)len<<=1,++bit;
	for(int i=1;i<len;++i)re[i]=(re[i>>1]>>1)|((i&1)<<(bit-1));
	w[0][0]=w[1][0]=1,w[0][1]=qpow(3,(mod-1)/len),w[1][1]=inv(w[0][1]);
	for(int o=0;o<2;++o)for(int i=2;i<=len;++i)
		w[o][i]=w[o][i-1]*w[o][1];
	return len;
}
void NTT(M* a,int n,int o=0){
	for(int i=1;i<n;++i)if(i<re[i])swap(a[i],a[re[i]]);
	M L,R;
	for(int k=1;k<n;k<<=1)
		for(int i=0,st=n/(k<<1);i<n;i+=k<<1)
			for(int j=0,nw=0;j<k;++j,nw+=st){
				L=a[i+j],R=a[i+j+k]*w[o][nw];
				a[i+j]=L+R,a[i+j+k]=L-R;
			}
	if(o){
		L=inv(n);
		for(int i=0;i<n;++i)a[i]=a[i]*L;
	}
}

M t0[N],t1[N],t2[N];
void mul(const M* a,const M* b,M* c,int n,int m){
	int len=getre(n+m+1);
	memset(t0,0,sizeof(int)*len),memcpy(t0,a,sizeof(int)*(n+1));
	memset(t1,0,sizeof(int)*len),memcpy(t1,b,sizeof(int)*(m+1));
	NTT(t0,len),NTT(t1,len);
	for(int i=0;i<len;++i)t0[i]=t0[i]*t1[i];
	NTT(t0,len,1);
	memcpy(c,t0,sizeof(int)*(n+m+1));
}
void inv(const M* a,M* b,int n){
	int len=1;
	while(len<=n)len<<=1;
	memset(t0,0,sizeof(int)*len),memcpy(t0,a,sizeof(int)*(n+1));
	memset(t1,0,sizeof(int)*(len<<1));
	memset(t2,0,sizeof(int)*(len<<1));
	t2[0]=inv(t0[0]);
	for(int k=1;k<=len;k<<=1){
		memcpy(t1,t0,sizeof(int)*k);
		getre(k<<1);
		NTT(t1,k<<1),NTT(t2,k<<1);
		for(int i=0;i<(k<<1);++i)t2[i]*=(-t1[i]*t2[i]+2);
		NTT(t2,k<<1,1);
		for(int i=k;i<(k<<1);++i)t2[i]=0;
	}
	memcpy(b,t2,sizeof(int)*(n+1));
}
} //namespace NTT

struct poly:public vector<M>{
	int time()const{return size()-1;}
	poly(int tim=0,int c=0){
		resize(tim+1);
		if(tim>=0)at(0)=c;
	}
	poly operator%(const int& n)const{
		poly r(*this);
		r.resize(n);
		return r;
	}
	poly operator%=(const int& n){
		resize(n);
		return *this;
	}
	poly operator+(const poly& p)const{
		int n=time(),m=p.time();
		poly r(*this);
		if(n<m)r.resize(m+1);
		for(int i=0;i<=m;++i)r[i]+=p[i];
		return r;
	}
	poly operator-(const poly& p)const{
		int n=time(),m=p.time();
		poly r(*this);
		if(n<m)r.resize(m+1);
		for(int i=0;i<=m;++i)r[i]-=p[i];
		return r;
	}
	poly operator*(const poly& p)const{
		poly r(time()+p.time());
		NTT::mul(&((*this)[0]),&p[0],&r[0],time(),p.time());
		return r;
	}
};

poly inv(const poly& a){
	poly r(a.time());
	NTT::inv(&a[0],&r[0],a.time());
	return r;
}

poly der(const poly& a){
	int n=a.time();
	poly r(n-1);
	for(int i=1;i<=n;++i)r[i-1]=a[i]*i;
	return r;
}
M _[N];
poly itr(const poly& a){
	int n=a.time();
	poly r(n+1);
	_[1]=1;
	for(int i=2;i<=n+1;++i)_[i]=_[mod%i]*(mod-mod/i);
	for(int i=0;i<=n;++i)r[i+1]=a[i]*_[i+1];
	return r;
}
poly ln(const poly& a){
	return itr(der(a)*inv(a)%a.time());
}

poly exp(const poly& a){
	poly r(0,1);
	int n=a.time(),k=1;
	while(r.time()<n)
		r%=k,r=r*(a%k-ln(r)+poly(0,1))%k,k<<=1;
	return r%(n+1);
}

void read(poly& a,int n=-1){
	if(!~n)n=a.time();
	else a.resize(n+1);
	for(int i=0;i<=n;++i)a[i]=read();
}
void write(const poly& a,int n=-1){
	if(!~n)n=a.time();
	else n=min(n,a.time());
	for(int i=0;i<n;++i)space(a[i]);
	enter(a[n]);
}
\end{lstlisting}

\subsection{Miller Rabin/Pollard Rho}

1s:$200$ 组 $10^{18}$。

\begin{lstlisting}
namespace pr
{
	typedef long long ll;
	typedef __int128 lll;
	typedef pair<ll,int> pa;
	ll ksm(ll x,ll y,const ll p)
	{
		ll r=1;
		while (y)
		{
			if (y&1) r=(lll)r*x%p;
			x=(lll)x*x%p; y>>=1;
		}
		return r;
	}
	namespace miller
	{
		const int p[7]={2,3,5,7,11,61,24251};
		ll s,t;
		bool test(ll n,int p)
		{
			if (p>=n) return 1;
			ll r=ksm(p,t,n),w;
			for (int j=0; j<s&&r!=1; j++)
			{
				w=(lll)r*r%n;
				if (w==1&&r!=n-1) return 0;
				r=w;
			}
			return r==1;
		}
		bool prime(ll n)
		{
			if (n<2||n==46'856'248'255'981ll) return 0;
			for (int i=0; i<7; ++i) if (n%p[i]==0) return n==p[i];
			s=__builtin_ctz(n-1); t=n-1>>s;
			for (int i=0; i<7; ++i) if (!test(n,p[i])) return 0;
			return 1;
		}
	}
	using miller::prime;
	mt19937_64 rnd(chrono::steady_clock::now().time_since_epoch().count());
	namespace rho
	{
		void nxt(ll &x,ll &y,ll &p) { x=((lll)x*x+y)%p; }
		ll find(ll n,ll C)
		{
			ll l,r,d,p=1;
			l=rnd()%(n-2)+2,r=l;
			nxt(r,C,n);
			int cnt=0;
			while (l^r)
			{
				p=(lll)p*llabs(l-r)%n;
				if (!p) return gcd(n,llabs(l-r));
				++cnt;
				if (cnt==127)
				{
					cnt=0;
					d=gcd(llabs(l-r),n);
					if (d>1) return d;
				}
				nxt(l,C,n); nxt(r,C,n); nxt(r,C,n);
			}
			return gcd(n,p);
		}
		vector<pa> w;
		vector<ll> d;
		void dfs(ll n,int cnt)
		{
			if (n==1) return;
			if (prime(n)) return w.emplace_back(n,cnt),void();
			ll p=n,C=rnd()%(n-1)+1;
			while (p==1||p==n) p=find(n,C++);
			int r=1; n/=p;
			while (n%p==0) n/=p,++r;
			dfs(p,r*cnt); dfs(n,cnt);
		}
		vector<pa> getw(ll n)
		{
			w=vector<pa>(0); dfs(n,1);
			if (n==1) return w;
			sort(w.begin(),w.end());
			int i,j;
			for (i=1,j=0; i<w.size(); i++) if (w[i].first==w[j].first) w[j].second+=w[i].second; else w[++j]=w[i];
			w.resize(j+1);
			return w;
		}
		void dfss(int x,ll n)
		{
			if (x==w.size()) return d.push_back(n),void();
			dfss(x+1,n);
			for (int i=1; i<=w[x].second; i++) dfss(x+1,n*=w[x].first);
		}
		vector<ll> getd(ll n)
		{
			getw(n); d=vector<ll>(0); dfss(0,1);
			sort(d.begin(),d.end());
			return d;
		}
	}
	using rho::getw,rho::getd;
	using miller::prime;
}
using pr::getw,pr::getd;
\end{lstlisting}

\subsection{半平面交}

\begin{lstlisting}
const int N=305;
const db inf=1e15,eps=1e-10;
int sign(db x){
	if(fabs(x)<eps)return 0;
	return x>0?1:-1;
}

struct vec{
	db x,y;
	vec(){}
	vec(db a,db b){x=a,y=b;}
	vec operator+(const vec& p)const{
		return vec(x+p.x,y+p.y);
	}
	vec operator-(const vec& p)const{
		return vec(x-p.x,y-p.y);
	}
	db operator*(const vec& p)const{
		return x*p.y-y*p.x;
	}
	vec operator*(const db& p)const{
		return vec(x*p,y*p);
	}
}p1[N],p2[N];

struct line{
	vec s,t;
	line(){}
	line(vec a,vec b){s=a,t=b;}
}a[N],q[N];
db ang(vec v){
	return atan2(v.y,v.x);
}
db ang(line l){
	return ang(l.t-l.s);
}
bool cmp(line x,line y){
	int s=sign(ang(x)-ang(y));
	return s?s<0:sign((x.t-x.s)*(y.t-x.s))>0;
}

vec inter(line x,line y){
	vec a=y.s-x.s,b=x.t-x.s,c=y.t-y.s;
	return y.s+c*((a*b)/(b*c));
}
bool out(line l,vec p){
	return sign((l.t-l.s)*(p-l.s))<0;
}

int n,tot=0;
db ans=inf;
int main(){
	scanf("%d",&n);
	for(int i=1;i<=n;++i)scanf("%lf",&p1[i].x);
	for(int i=1;i<=n;++i)scanf("%lf",&p1[i].y);
	for(int i=1;i<n;++i)a[i]=line(p1[i],p1[i+1]);
	a[n]=line(vec(p1[1].x,inf),vec(p1[1].x,p1[1].y));
	a[n+1]=line(vec(p1[n].x,p1[n].y),vec(p1[n].x,inf));
	
	sort(a+1,a+n+2,cmp);
	for(int i=1;i<=n;++i){
		if(!sign(ang(a[i])-ang(a[i+1])))continue;
		a[++tot]=a[i];
	}a[++tot]=a[n+1];
	
	int l=1,r=0;
	q[++r]=a[1],q[++r]=a[2];
	for(int i=3;i<=tot;++i){
		while(l<r&&out(a[i],inter(q[r],q[r-1])))--r;
		while(l<r&&out(a[i],inter(q[l],q[l+1])))++l;
		q[++r]=a[i];
	}
	while(l<r&&out(q[l],inter(q[r],q[r-1])))--r;
	while(l<r&&out(q[r],inter(q[l],q[l+1])))++l;
//......
}
\end{lstlisting}

\subsection{旋转卡壳}

\begin{lstlisting}
if(top==3)return !printf("%d\n",dis(a[sta[1]],a[sta[2]]));
for(int i=1,j=2;i<top;++i){
	while(area(a[sta[i]],a[sta[i+1]],a[sta[j]])>=area(a[sta[i]],a[sta[i+1]],a[sta[j%top+1]]))j=j%top+1;
	ans=max(ans,max(dis(a[sta[i]],a[sta[j]]),dis(a[sta[i+1]],a[sta[j]])));
}printf("%d\n",ans);
\end{lstlisting}

\subsection{l1ll5 trac}

题意:

给一个n个点的完全图,边有颜色。将其扩展到(m+1)阶完全图,并且m染色,判断是否可行并输出方案。

题解:

Lemma 1:考虑最终所有 m(m+1)/2 条边,按照颜色分组,仅当每组都有(m+1)/2条边且m为偶数时有解

Proof:考虑边数最多的一组,令其有 x 条边,则其覆盖了 2x 个点,min(x) = (m+1)/2 ,当m为偶数时,2x=m,否则组内有重复点。

现在考虑n个点的情况,同样考虑最终的m组,将每组的(m+1)/2条边分为三类:

2-set:两个端点均已经在目前的图中出现

1-set:仅有一个端点在目前的图中出现(某点x的目前所有边中没有这个颜色,则在最终的图中必然有这种颜色的一条边)

0-set:以上两种情况之外的边,即完全没有出现。

Lemma 2:对于给出的情况,若没有某一组的1-set边数加2-set边数超过(m+1)/2,则有解。否则无解。

Proof:无解显然,为了证明有解,只需证明一个满足如上性质的图一定可以加一个点。(归纳,加一个点的影响是将边从1-set变成2-set(加某个颜色的边)或者从0-set变成1-set(不加某个颜色的边),不影响组内边数)

考虑用网络流解决该问题:对于每种颜色,将其和对应的1-set(n个点)或0-set(1个点)连边。其中0-set到T连一条m-n,其余都是1

显然源点总共发出m的流量,汇点最多收到m的流量。

不妨令每个颜色的点发出去的所有流量均相等,此时发现该网络满流。(不会证)

则最大流为满流,任取一整数最大流则对应一个方案。依次扩展到m+1个点即可。

感觉非常的玄妙。

因为l1ll5博客炸了,平时补的一些题就放在这里了。

CF 717 E

长度为n的排列,swap k次,能得到的不同排列数。

一个思路是考虑这个的逆过程,对所有swap j次能得到一个1-n的排列的排列计数,只需要考虑最后一个元素是否为错排即可

$dp_{i,j}=dp_{i-1,j}+(i-1)dp_{i-1,j-1}$

$ans_i = dp_{i,k}+dp_{i-2,k} + \ldots$ 这里考虑到浪费操作数就是交换相同对

但是复杂度是与 $n$ 有关的,不妨考虑枚举做了 $k$ 次swap,影响到了i个元素,答案乘一个C(n,i)即可

显然有重复,怎么避免呢,考虑只要枚举的i个元素都是错排即可不重不漏,容斥这个过程即可。

\subsection{多项式复合 (yurzhang)}

$O(n\log n\sqrt{n\log n})$,奇慢无比,慎用

\begin{lstlisting}
#pragma GCC optimize("Ofast,inline")
#pragma GCC target("sse,sse2,sse3,ssse3,sse4,sse4.1,sse4.2,popcnt,abm,mmx,avx,avx2,tune=native")
#include <cstdio>
#include <cstring>
#include <cmath>
#include <algorithm>

#define MOD 998244353
#define G 332748118
#define N 262210
#define re register
#define gc pa==pb&&(pb=(pa=buf)+fread(buf,1,100000,stdin),pa==pb)?EOF:*pa++
typedef long long ll;
static char buf[100000],*pa(buf),*pb(buf);
static char pbuf[3000000],*pp(pbuf),st[15];
int read() {
    re int x(0);re char c(gc);
    while(c<'0'||c>'9')c=gc;
    while(c>='0'&&c<='9')
        x=x*10+c-48,c=gc;
    return x;
}
void write(re int v) {
    if(v==0)
        *pp++=48;
    else {
        re int tp(0);
        while(v)
            st[++tp]=v%10+48,v/=10;
        while(tp)
            *pp++=st[tp--];
    }
    *pp++=32;
}

int pow(re int a,re int b) {
    re int ans(1);
    while(b)
        ans=b&1?(ll)ans*a%MOD:ans,a=(ll)a*a%MOD,b>>=1;
    return ans;
}

int inv[N],ifac[N];
void pre(re int n) {
    inv[1]=ifac[0]=1;
    for(re int i(2);i<=n;++i)
        inv[i]=(ll)(MOD-MOD/i)*inv[MOD%i]%MOD;
    for(re int i(1);i<=n;++i)
        ifac[i]=(ll)ifac[i-1]*inv[i]%MOD;
}

int getLen(re int t) {
	return 1<<(32-__builtin_clz(t));
}

int lmt(1),r[N],w[N];
void init(re int n) {
	re int l(0);
	while(lmt<=n)
		lmt<<=1,++l;
	for(re int i(1);i<lmt;++i)
		r[i]=(r[i>>1]>>1)|((i&1)<<(l-1));
	re int wn(pow(3,(MOD-1)/lmt));
	w[lmt>>1]=1;
	for(re int i((lmt>>1)+1);i<lmt;++i)
		w[i]=(ll)w[i-1]*wn%MOD;
	for(re int i((lmt>>1)-1);i;--i)
		w[i]=w[i<<1];
}

void DFT(int*a,re int l) {
	static unsigned long long tmp[N];
	re int u(__builtin_ctz(lmt)-__builtin_ctz(l)),t;
	for(re int i(0);i<l;++i)
		tmp[i]=(a[r[i]>>u])%MOD;
	for(re int i(1);i<l;i<<=1)
		for(re int j(0),step(i<<1);j<l;j+=step)
			for(re int k(0);k<i;++k)
				t=(ll)w[i+k]*tmp[i+j+k]%MOD,
				tmp[i+j+k]=tmp[j+k]+MOD-t,
				tmp[j+k]+=t;
	for(re int i(0);i<l;++i)
		a[i]=tmp[i]%MOD;
}

void IDFT(int*a,re int l) {
	std::reverse(a+1,a+l);DFT(a,l);
	re int bk(MOD-(MOD-1)/l);
	for(re int i(0);i<l;++i)
		a[i]=(ll)a[i]*bk%MOD;
}

int n,m;
int a[N],b[N],c[N];

void getInv(int*a,int*b,int deg) {
    if(deg==1)
        b[0]=pow(a[0],MOD-2);
    else {
        static int tmp[N];
        getInv(a,b,(deg+1)>>1);
        re int l(getLen(deg<<1));
        for(re int i(0);i<l;++i)
            tmp[i]=i<deg?a[i]:0;
        DFT(tmp,l),DFT(b,l);
        for(re int i(0);i<l;++i)
            b[i]=(2ll-(ll)tmp[i]*b[i]%MOD+MOD)%MOD*b[i]%MOD;
        IDFT(b,l);
        for(re int i(deg);i<l;++i)
            b[i]=0;
    }
}

void getDer(int*a,int*b,int deg) {
    for(re int i(0);i+1<deg;++i)
        b[i]=(ll)a[i+1]*(i+1)%MOD;
    b[deg-1]=0;
}

void getComp(int*a,int*b,int k,int m,int&n,int*c,int*d) {
    if(k==1) {
        for(re int i(0);i<m;++i)
            c[i]=0,d[i]=b[i];
        n=m,c[0]=a[0];
    } else {
        static int t1[N],t2[N];
        int nl(n),nr(n),*cl,*cr,*dl,*dr;
        getComp(a,b,k>>1,m,nl,cl=c,dl=d);
        getComp(a+(k>>1),b,(k+1)>>1,m,nr,cr=c+nl,dr=d+nl);
        n=std::min(n,nl+nr-1);
        re int _l(getLen(nl+nr));
        for(re int i(0);i<_l;++i)
            t1[i]=i<nl?dl[i]:0;
        for(re int i(0);i<_l;++i)
            t2[i]=i<nr?cr[i]:0;
        DFT(t1,_l),DFT(t2,_l);
        for(re int i(0);i<_l;++i)
            t2[i]=(ll)t1[i]*t2[i]%MOD;
        IDFT(t2,_l);
        for(re int i(0);i<n;++i)
            c[i]=((i<nl?cl[i]:0)+t2[i])%MOD;
        for(re int i(0);i<_l;++i)
            t2[i]=i<nr?dr[i]:0;
        DFT(t2,_l);
        for(re int i(0);i<_l;++i)
            t2[i]=(ll)t1[i]*t2[i]%MOD;
        IDFT(t2,_l);
        for(re int i(0);i<n;++i)
            d[i]=t2[i];
    }
}

void getComp(int*a,int*b,int*c,int deg) {
    static int ts[N],ps[N],c0[N],_t1[N],idM[N];
    int M(std::max((int)ceil(sqrt(deg/log2(deg))*2.5),2)),_n(deg+deg/M);
    getComp(a,b,deg,M,_n,c0,_t1);
    re int _l(getLen(_n+deg));
    for(re int i(_n);i<_l;++i)
        c0[i]=0;
    for(re int i(0);i<_l;++i)
        ps[i]=i==0;
    for(re int i(0);i<_l;++i)
        ts[i]=M<=i&&i<deg?b[i]:0;
    getDer(b,_t1,M);
    for(re int i(M-1);i<deg;++i)
        _t1[i]=0; /// Important!!!
    getInv(_t1,idM,deg);
    for(int i=deg;i<_l;++i)
    	idM[i]=0;
    DFT(ts,_l),DFT(idM,_l);
    for(re int t(0);t*M<deg;++t) {
        for(re int i(0);i<_l;++i)
            _t1[i]=i<deg?c0[i]:0;
        DFT(ps,_l),DFT(_t1,_l);
        for(re int i(0);i<_l;++i)
            _t1[i]=(ll)_t1[i]*ps[i]%MOD,
            ps[i]=(ll)ps[i]*ts[i]%MOD;
        IDFT(ps,_l),IDFT(_t1,_l);
        for(re int i(deg);i<_l;++i)
            ps[i]=0;
        for(re int i(0);i<deg;++i)
            c[i]=((ll)_t1[i]*ifac[t]+c[i])%MOD;
        getDer(c0,c0,_n);
        for(re int i(_n-1);i<_l;++i)
            c0[i]=0;
        DFT(c0,_l);
        for(re int i(0);i<_l;++i)
            c0[i]=(ll)c0[i]*idM[i]%MOD;
        IDFT(c0,_l);
        for(re int i(_n-1);i<_l;++i)
            c0[i]=0;
    }
}

int main() {
    n=read(),m=read();
    for(re int i(0);i<=n;++i)
        a[i]=read();
    for(re int i(0);i<=m;++i)
        b[i]=read();
    
    m=(n>m?n:m)+1;
    pre(m);init(m*5);
    getComp(a,b,c,m);
    
    for(re int i(0);i<=n;++i)
        write(c[i]);
    fwrite(pbuf,1,pp-pbuf,stdout);
    return 0;
}
\end{lstlisting}

\subsection{下降幂多项式乘法}

$O(n\log n)$。

\begin{lstlisting}
#include<cstdio>
#include<algorithm>
const int N=524288,md=998244353,g3=(md+1)/3;
typedef long long LL;
int n,m,A[N],B[N],fac[N],iv[N],rev[N],C[N],g[20][N],lim,M;
int pow(int a,int b){
    int ret=1;
    for(;b;b>>=1,a=(LL)a*a%md)if(b&1)ret=(LL)ret*a%md;
    return ret;
}
void upd(int&a){a+=a>>31&md;}
void init(int n){
    int l=-1;
    for(lim=1;lim<n;lim<<=1)++l;M=l+1;
    for(int i=1;i<lim;++i)
    rev[i]=((rev[i>>1])>>1)|((i&1)<<l);
}
void NTT(int*a,int f){
    for(int i=1;i<lim;++i)if(i<rev[i])std::swap(a[i],a[rev[i]]);
    for(int i=0;i<M;++i){
        const int*G=g[i],c=1<<i;
        for(int j=0;j<lim;j+=c<<1)
        for(int k=0;k<c;++k){
            const int x=a[j+k],y=a[j+k+c]*(LL)G[k]%md;
            upd(a[j+k]+=y-md),upd(a[j+k+c]=x-y);
        }
    }
    if(!f){
        const int iv=pow(lim,md-2);
        for(int i=0;i<lim;++i)a[i]=(LL)a[i]*iv%md;
        std::reverse(a+1,a+lim);
    }
}
int main(){
    scanf("%d%d",&n,&m);++n,++m;
    for(int i=0;i<20;++i){
        int*G=g[i];
        G[0]=1;
        const int gi=G[1]=pow(3,(md-1)/(1<<i+1));
        for(int j=2;j<1<<i;++j)G[j]=(LL)G[j-1]*gi%md;
    }
    for(int i=0;i<n;++i)scanf("%d",A+i);
    for(int i=0;i<m;++i)scanf("%d",B+i);
    for(int i=*fac=1;i<N;++i)
    fac[i]=fac[i-1]*(LL)i%md;
    iv[N-1]=pow(fac[N-1],md-2);
    for(int i=N-2;~i;--i)iv[i]=(i+1LL)*iv[i+1]%md;
    init(n+m<<1);
    for(int i=0;i<n+m-1;++i)C[i]=iv[i];
    NTT(A,1),NTT(B,1),NTT(C,1);
    for(int i=0;i<lim;++i)A[i]=(LL)A[i]*C[i]%md,B[i]=(LL)B[i]*C[i]%md;
    NTT(A,0),NTT(B,0);
    for(int i=0;i<lim;++i)C[i]=0;
    for(int i=0;i<n+m-1;++i)
    C[i]=(i&1)?md-iv[i]:iv[i];
    for(int i=0;i<lim;++i)A[i]=(LL)A[i]*B[i]%md*fac[i]%md;
    for(int i=n+m-1;i<lim;++i)A[i]=0;
    NTT(A,1),NTT(C,1);
    for(int i=0;i<lim;++i)A[i]=(LL)A[i]*C[i]%md;
    NTT(A,0);
    for(int i=0;i<n+m-1;++i)printf("%d%c",A[i]," \n"[i==n+m-2]);
    return 0;
}
\end{lstlisting}

\subsection{平面欧几里得距离最小生成树}

$10^5$,400ms。

By Claris.

\begin{lstlisting}
#include<cstdio>
#include<algorithm>
#include<cmath>
using namespace std;
typedef long long ll;
const int N=100010;
const ll inf=2000000000000000001LL;
const double eps=1e-9;
inline int sgn(double x){
  if(x>eps)return 1;
  if(x<-eps)return -1;
  return 0;
}
struct P{
  double x,y;
  P(){}
  P(double _x,double _y){x=_x,y=_y;}
  bool operator<(const P&a)const{return sgn(x-a.x)<0||sgn(x-a.x)==0&&sgn(y-a.y)<0;}
  P operator-(const P&a)const{return P(x-a.x,y-a.y);}
  double operator&(const P&a)const{return x*a.y-y*a.x;}
  double operator|(const P&a)const{return x*a.x+y*a.y;}
}p[N];
struct PI{
  ll x,y;
  PI(){}
  PI(ll _x,ll _y){x=_x,y=_y;}
}loc[N],pool[N];
inline double check(const P&a,const P&b,const P&c){return (b-a)&(c-a);}
inline double dis2(const P&a){return a.x*a.x+a.y*a.y;}
inline bool cross(int a,int b,int c,int d){
  return sgn(check(p[a],p[c],p[d])*check(p[b],p[c],p[d]))<0&&sgn(check(p[c],p[a],p[b])*check(p[d],p[a],p[b]))<0;
}
inline ll dis(const PI&a,const PI&b){return (a.x-b.x)*(a.x-b.x)+(a.y-b.y)*(a.y-b.y);}
inline bool cmpx(const PI&a,const PI&b){return a.x<b.x;}
inline bool cmpy(int a,int b){return pool[a].y<pool[b].y;}
struct P3{
  double x,y,z;
  P3(){}
  P3(double _x,double _y,double _z){x=_x,y=_y,z=_z;}
  bool operator<(const P3&a)const{return sgn(x-a.x)<0||sgn(x-a.x)==0&&sgn(y-a.y)<0;}
  P3 operator-(const P3&a)const{return P3(x-a.x,y-a.y,z-a.z);}
  double operator|(const P3&a)const{return x*a.x+y*a.y+z*a.z;}
  P3 operator&(const P3&a)const{return P3(y*a.z-z*a.y,z*a.x-x*a.z,x*a.y-y*a.x);}
}ori[N];
inline P3 check(const P3&a,const P3&b,const P3&c){return (b-a)&(c-a);}
inline P3 gp3(const P&a){return P3(a.x,a.y,a.x*a.x+a.y*a.y);}
inline int cal(double x){
  int y=x;
  for(int i=y-2;i<=y+2;i++)if(!sgn(x-i))return i;
}
bool incir(int a,int b,int c,int d){
  P3 aa=gp3(p[a]),bb=gp3(p[b]),cc=gp3(p[c]),dd=gp3(p[d]);
  if(sgn(check(p[a],p[b],p[c]))<0)swap(bb,cc);
  return sgn(check(aa,bb,cc)|(dd-aa))<0;
}
int n,i,j,et,la[N],tot,l,r,q[N<<2];
struct E{
  int to,l,r;
  E(){}
  E(int _to,int _l,int _r=0){to=_to,l=_l,r=_r;}
}e[N<<5];
inline void add(int x,int y){
  e[++et]=E(y,la[x]),e[la[x]].r=et,la[x]=et;
  e[++et]=E(x,la[y]),e[la[y]].r=et,la[y]=et;
}
inline void del(int x){
  e[e[x].r].l=e[x].l;
  e[e[x].l].r=e[x].r;
  la[e[x^1].to]==x?la[e[x^1].to]=e[x].l:1;
}
void delaunay(int l,int r){
  if(r-l<=2){
    for(int i=l;i<r;i++)for(int j=i+1;j<=r;j++)add(i,j);
    return;
  }
  int i,j,mid=(l+r)>>1,ld=0,rd=0,id,op;
  delaunay(l,mid),delaunay(mid+1,r);
  for(tot=0,i=l;i<=r;q[++tot]=i++)
    while(tot>1&&sgn(check(p[q[tot-1]],p[q[tot]],p[i]))<0)tot--;
  for(i=1;i<tot&&!ld;i++)if(q[i]<=mid&&mid<q[i+1])ld=q[i],rd=q[i+1];
  for(;add(ld,rd),1;){
    id=op=0;
    for(i=la[ld];i;i=e[i].l)
      if(sgn(check(p[ld],p[rd],p[e[i].to]))>0)
        if(!id||incir(ld,rd,id,e[i].to))op=-1,id=e[i].to;
    for(i=la[rd];i;i=e[i].l)
      if(sgn(check(p[rd],p[ld],p[e[i].to]))<0)
        if(!id||incir(ld,rd,id,e[i].to))op=1,id=e[i].to;
    if(op==0)break;
    if(op==-1){
      for(i=la[ld];i;i=e[i].l)
      if(cross(rd,id,ld,e[i].to))del(i),del(i^1),i=e[i].r;
      ld=id;
    }else{
      for(i=la[rd];i;i=e[i].l)
      if(cross(ld,id,rd,e[i].to))del(i),del(i^1),i=e[i].r;
      rd=id;
    }
  }
}
namespace DS{
int m,tot,a[N],f[N],g[N],v[N<<1],nxt[N<<1],ed,col[N];ll w[N<<1];
double ans;
struct E{int x,y;ll w;E(){}E(int _x,int _y,ll _w){x=_x,y=_y,w=_w;}}e[N<<3];
inline bool cmp(const E&a,const E&b){return a.w<b.w;}
inline void newedge(int x,int y,ll z){e[++tot]=E(x,y,z);}
int F(int x){return f[x]==x?x:f[x]=F(f[x]);}
inline void merge(int x,int y,ll z){
  if(F(x)==F(y))return;
  f[f[x]]=f[y];
  v[++ed]=y;w[ed]=z;nxt[ed]=g[x];g[x]=ed;
  v[++ed]=x;w[ed]=z;nxt[ed]=g[y];g[y]=ed;
  ans+=sqrt(z);
}
inline void work(){
  sort(e+1,e+tot+1,cmp);
  for(ed=0,i=1;i<=n;i++)f[i]=i,g[i]=0;
  for(i=1;i<=tot;i++)merge(e[i].x,e[i].y,e[i].w);
  printf("%.15f\n",ans);
}
}
int main(){
  while(~scanf("%d",&n)){
    for(i=0;i<=n+1;i++)la[i]=0;
    et=1;
    DS::tot=0;
    for(i=1;i<=n;i++){
      ll x,y;
      scanf("%lld%lld",&x,&y);
      p[i]=P(x,y);
      loc[i]=PI(x,y);
      ori[i]=P3(x,y,i);
    }
    sort(p+1,p+n+1);
    sort(ori+1,ori+n+1);
    delaunay(1,n);
    for(i=1;i<=n;i++)for(j=la[i];j;j=e[j].l){
      int x=cal(ori[i].z),y=cal(ori[e[j].to].z);
      DS::newedge(x,y,dis(loc[x],loc[y]));
    }
    DS::work();
  }
}
\end{lstlisting}

\subsection{析合树}

解释一下本文可能用到的符号:$\wedge$ 逻辑与,$\vee$ 逻辑或。

\subsubsection{关于段的问题}

我们由一个小清新的问题引入:

> 对于一个 $1-n$ 的排列,我们称一个值域连续的区间为段。问一个排列的段的个数。比如,$\{5 ,3 ,4, 1 ,2\}$ 的段有:$[1,1],[2,2],[3,3],[4,4],[5,5],[2,3],[4,5],[1,3],[2,5],[1,5]$。

看到这个东西,感觉要维护区间的值域集合,复杂度好像挺不友好的。线段树可以查询某个区间是否为段,但不太能统计段的个数。

这里我们引入这个神奇的数据结构——析合树!

\subsubsection{连续段}

在介绍析合树之前,我们先做一些前提条件的限定。鉴于 LCA 的课件中给出的定义不易理解,为方便读者理解,这里给出一些不太严谨(但更容易理解)的定义。

\subsubsection{排列与连续段}

**排列**:定义一个 $n$ 阶排列 $P$ 是一个大小为 $n$ 的序列,使得 $P_i$ 取遍 $1,2,\cdots,n$。说得形式化一点,$n$ 阶排列 $P$ 是一个有序集合满足:

1. $|P|=n$.

2. $\forall i,P_i\in[1,n]$.

3. $\nexists i,j\in[1,n],P_i=P_j$.

   **连续段**:对于排列 $P$,定义连续段 $(P,[l,r])$ 表示一个区间 $[l,r]$,要求 $P_{l\sim r}$ 值域是连续的。说得更形式化一点,对于排列 $P$,连续段表示一个区间 $[l,r]$ 满足:

$$
(\nexists\ x,z\in[l,r],y\notin[l,r],\ P_x<P_y<P_z)
$$

特别地,当 $l>r$ 时,我们认为这是一个空的连续段,记作 $(P,\varnothing)$。

我们称排列 $P$ 的所有连续段的集合为 $I_P$,并且我们认为 $(P,\varnothing)\in I_P$。

\subsubsection{连续段的运算}

连续段是依赖区间和值域定义的,于是我们可以定义连续段的交并差的运算。

定义 $A=(P,[a,b]),B=(P,[x,y])$,且 $A,B\in I_P$。于是连续段的关系和运算可以表示为:

1. $A\subseteq B\iff x\le a\wedge b\le y$.
2. $A=B\iff a=x\wedge b=y$.
3. $A\cap B=(P,[\max(a,x),\min(b,y)])$.
4. $A\cup B=(P,[\min(a,x),\max(b,y)])$.
5. $A\setminus B=(P,\{i|i\in[a,b]\wedge i\notin[x,y]\})$.

其实这些运算就是普通的集合交并差放在区间上而已。

\subsubsection{连续段的性质}

连续段的一些显而易见的性质。我们定义 $A,B\in I_P,A \cap B \neq \varnothing,A \notin B,B \notin A$,那么有 $A\cup B,A\cap B,A\setminus B,B\setminus A\in I_P$。

证明?证明的本质就是集合的交并差的运算。

\subsubsection{析合树}

好的,现在讲到重点了。你可能已经猜到了,析合树正是由连续段组成的一棵树。但是要知道一个排列可能有多达 $O(n^2)$ 个连续段,因此我们就要抽出其中更基本的连续段组成析合树。

\subsubsection{本原段}

其实这个定义全称叫作 **本原连续段**。但笔者认为本原段更为简洁。

对于排列 $P$,我们认为一个本原段 $M$ 表示在集合 $I_P$ 中,不存在与之相交且不包含的连续段。形式化地定义,我们认为 $X\in I_P$ 且满足 $\forall A\in I_P,\ X\cap A= (P,\varnothing)\vee X\subseteq A\vee A\subseteq X$。

所有本原段的集合为 $M_P$. 显而易见,$(P,\varnothing)\in M_P$。

显然,本原段之间只有相离或者包含关系。并且你发现 **一个连续段可以由几个互不相交的本原段构成**。最大的本原段就是整个排列本身,它包含了其他所有本原段,因此我们认为本原段可以构成一个树形结构,我们称这个结构为 **析合树**。更严格地说,排列 $P$ 的析合树由排列 $P$ 的 **所有本原段** 组成。

前面干讲这么多的定义,不来点图怎么行。考虑排列 $P=\{9,1,10,3,2,5,7,6,8,4\}$. 它的本原段构成的析合树如下:

![p1](./images/div-com1.png)

在图中我们没有标明本原段。而图中 **每个结点都代表一个本原段**。我们只标明了每个本原段的值域。举个例子,结点 $[5,8]$ 代表的本原段就是 $(P,[6,9])=\{5,7,6,8\}$。于是这里就有一个问题:**什么是析点合点?**

\subsubsection{析点与合点}

这里我们直接给出定义,稍候再来讨论它的正确性。

1. **值域区间**:对于一个结点 $u$,用 $[u_l,u_r]$ 表示该结点的值域区间。
2. **儿子序列**:对于析合树上的一个结点 $u$,假设它的儿子结点是一个 **有序** 序列,该序列是以值域区间为元素的(单个的数 $x$ 可以理解为 $[x,x]$ 的区间)。我们把这个序列称为儿子序列。记作 $S_u$。
3. **儿子排列**:对于一个儿子序列 $S_u$,把它的元素离散化成正整数后形成的排列称为儿子排列。举个例子,对于结点 $[5,8]$,它的儿子序列为 $\{[5,5],[6,7],[8,8]\}$,那么把区间排序标个号,则它的儿子排列就为 $\{1,2,3\}$;类似的,结点 $[4,8]$ 的儿子排列为 $\{2,1\}$。结点 $u$ 的儿子排列记为 $P_u$。
4. **合点**:我们认为,儿子排列为顺序或者逆序的点为合点。形式化地说,满足 $P_u=\{1,2,\cdots,|S_u|\}$ 或者 $P_u=\{|S_u|,|S_u-1|,\cdots,1\}$ 的点称为合点。**叶子结点没有儿子排列,我们也认为它是合点**。
5. **析点**:不是合点的就是析点。

从图中可以看到,只有 $[1,10]$ 不是合点。因为 $[1,10]$ 的儿子排列是 $\{3,1,4,2\}$。

\subsubsection{析点与合点的性质}

析点与合点的命名来源于他们的性质。首先我们有一个非常显然的性质:对于析合树中任何的结点 $u$,其儿子序列区间的并集就是结点 $u$ 的值域区间。即 $\bigcup_{i=1}^{|S_u|}S_u[i]=[u_l,u_r]$。

对于一个合点 $u$:其儿子序列的任意 **子区间** 都构成一个 **连续段**。形式化地说,$\forall S_u[l\sim r]$,有 $\bigcup_{i=l}^rS_u[i]\in I_P$。

对于一个析点 $u$:其儿子序列的任意 **长度大于 1(这里的长度是指儿子序列中的元素数,不是下标区间的长度)** 的子区间都 **不** 构成一个 **连续段**。形式化地说,$\forall S_u[l\sim r],l<r$,有 $\bigcup_{i=l}^rS_u[i]\notin I_P$。

合点的性质不难证明。因为合点的儿子排列要么是顺序,要么是倒序,而值域区间也是首位相接,因此只要是连续的一段子序列(区间)都是一个连续段。

对于析点的性质可能很多读者就不太能理解了:为什么 **任意** 长度大于 $1$ 的子区间都不构成连续段?

使用反证法。假设对于一个点 $u$,它的儿子序列中有一个 **最长的** 区间 $S_u[l\sim r]$ 构成了连续段。那么这个 $A=\bigcup_{i=l}^rS_u[i]\in I_P$,也就意味着 $A$ 是一个本原段!(因为 $A$ 是儿子序列中最长的,因此找不到一个与它相交又不包含的连续段)于是你就没有使用所有的本原段构成这个析合树。矛盾。

\subsubsection{析合树的构造}

前面讲了这么多零零散散的东西,现在就来具体地讲如何构造析合树。LCA 大佬的线性构造算法我是没看懂的,今天就讲一下比较好懂的 $O(n\log n)$ 的算法。

我们考虑增量法。用一个栈维护前 $i-1$ 个元素构成的析合森林。在这里我需要 **着重强调**,析合森林的意思是,在任何时侯,栈中结点要么是析点要么是合点。现在考虑当前结点 $P_i$。

1. 我们先判断它能否成为栈顶结点的儿子,如果能就变成栈顶的儿子,然后把栈顶取出,作为当前结点。重复上述过程直到栈空或者不能成为栈顶结点的儿子。
2. 如果不能成为栈顶的儿子,就看能不能把栈顶的若干个连续的结点都合并成一个结点(判断能否合并的方法在后面),把合并后的点,作为当前结点。
3. 重复上述过程直到不能进行为止。然后结束此次增量,直接把当前结点压栈。

接下来我们仔细解释一下。

我们认为,如果当前点能够成为栈顶结点的儿子,那么栈顶结点是一个合点。如果是析点,那么你合并后这个析点就存在一个子连续段,不满足析点的性质。因此一定是合点。

如果无法成为栈顶结点的儿子,那么我们就看栈顶连续的若干个点能否与当前点一起合并。设 $l$ 为当前点所在区间的左端点。我们计算 $L_i$ 表示右端点下标为 $i$ 的连续段中,左端点 $< l$ 的最大值。当前结点为 $P_i$,栈顶结点记为 $t$。

1. 如果 $L_i$ 不存在,那么显然当前结点无法合并;
2. 如果 $t_l=L_i$,那么这就是两个结点合并,合并后就是一个 **合点**;
3. 否则在栈中一定存在一个点 $t'$ 的左端点 ${t'}_l=L_i$,那么一定可以从当前结点合并到 $t’$ 形成一个 **析点**;

最后,我们考虑如何处理 $L_i$。事实上,一个连续段 $(P,[l,r])$ 等价于区间极差与区间长度 -1 相等。即

$$
\max_{l\le i\le r}P_i-\min_{l\le i\le r}P_i=r-l
$$

而且由于 P 是一个排列,因此对于任意的区间 $[l,r]$ 都有

$$
\max_{l\le i\le r}P_i-\min_{l\le i\le r}P_i\ge r-l
$$

于是我们就维护 $\max_{l\le i\le r}P_i-\min_{l\le i\le r}P_i-(r-l)$,那么要找到一个连续段相当于查询一个最小值!

有了上述思路,不难想到这样的算法。对于增量过程中的当前的 $i$,我们维护一个数组 $Q$ 表示区间 $[j,i]$ 的极差减长度。即

$$
Q_j=\max_{j\le k\le i}P_k-\min_{j\le k\le i}P_k-(i-j),\ \ 0<j<i
$$

现在我们想知道在 $1\sim i-1$ 中是否存在一个最小的 $j$ 使得 $Q_j=0$。这等价于求 $Q_{1\sim i-1}$ 的最小值。求得最小的 $j$ 就是 $L_i$。如果没有,那么 $L_i=i$。

但是当第 $i$ 次增量结束时,我们需要快速把 $Q$ 数组更新到 i+1 的情况。原本的区间从 $[j,i]$ 变成 $[j,i+1]$,如果 $P_{i+1}>\max$ 或者 $P_{i+1}<\min$ 都会造成 $Q_j$ 发生变化。如何变化?如果 $P_{i+1}>\max$,相当于我们把 $Q_j$ 先减掉 $\max$ 再加上 $P_{i+1}$ 就完成了 $Q_j$ 的更新;$P_{i+1}<\min$ 同理,相当于 $Q_j=Q_j+\min-P_{i+1}$.

那么如果对于一个区间 $[x,y]$,满足 $P_{x\sim i},P_{x+1\sim i},P_{x+2\sim i},\cdots,P_{y\sim i}$ 的区间 $\max$ 都相同呢?你已经发现了,那么相当于我们在做一个区间加的操作;同理,当 $P_{x\sim i},P_{x+1\sim i},\cdots,P_{y\sim i}$ 的区间 $\min$ 都想同时也是一个区间加的操作。同时,$\max$ 和 $\min$ 的更新是相互独立的,因此可以各自更新。

因此我们对 $Q$ 的维护可以这样描述:

1. 找到最大的 $j$ 使得 $P_{j}>P_{i+1}$,那么显然,$P_{j+1\sim i}$ 这一段数全部小于 $P_{i+1}$,于是就需要更新 $Q_{j+1\sim i}$ 的最大值。由于 $P_{i},\max(P_i,P_{i-1}),\max(P_i,P_{i-1},P_{i-2}),\cdots,\max(P_i,P_{i-1},\cdots,P_{j+1})$ 是(非严格)单调递增的,因此可以每一段相同的 $\max$ 做相同的更新,即区间加操作。
2. 更新 $\min$ 同理。
3. 把每一个 $Q_j$ 都减 $1$。因为区间长度加 $1$。
4. 查询 $L_i$:即查询 $Q$ 的最小值的所在的 **下标**。

没错,我们可以使用线段树维护 $Q$!现在还有一个问题:怎么找到相同的一段使得他们的 $\max/\min$ 都相同?使用单调栈维护!维护两个单调栈分别表示 $\max/\min$。那么显然,栈中以相邻两个元素为端点的区间的 $\max/\min$ 是相同的,于是在维护单调栈的时侯顺便更新线段树即可。

具体的维护方法见代码。

讲这么多干巴巴的想必小伙伴也听得云里雾里的,那么我们就先上图吧。长图警告!

\begin{lstlisting}
#include <bits/stdc++.h>
#define rg register
using namespace std;
const int N = 200010;

int n, m, a[N], st1[N], st2[N], tp1, tp2, rt;
int L[N], R[N], M[N], id[N], cnt, typ[N], bin[20], st[N], tp;

// 本篇代码原题应为 CERC2017 Intrinsic Interval
// a 数组即为原题中对应的排列
// st1 和 st2 分别两个单调栈,tp1、tp2 为对应的栈顶,rt 为析合树的根
// L、R 数组表示该析合树节点的左右端点,M 数组的作用在析合树构造时有提到
// id 存储的是排列中某一位置对应的节点编号,typ 用于标记析点还是合点
// st 为存储析合树节点编号的栈,tp为其栈顶
struct RMQ {  // 预处理 RMQ(Max & Min)
  int lg[N], mn[N][17], mx[N][17];

  void chkmn(int& x, int y) {
    if (x > y) x = y;
  }

  void chkmx(int& x, int y) {
    if (x < y) x = y;
  }

  void build() {
    for (int i = bin[0] = 1; i < 20; ++i) bin[i] = bin[i - 1] << 1;
    for (int i = 2; i <= n; ++i) lg[i] = lg[i >> 1] + 1;
    for (int i = 1; i <= n; ++i) mn[i][0] = mx[i][0] = a[i];
    for (int i = 1; i < 17; ++i)
      for (int j = 1; j + bin[i] - 1 <= n; ++j)
        mn[j][i] = min(mn[j][i - 1], mn[j + bin[i - 1]][i - 1]),
        mx[j][i] = max(mx[j][i - 1], mx[j + bin[i - 1]][i - 1]);
  }

  int ask_mn(int l, int r) {
    int t = lg[r - l + 1];
    return min(mn[l][t], mn[r - bin[t] + 1][t]);
  }

  int ask_mx(int l, int r) {
    int t = lg[r - l + 1];
    return max(mx[l][t], mx[r - bin[t] + 1][t]);
  }
} D;

// 维护 L_i

struct SEG {  // 线段树
#define ls (k << 1)
#define rs (k << 1 | 1)
  int mn[N << 1], ly[N << 1];  // 区间加;区间最小值

  void pushup(int k) { mn[k] = min(mn[ls], mn[rs]); }

  void mfy(int k, int v) { mn[k] += v, ly[k] += v; }

  void pushdown(int k) {
    if (ly[k]) mfy(ls, ly[k]), mfy(rs, ly[k]), ly[k] = 0;
  }

  void update(int k, int l, int r, int x, int y, int v) {
    if (l == x && r == y) {
      mfy(k, v);
      return;
    }
    pushdown(k);
    int mid = (l + r) >> 1;
    if (y <= mid)
      update(ls, l, mid, x, y, v);
    else if (x > mid)
      update(rs, mid + 1, r, x, y, v);
    else
      update(ls, l, mid, x, mid, v), update(rs, mid + 1, r, mid + 1, y, v);
    pushup(k);
  }

  int query(int k, int l, int r) {  // 询问 0 的位置
    if (l == r) return l;
    pushdown(k);
    int mid = (l + r) >> 1;
    if (!mn[ls])
      return query(ls, l, mid);
    else
      return query(rs, mid + 1, r);
    // 如果不存在 0 的位置就会自动返回当前你查询的位置
  }
} T;

int o = 1, hd[N], dep[N], fa[N][18];

struct Edge {
  int v, nt;
} E[N << 1];

void add(int u, int v) {  // 树结构加边
  E[o] = (Edge){v, hd[u]};
  hd[u] = o++;
}

void dfs(int u) {
  for (int i = 1; bin[i] <= dep[u]; ++i) fa[u][i] = fa[fa[u][i - 1]][i - 1];
  for (int i = hd[u]; i; i = E[i].nt) {
    int v = E[i].v;
    dep[v] = dep[u] + 1;
    fa[v][0] = u;
    dfs(v);
  }
}

int go(int u, int d) {
  for (int i = 0; i < 18 && d; ++i)
    if (bin[i] & d) d ^= bin[i], u = fa[u][i];
  return u;
}

int lca(int u, int v) {
  if (dep[u] < dep[v]) swap(u, v);
  u = go(u, dep[u] - dep[v]);
  if (u == v) return u;
  for (int i = 17; ~i; --i)
    if (fa[u][i] != fa[v][i]) u = fa[u][i], v = fa[v][i];
  return fa[u][0];
}

// 判断当前区间是否为连续段
bool judge(int l, int r) { return D.ask_mx(l, r) - D.ask_mn(l, r) == r - l; }

// 建树
void build() {
  for (int i = 1; i <= n; ++i) {
    // 单调栈
    // 在区间 [st1[tp1-1]+1,st1[tp1]] 的最小值就是 a[st1[tp1]]
    // 现在把它出栈,意味着要把多减掉的 Min 加回来。
    // 线段树的叶结点位置 j 维护的是从 j 到当前的 i 的
    // Max{j,i}-Min{j,i}-(i-j)
    // 区间加只是一个 Tag。
    // 维护单调栈的目的是辅助线段树从 i-1 更新到 i。
    // 更新到 i 后,只需要查询全局最小值即可知道是否有解

    while (tp1 && a[i] <= a[st1[tp1]])  // 单调递增的栈,维护 Min
      T.update(1, 1, n, st1[tp1 - 1] + 1, st1[tp1], a[st1[tp1]]), tp1--;
    while (tp2 && a[i] >= a[st2[tp2]])
      T.update(1, 1, n, st2[tp2 - 1] + 1, st2[tp2], -a[st2[tp2]]), tp2--;

    T.update(1, 1, n, st1[tp1] + 1, i, -a[i]);
    st1[++tp1] = i;
    T.update(1, 1, n, st2[tp2] + 1, i, a[i]);
    st2[++tp2] = i;

    id[i] = ++cnt;
    L[cnt] = R[cnt] = i;  // 这里的 L,R 是指节点所对应区间的左右端点
    int le = T.query(1, 1, n), now = cnt;
    while (tp && L[st[tp]] >= le) {
      if (typ[st[tp]] && judge(M[st[tp]], i)) {
        // 判断是否能成为儿子,如果能就做
        R[st[tp]] = i, M[st[tp]] = L[now], add(st[tp], now), now = st[tp--];
      } else if (judge(L[st[tp]], i)) {
        typ[++cnt] = 1;  // 合点一定是被这样建出来的
        L[cnt] = L[st[tp]], R[cnt] = i, M[cnt] = L[now];
        // 这里M数组是记录节点最右面的儿子的左端点,用于上方能否成为儿子的判断
        add(cnt, st[tp--]), add(cnt, now);
        now = cnt;
      } else {
        add(++cnt, now);  // 新建一个结点,把 now 添加为儿子
        // 如果从当前结点开始不能构成连续段,就合并。
        // 直到找到一个结点能构成连续段。而且我们一定能找到这样
        // 一个结点。
        do add(cnt, st[tp--]);
        while (tp && !judge(L[st[tp]], i));
        L[cnt] = L[st[tp]], R[cnt] = i, add(cnt, st[tp--]);
        now = cnt;
      }
    }
    st[++tp] = now;  // 增量结束,把当前点压栈

    T.update(1, 1, n, 1, i, -1);  // 因为区间右端点向后移动一格,因此整体 -1
  }

  rt = st[1];  // 栈中最后剩下的点是根结点
}

void query(int l, int r) {
  int x = id[l], y = id[r];
  int z = lca(x, y);
  if (typ[z] & 1)
    l = L[go(x, dep[x] - dep[z] - 1)], r = R[go(y, dep[y] - dep[z] - 1)];
  // 合点这里特判的原因是因为这个合点不一定是最小的包含l,r的连续段.
  // 因为合点所代表的区间的子区间也都是连续段,而我们只需要其中的一段就够了。
  else
    l = L[z], r = R[z];
  printf("%d %d\n", l, r);
}  // 分 lca 为析或和,这里把叶子看成析的

int main() {
  scanf("%d", &n);
  for (int i = 1; i <= n; ++i) scanf("%d", &a[i]);
  D.build();
  build();
  dfs(rt);
  scanf("%d", &m);
  for (int i = 1; i <= m; ++i) {
    int x, y;
    scanf("%d%d", &x, &y);
    query(x, y);
  }
  return 0;
}

// 20190612
// 析合树
\end{lstlisting}

\subsection{弦图找错}

\begin{lstlisting}
#include <bits/stdc++.h>
using namespace std;
const int MAXN = 200005;
using lint = long long;
using pi = pair<int, int>;

// the algorithm may be wrong. if you have any ideas for proving / disproving this, please contact me.

vector<int> gph[MAXN];
int n, m, cnt[MAXN], idx[MAXN];
int mark[MAXN], vis[MAXN], par[MAXN];

void report(int x, int y){
	gph[x].erase(find(gph[x].begin(), gph[x].end(), y));
	gph[y].erase(find(gph[y].begin(), gph[y].end(), x));
	for(int i=1; i<=n; i++){
		if(binary_search(gph[i].begin(), gph[i].end(), x) && 
			binary_search(gph[i].begin(), gph[i].end(), y)){
			mark[i] = 1;
		}
	}
	queue<int> que;
	vis[x] = 1;
	que.push(x);
	while(!que.empty()){
		int x = que.front(); que.pop();
		for(auto &i : gph[x]){
			if(!mark[i] && !vis[i]){
				par[i] = x;
				vis[i] = 1;
				que.push(i);
			}
		}
	}
	assert(vis[y]);
	vector<int> v;
	while(y){
		v.push_back(y);
		y = par[y];
	}
	printf("NO\n%d\n", v.size());
	for(auto &i : v) printf("%d ", i-1);
}

int main(){
	scanf("%d %d",&n,&m);
	for(int i=0; i<m; i++){
		int s, e; scanf("%d %d",&s,&e);
		s++, e++;
		gph[s].push_back(e);
		gph[e].push_back(s);
	}
	for(int i=1; i<=n; i++) sort(gph[i].begin(), gph[i].end());
	priority_queue<pi> pq;
	for(int i=1; i<=n; i++) pq.emplace(cnt[i], i);
	vector<int> ord;
	while(!pq.empty()){
		int x = pq.top().second, y = pq.top().first;
		pq.pop();
		if(cnt[x] != y || idx[x]) continue;
		ord.push_back(x);
		idx[x] = n + 1 - ord.size();
		for(auto &i : gph[x]){
			if(!idx[i]){
				cnt[i]++;
				pq.emplace(cnt[i], i);
			}
		}
	}
	reverse(ord.begin(), ord.end());
	for(auto &i : ord){
		int minBef = 1e9;
		for(auto &j : gph[i]){
			if(idx[j] > idx[i]) minBef = min(minBef, idx[j]);
		}
		minBef--;
		if(minBef < n){
			minBef = ord[minBef];
			for(auto &j : gph[i]){
				if(idx[j] > idx[minBef] && !binary_search(gph[minBef].begin(), gph[minBef].end(), j)){
					report(minBef, i);
					return 0;
				}
			}
		}
	}
	puts("YES");
	for(auto &i : ord) printf("%d ", i-1);
}


\end{lstlisting}

\subsection{$O(\frac{nm}{\omega})$ LCS}

\begin{lstlisting}
/*
 * Author : _Wallace_
 * Source : https://www.cnblogs.com/-Wallace-/
 * Problem : LOJ #6564. 最长公共子序列
 * Standard : GNU C++ 03
 * Optimal : -Ofast
 */
#include <algorithm>
#include <cstddef>
#include <cstdio>
#include <cstring>

typedef unsigned long long ULL;

const int N = 7e4 + 5;
int n, m, u;

struct bitset {
  ULL t[N / 64 + 5];

  bitset() {
    memset(t, 0, sizeof(t));
  }
  bitset(const bitset &rhs) {
    memcpy(t, rhs.t, sizeof(t));
  }

  bitset& set(int p) {
    t[p >> 6] |= 1llu << (p & 63);
    return *this;
  }
  bitset& shift() {
    ULL last = 0llu;
    for (int i = 0; i < u; i++) {
      ULL cur = t[i] >> 63;
      (t[i] <<= 1) |= last, last = cur;
    }
    return *this;
  }
  int count() {
    int ret = 0;
    for (int i = 0; i < u; i++)
      ret += __builtin_popcountll(t[i]);
    return ret;
  }

  bitset& operator = (const bitset &rhs) {
    memcpy(t, rhs.t, sizeof(t));
    return *this;
  }
  bitset& operator &= (const bitset &rhs) {
    for (int i = 0; i < u; i++) t[i] &= rhs.t[i];
    return *this;
  }
  bitset& operator |= (const bitset &rhs) {
    for (int i = 0; i < u; i++) t[i] |= rhs.t[i];
    return *this;
  }
  bitset& operator ^= (const bitset &rhs) {
    for (int i = 0; i < u; i++) t[i] ^= rhs.t[i];
    return *this;
  }

  friend bitset operator - (const bitset &lhs, const bitset &rhs) {
    ULL last = 0llu; bitset ret;
    for (int i = 0; i < u; i++){
      ULL cur = (lhs.t[i] < rhs.t[i] + last);
      ret.t[i] = lhs.t[i] - rhs.t[i] - last;
      last = cur;
    }
    return ret;
  }
} p[N], f, g;

signed main() {
  scanf("%d%d", &n, &m), u = n / 64 + 1;
  for (int i = 1, c; i <= n; i++)
    scanf("%d", &c), p[c].set(i);
  for (int i = 1, c; i <= m; i++) {
    scanf("%d", &c), (g = f) |= p[c];
    f.shift(), f.set(0);
    ((f = g - f) ^= g) &= g;
  }
  printf("%d\n", f.count());
  return 0;
}
\end{lstlisting}

另一个实现

\begin{lstlisting}

#include <bits/stdc++.h>
#pragma GCC target("popcnt,bmi")

using namespace std;
using ull = uint64_t;

const int N = 70005, M = 1136;

int n, m;
ull g[N][M], f[M];

int read() {
    const int M = 1e6;
    static streambuf *in = cin.rdbuf();
#define gc (p1 == p2 && (p2 = (p1 = buf) + in -> sgetn(buf, M), p1 == p2) ? -1 : *p1++)
    static char buf[M], *p1, *p2;
    int c = gc, r = 0;

    while (c < 48)
        c = gc;

    while (c > 47)
        r = r * 10 + (c & 15), c = gc;

    return r;
}
int main() {
    cin.tie(0)->sync_with_stdio(0);
    cin >> n >> m;

    for (int i = 0; i < n; i++)
        g[read()][i / 62] |= 1ULL << (i % 62);

    int lim = (n - 1) / 62;

    for (int i = 0; i < m; i++) {
        int c = 1;
        auto can = g[read()];

        for (int j = 0; j <= lim; j++) {
            ull x = f[j], y = x | can[j];
            x += x + c + (~y & (1ULL << 62) - 1);
            f[j] = x & y, c = x >> 62;
        }
    }

    int ans = 0;

    for (int i = 0; i <= lim; i++)
        ans += __builtin_popcountll(f[i]);

    cout << ans;
}
\end{lstlisting}

\subsection{区间 LIS(排列)}

\begin{lstlisting}


#include<bits/stdc++.h>
using namespace std;
//dengyaotriangle!

const int maxn=100005;

int pool[(int)5e7];int ps;
inline int *aloc(int x){
    ps+=x;return pool+ps-x;
}
void unit_monge_mult(int *a,int *b,int *r,int n){
    if(n==2){
        if(a[0]==0&&b[0]==0)r[0]=0,r[1]=1;
        else r[0]=1,r[1]=0;
        return;
    }
    if(n==1){r[0]=0;return;}
    int lps=ps;
    int d=n/2;
    int *a1=aloc(d),*a2=aloc(n-d),*b1=aloc(d),*b2=aloc(n-d);
    int *mpa1=aloc(d),*mpa2=aloc(n-d),*mpb1=aloc(d),*mpb2=aloc(n-d);
    int p[2]={0,0};
    for(int i=0;i<n;i++){
        if(a[i]<d)a1[p[0]]=a[i],mpa1[p[0]]=i,p[0]++;
        else a2[p[1]]=a[i]-d,mpa2[p[1]]=i,p[1]++;
    }
    p[0]=p[1]=0;
    for(int i=0;i<n;i++){
        if(b[i]<d)b1[p[0]]=b[i],mpb1[p[0]]=i,p[0]++;
        else b2[p[1]]=b[i]-d,mpb2[p[1]]=i,p[1]++;
    }
    int *c1=aloc(d),*c2=aloc(n-d);
    unit_monge_mult(a1,b1,c1,d),unit_monge_mult(a2,b2,c2,n-d);
    int *cpx=aloc(n),*cpy=aloc(n),*cqx=aloc(n),*cqy=aloc(n);
    for(int i=0;i<d;i++)cpx[mpa1[i]]=mpb1[c1[i]],cpy[mpa1[i]]=0;
    for(int i=0;i<n-d;i++)cpx[mpa2[i]]=mpb2[c2[i]],cpy[mpa2[i]]=1;
    for(int i=0;i<n;i++)r[i]=cpx[i];
    for(int i=0;i<n;i++)cqx[cpx[i]]=i,cqy[cpx[i]]=cpy[i];
    int hi=n,lo=n,his=0,los=0;
    for(int i=0;i<n;i++){
        if(cqy[i]^(cqx[i]>=hi))his--;
        while(hi>0&&his<0){
            hi--;
            if(cpy[hi]^(cpx[hi]>i))his++;
        }
        while(lo>0&&los<=0){
            lo--;
            if(cpy[lo]^(cpx[lo]>=i))los++;
        }
        if(los>0&&hi==lo)r[lo]=i;
        if(cqy[i]^(cqx[i]>=lo))los--;
    }
    ps=lps;
}
void subunit_monge_mult(int*a,int*b,int*c,int n){
    int lps=ps;
    int *za=aloc(n),*zb=aloc(n),*res=aloc(n),*vis=aloc(n),*mpa=aloc(n),*mpb=aloc(n),*rb=aloc(n);
    memset(vis,0,sizeof(int)*n);
    memset(mpa,-1,sizeof(int)*n);
    memset(mpb,-1,sizeof(int)*n);
    memset(rb,-1,sizeof(int)*n);
    int ca=n;
    for(int i=n-1;i>=0;i--)if(a[i]!=-1){
        vis[a[i]]=1;ca--;za[ca]=a[i];mpa[ca]=i;
    }
    for(int i=n-1;i>=0;i--)if(!vis[i])za[--ca]=i;
    memset(vis,-1,sizeof(int)*n);
    for(int i=0;i<n;i++)if(b[i]!=-1)vis[b[i]]=i;
    ca=0;
    for(int i=0;i<n;i++)if(vis[i]!=-1){
        mpb[ca]=i;rb[vis[i]]=ca++;
    }
    for(int i=0;i<n;i++)if(rb[i]==-1)rb[i]=ca++;
    for(int i=0;i<n;i++)zb[rb[i]]=i;
    unit_monge_mult(za,zb,res,n);
    memset(c,-1,sizeof(int)*n);
    for(int i=0;i<n;i++)if(mpa[i]!=-1&&mpb[res[i]]!=-1)c[mpa[i]]=mpb[res[i]];
    ps=lps;
}


void solve(int *p,int *ret,int n){
    if(n==1){ret[0]=-1;return;}
    int lps=ps,d=n/2;
    int *pl=aloc(d),*pr=aloc(n-d);
    for(int i=0;i<d;i++)pl[i]=p[i];
    for(int i=0;i<n-d;i++)pr[i]=p[i+d];
    int *vis=aloc(n);memset(vis,-1,sizeof(int)*n);
    for(int i=0;i<d;i++)vis[pl[i]]=i;
    int *tl=aloc(d),*tr=aloc(n-d),*mpl=aloc(d),*mpr=aloc(n-d);
    int ca=0;
    for(int i=0;i<n;i++)if(vis[i]!=-1)mpl[ca]=i,tl[vis[i]]=ca++;
    ca=0;memset(vis,-1,sizeof(int)*n);
    for(int i=0;i<n-d;i++)vis[pr[i]]=i;
    for(int i=0;i<n;i++)if(vis[i]!=-1)mpr[ca]=i,tr[vis[i]]=ca++;
    int *vl=aloc(d),*vr=aloc(n-d);
    solve(tl,vl,d),solve(tr,vr,n-d);
    int *sl=aloc(n),*sr=aloc(n);
    iota(sl,sl+n,0);iota(sr,sr+n,0);
    for(int i=0;i<d;i++)sl[mpl[i]]=(vl[i]==-1?-1:mpl[vl[i]]);
    for(int i=0;i<n-d;i++)sr[mpr[i]]=(vr[i]==-1?-1:mpr[vr[i]]);
    subunit_monge_mult(sl,sr,ret,n);
    ps=lps;
}
int invp[maxn],res_monge[maxn];
int main(){
    ios::sync_with_stdio(0);cin.tie(0);
    int n,q;
    cin>>n>>q;
    vector<int> a(n);
    for(int i=0;i<n;i++)cin>>a[i],invp[a[i]]=i;
    solve(invp,res_monge,n);
    vector<int> fwk(n+1),ans(q);
    vector<vector<pair<int,int> > > qry(n+1);
    for(int i=0;i<q;i++){
        int l,r;
        cin>>l>>r;
        qry[l].push_back({r,i});
        ans[i]=r-l;
    }
    for(int i=n-1;i>=0;i--){
        if(res_monge[i]!=-1){
            for(int p=res_monge[i]+1;p<=n;p+=p&-p)fwk[p]++;
        }
        for(auto& z:qry[i]){
            int id,c;tie(id,c)=z;
            for(int p=id;p;p-=p&-p)ans[c]-=fwk[p];
        }
    }
    for(int i=0;i<q;i++)cout<<ans[i]<<'\n';
    return 0;
}
\end{lstlisting}

\subsection{区间 LCS}

$s_{[0,a)}$ 和 $t_{[b,c)}$ 的 LCS

\begin{lstlisting}


#include<bits/stdc++.h>
using namespace std;
//dengyaotriangle!

const int maxn=1005;
const int maxq=500005;
int n,m,q;
char a[maxn],b[maxn];
struct qryt{
    int x,nxt;
}z[maxq];
int qry[maxn][maxn];
int ans[maxq];
int r[maxn];
int bit[maxn];

int main(){
    ios::sync_with_stdio(0);cin.tie(0);
    cin>>q>>b>>a;n=strlen(a);m=strlen(b);
	//q,s,t
    for(int i=1;i<=q;i++){
        int a,b,c;
        cin>>a>>b>>c;
        if(a){
            ans[i]=c-b;
            z[i].x=b;z[i].nxt=qry[a][c];
            qry[a][c]=i;
        }
    }    
    for(int i=0;i<n;i++)r[i]=i;
    for(int i=0;i<m;i++){
        int lp=-1;
        for(int j=0;j<n;j++)if(a[j]==b[i]){lp=j;break;}
        if(lp!=-1){
            for(int j=lp+1;j<n;j++){
                if(a[j]!=b[i]){
                    if(r[j-1]<r[j])swap(r[j-1],r[j]);
                }
            }
            for(int i=n-1;i>lp;i--)r[i]=r[i-1];
            r[lp]=-1;
        }
        for(int i=0;i<=n;i++)bit[i]=0;
        for(int j=0;j<n;j++){
            if(r[j]!=-1){
                for(int p=n-r[j];p<=n;p+=p&-p)bit[p]++;
            }
            for(int y=qry[i+1][j+1];y;y=z[y].nxt){
                for(int p=n-z[y].x;p;p-=p&-p)ans[y]-=bit[p];
            }
        }
    }
    for(int i=1;i<=q;i++)cout<<ans[i]<<'\n';
    return 0;
}
\end{lstlisting}

\subsection{毛毛虫剖分}
毛毛虫剖分,一种由轻重链剖分(HLD)推广而成的树上结点重标号方法,支持修改 / 查询一只毛毛虫的信息,并且可以对毛毛虫的身体和足分别修改 / 查询不同信息 .

严格强于树剖,而且复杂度和树剖一样哦!

一些定义(默认在一棵树上):

毛毛虫:一条链和与这条链邻接的所有结点构成的集合 .
虫身(身体):毛毛虫的链部分 .
虫足(足):毛毛虫除虫身的部分 .
重标号方法
首先重剖求出重链 .
DFS,若现在处理到结点 u:
若 u 还未被标号,则为其标号 .
若 u 是重链头,遍历这条重链,将邻接这条链的结点依次标号 .
先递归重儿子,再递归轻儿子 .
重标号性质
对于重链,除链头外的结点标号连续 .
对于任意结点,其轻儿子标号连续 .
对于以重链头为根的子树,与这条重链邻接的所有结点标号连续 .
这样就可以随便维护毛毛虫信息了,顺便还能维护链信息,子树信息等 .

时间复杂度同轻重链剖分 .
\end{document}
